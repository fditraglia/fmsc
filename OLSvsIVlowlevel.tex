%!TEX root = main.tex
\subsection{Example: OLS versus TSLS}
\label{sec:OLSvsIVlowlevel}
The simplest interesting application of the FMSC is choosing between ordinary least squares (OLS) and two-stage least squares (TSLS) estimators of the effect $\beta$ of a single endogenous regressor $x$ on an outcome of interest $y$. 
To keep the presentation transparent, I work within an iid, homoskedastic setting. 
Neither of these restrictions, however, is necessary. 
Without loss of generality assume that there are no exogenous regressors, or equivalently that they have been ``projected out of the system'' so that the data generating process is given by
    \begin{eqnarray}
			y_{i} &=& \beta x_{i}  + \epsilon_{i}\\
	x_{i} &=& \mathbf{z}_{i}' \boldsymbol{\pi} + v_{i}
	\end{eqnarray}
where $\beta$ and $\boldsymbol{\pi}$ are unknown constants, $\mathbf{z}_i$ is a vector of exogenous and relevant instruments, $x_i$ is the endogenous regressor, $y_i$ is the outcome of interest, and $\epsilon_i, v_i$ are unobservable error terms. 
All random variables in this system are mean zero, or equivalently all constant terms have been projected out. 
Stacking observations in the usual way, let 
$\mathbf{z}' = (z_{1}, \hdots, z_{n})$, $Z' = (\mathbf{z}_{1}, \hdots, \mathbf{z}_{n})$, $\mathbf{x}' = (x_{1}, \hdots, x_{n})$ and so on. The two estimators under consideration are $\widehat{\beta}_{OLS}$ and $\widetilde{\beta}_{TSLS}$, given by
  \begin{eqnarray} 
  \label{eq:OLS}
		\widehat{\beta}_{OLS} &=& \left(\mathbf{x}'\mathbf{x}\right)^{-1}\mathbf{x}'\mathbf{y}\\
	\label{eq:TSLS}
		\widetilde{\beta}_{TSLS} &=& \left(\mathbf{x}'P_Z\mathbf{x}\right)^{-1}\mathbf{x}'P_Z\mathbf{y}
	\end{eqnarray}
where $P_Z = Z(Z'Z)^{-1}Z'$. 
But why should we ever \emph{consider} using OLS if $x$ is endogenous and we have valid instruments at our disposal? 
The answer is simple: TSLS is a high variance estimator. 
Depending on the degree of endogeneity present in $x$ and the strength of the instruments, $\widehat{\beta}_{OLS}$ could easily have the lower mean-squared error of the two estimators.\footnote{Because the moments of the TSLS estimator only exist up to the order of overidentificiation \citep{Phillips1980} this statement should be understood to refer to ``trimmed'' mean-squared error when the number of instruments is two or fewer. See, e.g., \cite{Hansen2013}.} 
The appropriate version of the local mis-specification assumption for this example is as follows
  \begin{equation}
     E \left[\begin{array}{c} \mathbf{z}_{ni} \epsilon_{ni} \\ x_{ni} \epsilon_{ni} \end{array}\right] = \left[\begin{array}{c} \mathbf{0} \\ \tau/\sqrt{n} \end{array}\right]
  \end{equation}
where $\mathbf{0}$ is a vector of zeros, and $\tau$ is an unknown constant that encodes the endogeneity of $\tau$. 
In this setting, there is only a single moment condition in the $h$--block, namely $E[x_{ni}\epsilon_{ni}] = \tau/\sqrt{n}$. 
The question is not whether we should use this moment condition \emph{in addition} to the TSLS moment conditions written above it, but rather whether we should use it \emph{instead} of them.
The following simple low-level conditions are sufficient for the asymptotic normality of the OLS and TSLS estimators in this example.

\begin{assump}[OLS versus TSLS]
\label{assump:OLSvsIV}
	Let $\{(\mathbf{z}_{ni}, v_{ni}, \epsilon_{ni})\colon 1\leq i \leq n, n = 1, 2, \hdots\}$ be a triangular array of random variables such that
	\begin{enumerate}[(a)]
		\item $(\mathbf{z}_{ni}, v_{ni}, \epsilon_{ni}) \sim$ iid and mean zero within each row of the array (i.e.\ for fixed $n$)
		\item $E[\mathbf{z}_{ni} \epsilon_{ni}]=\mathbf{0}$, $E[\mathbf{z}_{ni} v_{ni}]=\mathbf{0}$, and $E[\epsilon_{ni}v_{ni}] = \tau/\sqrt{n}$ for all $n$
		\item $E[\left|\mathbf{z}_{ni}\right|^{4+\eta}] <C$, $E[\left|\epsilon_{ni}\right|^{4+\eta}] <C$, and $E[\left|v_{ni}\right|^{4+\eta}] <C$ for some $\eta >0$, $C <\infty$
		\item $E[\mathbf{z}_{ni} \mathbf{z}_{ni}'] \rightarrow Q>0$, $E[v_{ni}^2]\rightarrow \sigma_v^2 >0$, and $E[\epsilon_{ni}^2] \rightarrow \sigma_\epsilon^2 >0$ as $n\rightarrow \infty$
		\item As $n\rightarrow \infty$, $E[\epsilon_{ni}^2 \mathbf{z}_{ni} \mathbf{z}_{ni}']- E[\epsilon_{ni}^2]E[ \mathbf{z}_{ni} \mathbf{z}_{ni}'] \rightarrow 0$, $E[\epsilon_i^2 v_{ni} \mathbf{z}_{ni}'] - E[\epsilon_{ni}^2]E[v_{ni} \mathbf{z}_{ni}'] \rightarrow 0$, and $E[\epsilon_{ni}^2 v_{ni}^2] - E[\epsilon_{ni}^2]E[v_{ni}^2] \rightarrow 0$
		\item $x_{ni} = \mathbf{z}_{ni}'\boldsymbol{\pi} + v_i$ where $\boldsymbol{\pi} \neq \mathbf{0}$, and $y_{ni} = \beta x_{ni} + \epsilon_{ni}$
	\end{enumerate}
\end{assump}

Parts (a), (b) and (d) correspond to the local mis-specification assumption, part (c) is a set of moment restrictions, and (f) is simply the DGP.
Part (e) is the homoskedasticity assumption: an \emph{asymptotic} restriction on the joint distribution of $v_{ni}$, $\epsilon_{ni}$, and $\mathbf{z}_{ni}$. 
This condition holds automatically, given the other asssumptions, if $(\mathbf{z}_{ni}, v_{ni}, \epsilon_{ni})$ are jointly normal, as in our simulation experiment described below. 

\begin{thm}[OLS and TSLS Limit Distributions]
	\label{thm:OLSvsIV} Under Assumption \ref{assump:OLSvsIV},
	$$
\left[
\begin{array}{c}
  \sqrt{n}(\widehat{\beta}_{OLS} - \beta) \\
  \sqrt{n}(\widetilde{\beta}_{TSLS} - \beta)
\end{array}
\right] \overset{d}{\rightarrow}
N\left(
\left[
\begin{array}{c}
\tau/\sigma_x^2 \\ 
0
\end{array}
\right],\;
\sigma_\epsilon^2 \left[ \begin{array}{cc}
  1/\sigma_x^2 & 1/\sigma_x^2\\
  1/\sigma_x^2 & 1/\gamma^2 
  \end{array}\right]
  \right)
$$
where $\sigma_x^2 = \gamma^2 + \sigma_v^2$, $\gamma^2 = \boldsymbol{\pi}'Q \boldsymbol{\pi}$, and $Q$, $ \sigma_\epsilon^2$, and $\sigma_v^2$ are defined in Assumption \ref{assump:OLSvsIV}.
\end{thm}

We see from the preceding result that the variance of the OLS estimator is always strictly lower than that of the TSLS estimator since $\sigma^2_\epsilon/\sigma_x^2 = \sigma^2_\epsilon/(\gamma^2 + \sigma_v^2)$. 
Unless $\tau = 0$, however, OLS shows an asymptotic bias. 
In contrast, the TSLS estimator is asymptotically unbiased regardless of the value of $\tau$.  