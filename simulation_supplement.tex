%!TEX root = main.tex
\section{Supplementary Simulation Results}

\subsection{Canonical Correlations Information Criterion}
\label{sec:CCIC}
Because the GMM moment selection criteria suggested by \cite{Andrews1999} consider only instrument exogeneity, not relevance, \cite{HallPeixe2003} suggest combining them with their canonical correlations information criterion (CCIC), which aims to detect and eliminate ``redundant instruments.''
Including such instruments, which add no information beyond that already contained in the other instruments, can lead to poor finite-sample performance in spite of the fact that the first-order limit distribution is unchanged.
For the choosing instrumental variables simulation example, presented in Section \ref{sec:chooseIVsim}, the CCIC takes the following simple form
	\begin{equation}
	\mbox{CCIC}(S) = n \log\left[1 - R_n^2(S) \right] + h(p + |S|)\kappa_n
	\end{equation}
where $R_n^2(S)$ is the first-stage $R^2$ based on instrument set $S$ and $h(p + |S|)\kappa_n$ is a penalty term \citep{Jana2005}. 
Instruments are chosen to \emph{minimize} this criterion.
If we define $h(p + |S|) = (p + |S| - r)$, setting $\kappa_n = \log{n}$ gives the CCIC-BIC, while $\kappa_n = 2.01 \log{\log{n}}$ gives the CCIC-HQ and $\kappa_n = 2$ gives the CCIC-AIC.
By combining the CCIC with an Andrews-type criterion, \cite{HallPeixe2003} propose to first eliminate invalid instruments and then redundant ones.
A combined GMM-BIC/CCIC-BIC criterion for the simulation example from section \ref{sec:chooseIVsim} uses the valid estimator unless both the GMM-BIC \emph{and} CCIC-BIC select the full estimator.
Combined HQ and AIC-type procedures can be defined analogously.
In the simulation design from this paper, however, \emph{each} of these combined criteria gives results that are practically identical to those of the valid estimator.
This hold true across all parameter values and sample sizes.
Full simulation results are available upon request.
