%!TEX root = main.tex
\section{Conclusion}
\label{sec:conclude}
This paper has introduced the FMSC, a proposal to choose moment conditions using AMSE. 
The criterion performs well in simulations, and the framework under which it is derived can be used produce confidence intervals that adjust for the effects of moment selection on subsequent inference. 
Moment selection is not a panacea, but the FMSC and related confidence interval procedures can yield sizeable benefits in empirically relevant settings.
While the discussion here concentrates on two cross-section examples, the FMSC could prove useful in any context in which moment conditions arise from more than one source. 
In a panel model, for example, the assumption of contemporaneously exogenous instruments may be plausible while that of predetermined instruments is more dubious.
Using the FMSC, we could assess whether the extra information contained in the lagged instruments outweighs their potential invalidity. 
Work in progress explores this idea in both static and dynamic panel settings by extending the FMSC to allow for simultaneous moment and model selection.
Other potentially fruitful extensions include the consideration of risk functions other than MSE, and an explicit treatment of weak identification and many moment conditions.
% In a macro model, measurement error could be present in the intra--Euler equation but not the \emph{inter}--Euler equation, as considered by \cite{Eichenbaum}. 
% The FMSC could be used to select over the intra-Euler moment conditions.
