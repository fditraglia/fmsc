%!TEX root = main.tex
\section{The Focused Moment Selection Criterion}
\label{sec:FMSC}

\subsection{The General Case}
The FMSC chooses among the potentially invalid moment conditions contained in $h$ to minimize estimator AMSE for a target parameter. 
Denote this target parameter by $\mu$, a real-valued, $Z$-almost continuous function of the parameter vector $\theta$ that is differentiable in a neighborhood of $\theta_0$. 
Further, define the GMM estimator of $\mu$ based on $\widehat{\theta}_S$ by $\widehat{\mu}_S = \mu(\widehat{\theta}_S)$ and the true value of $\mu$ by $\mu_0 = \mu(\theta_0)$. 
Applying the Delta Method to Theorem \ref{thm:normality} gives the AMSE of $\widehat{\mu}_S$.

\begin{cor}[AMSE of Target Parameter]
\label{cor:target}
Under the hypotheses of Theorem \ref{thm:normality}, 
$$\sqrt{n}\left(\widehat{\mu}_S - \mu_0\right)\rightarrow_d-\nabla_\theta\mu(\theta_0)'K_S \Xi_S \left(M +  \left[\begin{array}
	{c} 0 \\ \tau
\end{array} \right]\right)$$ 
where $M$ is defined in Theorem \ref{thm:normality}.
Hence,
	$$\mbox{AMSE}\left(\widehat{\mu}_S\right) = \nabla_\theta\mu(\theta_0)'K_S \Xi_S \left\{\left[\begin{array}{cc}0&0\\0&\tau\tau'\end{array}\right] + \Omega\right\}\Xi_S'K_S'\nabla_\theta\mu(\theta_0).$$
\end{cor}

For the valid estimator $\widehat{\theta}_v$ we have $K_v = \left[G'W_{v}G\right]^{-1}G' W_{v}$ and $\Xi_v =\left[\begin{array}{cc} \mathbf{I}_p& \mathbf{0}_{p\times q} \end{array} \right]$. 
Thus, the valid estimator $\widehat{\mu}_v$ of $\mu$ has zero asymptotic bias. 
In contrast, any candidate estimator $\widehat{\mu}_S$ that includes moment conditions from $h$ inherits an asymptotic bias from the corresponding elements of $\tau$. 
We see that the extent and direction of this bias depends both on $K_S$ and $\nabla_\theta\mu(\theta_0)$. 
Adding moment conditions from $h$, however, generally decreases asymptotic variance. 
In particular, the usual proof that adding moment conditions cannot increase asymptotic variance under efficient GMM \citep[see for example][ch.\ 6]{Hallbook} continues to hold under local mis-specification, because all moment conditions are correctly specified in the limit. 
Thus, we see that local mis-specification gives an asymptotic analogue of the bias-variance tradeoff that we encounter in finite samples.\footnote{The general result for adding moment conditions in GMM is only relevant in situations where the valid moment set is strictly nested inside of all other candidate moment sets. When this does not hold, such as in the OLS verus IV example, we establish an analogous ordering of asymptotic variances by direct calculation.} 

To use this framework for moment selection, we need to construct estimators of the unknown quantities: $\theta_0$, $K_S$, $\Omega$, and $\tau$. 
Under local mis-specification, the estimator of $\theta$ under \emph{any} moment set is consistent. 
A natural estimator is $\widehat{\theta}_v$, although there are other possibilities. 
Recall that $K_S = [F_S'W_SF_S]^{-1} F_S'W_S \Xi_S$.
Now, $\Xi_S$ is known because it is simply the selection matrix defining moment set $S$. 
The remaining quantities $F_S$ and $W_S$ that make up $K_S$ are consistently estimated by their sample analogues under Assumption \ref{assump:highlevel}.
Similarly, consistent estimators of $\Omega$ are readily available under local mis-specification, although the precise form depends on the situation.
We consider this point further below as it relates to our two running examples.

The only remaining unknown is $\tau$. Local mis-specification is essential for making meaningful comparisons of AMSE because it prevents the bias term from dominating the comparison. 
Unfortunately, it also prevents us from consistently estimating this asymptotic bias parameter. 
Under Assumption \ref{assump:Identification}, however, we can construct an \emph{asymptotically unbiased} estimator $\widehat{\tau}$ of $\tau$ by substituting $\widehat{\theta}_v$, the estimator of $\theta_0$ that uses only correctly specified moment conditions, into $h_n$, the sample analogue of the potentially mis-specified moment conditions. 
In other words,  $\widehat{\tau} = \sqrt{n} h_n(\widehat{\theta}_v)$. 

\begin{thm}[Asymptotic Distribution of $\widehat{\tau}$] 
\label{thm:tau}
Let $\widehat{\tau} = \sqrt{n} h_n(\widehat{\theta}_v)$ where $\widehat{\theta}_v$ is the valid estimator, based only on the moment conditions contained in $g$. 
Then under Assumptions \ref{assump:drift}, \ref{assump:highlevel} and \ref{assump:Identification}
$$\widehat{\tau} \rightarrow_d \Psi\left( M + \left[\begin{array}
	{c} 0 \\ \tau
\end{array} \right]\right), \quad \Psi = \left[\begin{array}{cc} -HK_v & \mathbf{I}_q \end{array}\right]$$ 
where $K_v$ is defined in Corollary \ref{cor:valid}.
Thus, $\widehat{\tau}\rightarrow_d (\Psi M + \tau) \sim N_q(\tau, \Psi \Omega \Psi')$.
\end{thm}

Returning to Corollary $\ref{cor:target}$, however, we see that it is $\tau \tau'$ rather than $\tau$ that enters the expression for AMSE. 
Although $\widehat{\tau}$ is an asymptotically unbiased estimator of $\tau$, the limiting expectation of $\widehat{\tau} \widehat{\tau}'$ is not $\tau\tau'$ because $\widehat{\tau}$ has an asymptotic variance.  
To obtain an asymptotically unbiased estimator of $\tau\tau'$ we proceed as follows, subtracting a consistent estimate of the asymptotic variance.


\begin{cor}[Asymptotically Unbiased Estimator of $\tau \tau'$]
\label{cor:tautau}
If $\widehat{\Omega}$ and $\widehat{\Psi}$ are consistent for $\Omega$ and $\Psi$, then $ \widehat{\tau}\widehat{\tau}' - \widehat{\Psi}\widehat{\Omega}\widehat{\Psi}$ is an asymptotically unbiased estimator of $\tau\tau'$.
\end{cor}
It follows that
\begin{equation}
\label{eq:fmsc}
	\mbox{FMSC}_n(S) = \nabla_\theta\mu(\widehat{\theta})'\widehat{K}_S\Xi_S \left\{\left[\begin{array}{cc}0&0\\0&\widehat{\tau}\widehat{\tau}' - \widehat{\Psi}\widehat{\Omega}\widehat{\Psi}'\end{array}\right] + \widehat{\Omega}\right\}\Xi_S'\widehat{K}_S' \nabla_\theta\mu(\widehat{\theta})
\end{equation}
provides an asymptotically unbiased estimator of AMSE.
Given a set $\mathscr{S}$ of candidate specifications, the FMSC selects the candidate $S^*$ that \emph{minimizes} the expression given in Equation \ref{eq:fmsc}, that is $S^*_{FMSC} =  \arg \min_{S\in \mathscr{S}} \;\mbox{FMSC}_n(S)$.

At this point, it is worth taking a brief pause to survey the ground covered thus far.
We began with a target parameter, $\mu$, a risk function, mean-squared error, and a collection of candidate estimators, $\widehat{\mu}_S$ for $S\in\mathcal{S}$. 
Our goal was to choose the estimator with the lowest risk.
Because finite-sample distributions were unavailable, we resorted to an asymptotic experiment, local mis-specification, that preserved the bias-variance tradeoff embodied in our chosen risk function.
We then calculated the risk of the \emph{limit distribution} of $\widehat{\mu}_S$ to use as a stand-in for the finite-sample risk. 
This quantity involved several unknown parameters.
We estimated these in such a way that the resulting asymptotic risk estimate would converge in distribution to a random variable with mean equal to the true asymptotic risk.
The result was the FMSC: an asymptotically unbiased estimator of the AMSE of $\widehat{\mu}_S$.
Viewing the FMSC at this level of abstraction raises two questions.
First, could we have chosen a risk function other than mean-squared error?
Second, why should we use an \emph{asymptotically unbiased} risk estimator? 

The answer to the first question is a straightforward yes.
The idea of using asymptotic risk to stand-in for finite sample risk requires only that we can characterize the limit distribution of each $\widehat{\mu}_S$ and use it to evaluate the chosen risk function.
\cite{Claeskens2006} and \cite{ClaeskensHjort2008}, for example, show how the FIC for model selection in maximum likelihood models can be extended from squared error to $L_p$ and linex loss, respectively, in precisely this way.
One could easily do the same for the FMSC although we will not consider this possibility futher here.

Answering the second question is more difficult.
Under local mis-specification it is impossible to consistently estimate AMSE.\footnote{This is not a defect of the FMSC: there is a fundamental trade-off between consistency and desirable risk properties. See Section \ref{sec:avg} for a discussion of this point.}
If we merely use the plug-in estimator of the squared asymptotic bias based on $\widehat{\tau}$, the resulting AMSE estimate will ``overshoot'' asymptotically.
Accordingly, it seems natural to correct this bias as explained in Corollary \ref{cor:tautau}.
This is the same heuristic that underlies the classical AIC and TIC model selection criteria as well as more recent procedures such as those described in \cite{ClaeskensHjort2003} and \cite{Schorfheide2005}.
A heuristic, however, is not a rigorous justification: there could certainly be situations in which it makes sense to use a risk estimator other than the asymptotically unbiased one suggested here.
If we wish to consider risk functions other than MSE, to take a simple example, it may not be possible to derive an asymptotically unbiased risk estimator.
The plug-in estimator, however, is always available.
Although we will not consider them further below, alternative risk estimators in the FMSC framework could be an interesting topic for future research.

\subsection{Digression: Failure of the Identification Condition}
\label{subsec:digress}
When $r > p$, Assumption \ref{assump:Identification} fails: $\theta_0$ is not estimable by $\widehat{\theta}_v$ so $\widehat{\tau}$ is an infeasible estimator of $\tau$. 
A na\"{i}ve approach to this problem would be to substitute another consistent estimator of $\theta_0$ and proceed analogously. 
Unfortunately, this approach fails. To understand why, consider the case in which all moment conditions are potentially invalid so that the $g$--block is empty. 
Letting $\widehat{\theta}_f$ denote the estimator based on the full set of moment conditions in $h$,  $\sqrt{n}h_n(\widehat{\theta}_f) \rightarrow_d\Gamma  \mathcal{N}_q(\tau, \Omega)$ where $\Gamma = \mathbf{I}_q - H \left(H'WH\right)^{-1}H'W$, using an argument similar to that in the proof of Theorem \ref{thm:tau}. 
The mean, $\Gamma \tau$, of the resulting limit distribution does not equal $\tau$, and because $\Gamma$ has rank $q-r$ we cannot pre-multiply by its inverse to extract an estimate of $\tau$.
Intuitively, $q-r$ over-identifying restrictions are insufficient to estimate a $q$-vector: $\tau$ is not identified unless we have a minimum of $r$ valid moment conditions. 
However, the limiting distribution of $\sqrt{n}h_n(\widehat{\theta}_f)$ partially identifies $\tau$ even when we have no valid moment conditions at our disposal. 
A combination of this information with prior restrictions on the magnitude of the components of $\tau$ allows the use of the FMSC framework to carry out a sensitivity analysis when $r>p$. 
For example, the worst-case estimate of AMSE over values of $\tau$ in the identified region could still allow certain moment sets to be ruled out.
This idea shares similarities with \citet{Kraay} and \citet{Conleyetal}, two recent papers that suggest methods for evaluating the robustness of conclusions drawn from IV regressions when the instruments used may be invalid.