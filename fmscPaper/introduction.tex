%!TEX root = main.tex
\section{Introduction}
In finite samples, the addition of an slightly endogenous but highly relevant instrument can yield a substantial improvement, reducing estimator variance by far more than bias is increased. 
Building on this observation, I propose a novel moment selection criterion for generalized method of moments (GMM) estimation: the focused moment selection criterion (FMSC). 
Rather than selecting only valid moment conditions, the FMSC chooses from a set of potentially mis-specified moment conditions to yield the minimum asymptotic mean squared error (AMSE) GMM estimator of a user-specified target parameter.
To ensure a meaningful bias-variance tradeoff in the limit, I employ a drifting asymptotic framework in which mis-specification, while present for any fixed sample size, vanishes in the limit.
In the presence of such \emph{locally mis-specified} moment conditions, GMM remains consistent although, centered and rescaled, its limiting distribution displays an asymptotic bias. Adding an additional mis-specified moment condition introduces a further source of bias while reducing asymptotic variance. 
The idea behind the FMSC is to trade off these two effects in the limit as an approximation to finite sample behavior.\footnote{When finite-sample mean-squared error (MSE) is undefined or infinite, AMSE comparisons remain meaningful. In this case, one can view AMSE as the limit of a sequence of ``trimmed'' squared loss functions, as in \cite{Hansen2013}. Trimmed MSE is always well-defined and the trimming fraction can be made asymptotically negligible.} 
While estimating asymptotic variance is straightforward, even under local mis-specification, estimating asymptotic bias requires an identification assumption. 
I consider a setting in which two blocks of moment conditions are available: one that is assumed correctly specified, and another that may not be.
When the correctly specified block identifies the model, I derive an asymptotically unbiased estimator of AMSE: the FMSC.\footnote{This identifying assumption is shared by \cite{Liao} and \cite{ChengLiao}: see Section \ref{sec:ident}.}
When this is not the case, it remains possible to use the AMSE framework to carry out a sensitivity analysis. 

Continuing under the local mis-specification assumption, I show how the framework employed in the derivation of the FMSC can be used to address the important problem of inference post-moment selection.
I treat post-selection estimators as a special case of moment averaging: combining estimators based on different moment sets with data-dependent weights.
By deriving the limiting distribution of this class of estimators in general, I propose a simulation-based procedure for constructing valid confidence intervals. 
This technique can be applied to a variety of moment averaging and post-selection estimators including the FMSC. 

While the methods described here apply to any model estimated by GMM, subject to standard regularity conditions, I focus on two simple but empirically relevant examples: choosing between ordinary least squares (OLS) and two-stage least squares (TSLS) estimators, and selecting instruments in linear instrumental variables (IV) models. 
In the OLS versus TSLS example the FMSC takes a particularly intuitive and transparent form, providing a novel justification for the Durbin-Hausman-Wu (DHW) test, and leading to a novel ``minimum-AMSE'' estimator that optimally combines the information contained in the OLS and TSLS estimators.
In simulation studies for both examples, the FMSC and related procedures perform well relative to alternatives from the literature. 
I conclude with an empirical application from development economics, exploring the effect of instrument selection on the estimated relationship between malaria transmission and income.

My approach to moment selection under mis-specification is inspired by the focused information criterion of \citet{ClaeskensHjort2003}, a model selection criterion for models estimated by maximum likelihood. 
Like them, I allow for mis-specification and study AMSE-based selection in a drifting asymptotic framework. 
In contradistinction, however, I consider moment rather than model selection, and general GMM rather than maximum likelihood estimators.
Although developed independently of the FIC, \cite{Schorfheide2005} uses a similar approach to study forecasting with mis-specified vector autoregression models.
Mine is by no means the first paper to consider GMM asymptotics under locally mis-specified moment conditions, an idea that dates at least as far back as \cite{Newey1985}.
The idea of using this framework for AMSE-based moment selection, however, is new to the literature. 

The FMSC differs substantially from other moment selection criteria from the literature, whose primary aim is to consistently select all correctly specified moment conditions while eliminating all invalid ones.
This idea begins with \cite{Andrews1999}, who proposes a family of consistent moment selection criteria for GMM by adding an appropriate penalty term to the J-test statistic. 
\cite{AndrewsLu} extend this work to allow simultaneous GMM moment and model selection, while \cite{HongPrestonShum} derive analogous results for generalized empirical likelihood. 
More recently, \cite{Liao} proposes a shrinkage procedure for simultaneous GMM moment selection and estimation. 
Given a set of correctly specified moment conditions that identifies the model, this method consistently chooses all valid conditions from a second set of potentially mis-specified conditions.
In contrast to these proposals, which examine only the validity of the moment conditions under consideration, the FMSC balances validity against relevance to minimize AMSE.
Although \cite{HallPeixe2003} and \cite{ChengLiao} do consider relevance, they first consistently eliminate any invalid moment conditions and examine the relevance of only those that remain.
Unlike the FMSC, which trades bias against variance, these papers consider relevance only as a means of eliminating ``redundant'' moment conditions, after first eliminating any invalid ones.



Moreover, the FMSC is not a single criterion but rather a \emph{method} for constructing application-specific criteria.
Consider, for example, a simple dynamic panel model. If your target parameter is a long-run effect while mine is a contemporaneous effect, there is no reason to suppose a priori that we should use the same moment conditions in estimation, even if we share the same dataset.
The FMSC explicitly takes this difference of research goals into account, inlike other moment selection criteria from the literature.

FMSC is conservative, like AIC, not consistent, like BIC. 
But isn't consistency a good idea? 
It depends on what you're trying to do.
No such thing as selection that is always best and always beats others.
FMSC is intended for a particular setting.
Two kinds of assumptions etc.
Risk-based.
In this setting, as we see from simulations, the gains can be very large.
Even if you may not be in this setting, conservative criteria have better worst-case performance and the FMSC is no exception.
Consistent criteria have, in general, unbounded minimax risk.
Cite \cite{LeebPoetscher2008}.
Also, need to use conservative selection to get an accurate picture of effects of selection on inference.

\todo[inline]{Cite \cite{ChengLiao} here. Similar to Hall and Peixi but Lasso-based.}
The only other proposal from the literature to consider both validity and relevance in moment selection is a suggestion by \cite{HallPeixe2003} to combine their canonical correlations information criterion (CCIC) -- a relevance criterion that seeks to avoid including redundant instruments -- with Andrews' GMM moment selection criteria. 
This procedure, however, merely seeks to avoid including redundant instruments after eliminating invalid ones: it does not allow for the intentional inclusion of a slightly invalid but highly relevant instrument to reduce MSE. 

\todo[inline]{Need some mention of the fact that FMSC is conservative rather tha consistent. Briefly mention why this is a good thing: consistent selection leads to bad risk properties, and we want to do inference.}


\todo[inline]{Update preceding to make it clear what this criterion is for, and mention the examples used.}

\todo[inline]{Make sure to stress the contributions. Moment selection better targeted at empirical practice. Novel justification for DHW test, novel estimator that combines OLS and 2SLS. General framework for inference that accounts for moment selection, including various Hausman and J pre-tests}

\todo[inline]{Important limitation: do not consider weak identification. Topic for future work. Sim results partially address this.}

The idea of choosing instruments to minimize MSE is shared by the procedures in \cite{DonaldNewey2001} and \cite{DonaldImbensNewey2009}. 
\cite{KuersteinerOkui2010} also aim to minimize MSE but, rather than choosing a particular instrument set, suggest averaging over the first-stage predictions implied by many instrument sets and using this average in the second stage. 
Unlike FMSC, these papers consider the higher-order bias that arises from including many valid instruments rather than the first-order bias that arises from the use of invalid instruments.

The literature on post-selection, or ``pre-test'' estimators is vast. 
\citet{LeebPoetscher2005, LeebPoetscher2009}  give a theoretical overview, while \cite{Demetrescu} illustrate the practical consequences via a simulation experiment. 
There are several proposals to construct valid confidence intervals post-model selection, including \cite{Kabaila1998}, \cite{HjortClaeskens} and \cite{KabailaLeeb2006}. 
To my knowledge, however, this is the first paper to give a general treatment of the problem specifically from the perspective of moment selection. 
\todo[inline]{Cite the various pre-test and IV papers here: \cite{Berkowitz}, \cite{Guggenberger2010}, \cite{Guggenberger2012}, \cite{GuggenbergerKumar}}
The approach adopted here, treating post-moment selection estimators as a specific example of moment averaging, is adapted from the frequentist model average estimators of \cite{HjortClaeskens}.
\todo[inline]{Might want to mention the Leeb and Poetscher stuff here.}
Another paper that considers weighting GMM estimators based on different moment sets is \cite{Xiao}. 
While Xiao combines estimators based on valid moment conditions to achieve a minimum variance estimator, I combine estimators based on potentially invalid conditions to minimize MSE. 
\todo[inline]{Also cite \cite{ChenChavezLinton}.}
A similar idea underlies the combined moments (CM) estimator of \cite{Judge2007}, who emphasize that incorporating the information from an incorrect specification could lead to favorable bias-variance tradeoff. 
%Their proposal uses a Cressie-Read divergence measure to combine the information from competing moment specifications, for example OLS versus two-stage least squares (2SLS), yielding a data-driven compromise estimator. Unlike the FMSC, however, the CM estimator is not targeted to a particular research goal and does not explicitly aim to minimize MSE.

The remainder of the paper is organized as follows.
Section \ref{sec:asymp} describes the local mis-specification framework and gives the main limiting results used later in the paper. 
Section \ref{sec:FMSC} derives FMSC as an asymptotically unbiased estimator of AMSE, presents specialized results for 2SLS, and examines their performance in a Monte Carlo experiment. 
Section \ref{sec:avg} describes a simulation-based procedure to construct valid confidence intervals for moment average estimators and examines its performance in a Monte Carlo experiment. Section \ref{sec:application} presents the empirical application and Section \ref{sec:conclude} concludes.
Proofs and computational details appear in the Appendix.