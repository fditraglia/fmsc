\documentclass[12pt,letterpaper]{article}
\usepackage[margin=1in]{geometry}
\usepackage{amsmath, amssymb}
\usepackage{setspace}
\usepackage{caption}
\usepackage{subcaption}
\usepackage{morefloats}
\usepackage{todonotes}

\title{Additional Simulations for JOE Revision}
\author{Francis J.\ DiTraglia}

\begin{document}
\maketitle

\section{New Section on Confidence Intervals}
Corollary 4.2 is sufficiently general to cover a wide range of moment selection and averaging procedures, but this same generality makes the confidence interval procedure given in Algorithm 4.1 somewhat less than intuitive.
In this section I specialize the results on moment average estimators from above to the two examples of FMSC selection that appear in the simulation studies described below: OLS versus TSLS and choosing instrumental variables.
The structure of these examples allows us to bypass Algorithm 4.1 and characterize the asymptotic properties of various proposals for post-FMSC inference \emph{directly}, clearly illustrating the relevant trade-offs between coverage and width.
Because this section presents asymptotic results only, I treat any consistently estimable quantity that appears in a limit distribution as known.
Recall that the bias parameter $\tau$ remains unknown \emph{even in the limit}.
This is the main complication of post-FMSC inference and the focus of this section.

First recall the OLS versus TSLS example from Section 3.2.
The joint limit distribution for this case is as follows
\begin{equation*}
  \left[ 
  \begin{array}{c}
    \sqrt{n} \left( \widehat{\beta}_{OLS} - \beta \right)\\
    \sqrt{n} \left( \widetilde{\beta}_{TSLS} - \beta \right)\\
    \widehat{\tau}
\end{array}
\right] \overset{d}{\rightarrow} N\left( \left[
\begin{array}{c}
  \tau/\sigma_x^2 \\ 0 \\ \tau
\end{array}
\right], \sigma_{\epsilon}^2 
\left[
\begin{array}{ccc}
  1/\sigma_x^2 & 1/\sigma_x^2 & 0\\
  1/\sigma_x^2 & 1/\gamma^2 & -\sigma_v^2/\gamma^2\\
  0 & -\sigma_v^2/\gamma^2 & \sigma_x^2 \sigma_v^2/\gamma^2
\end{array}
\right]\right).
\end{equation*}
Second, consider a slightly simplified version of the choosing instrumental variables example from Section 3.3, namely
\begin{eqnarray*}
  y_{ni} &=& \beta x_{ni} + \epsilon_{ni}\\
  x_{ni} &=& \gamma w_{ni} + \mathbf{z}_{ni}' \boldsymbol{\pi} + v_{ni}
\end{eqnarray*}
where $x$ is the endogenous regressor of interest, $z$ is a vector of exogenous instruments, and $w$ is a single potentially endogenous instrument.
Without loss of generality I assume that $w$ and $\mathbf{z}$ are uncorrelated and that all random variables are mean zero.
For simplicity, I further assume that the errors satisfy the same kind of asymptotic homoskedasticity condition used in the OLS versus TSLS example so that TSLS is the efficient GMM estimator.
Let the ``Full'' estimator denote the TSLS estimator using $w$ and $\mathbf{z}$ and the ``Valid'' estimator denote the TSLS estimator using only $\mathbf{z}$.
Then we have,
\begin{equation*}
  \left[ 
  \begin{array}{c}
    \sqrt{n} \left( \widehat{\beta}_{Full} - \beta \right)\\
    \sqrt{n} \left( \widetilde{\beta}_{Valid} - \beta \right)\\
    \widehat{\tau}
\end{array}
\right] \overset{d}{\rightarrow} N\left( \left[
\begin{array}{c}
  \tau\gamma/q^2_{F} \\ 0 \\ \tau
\end{array}
\right], \sigma_{\epsilon}^2 
\left[
\begin{array}{ccc}
  1/q_{F}^2 & 1/q_{F}^2 & 0\\
  1/q_{F}^2 & 1/q_{V}^2 & -\gamma\sigma_w^2/q^2_{V}\\ 
  0 & -\gamma\sigma_w^2/q^2_{V} & \sigma_w^2 q^2_{F}/q^2_{V}
\end{array}
\right]\right)
\end{equation*}
where $q^2_{F} = \gamma^2 \sigma_w^2 + q^2_{V}$, $q^2_{V} = \boldsymbol{\pi}'\Sigma_{zz}\boldsymbol{\pi}$, $\Sigma_{zz}$ is the covariance matrix of the valid instruments $\mathbf{z}$, and $\sigma_w^2$ is the variance of the ``suspect'' instrument $w$.
After some algebraic manipulations we see that \emph{both} of these examples share the same structure, namely
\begin{equation}
  \left[ 
  \begin{array}{c}
    \sqrt{n} \left( \widehat{\beta} - \beta \right)\\
    \sqrt{n} \left( \widetilde{\beta} - \beta \right)\\
    \widehat{\tau}
\end{array}
\right] \overset{d}{\rightarrow} 
\left[
\begin{array}{c}
  U \\ V \\ T
\end{array}
\right] \sim
N\left( \left[
\begin{array}{c}
  c\tau\\ 0 \\ \tau
\end{array}
\right], 
\left[
\begin{array}{ccc}
  \eta^2 & \eta^2 & 0\\
  \eta^2 & \eta^2 + c^2 \sigma^2 & -c\sigma^2\\ 
  0 & -c\sigma^2 & \sigma^2 
\end{array}
\right]\right)
\label{eq:LimitExperiment}
\end{equation}
where $\widehat{\beta}$ denotes the low variance but possibly biased estimator, and $\widetilde{\beta}$ denotes the higher variance but unbiased estimator.
For any example with a limit distribution that takes this form, simple algebra shows that FMSC selection amounts to choosing $\widehat{\beta}$ whenever $|\widehat{\tau}|<\sqrt{2}\sigma$, and choosing $\widetilde{\beta}$ otherwise, in other words
\begin{eqnarray*}
  \sqrt{n}(\widehat{\beta}_{FMSC} - \beta) = \mathbf{1}\left\{ |\widehat{\tau}|<\sigma \sqrt{2} \right\} \sqrt{n}(\widehat{\beta} - \beta) +  \mathbf{1}\left\{ |\widehat{\tau}|\geq\sigma \sqrt{2} \right\}\sqrt{n}(\widetilde{\beta} - \beta)
\end{eqnarray*}
and so by the Continuous Mapping Theorem,
\begin{equation}
  \sqrt{n}(\widehat{\beta}_{FMSC} - \beta) \overset{d}{\rightarrow}  \mathbf{1}\left\{ |T|<\sigma \sqrt{2} \right\} U +  \mathbf{1}\left\{ |T|\geq\sigma \sqrt{2} \right\} V.
  \label{eq:FMSCLimitExperiment}
\end{equation}

To better understand the implications of Equation \ref{eq:FMSCLimitExperiment}, it is helpful to re-express the limit distribution from Equation \ref{eq:LimitExperiment} in terms of the marginal distribution of $T$ and the conditional distribution of $U$ and $V$ given $T$. 
We have $T \sim N(\tau, \sigma^2)$ and by direct calculation
\begin{equation}
  \left.\left[
  \begin{array}{c}
   U \\ V 
  \end{array}
\right]\right| (T = t) \sim N\left(
\left[
\begin{array}{c}
  c \tau \\ c\tau - ct
\end{array}
\right], \eta^2
\left[
\begin{array}{cc}
  1 & 1 \\ 1 & 1
\end{array}
\right]
\right)
\label{eq:LimitExperimentConditional}
\end{equation}
which is a \emph{singular distribution}.
Crucially $U$ is independent of $T$, but conditional on $T$ the random variables $U$ and $V$ are perfectly correlated with the same variance.
Given $T$, the only difference between $U$ and $V$ is that the mean of $V$ is shifted by a distance that depends on the realization $t$ of $T$.
Thus, the conditional distribution of $V$ shows a \emph{random bias}: on average $V$ has mean zero because the mean of $T$ is $\tau$ but any particular realization $t$ of $T$ will not in general equal $\tau$.
Using the form of the conditional distributions we can express the distribution of $(U,V,T)'$ from Equation \ref{eq:LimitExperiment} in a more transparent form as
\begin{eqnarray*}
  T &=& \sigma Z_1 + \tau\\
  U &=& \eta Z_2 + c\tau\\
  V &=& \eta Z_2 - c\sigma Z_1
\end{eqnarray*}
where $Z_1, Z_2$ are iid standard normal random variables.
Combined with Equation \ref{eq:FMSCLimitExperiment}, this representation allows us to tabulate the asymptotic distribution, $F_{FMSC}$, of the post-FMSC estimator \emph{directly} for any example that takes the form of Equation \ref{eq:LimitExperiment} rather than approximating it by simulation as in Algorithm 4.1.
We have:
\begin{eqnarray}
  F_{FMSC}(x) &=& G(x) + H_1(x) + H_2(x) \\
  \label{eq:FFMSC}
  G(x) &=& \Phi\left( \frac{x - c\tau}{\eta} \right)\left[ \Phi( \sqrt{2} - \tau/\sigma) -  \Phi( -\sqrt{2} - \tau/\sigma )\right]\\
  \label{eq:GFMSC}
  H_1(x) &=& \frac{1}{\sigma}\int_{-\infty}^{-\sigma\sqrt{2} - \tau} \Phi\left( \frac{x + ct}{\eta}\right)\varphi(t/\sigma)\; dt\\
  \label{eq:H1FMSC}
  H_2(x) &=& \frac{1}{\sigma}\int^{+\infty}_{\sigma\sqrt{2} - \tau} \Phi\left( \frac{x + ct}{\eta}\right)\varphi(t/\sigma)\; dt
  \label{eq:H2FMSC}
\end{eqnarray}
where $\Phi$ is the CDF and $\varphi$ the pdf of a standard normal random variable.
Note that the limit distribution of the post-FMSC distribution depends on $\tau$ in addition to the consistently estimable quantities $\sigma, \eta, c$ although I supress this dependence to simplify the notation.
While these expressions lack a closed form, there are various fast and extremely accurate numerical routines for evaluating integrals of the form taken by $H_1$ and $H_2$, for example those provided in the \texttt{mvtnorm} package in R. \todo{Cite!}
Armed with a numerical routine to evaluate $F_{FMSC}$ it is straightforward to evaluate the corresponding quantile function $Q_{FMSC}$ using a root-finder by analytically bounding $G$, $H_1$ and $H_2$.
I provide numerical routines to evaluate both $F_{FMSC}$ and $Q_{FMSC}$ along with various related quantities in my R package \texttt{fmscr}. \todo{Cite!}

The ability to compute $F_{FMSC}$ and $Q_{FMSC}$ allows us to answer a number of important questions about post-FMSC inference for any example of the form given in Equation \ref{eq:LimitExperiment}.
To begin, suppose that we were to carry out FMSC selection and then construct a $(1 - \alpha) \times 100\%$ confidence interval \emph{conditional} in the selected estimator, completely ignoring the effects of the moment selection step.
What would be the resulting asymptotic coverage probability and width of such a ``naive'' confidence interval procedure?
Using calculations similar to those used above in the expression for $F_{FMSC}$, we find that the coverage probability of this naive interval is given by
\begin{eqnarray*}
  \mbox{CP}_{Naive}(\alpha) &=& G(u_\alpha) - G(-u_\alpha) +  H_1(\ell_\alpha) - H_2(-\ell_\alpha) +  H_2(\ell_\alpha) - H_2(-\ell_\alpha) \\
  u_\alpha &=& z_{1-\alpha/2}\; \eta\\
  \ell_{\alpha} &=& z_{1-\alpha/2} \sqrt{\eta^2 + c^2\sigma^2}
\end{eqnarray*}
where $z_{1-\alpha/2} = \Phi^{-1}(1 -\alpha/2)$ and $G$, $H_1$, $H_2$ are as defined in Equations \ref{eq:GFMSC}--\ref{eq:H2FMSC}.
And since the width of this naive CI equals that of the textbook interval for $\widehat{\beta}$ when $|\widehat{\tau}|<\sigma\sqrt{2}$ and that of the textbook interval for $\widetilde{\beta}$ otherwise, we have
\begin{equation*}
  \frac{E\left[ \mbox{Width}_{Naive}(\alpha) \right]}{\mbox{Width}_{Valid}(\alpha)} = 1 + \left[ \Phi( \sqrt{2} - \tau/\sigma) -  \Phi( -\sqrt{2} - \tau/\sigma )\right]\left( \sqrt{\frac{\eta^2}{\eta^2 + c^2 \sigma^2}} - 1 \right)
\end{equation*}
where $\mbox{Width}_{Valid}(\alpha)$ is the width of a standard, textbook confidence interval for $\widetilde{\beta}$.

To evaluate these expressions we need values for $c, \eta^2, \sigma^2$ and $\tau$.
For the remainder of this section I will consider the parameter values that correspond to the simulation experiments presented below in Section 5.3.
For the OLS versus TSLS simulation experiment we have $c=1$, $\eta^2=1$ and $\sigma^2 = (1-\pi^2)/\pi^2$ where $\pi^2$ denotes the population first-stage R-squared for the TSLS estimator. 
For the choosing IVs simulation experiment we have $c =\gamma/(\gamma^2 +1/9)$, $\eta^2 = 1/(\gamma^2 + 1/9)$ and $\sigma^2 = 1 + 9\gamma^2$ where $\gamma^2$ is the increase in the population first-stage R-squared of the TSLS estimator from \emph{adding} $w$ to the instrument set.
The population first-stage R-squared without $w$ in this example is $1/9$.

Table \ref{tab:LimitNaiveCover} presents the coverage probability and \ref{tab:LimitNaiveWidth} the expected relative width of the naive confidence interval procedure for a variety of values of $\tau$ and $\alpha$ for each of the two examples.
For the OLS versus TSLS example, I allow $\pi^2$ to vary while for the choosing IV example I allow $\gamma^2$ to vary.
Note that the relative expected width does not depends on $\alpha$.
In terms of coverage probability, the naive interval performs very badly: in some regions of the parameter space the actual coverage is very close to the nominal level, while in others it is far lower.
These striking size distortions, which echo the findings of Guggenberger \todo{Cite!!!} provide a strong argument against the use of the naive interval.
Its attraction, of course, is width: the naive interval can be dramatically shorter than the corresponding ``textbook'' confidence interval for the valid estimator.

\begin{table}[h]
  \centering
  \begin{subtable}{0.48\textwidth}
    \caption{OLS versus TSLS}
    \begin{tabular}{r|rrrrrr}
\hline\hline
 &\multicolumn{6}{c}{$\tau$} \\ 
 $\alpha = 0.05$ & $0$ & $1$ & $2$ & $3$ & $4$ & $5$ \\ 
 \hline$0.1$ & $91$ & $81$ & $57$ & $41$ & $45$ & $58$\\ 
$\pi^2\;\;\;$ $0.2$ & $91$ & $83$ & $63$ & $58$ & $70$ & $84$\\ 
$0.3$ & $92$ & $84$ & $69$ & $73$ & $86$ & $93$\\ 
$0.4$ & $92$ & $85$ & $76$ & $84$ & $93$ & $95$\\ 
 \hline 
 \end{tabular}
 
 \vspace{2em} 
 
\begin{tabular}{r|rrrrrr}
\hline\hline
 &\multicolumn{6}{c}{$\tau$} \\ 
 $\alpha = 0.1$ & $0$ & $1$ & $2$ & $3$ & $4$ & $5$ \\ 
 \hline$0.1$ & $83$ & $70$ & $45$ & $35$ & $42$ & $55$\\ 
$\pi^2\;\;\;$ $0.2$ & $84$ & $72$ & $53$ & $52$ & $67$ & $81$\\ 
$0.3$ & $85$ & $74$ & $60$ & $68$ & $83$ & $89$\\ 
$0.4$ & $86$ & $76$ & $68$ & $80$ & $89$ & $90$\\ 
 \hline 
 \end{tabular}
 
 \vspace{2em} 
 
\begin{tabular}{r|rrrrrr}
\hline\hline
 &\multicolumn{6}{c}{$\tau$} \\ 
 $\alpha = 0.2$ & $0$ & $1$ & $2$ & $3$ & $4$ & $5$ \\ 
 \hline$0.1$ & $70$ & $54$ & $31$ & $27$ & $37$ & $50$\\ 
$\pi^2\;\;\;$ $0.2$ & $71$ & $57$ & $39$ & $45$ & $62$ & $74$\\ 
$0.3$ & $73$ & $59$ & $49$ & $61$ & $75$ & $79$\\ 
$0.4$ & $74$ & $62$ & $58$ & $72$ & $79$ & $80$\\ 
 \hline 
 \end{tabular}
  \end{subtable}
  ~
  \begin{subtable}{0.48\textwidth}
    \caption{Choosing IVs}
    \begin{tabular}{r|rrrrrr}
\hline\hline
 &\multicolumn{6}{c}{$\tau$} \\ 
 $\alpha = 0.05$ & $0$ & $1$ & $2$ & $3$ & $4$ & $5$ \\ 
 \hline$0.1$ & $93$ & $89$ & $84$ & $85$ & $91$ & $94$\\ 
$\gamma^2\;\;\;$ $0.2$ & $92$ & $87$ & $76$ & $74$ & $83$ & $91$\\ 
$0.3$ & $92$ & $85$ & $71$ & $65$ & $74$ & $86$\\ 
$0.4$ & $91$ & $85$ & $68$ & $59$ & $67$ & $80$\\ 
 \hline 
 \end{tabular}
 
 \vspace{2em} 
 
\begin{tabular}{r|rrrrrr}
\hline\hline
 &\multicolumn{6}{c}{$\tau$} \\ 
 $\alpha = 0.1$ & $0$ & $1$ & $2$ & $3$ & $4$ & $5$ \\ 
 \hline$0.1$ & $87$ & $82$ & $76$ & $79$ & $86$ & $89$\\ 
$\gamma^2\;\;\;$ $0.2$ & $85$ & $78$ & $66$ & $67$ & $79$ & $87$\\ 
$0.3$ & $84$ & $76$ & $61$ & $59$ & $71$ & $82$\\ 
$0.4$ & $84$ & $75$ & $57$ & $52$ & $63$ & $77$\\ 
 \hline 
 \end{tabular}
 
 \vspace{2em} 
 
\begin{tabular}{r|rrrrrr}
\hline\hline
 &\multicolumn{6}{c}{$\tau$} \\ 
 $\alpha = 0.2$ & $0$ & $1$ & $2$ & $3$ & $4$ & $5$ \\ 
 \hline$0.1$ & $75$ & $69$ & $64$ & $70$ & $77$ & $80$\\ 
$\gamma^2\;\;\;$ $0.2$ & $73$ & $64$ & $53$ & $59$ & $71$ & $78$\\ 
$0.3$ & $72$ & $62$ & $47$ & $50$ & $64$ & $75$\\ 
$0.4$ & $72$ & $60$ & $43$ & $44$ & $58$ & $71$\\ 
 \hline 
 \end{tabular}
  \end{subtable}
  \caption{Asymptotic coverage probability of Naive $(1-\alpha)\times 100\%$ confidence interval.}
  \label{tab:LimitNaiveCover}
\end{table}

\begin{table}[h]
  \centering
  \begin{subtable}{0.48\textwidth}
    \caption{OLS versus TSLS}
    \begin{tabular}{r|rrrrrr}
\hline\hline
 &\multicolumn{6}{c}{$\tau$} \\ 
  & $0$ & $1$ & $2$ & $3$ & $4$ & $5$ \\ 
 \hline$0.1$ & $ 42$ & $ 44$ & $ 48$ & $ 55$ & $ 64$ & $ 73$\\ 
$\pi^2\;\;\;$ $0.2$ & $ 53$ & $ 56$ & $ 64$ & $ 74$ & $ 85$ & $ 92$\\ 
$0.3$ & $ 62$ & $ 66$ & $ 76$ & $ 87$ & $ 95$ & $ 99$\\ 
$0.4$ & $ 69$ & $ 74$ & $ 85$ & $ 94$ & $ 99$ & $100$\\ 
 \hline 
 \end{tabular}
  \end{subtable}
  ~
  \begin{subtable}{0.48\textwidth}
    \caption{Choosing IVs}
    \begin{tabular}{r|rrrrrr}
\hline\hline
 &\multicolumn{6}{c}{$\tau$} \\ 
  & $0$ & $1$ & $2$ & $3$ & $4$ & $5$ \\ 
 \hline$0.1$ & $ 77$ & $ 80$ & $ 87$ & $ 94$ & $ 98$ & $100$\\ 
$\gamma^2\;\;\;$ $0.2$ & $ 66$ & $ 69$ & $ 77$ & $ 86$ & $ 93$ & $ 98$\\ 
$0.3$ & $ 60$ & $ 62$ & $ 69$ & $ 79$ & $ 88$ & $ 94$\\ 
$0.4$ & $ 55$ & $ 57$ & $ 64$ & $ 73$ & $ 83$ & $ 90$\\ 
 \hline 
 \end{tabular}
  \end{subtable}
  \caption{Asymptotic expected width of naive confidence interval relative to that of the valid estimator. Values are given in percentage points.}
  \label{tab:LimitNaiveWidth}
\end{table}

Is there any way to construct a post-FMSC confidence interval that does not suffer from the egregious size distortions of the naive interval but is still shorter than the textbook interval for the valid estimator?
As a first step towards answering this question, Table \ref{tab:WidthInfeasible} presents the relative width of the shortest possible \emph{infeasible} post-FMSC confidence interval constructed directly from $Q_{FMSC}$.
This interval has asymptotic coverage probability \emph{exactly} equal to its nominal level as it correctly accounts for the effect of moment selection on the asymptotic distribution of the estimators.
Unfortunately it cannot be used in practice because it requires knowledge of $\tau$, for which no consistent estimator exists.
As such, this interval serves as a benchmark against which to judge various feasible procedures that do not require knowledge of $\tau$.
For certain parameter values this interval is indeed shorter than the valid interval but the improvement is not uniform.
Just as the FMSC itself cannot provide a uniform reduction in AMSE relative to the valid estimator, the infeasible post-FMSC cannot provide a corresponding reduction in width.
In both cases, however, improvements are possible when $\tau$ is expected to be small, the setting in which this paper assumes that an applied researcher finds herself. 
The potential reductions in width can be particularly dramatic for larger values of $\alpha$.
The question remains: is there any way to capture these gains using a \emph{feasible} procedure?


\begin{table}[h]
  \centering
  \begin{subtable}{0.48\textwidth}
    \caption{OLS versus TSLS}
    \begin{tabular}{r|rrrrrr}
\hline\hline
 &\multicolumn{6}{c}{$\tau$} \\ 
 $\alpha = 0.05$ & $0$ & $1$ & $2$ & $3$ & $4$ & $5$ \\ 
 \hline$0.1$ & $ 99$ & $ 92$ & $ 85$ & $ 89$ & $ 95$ & $101$\\ 
$\pi^2\;\;\;$ $0.2$ & $ 97$ & $ 91$ & $ 94$ & $102$ & $110$ & $117$\\ 
$0.3$ & $ 94$ & $ 94$ & $102$ & $111$ & $117$ & $109$\\ 
$0.4$ & $ 92$ & $ 97$ & $107$ & $114$ & $107$ & $100$\\ 
 \hline 
 \end{tabular}
 
 \vspace{2em} 
 
\begin{tabular}{r|rrrrrr}
\hline\hline
 &\multicolumn{6}{c}{$\tau$} \\ 
 $\alpha = 0.1$ & $0$ & $1$ & $2$ & $3$ & $4$ & $5$ \\ 
 \hline$0.1$ & $ 88$ & $ 81$ & $ 85$ & $ 91$ & $ 99$ & $107$\\ 
$\pi^2\;\;\;$ $0.2$ & $ 89$ & $ 88$ & $ 97$ & $107$ & $116$ & $123$\\ 
$0.3$ & $ 86$ & $ 93$ & $105$ & $115$ & $119$ & $103$\\ 
$0.4$ & $ 87$ & $ 98$ & $111$ & $116$ & $104$ & $100$\\ 
 \hline 
 \end{tabular}
 
 \vspace{2em} 
 
\begin{tabular}{r|rrrrrr}
\hline\hline
 &\multicolumn{6}{c}{$\tau$} \\ 
 $\alpha = 0.2$ & $0$ & $1$ & $2$ & $3$ & $4$ & $5$ \\ 
 \hline$0.1$ & $ 48$ & $ 55$ & $ 84$ & $ 96$ & $106$ & $116$\\ 
$\pi^2\;\;\;$ $0.2$ & $ 65$ & $ 80$ & $101$ & $114$ & $125$ & $117$\\ 
$0.3$ & $ 74$ & $ 90$ & $111$ & $123$ & $112$ & $101$\\ 
$0.4$ & $ 80$ & $ 97$ & $116$ & $115$ & $102$ & $100$\\ 
 \hline 
 \end{tabular}
  \end{subtable}
  ~
  \begin{subtable}{0.48\textwidth}
    \caption{Choosing IVs}
    \begin{tabular}{r|rrrrrr}
\hline\hline
 &\multicolumn{6}{c}{$\tau$} \\ 
 $\alpha = 0.05$ & $0$ & $1$ & $2$ & $3$ & $4$ & $5$ \\ 
 \hline$0.1$ & $ 92$ & $ 97$ & $106$ & $111$ & $109$ & $102$\\ 
$\gamma^2\;\;\;$ $0.2$ & $ 93$ & $ 94$ & $101$ & $109$ & $115$ & $114$\\ 
$0.3$ & $ 95$ & $ 93$ & $ 97$ & $105$ & $112$ & $117$\\ 
$0.4$ & $ 97$ & $ 92$ & $ 94$ & $101$ & $108$ & $115$\\ 
 \hline 
 \end{tabular}
 
 \vspace{2em} 
 
\begin{tabular}{r|rrrrrr}
\hline\hline
 &\multicolumn{6}{c}{$\tau$} \\ 
 $\alpha = 0.1$ & $0$ & $1$ & $2$ & $3$ & $4$ & $5$ \\ 
 \hline$0.1$ & $ 89$ & $ 97$ & $108$ & $113$ & $108$ & $101$\\ 
$\gamma^2\;\;\;$ $0.2$ & $ 86$ & $ 93$ & $104$ & $113$ & $118$ & $109$\\ 
$0.3$ & $ 86$ & $ 90$ & $100$ & $109$ & $117$ & $121$\\ 
$0.4$ & $ 88$ & $ 88$ & $ 96$ & $105$ & $114$ & $121$\\ 
 \hline 
 \end{tabular}
 
 \vspace{2em} 
 
\begin{tabular}{r|rrrrrr}
\hline\hline
 &\multicolumn{6}{c}{$\tau$} \\ 
 $\alpha = 0.2$ & $0$ & $1$ & $2$ & $3$ & $4$ & $5$ \\ 
 \hline$0.1$ & $ 86$ & $ 96$ & $111$ & $115$ & $105$ & $101$\\ 
$\gamma^2\;\;\;$ $0.2$ & $ 78$ & $ 89$ & $108$ & $119$ & $118$ & $104$\\ 
$0.3$ & $ 72$ & $ 84$ & $103$ & $116$ & $125$ & $112$\\ 
$0.4$ & $ 67$ & $ 79$ & $ 99$ & $112$ & $123$ & $128$\\ 
 \hline 
 \end{tabular}
  \end{subtable}
  \caption{Width of shortest possible $(1-\alpha)\times 100\%$ post-FMSC confidence interval constructed directly from $Q_{FMSC}$ using knowledge of $\tau$. This interval is infeasible as no consistent estimator of $\tau$ exists. Values are given in percentage points.}
  \label{tab:WidthInfeasible}
\end{table}

Although no consistent estimator of $\tau$ exists, $\widehat{\tau}$ provides an asymptotically unbiased estimator.
I now consider two different ways to use $\widehat{\tau}$ to construct a post-FMSC confidence interval.
The first is equivalent to the two-step procedure given in Algorithm 4.1 but uses exact calculations rather than simulations to calculate $Q_{FMSC}$, following the derivations given above.
This procedure first constructs a $(1-\alpha_1)\times 100\%$ confidence interval for $\widehat{\tau}$.
For each $\tau^*$ in this interval it then constructs a $(1-\alpha_2)\times 100\%$ based on $Q_{FMSC}$ and then takes the upper and lower bounds over all of the resulting intervals.
This interval is guaranteed to have asymptotic coverage probability of at least $1 - (\alpha_1 + \alpha_2)$ by Theorem 4.4.
As we are free to choose any $\alpha_1, \alpha_2$ such that $\alpha_1 + \alpha_2 = \alpha$ when using this method, I experimented with three possibilities: $\alpha_1 = \alpha_2 = \alpha/2$, followed by $\alpha_1 = \alpha/4, \alpha_2 = 3\alpha/4$ and $\alpha_1 = 3\alpha/4, \alpha_2 = \alpha/4$.
As setting $\alpha_1 = \alpha/4$ produced the shortest intervals I report only results for the middle configuration here.
Additional results are available on request.
As we see from Table \ref{tab:Limit2StepWideTauOLSvsIV} for the OLS versus TSLS example and Table \ref{tab:Limit2StepWideTauOLSvsIV} for the choosing IVs example, this procedure delivers on its promise that asymptotic coverage will never fall below $1-\alpha$.
Protection against under-coverage, however, comes at the expense of extreme conservatism, particularly for larger values of $\alpha$.
The two-step procedure yields asymptotic coverage that is systematically \emph{too large} and hence \emph{cannot} produce an interval shorter than the textbook CI for the valid estimator.


\begin{table}[h]
  \centering
  \begin{subtable}{0.48\textwidth}
    \caption{Coverage Probability}
    \begin{tabular}{r|rrrrrr}
\hline\hline
 &\multicolumn{6}{c}{$\tau$} \\ 
 $\alpha = 0.05$ & $0$ & $1$ & $2$ & $3$ & $4$ & $5$ \\ 
 \hline$0.1$ & $97$ & $97$ & $97$ & $98$ & $98$ & $98$\\ 
$\pi^2\;\;\;$ $0.2$ & $97$ & $97$ & $98$ & $97$ & $97$ & $97$\\ 
$0.3$ & $98$ & $98$ & $98$ & $97$ & $96$ & $97$\\ 
$0.4$ & $98$ & $98$ & $97$ & $96$ & $97$ & $98$\\ 
 \hline 
 \end{tabular}
 
 \vspace{2em} 
 
\begin{tabular}{r|rrrrrr}
\hline\hline
 &\multicolumn{6}{c}{$\tau$} \\ 
 $\alpha = 0.1$ & $0$ & $1$ & $2$ & $3$ & $4$ & $5$ \\ 
 \hline$0.1$ & $94$ & $95$ & $96$ & $96$ & $95$ & $94$\\ 
$\pi^2\;\;\;$ $0.2$ & $95$ & $96$ & $96$ & $95$ & $94$ & $93$\\ 
$0.3$ & $96$ & $96$ & $95$ & $94$ & $92$ & $94$\\ 
$0.4$ & $96$ & $95$ & $94$ & $92$ & $94$ & $95$\\ 
 \hline 
 \end{tabular}
 
 \vspace{2em} 
 
\begin{tabular}{r|rrrrrr}
\hline\hline
 &\multicolumn{6}{c}{$\tau$} \\ 
 $\alpha = 0.2$ & $0$ & $1$ & $2$ & $3$ & $4$ & $5$ \\ 
 \hline$0.1$ & $91$ & $92$ & $92$ & $91$ & $90$ & $90$\\ 
$\pi^2\;\;\;$ $0.2$ & $93$ & $92$ & $91$ & $89$ & $87$ & $85$\\ 
$0.3$ & $93$ & $92$ & $89$ & $86$ & $85$ & $89$\\ 
$0.4$ & $93$ & $91$ & $86$ & $85$ & $88$ & $89$\\ 
 \hline 
 \end{tabular}
  \end{subtable}
  ~
  \begin{subtable}{0.48\textwidth}
    \caption{Relative Width}
    \begin{tabular}{r|rrrrrr}
\hline\hline
 &\multicolumn{6}{c}{$\tau$} \\ 
 $\alpha = 0.05$ & $0$ & $1$ & $2$ & $3$ & $4$ & $5$ \\ 
 \hline$0.1$ & $114$ & $115$ & $117$ & $119$ & $123$ & $126$\\ 
$\pi^2\;\;\;$ $0.2$ & $116$ & $117$ & $120$ & $121$ & $125$ & $126$\\ 
$0.3$ & $117$ & $117$ & $120$ & $122$ & $123$ & $123$\\ 
$0.4$ & $116$ & $118$ & $120$ & $121$ & $121$ & $120$\\ 
 \hline 
 \end{tabular}
 
 \vspace{2em} 
 
\begin{tabular}{r|rrrrrr}
\hline\hline
 &\multicolumn{6}{c}{$\tau$} \\ 
 $\alpha = 0.1$ & $0$ & $1$ & $2$ & $3$ & $4$ & $5$ \\ 
 \hline$0.1$ & $121$ & $123$ & $125$ & $128$ & $129$ & $131$\\ 
$\pi^2\;\;\;$ $0.2$ & $122$ & $124$ & $126$ & $129$ & $130$ & $131$\\ 
$0.3$ & $123$ & $125$ & $126$ & $127$ & $128$ & $128$\\ 
$0.4$ & $123$ & $123$ & $124$ & $125$ & $125$ & $123$\\ 
 \hline 
 \end{tabular}
 
 \vspace{2em} 
 
\begin{tabular}{r|rrrrrr}
\hline\hline
 &\multicolumn{6}{c}{$\tau$} \\ 
 $\alpha = 0.2$ & $0$ & $1$ & $2$ & $3$ & $4$ & $5$ \\ 
 \hline$0.1$ & $135$ & $139$ & $140$ & $140$ & $144$ & $145$\\ 
$\pi^2\;\;\;$ $0.2$ & $136$ & $136$ & $137$ & $139$ & $141$ & $141$\\ 
$0.3$ & $135$ & $135$ & $136$ & $137$ & $136$ & $135$\\ 
$0.4$ & $133$ & $133$ & $133$ & $133$ & $131$ & $128$\\ 
 \hline 
 \end{tabular}
  \end{subtable}
  \caption{OLS versus TSLS Example: Asymptotic coverage and expected relative width of two-step confidence interval with $\alpha_1 = \alpha/4,  \alpha_2 = 3\alpha/4$.}
  \label{tab:Limit2StepWideTauOLSvsIV}
\end{table}

\begin{table}[h]
  \centering
  \begin{subtable}{0.48\textwidth}
    \caption{Coverage Probability}
    \begin{tabular}{r|rrrrrr}
\hline\hline
 &\multicolumn{6}{c}{$\tau$} \\ 
 $\alpha = 0.05$ & $0$ & $1$ & $2$ & $3$ & $4$ & $5$ \\ 
 \hline$0.1$ & $98$ & $98$ & $97$ & $96$ & $96$ & $97$\\ 
$\gamma^2\;\;\;$ $0.2$ & $98$ & $98$ & $98$ & $97$ & $96$ & $96$\\ 
$0.3$ & $98$ & $98$ & $98$ & $97$ & $97$ & $96$\\ 
$0.4$ & $97$ & $97$ & $98$ & $98$ & $97$ & $97$\\ 
 \hline 
 \end{tabular}
 
 \vspace{2em} 
 
\begin{tabular}{r|rrrrrr}
\hline\hline
 &\multicolumn{6}{c}{$\tau$} \\ 
 $\alpha = 0.1$ & $0$ & $1$ & $2$ & $3$ & $4$ & $5$ \\ 
 \hline$0.1$ & $96$ & $96$ & $94$ & $93$ & $93$ & $94$\\ 
$\gamma^2\;\;\;$ $0.2$ & $96$ & $96$ & $95$ & $94$ & $93$ & $93$\\ 
$0.3$ & $96$ & $96$ & $95$ & $95$ & $93$ & $92$\\ 
$0.4$ & $95$ & $96$ & $96$ & $95$ & $94$ & $93$\\ 
 \hline 
 \end{tabular}
 
 \vspace{2em} 
 
\begin{tabular}{r|rrrrrr}
\hline\hline
 &\multicolumn{6}{c}{$\tau$} \\ 
 $\alpha = 0.2$ & $0$ & $1$ & $2$ & $3$ & $4$ & $5$ \\ 
 \hline$0.1$ & $93$ & $91$ & $87$ & $85$ & $87$ & $88$\\ 
$\gamma^2\;\;\;$ $0.2$ & $93$ & $92$ & $89$ & $86$ & $85$ & $87$\\ 
$0.3$ & $93$ & $92$ & $90$ & $88$ & $85$ & $85$\\ 
$0.4$ & $93$ & $92$ & $91$ & $89$ & $87$ & $85$\\ 
 \hline 
 \end{tabular}
  \end{subtable}
  ~
  \begin{subtable}{0.48\textwidth}
    \caption{Relative Width}
    \begin{tabular}{r|rrrrrr}
\hline\hline
 &\multicolumn{6}{c}{$\tau$} \\ 
 $\alpha = 0.05$ & $0$ & $1$ & $2$ & $3$ & $4$ & $5$ \\ 
 \hline$0.1$ & $114$ & $115$ & $117$ & $119$ & $123$ & $126$\\ 
$\gamma^2\;\;\;$ $0.2$ & $116$ & $117$ & $120$ & $121$ & $125$ & $126$\\ 
$0.3$ & $117$ & $117$ & $120$ & $122$ & $123$ & $123$\\ 
$0.4$ & $116$ & $118$ & $120$ & $121$ & $121$ & $120$\\ 
 \hline 
 \end{tabular}
 
 \vspace{2em} 
 
\begin{tabular}{r|rrrrrr}
\hline\hline
 &\multicolumn{6}{c}{$\tau$} \\ 
 $\alpha = 0.1$ & $0$ & $1$ & $2$ & $3$ & $4$ & $5$ \\ 
 \hline$0.1$ & $121$ & $123$ & $125$ & $128$ & $129$ & $131$\\ 
$\gamma^2\;\;\;$ $0.2$ & $122$ & $124$ & $126$ & $129$ & $130$ & $131$\\ 
$0.3$ & $123$ & $125$ & $126$ & $127$ & $128$ & $128$\\ 
$0.4$ & $123$ & $123$ & $124$ & $125$ & $125$ & $123$\\ 
 \hline 
 \end{tabular}
 
 \vspace{2em} 
 
\begin{tabular}{r|rrrrrr}
\hline\hline
 &\multicolumn{6}{c}{$\tau$} \\ 
 $\alpha = 0.2$ & $0$ & $1$ & $2$ & $3$ & $4$ & $5$ \\ 
 \hline$0.1$ & $135$ & $139$ & $140$ & $140$ & $144$ & $145$\\ 
$\gamma^2\;\;\;$ $0.2$ & $136$ & $136$ & $137$ & $139$ & $141$ & $141$\\ 
$0.3$ & $135$ & $135$ & $136$ & $137$ & $136$ & $135$\\ 
$0.4$ & $133$ & $133$ & $133$ & $133$ & $131$ & $128$\\ 
 \hline 
 \end{tabular}
  \end{subtable}
  \caption{Choose IVs Example: 2-step $\alpha_1 = \alpha/4,  \alpha_2 = 3\alpha/4$ CI, limit sim.}
  \label{tab:Limit2StepWideTauChooseIVs}
\end{table}

To address this problem, I now consider an alternative one-step procedure.
Rather than first constructing a confidence region for $\tau$ and then taking upper and lower bounds, this method simply assumes that $\widehat{\tau}$ is exactly equal to $\tau$ and then constructs a confidence interval from $Q_{FMSC}$ exactly as in the infeasible interval described above.\footnote{As in the construction of the naive interval, I take the shortest possible interval based on $Q_{FMSC}$ rather than an equal-tailed interval. Additional results for an equal-tailed version of this one-step procedure are available upon request. Their performance is comparable.}
Note that Theorem 4.4 does \emph{not} apply to this one-step interval: it comes with no generic guarantee of uniform coverage performance.
However for any example that take the form of Equation \ref{eq:LimitExperiment} we can directly calculate the asymptotic coverage and expected relative width of this procedure to determine precisely how it performs over relevant regions of the parameter space.

\begin{table}[h]
  \centering
  \begin{subtable}{0.48\textwidth}
    \caption{Coverage Probability}
    \begin{tabular}{r|rrrrrr}
\hline\hline
 &\multicolumn{6}{c}{$\tau$} \\ 
 $\alpha = 0.05$ & $0$ & $1$ & $2$ & $3$ & $4$ & $5$ \\ 
 \hline$0.1$ & $93$ & $94$ & $95$ & $94$ & $91$ & $90$\\ 
$\pi^2\;\;\;$ $0.2$ & $95$ & $95$ & $95$ & $93$ & $91$ & $91$\\ 
$0.3$ & $95$ & $96$ & $94$ & $92$ & $92$ & $94$\\ 
$0.4$ & $96$ & $95$ & $94$ & $93$ & $95$ & $95$\\ 
 \hline 
 \end{tabular}
 
 \vspace{2em} 
 
\begin{tabular}{r|rrrrrr}
\hline\hline
 &\multicolumn{6}{c}{$\tau$} \\ 
 $\alpha = 0.1$ & $0$ & $1$ & $2$ & $3$ & $4$ & $5$ \\ 
 \hline$0.1$ & $89$ & $89$ & $88$ & $86$ & $82$ & $80$\\ 
$\pi^2\;\;\;$ $0.2$ & $91$ & $91$ & $88$ & $85$ & $83$ & $85$\\ 
$0.3$ & $92$ & $91$ & $87$ & $85$ & $87$ & $90$\\ 
$0.4$ & $92$ & $90$ & $87$ & $87$ & $90$ & $91$\\ 
 \hline 
 \end{tabular}
 
 \vspace{2em} 
 
\begin{tabular}{r|rrrrrr}
\hline\hline
 &\multicolumn{6}{c}{$\tau$} \\ 
 $\alpha = 0.2$ & $0$ & $1$ & $2$ & $3$ & $4$ & $5$ \\ 
 \hline$0.1$ & $84$ & $80$ & $71$ & $67$ & $65$ & $64$\\ 
$\pi^2\;\;\;$ $0.2$ & $85$ & $80$ & $71$ & $70$ & $70$ & $76$\\ 
$0.3$ & $84$ & $79$ & $73$ & $72$ & $78$ & $81$\\ 
$0.4$ & $84$ & $79$ & $74$ & $77$ & $81$ & $81$\\ 
 \hline 
 \end{tabular}
  \end{subtable}
  ~
  \begin{subtable}{0.48\textwidth}
    \caption{Relative Width}
    \begin{tabular}{r|rrrrrr}
\hline\hline
 &\multicolumn{6}{c}{$\tau$} \\ 
 $\alpha = 0.05$ & $0$ & $1$ & $2$ & $3$ & $4$ & $5$ \\ 
 \hline$0.1$ & $ 93$ & $ 93$ & $ 95$ & $ 97$ & $ 99$ & $102$\\ 
$\pi^2\;\;\;$ $0.2$ & $ 96$ & $ 97$ & $ 99$ & $104$ & $106$ & $109$\\ 
$0.3$ & $ 97$ & $ 99$ & $102$ & $106$ & $108$ & $107$\\ 
$0.4$ & $ 98$ & $100$ & $105$ & $108$ & $106$ & $103$\\ 
 \hline 
 \end{tabular}
 
 \vspace{2em} 
 
\begin{tabular}{r|rrrrrr}
\hline\hline
 &\multicolumn{6}{c}{$\tau$} \\ 
 $\alpha = 0.1$ & $0$ & $1$ & $2$ & $3$ & $4$ & $5$ \\ 
 \hline$0.1$ & $ 90$ & $ 91$ & $ 92$ & $ 97$ & $ 99$ & $102$\\ 
$\pi^2\;\;\;$ $0.2$ & $ 94$ & $ 96$ & $100$ & $105$ & $108$ & $110$\\ 
$0.3$ & $ 96$ & $100$ & $104$ & $108$ & $109$ & $106$\\ 
$0.4$ & $ 97$ & $101$ & $106$ & $108$ & $106$ & $103$\\ 
 \hline 
 \end{tabular}
 
 \vspace{2em} 
 
\begin{tabular}{r|rrrrrr}
\hline\hline
 &\multicolumn{6}{c}{$\tau$} \\ 
 $\alpha = 0.2$ & $0$ & $1$ & $2$ & $3$ & $4$ & $5$ \\ 
 \hline$0.1$ & $ 83$ & $ 84$ & $ 87$ & $ 93$ & $ 99$ & $103$\\ 
$\pi^2\;\;\;$ $0.2$ & $ 91$ & $ 92$ & $ 96$ & $105$ & $109$ & $110$\\ 
$0.3$ & $ 93$ & $ 97$ & $104$ & $109$ & $108$ & $106$\\ 
$0.4$ & $ 95$ & $100$ & $107$ & $108$ & $105$ & $102$\\ 
 \hline 
 \end{tabular}
  \end{subtable}
  \label{tab:Limit1StepShortOLSvsIV}
  \caption{OLS vs TSLS Example: 1-step shortest CI, limit sim.}
\end{table}


\begin{table}[h]
  \centering
  \begin{subtable}{0.48\textwidth}
    \caption{Coverage Probability}
    \begin{tabular}{r|rrrrrr}
\hline\hline
 &\multicolumn{6}{c}{$\tau$} \\ 
 $\alpha = 0.05$ & $0$ & $1$ & $2$ & $3$ & $4$ & $5$ \\ 
 \hline$0.1$ & $96$ & $95$ & $94$ & $93$ & $94$ & $95$\\ 
$\gamma^2\;\;\;$ $0.2$ & $96$ & $96$ & $95$ & $93$ & $93$ & $94$\\ 
$0.3$ & $95$ & $95$ & $95$ & $93$ & $92$ & $92$\\ 
$0.4$ & $95$ & $95$ & $95$ & $94$ & $92$ & $91$\\ 
 \hline 
 \end{tabular}
 
 \vspace{2em} 
 
\begin{tabular}{r|rrrrrr}
\hline\hline
 &\multicolumn{6}{c}{$\tau$} \\ 
 $\alpha = 0.1$ & $0$ & $1$ & $2$ & $3$ & $4$ & $5$ \\ 
 \hline$0.1$ & $92$ & $90$ & $88$ & $88$ & $89$ & $91$\\ 
$\gamma^2\;\;\;$ $0.2$ & $92$ & $91$ & $88$ & $86$ & $86$ & $89$\\ 
$0.3$ & $92$ & $91$ & $89$ & $86$ & $85$ & $87$\\ 
$0.4$ & $91$ & $91$ & $89$ & $86$ & $84$ & $84$\\ 
 \hline 
 \end{tabular}
 
 \vspace{2em} 
 
\begin{tabular}{r|rrrrrr}
\hline\hline
 &\multicolumn{6}{c}{$\tau$} \\ 
 $\alpha = 0.2$ & $0$ & $1$ & $2$ & $3$ & $4$ & $5$ \\ 
 \hline$0.1$ & $83$ & $80$ & $76$ & $77$ & $80$ & $81$\\ 
$\gamma^2\;\;\;$ $0.2$ & $84$ & $80$ & $75$ & $73$ & $76$ & $80$\\ 
$0.3$ & $85$ & $81$ & $75$ & $71$ & $72$ & $77$\\ 
$0.4$ & $84$ & $80$ & $73$ & $72$ & $69$ & $74$\\ 
 \hline 
 \end{tabular}
  \end{subtable}
  ~
  \begin{subtable}{0.48\textwidth}
    \caption{Relative Width}
    \begin{tabular}{r|rrrrrr}
\hline\hline
 &\multicolumn{6}{c}{$\tau$} \\ 
 $\alpha = 0.05$ & $0$ & $1$ & $2$ & $3$ & $4$ & $5$ \\ 
 \hline$0.1$ & $ 98$ & $100$ & $104$ & $106$ & $106$ & $104$\\ 
$\gamma^2\;\;\;$ $0.2$ & $ 97$ & $ 99$ & $103$ & $106$ & $108$ & $108$\\ 
$0.3$ & $ 97$ & $ 98$ & $101$ & $104$ & $107$ & $109$\\ 
$0.4$ & $ 97$ & $ 97$ & $ 99$ & $103$ & $106$ & $108$\\ 
 \hline 
 \end{tabular}
 
 \vspace{2em} 
 
\begin{tabular}{r|rrrrrr}
\hline\hline
 &\multicolumn{6}{c}{$\tau$} \\ 
 $\alpha = 0.1$ & $0$ & $1$ & $2$ & $3$ & $4$ & $5$ \\ 
 \hline$0.1$ & $ 98$ & $100$ & $104$ & $107$ & $106$ & $104$\\ 
$\gamma^2\;\;\;$ $0.2$ & $ 97$ & $ 98$ & $103$ & $107$ & $109$ & $107$\\ 
$0.3$ & $ 96$ & $ 97$ & $101$ & $105$ & $108$ & $109$\\ 
$0.4$ & $ 95$ & $ 96$ & $100$ & $103$ & $107$ & $109$\\ 
 \hline 
 \end{tabular}
 
 \vspace{2em} 
 
\begin{tabular}{r|rrrrrr}
\hline\hline
 &\multicolumn{6}{c}{$\tau$} \\ 
 $\alpha = 0.2$ & $0$ & $1$ & $2$ & $3$ & $4$ & $5$ \\ 
 \hline$0.1$ & $ 98$ & $100$ & $105$ & $107$ & $106$ & $103$\\ 
$\gamma^2\;\;\;$ $0.2$ & $ 94$ & $ 97$ & $104$ & $108$ & $109$ & $107$\\ 
$0.3$ & $ 93$ & $ 96$ & $101$ & $106$ & $109$ & $109$\\ 
$0.4$ & $ 89$ & $ 93$ & $ 97$ & $105$ & $108$ & $110$\\ 
 \hline 
 \end{tabular}
  \end{subtable}
  \label{tab:Limit1StepShortChooseIVs}
  \caption{Choose IVs Example: 1-step shortest CI, limit sim.}
\end{table}



%%%%%%%%%%%%%%%%%%%%%%%%%%%%%%%%%%%%%%%%%%%%%%%
\begin{table}[h]
  \centering
  \begin{subtable}{0.48\textwidth}
    \caption{Coverage Probability}
    \begin{tabular}{r|rrrrrr}
\hline\hline
 &\multicolumn{6}{c}{$\tau$} \\ 
 $\alpha = 0.05$ & $0$ & $1$ & $2$ & $3$ & $4$ & $5$ \\ 
 \hline$0.1$ & $94$ & $94$ & $96$ & $96$ & $94$ & $90$\\ 
$\pi^2\;\;\;$ $0.2$ & $95$ & $95$ & $95$ & $94$ & $91$ & $92$\\ 
$0.3$ & $95$ & $96$ & $94$ & $93$ & $93$ & $94$\\ 
$0.4$ & $96$ & $95$ & $93$ & $93$ & $95$ & $95$\\ 
 \hline 
 \end{tabular}
 
 \vspace{2em} 
 
\begin{tabular}{r|rrrrrr}
\hline\hline
 &\multicolumn{6}{c}{$\tau$} \\ 
 $\alpha = 0.1$ & $0$ & $1$ & $2$ & $3$ & $4$ & $5$ \\ 
 \hline$0.1$ & $90$ & $90$ & $91$ & $89$ & $85$ & $81$\\ 
$\pi^2\;\;\;$ $0.2$ & $91$ & $91$ & $89$ & $85$ & $84$ & $86$\\ 
$0.3$ & $92$ & $91$ & $88$ & $86$ & $87$ & $90$\\ 
$0.4$ & $92$ & $90$ & $87$ & $87$ & $90$ & $91$\\ 
 \hline 
 \end{tabular}
 
 \vspace{2em} 
 
\begin{tabular}{r|rrrrrr}
\hline\hline
 &\multicolumn{6}{c}{$\tau$} \\ 
 $\alpha = 0.2$ & $0$ & $1$ & $2$ & $3$ & $4$ & $5$ \\ 
 \hline$0.1$ & $86$ & $84$ & $77$ & $70$ & $68$ & $68$\\ 
$\pi^2\;\;\;$ $0.2$ & $86$ & $83$ & $75$ & $72$ & $72$ & $77$\\ 
$0.3$ & $86$ & $81$ & $74$ & $73$ & $78$ & $81$\\ 
$0.4$ & $86$ & $79$ & $75$ & $77$ & $81$ & $81$\\ 
 \hline 
 \end{tabular}
  \end{subtable}
  ~
  \begin{subtable}{0.48\textwidth}
    \caption{Relative Width}
    \begin{tabular}{r|rrrrrr}
\hline\hline
 &\multicolumn{6}{c}{$\tau$} \\ 
 $\alpha = 0.05$ & $0$ & $1$ & $2$ & $3$ & $4$ & $5$ \\ 
 \hline$0.1$ & $ 96$ & $ 96$ & $ 97$ & $ 99$ & $101$ & $103$\\ 
$\pi^2\;\;\;$ $0.2$ & $ 97$ & $ 98$ & $100$ & $104$ & $107$ & $109$\\ 
$0.3$ & $ 97$ & $ 99$ & $103$ & $107$ & $109$ & $107$\\ 
$0.4$ & $ 98$ & $100$ & $105$ & $108$ & $107$ & $104$\\ 
 \hline 
 \end{tabular}
 
 \vspace{2em} 
 
\begin{tabular}{r|rrrrrr}
\hline\hline
 &\multicolumn{6}{c}{$\tau$} \\ 
 $\alpha = 0.1$ & $0$ & $1$ & $2$ & $3$ & $4$ & $5$ \\ 
 \hline$0.1$ & $ 93$ & $ 94$ & $ 95$ & $ 99$ & $103$ & $106$\\ 
$\pi^2\;\;\;$ $0.2$ & $ 95$ & $ 96$ & $100$ & $105$ & $110$ & $110$\\ 
$0.3$ & $ 97$ & $ 98$ & $105$ & $108$ & $110$ & $108$\\ 
$0.4$ & $ 97$ & $101$ & $107$ & $109$ & $106$ & $102$\\ 
 \hline 
 \end{tabular}
 
 \vspace{2em} 
 
\begin{tabular}{r|rrrrrr}
\hline\hline
 &\multicolumn{6}{c}{$\tau$} \\ 
 $\alpha = 0.2$ & $0$ & $1$ & $2$ & $3$ & $4$ & $5$ \\ 
 \hline$0.1$ & $ 90$ & $ 90$ & $ 94$ & $ 99$ & $104$ & $109$\\ 
$\pi^2\;\;\;$ $0.2$ & $ 93$ & $ 97$ & $102$ & $108$ & $111$ & $113$\\ 
$0.3$ & $ 96$ & $100$ & $106$ & $109$ & $110$ & $107$\\ 
$0.4$ & $ 97$ & $101$ & $108$ & $109$ & $106$ & $103$\\ 
 \hline 
 \end{tabular}
  \end{subtable}
  \caption{OLS vs TSLS Example: 1-step equal-tailed CI, limit sim.}
  \label{tab:Limit1StepEqualOLSvsIV}
\end{table}

\begin{table}[h]
  \centering
  \begin{subtable}{0.48\textwidth}
    \caption{Coverage Probability}
    \begin{tabular}{r|rrrrrr}
\hline\hline
 &\multicolumn{6}{c}{$\tau$} \\ 
 $\alpha = 0.05$ & $0$ & $1$ & $2$ & $3$ & $4$ & $5$ \\ 
 \hline$0.1$ & $96$ & $95$ & $94$ & $94$ & $94$ & $95$\\ 
$\gamma^2\;\;\;$ $0.2$ & $96$ & $96$ & $95$ & $93$ & $93$ & $94$\\ 
$0.3$ & $95$ & $96$ & $95$ & $93$ & $93$ & $92$\\ 
$0.4$ & $95$ & $95$ & $96$ & $94$ & $92$ & $91$\\ 
 \hline 
 \end{tabular}
 
 \vspace{2em} 
 
\begin{tabular}{r|rrrrrr}
\hline\hline
 &\multicolumn{6}{c}{$\tau$} \\ 
 $\alpha = 0.1$ & $0$ & $1$ & $2$ & $3$ & $4$ & $5$ \\ 
 \hline$0.1$ & $92$ & $90$ & $88$ & $88$ & $89$ & $91$\\ 
$\gamma^2\;\;\;$ $0.2$ & $92$ & $91$ & $88$ & $86$ & $87$ & $89$\\ 
$0.3$ & $92$ & $91$ & $89$ & $86$ & $86$ & $87$\\ 
$0.4$ & $91$ & $91$ & $90$ & $86$ & $84$ & $85$\\ 
 \hline 
 \end{tabular}
 
 \vspace{2em} 
 
\begin{tabular}{r|rrrrrr}
\hline\hline
 &\multicolumn{6}{c}{$\tau$} \\ 
 $\alpha = 0.2$ & $0$ & $1$ & $2$ & $3$ & $4$ & $5$ \\ 
 \hline$0.1$ & $84$ & $81$ & $76$ & $77$ & $80$ & $81$\\ 
$\gamma^2\;\;\;$ $0.2$ & $86$ & $82$ & $75$ & $74$ & $77$ & $80$\\ 
$0.3$ & $86$ & $82$ & $76$ & $72$ & $73$ & $78$\\ 
$0.4$ & $86$ & $83$ & $77$ & $72$ & $72$ & $75$\\ 
 \hline 
 \end{tabular}
  \end{subtable}
  ~
  \begin{subtable}{0.48\textwidth}
    \caption{Relative Width}
    \begin{tabular}{r|rrrrrr}
\hline\hline
 &\multicolumn{6}{c}{$\tau$} \\ 
 $\alpha = 0.05$ & $0$ & $1$ & $2$ & $3$ & $4$ & $5$ \\ 
 \hline$0.1$ & $ 99$ & $100$ & $104$ & $106$ & $107$ & $104$\\ 
$\gamma^2\;\;\;$ $0.2$ & $ 98$ & $ 99$ & $103$ & $106$ & $108$ & $109$\\ 
$0.3$ & $ 97$ & $ 98$ & $101$ & $105$ & $109$ & $109$\\ 
$0.4$ & $ 97$ & $ 98$ & $100$ & $103$ & $106$ & $109$\\ 
 \hline 
 \end{tabular}
 
 \vspace{2em} 
 
\begin{tabular}{r|rrrrrr}
\hline\hline
 &\multicolumn{6}{c}{$\tau$} \\ 
 $\alpha = 0.1$ & $0$ & $1$ & $2$ & $3$ & $4$ & $5$ \\ 
 \hline$0.1$ & $ 98$ & $100$ & $105$ & $107$ & $106$ & $104$\\ 
$\gamma^2\;\;\;$ $0.2$ & $ 97$ & $ 99$ & $103$ & $107$ & $110$ & $108$\\ 
$0.3$ & $ 96$ & $ 97$ & $101$ & $106$ & $110$ & $110$\\ 
$0.4$ & $ 96$ & $ 97$ & $101$ & $105$ & $108$ & $110$\\ 
 \hline 
 \end{tabular}
 
 \vspace{2em} 
 
\begin{tabular}{r|rrrrrr}
\hline\hline
 &\multicolumn{6}{c}{$\tau$} \\ 
 $\alpha = 0.2$ & $0$ & $1$ & $2$ & $3$ & $4$ & $5$ \\ 
 \hline$0.1$ & $ 98$ & $101$ & $105$ & $107$ & $106$ & $103$\\ 
$\gamma^2\;\;\;$ $0.2$ & $ 98$ & $ 99$ & $104$ & $109$ & $110$ & $108$\\ 
$0.3$ & $ 96$ & $ 96$ & $103$ & $107$ & $110$ & $111$\\ 
$0.4$ & $ 94$ & $ 95$ & $102$ & $106$ & $111$ & $112$\\ 
 \hline 
 \end{tabular}
  \end{subtable}
  \caption{Choose IVs Example: 1-step equal-tailed CI, limit sim.}
  \label{tab:Limit1StepEqualChooseIVs}
\end{table}




\begin{table}[h]
  \centering
  \begin{subtable}{0.48\textwidth}
    \caption{Coverage Probability}
    \begin{tabular}{r|rrrrrr}
\hline\hline
 &\multicolumn{6}{c}{$\tau$} \\ 
 $\alpha = 0.05$ & $0$ & $1$ & $2$ & $3$ & $4$ & $5$ \\ 
 \hline$0.1$ & $98$ & $98$ & $98$ & $98$ & $99$ & $98$\\ 
$\pi^2\;\;\;$ $0.2$ & $98$ & $98$ & $98$ & $98$ & $98$ & $98$\\ 
$0.3$ & $98$ & $98$ & $98$ & $98$ & $98$ & $98$\\ 
$0.4$ & $98$ & $98$ & $98$ & $98$ & $98$ & $98$\\ 
 \hline 
 \end{tabular}
 
 \vspace{2em} 
 
\begin{tabular}{r|rrrrrr}
\hline\hline
 &\multicolumn{6}{c}{$\tau$} \\ 
 $\alpha = 0.1$ & $0$ & $1$ & $2$ & $3$ & $4$ & $5$ \\ 
 \hline$0.1$ & $96$ & $96$ & $97$ & $97$ & $97$ & $96$\\ 
$\pi^2\;\;\;$ $0.2$ & $96$ & $97$ & $97$ & $97$ & $96$ & $95$\\ 
$0.3$ & $97$ & $97$ & $97$ & $96$ & $95$ & $96$\\ 
$0.4$ & $97$ & $97$ & $96$ & $95$ & $96$ & $97$\\ 
 \hline 
 \end{tabular}
 
 \vspace{2em} 
 
\begin{tabular}{r|rrrrrr}
\hline\hline
 &\multicolumn{6}{c}{$\tau$} \\ 
 $\alpha = 0.2$ & $0$ & $1$ & $2$ & $3$ & $4$ & $5$ \\ 
 \hline$0.1$ & $93$ & $94$ & $94$ & $94$ & $94$ & $93$\\ 
$\pi^2\;\;\;$ $0.2$ & $94$ & $94$ & $94$ & $93$ & $92$ & $90$\\ 
$0.3$ & $95$ & $94$ & $93$ & $91$ & $90$ & $92$\\ 
$0.4$ & $95$ & $94$ & $92$ & $90$ & $92$ & $93$\\ 
 \hline 
 \end{tabular}
  \end{subtable}
  ~
  \begin{subtable}{0.48\textwidth}
    \caption{Relative Width}
    \begin{tabular}{r|rrrrrr}
\hline\hline
 &\multicolumn{6}{c}{$\tau$} \\ 
 $\alpha = 0.05$ & $0$ & $1$ & $2$ & $3$ & $4$ & $5$ \\ 
 \hline$0.1$ & $121$ & $121$ & $123$ & $125$ & $129$ & $131$\\ 
$\pi^2\;\;\;$ $0.2$ & $123$ & $124$ & $127$ & $129$ & $132$ & $134$\\ 
$0.3$ & $124$ & $125$ & $128$ & $130$ & $132$ & $132$\\ 
$0.4$ & $124$ & $125$ & $128$ & $129$ & $129$ & $128$\\ 
 \hline 
 \end{tabular}
 
 \vspace{2em} 
 
\begin{tabular}{r|rrrrrr}
\hline\hline
 &\multicolumn{6}{c}{$\tau$} \\ 
 $\alpha = 0.1$ & $0$ & $1$ & $2$ & $3$ & $4$ & $5$ \\ 
 \hline$0.1$ & $131$ & $130$ & $134$ & $136$ & $140$ & $142$\\ 
$\pi^2\;\;\;$ $0.2$ & $131$ & $132$ & $136$ & $139$ & $141$ & $142$\\ 
$0.3$ & $132$ & $133$ & $136$ & $138$ & $139$ & $139$\\ 
$0.4$ & $133$ & $134$ & $135$ & $136$ & $136$ & $134$\\ 
 \hline 
 \end{tabular}
 
 \vspace{2em} 
 
\begin{tabular}{r|rrrrrr}
\hline\hline
 &\multicolumn{6}{c}{$\tau$} \\ 
 $\alpha = 0.2$ & $0$ & $1$ & $2$ & $3$ & $4$ & $5$ \\ 
 \hline$0.1$ & $148$ & $150$ & $150$ & $153$ & $158$ & $160$\\ 
$\pi^2\;\;\;$ $0.2$ & $148$ & $149$ & $151$ & $155$ & $156$ & $158$\\ 
$0.3$ & $147$ & $149$ & $150$ & $152$ & $153$ & $151$\\ 
$0.4$ & $146$ & $148$ & $149$ & $149$ & $148$ & $144$\\ 
 \hline 
 \end{tabular}
  \end{subtable}
  \label{tab:Limit2StepEqualOLSvsIV}
  \caption{OLS vs TSLS Example: 2-step $\alpha_1 = \alpha_2 = \alpha/2$ CI, limit sim.}
\end{table}

\begin{table}[h]
  \centering
  \begin{subtable}{0.48\textwidth}
    \caption{Coverage Probability}
    \begin{tabular}{r|rrrrrr}
\hline\hline
 &\multicolumn{6}{c}{$\tau$} \\ 
 $\alpha = 0.05$ & $0$ & $1$ & $2$ & $3$ & $4$ & $5$ \\ 
 \hline$0.1$ & $99$ & $99$ & $98$ & $98$ & $98$ & $98$\\ 
$\gamma^2\;\;\;$ $0.2$ & $98$ & $98$ & $98$ & $98$ & $98$ & $97$\\ 
$0.3$ & $98$ & $98$ & $98$ & $98$ & $98$ & $98$\\ 
$0.4$ & $98$ & $98$ & $98$ & $98$ & $98$ & $98$\\ 
 \hline 
 \end{tabular}
 
 \vspace{2em} 
 
\begin{tabular}{r|rrrrrr}
\hline\hline
 &\multicolumn{6}{c}{$\tau$} \\ 
 $\alpha = 0.1$ & $0$ & $1$ & $2$ & $3$ & $4$ & $5$ \\ 
 \hline$0.1$ & $98$ & $97$ & $96$ & $95$ & $95$ & $96$\\ 
$\gamma^2\;\;\;$ $0.2$ & $97$ & $97$ & $97$ & $96$ & $95$ & $95$\\ 
$0.3$ & $97$ & $97$ & $97$ & $96$ & $96$ & $95$\\ 
$0.4$ & $97$ & $97$ & $97$ & $97$ & $96$ & $96$\\ 
 \hline 
 \end{tabular}
 
 \vspace{2em} 
 
\begin{tabular}{r|rrrrrr}
\hline\hline
 &\multicolumn{6}{c}{$\tau$} \\ 
 $\alpha = 0.2$ & $0$ & $1$ & $2$ & $3$ & $4$ & $5$ \\ 
 \hline$0.1$ & $95$ & $94$ & $91$ & $90$ & $91$ & $92$\\ 
$\gamma^2\;\;\;$ $0.2$ & $95$ & $94$ & $93$ & $91$ & $90$ & $91$\\ 
$0.3$ & $95$ & $94$ & $94$ & $93$ & $91$ & $90$\\ 
$0.4$ & $94$ & $94$ & $94$ & $93$ & $92$ & $90$\\ 
 \hline 
 \end{tabular}
  \end{subtable}
  ~
  \begin{subtable}{0.48\textwidth}
    \caption{Relative Width}
    \begin{tabular}{r|rrrrrr}
\hline\hline
 &\multicolumn{6}{c}{$\tau$} \\ 
 $\alpha = 0.05$ & $0$ & $1$ & $2$ & $3$ & $4$ & $5$ \\ 
 \hline$0.1$ & $124$ & $125$ & $126$ & $126$ & $126$ & $126$\\ 
$\gamma^2\;\;\;$ $0.2$ & $124$ & $125$ & $127$ & $129$ & $130$ & $130$\\ 
$0.3$ & $122$ & $124$ & $127$ & $129$ & $131$ & $132$\\ 
$0.4$ & $123$ & $124$ & $126$ & $129$ & $131$ & $133$\\ 
 \hline 
 \end{tabular}
 
 \vspace{2em} 
 
\begin{tabular}{r|rrrrrr}
\hline\hline
 &\multicolumn{6}{c}{$\tau$} \\ 
 $\alpha = 0.1$ & $0$ & $1$ & $2$ & $3$ & $4$ & $5$ \\ 
 \hline$0.1$ & $132$ & $132$ & $132$ & $133$ & $133$ & $132$\\ 
$\gamma^2\;\;\;$ $0.2$ & $132$ & $133$ & $135$ & $136$ & $138$ & $137$\\ 
$0.3$ & $132$ & $132$ & $135$ & $137$ & $140$ & $140$\\ 
$0.4$ & $132$ & $132$ & $135$ & $137$ & $140$ & $141$\\ 
 \hline 
 \end{tabular}
 
 \vspace{2em} 
 
\begin{tabular}{r|rrrrrr}
\hline\hline
 &\multicolumn{6}{c}{$\tau$} \\ 
 $\alpha = 0.2$ & $0$ & $1$ & $2$ & $3$ & $4$ & $5$ \\ 
 \hline$0.1$ & $145$ & $145$ & $145$ & $145$ & $144$ & $142$\\ 
$\gamma^2\;\;\;$ $0.2$ & $147$ & $148$ & $150$ & $150$ & $151$ & $150$\\ 
$0.3$ & $147$ & $148$ & $151$ & $153$ & $154$ & $154$\\ 
$0.4$ & $148$ & $148$ & $152$ & $154$ & $155$ & $157$\\ 
 \hline 
 \end{tabular}
  \end{subtable}
  \label{tab:Limit2StepEqualChooseIVs}
  \caption{Choose IVs Example: 2-step $\alpha_1 = \alpha_2 = \alpha/2$ CI, limit sim.}
\end{table}



\begin{table}[h]
  \centering
  \begin{subtable}{0.48\textwidth}
    \caption{Coverage Probability}
    \begin{tabular}{r|rrrrrr}
\hline\hline
 &\multicolumn{6}{c}{$\tau$} \\ 
 $\alpha = 0.05$ & $0$ & $1$ & $2$ & $3$ & $4$ & $5$ \\ 
 \hline$0.1$ & $99$ & $99$ & $99$ & $99$ & $99$ & $99$\\ 
$\pi^2\;\;\;$ $0.2$ & $99$ & $99$ & $99$ & $99$ & $99$ & $99$\\ 
$0.3$ & $99$ & $99$ & $99$ & $99$ & $99$ & $99$\\ 
$0.4$ & $99$ & $99$ & $99$ & $99$ & $99$ & $99$\\ 
 \hline 
 \end{tabular}
 
 \vspace{2em} 
 
\begin{tabular}{r|rrrrrr}
\hline\hline
 &\multicolumn{6}{c}{$\tau$} \\ 
 $\alpha = 0.1$ & $0$ & $1$ & $2$ & $3$ & $4$ & $5$ \\ 
 \hline$0.1$ & $98$ & $98$ & $98$ & $98$ & $99$ & $98$\\ 
$\pi^2\;\;\;$ $0.2$ & $98$ & $98$ & $98$ & $98$ & $98$ & $98$\\ 
$0.3$ & $98$ & $98$ & $98$ & $98$ & $98$ & $98$\\ 
$0.4$ & $98$ & $98$ & $98$ & $98$ & $98$ & $98$\\ 
 \hline 
 \end{tabular}
 
 \vspace{2em} 
 
\begin{tabular}{r|rrrrrr}
\hline\hline
 &\multicolumn{6}{c}{$\tau$} \\ 
 $\alpha = 0.2$ & $0$ & $1$ & $2$ & $3$ & $4$ & $5$ \\ 
 \hline$0.1$ & $96$ & $96$ & $97$ & $97$ & $97$ & $96$\\ 
$\pi^2\;\;\;$ $0.2$ & $97$ & $97$ & $97$ & $97$ & $96$ & $95$\\ 
$0.3$ & $97$ & $97$ & $97$ & $96$ & $95$ & $95$\\ 
$0.4$ & $97$ & $97$ & $96$ & $95$ & $96$ & $96$\\ 
 \hline 
 \end{tabular}
  \end{subtable}
  ~
  \begin{subtable}{0.48\textwidth}
    \caption{Relative Width}
    \begin{tabular}{r|rrrrrr}
\hline\hline
 &\multicolumn{6}{c}{$\tau$} \\ 
 $\alpha = 0.05$ & $0$ & $1$ & $2$ & $3$ & $4$ & $5$ \\ 
 \hline$0.1$ & $132$ & $132$ & $134$ & $136$ & $140$ & $143$\\ 
$\pi^2\;\;\;$ $0.2$ & $133$ & $134$ & $138$ & $142$ & $145$ & $147$\\ 
$0.3$ & $135$ & $136$ & $140$ & $143$ & $145$ & $145$\\ 
$0.4$ & $136$ & $137$ & $141$ & $142$ & $143$ & $142$\\ 
 \hline 
 \end{tabular}
 
 \vspace{2em} 
 
\begin{tabular}{r|rrrrrr}
\hline\hline
 &\multicolumn{6}{c}{$\tau$} \\ 
 $\alpha = 0.1$ & $0$ & $1$ & $2$ & $3$ & $4$ & $5$ \\ 
 \hline$0.1$ & $143$ & $144$ & $146$ & $149$ & $152$ & $157$\\ 
$\pi^2\;\;\;$ $0.2$ & $145$ & $146$ & $150$ & $154$ & $157$ & $159$\\ 
$0.3$ & $146$ & $148$ & $152$ & $155$ & $156$ & $157$\\ 
$0.4$ & $147$ & $150$ & $152$ & $154$ & $153$ & $152$\\ 
 \hline 
 \end{tabular}
 
 \vspace{2em} 
 
\begin{tabular}{r|rrrrrr}
\hline\hline
 &\multicolumn{6}{c}{$\tau$} \\ 
 $\alpha = 0.2$ & $0$ & $1$ & $2$ & $3$ & $4$ & $5$ \\ 
 \hline$0.1$ & $165$ & $166$ & $168$ & $173$ & $177$ & $179$\\ 
$\pi^2\;\;\;$ $0.2$ & $168$ & $168$ & $173$ & $177$ & $180$ & $182$\\ 
$0.3$ & $167$ & $169$ & $173$ & $177$ & $178$ & $177$\\ 
$0.4$ & $168$ & $170$ & $174$ & $174$ & $173$ & $169$\\ 
 \hline 
 \end{tabular}
  \end{subtable}
  \label{tab:Limit2StepNarrowTauOLSvsIV}
  \caption{OLS vs TSLS Example: 2-step $\alpha_1 = 3\alpha/4,  \alpha_2 = \alpha/4$ CI, limit sim.}
\end{table}

\begin{table}[h]
  \centering
  \begin{subtable}{0.48\textwidth}
    \caption{Coverage Probability}
    \begin{tabular}{r|rrrrrr}
\hline\hline
 &\multicolumn{6}{c}{$\tau$} \\ 
 $\alpha = 0.05$ & $0$ & $1$ & $2$ & $3$ & $4$ & $5$ \\ 
 \hline$0.1$ & $99$ & $99$ & $99$ & $99$ & $99$ & $99$\\ 
$\gamma^2\;\;\;$ $0.2$ & $99$ & $99$ & $99$ & $99$ & $99$ & $99$\\ 
$0.3$ & $99$ & $99$ & $99$ & $99$ & $99$ & $99$\\ 
$0.4$ & $99$ & $99$ & $99$ & $99$ & $99$ & $99$\\ 
 \hline 
 \end{tabular}
 
 \vspace{2em} 
 
\begin{tabular}{r|rrrrrr}
\hline\hline
 &\multicolumn{6}{c}{$\tau$} \\ 
 $\alpha = 0.1$ & $0$ & $1$ & $2$ & $3$ & $4$ & $5$ \\ 
 \hline$0.1$ & $99$ & $98$ & $98$ & $98$ & $98$ & $98$\\ 
$\gamma^2\;\;\;$ $0.2$ & $98$ & $98$ & $98$ & $98$ & $98$ & $97$\\ 
$0.3$ & $98$ & $98$ & $98$ & $98$ & $98$ & $98$\\ 
$0.4$ & $98$ & $98$ & $98$ & $98$ & $98$ & $98$\\ 
 \hline 
 \end{tabular}
 
 \vspace{2em} 
 
\begin{tabular}{r|rrrrrr}
\hline\hline
 &\multicolumn{6}{c}{$\tau$} \\ 
 $\alpha = 0.2$ & $0$ & $1$ & $2$ & $3$ & $4$ & $5$ \\ 
 \hline$0.1$ & $97$ & $97$ & $96$ & $95$ & $95$ & $96$\\ 
$\gamma^2\;\;\;$ $0.2$ & $97$ & $97$ & $97$ & $96$ & $95$ & $95$\\ 
$0.3$ & $97$ & $97$ & $97$ & $97$ & $96$ & $95$\\ 
$0.4$ & $97$ & $97$ & $97$ & $97$ & $96$ & $96$\\ 
 \hline 
 \end{tabular}
  \end{subtable}
  ~
  \begin{subtable}{0.48\textwidth}
    \caption{Relative Width}
    \begin{tabular}{r|rrrrrr}
\hline\hline
 &\multicolumn{6}{c}{$\tau$} \\ 
 $\alpha = 0.05$ & $0$ & $1$ & $2$ & $3$ & $4$ & $5$ \\ 
 \hline$0.1$ & $136$ & $137$ & $138$ & $139$ & $140$ & $140$\\ 
$\gamma^2\;\;\;$ $0.2$ & $135$ & $137$ & $139$ & $142$ & $143$ & $144$\\ 
$0.3$ & $134$ & $134$ & $138$ & $142$ & $144$ & $145$\\ 
$0.4$ & $133$ & $134$ & $138$ & $141$ & $144$ & $145$\\ 
 \hline 
 \end{tabular}
 
 \vspace{2em} 
 
\begin{tabular}{r|rrrrrr}
\hline\hline
 &\multicolumn{6}{c}{$\tau$} \\ 
 $\alpha = 0.1$ & $0$ & $1$ & $2$ & $3$ & $4$ & $5$ \\ 
 \hline$0.1$ & $148$ & $148$ & $150$ & $150$ & $150$ & $150$\\ 
$\gamma^2\;\;\;$ $0.2$ & $147$ & $147$ & $150$ & $153$ & $155$ & $155$\\ 
$0.3$ & $145$ & $147$ & $150$ & $153$ & $156$ & $158$\\ 
$0.4$ & $145$ & $146$ & $150$ & $153$ & $156$ & $158$\\ 
 \hline 
 \end{tabular}
 
 \vspace{2em} 
 
\begin{tabular}{r|rrrrrr}
\hline\hline
 &\multicolumn{6}{c}{$\tau$} \\ 
 $\alpha = 0.2$ & $0$ & $1$ & $2$ & $3$ & $4$ & $5$ \\ 
 \hline$0.1$ & $168$ & $168$ & $169$ & $170$ & $170$ & $168$\\ 
$\gamma^2\;\;\;$ $0.2$ & $167$ & $169$ & $171$ & $174$ & $176$ & $175$\\ 
$0.3$ & $167$ & $168$ & $171$ & $176$ & $178$ & $179$\\ 
$0.4$ & $167$ & $167$ & $170$ & $175$ & $178$ & $181$\\ 
 \hline 
 \end{tabular}
  \end{subtable}
  \label{tab:Limit2StepNarrowTauChooseIVs}
  \caption{Choose IVs Example: 2-step $\alpha_1 = 3\alpha/4,  \alpha_2 = \alpha/4$ CI, limit sim.}
\end{table}


%%%%%%%%%%%%%%%%%%%%%%%%%%%%%%%%%%%%%%%%%%%%%%%%%%%%%%%%%%


\begin{table}[h]
  \centering
  \begin{tabular}{r|rrrrrr}
\hline\hline
 &\multicolumn{6}{c}{$\rho$} \\ 
 $\alpha = 0.05$ & $0$ & $0.1$ & $0.2$ & $0.3$ & $0.4$ & $0.5$ \\ 
 \hline$0.1$ & $99$ & $98$ & $98$ & $96$ & $93$ & $88$\\ 
$\pi^2\;\;\;$ $0.2$ & $96$ & $97$ & $96$ & $94$ & $93$ & $90$\\ 
$0.3$ & $96$ & $96$ & $96$ & $94$ & $93$ & $92$\\ 
$0.4$ & $96$ & $96$ & $95$ & $94$ & $93$ & $92$\\ 
 \hline 
 \end{tabular}
 
 \vspace{2em} 
 
\begin{tabular}{r|rrrrrr}
\hline\hline
 &\multicolumn{6}{c}{$\rho$} \\ 
 $\alpha = 0.1$ & $0$ & $0.1$ & $0.2$ & $0.3$ & $0.4$ & $0.5$ \\ 
 \hline$0.1$ & $96$ & $95$ & $93$ & $91$ & $89$ & $84$\\ 
$\pi^2\;\;\;$ $0.2$ & $93$ & $91$ & $92$ & $91$ & $89$ & $85$\\ 
$0.3$ & $91$ & $91$ & $90$ & $90$ & $90$ & $86$\\ 
$0.4$ & $90$ & $90$ & $89$ & $90$ & $89$ & $88$\\ 
 \hline 
 \end{tabular}
 
 \vspace{2em} 
 
\begin{tabular}{r|rrrrrr}
\hline\hline
 &\multicolumn{6}{c}{$\rho$} \\ 
 $\alpha = 0.2$ & $0$ & $0.1$ & $0.2$ & $0.3$ & $0.4$ & $0.5$ \\ 
 \hline$0.1$ & $87$ & $86$ & $84$ & $83$ & $80$ & $76$\\ 
$\pi^2\;\;\;$ $0.2$ & $82$ & $82$ & $83$ & $80$ & $81$ & $76$\\ 
$0.3$ & $83$ & $83$ & $80$ & $78$ & $77$ & $80$\\ 
$0.4$ & $80$ & $82$ & $78$ & $81$ & $79$ & $78$\\ 
 \hline 
 \end{tabular}
  \label{tab:CISim50TSLS_OLSvsIV}
  \caption{Coverage prob of TSLS Estimator, OLS vs IV Example, $N=50$}
\end{table}

\begin{table}[h]
  \centering
  \begin{tabular}{r|rrrrrr}
\hline\hline
 &\multicolumn{6}{c}{$\rho$} \\ 
 $\alpha = 0.05$ & $0$ & $0.1$ & $0.2$ & $0.3$ & $0.4$ & $0.5$ \\ 
 \hline$0.1$ & $97$ & $97$ & $97$ & $95$ & $93$ & $91$\\ 
$\pi^2\;\;\;$ $0.2$ & $96$ & $97$ & $96$ & $96$ & $94$ & $91$\\ 
$0.3$ & $96$ & $96$ & $95$ & $95$ & $94$ & $93$\\ 
$0.4$ & $96$ & $96$ & $96$ & $94$ & $94$ & $94$\\ 
 \hline 
 \end{tabular}
 
 \vspace{2em} 
 
\begin{tabular}{r|rrrrrr}
\hline\hline
 &\multicolumn{6}{c}{$\rho$} \\ 
 $\alpha = 0.1$ & $0$ & $0.1$ & $0.2$ & $0.3$ & $0.4$ & $0.5$ \\ 
 \hline$0.1$ & $95$ & $94$ & $93$ & $90$ & $90$ & $86$\\ 
$\pi^2\;\;\;$ $0.2$ & $89$ & $91$ & $90$ & $89$ & $89$ & $87$\\ 
$0.3$ & $90$ & $92$ & $90$ & $89$ & $89$ & $88$\\ 
$0.4$ & $90$ & $89$ & $89$ & $91$ & $90$ & $89$\\ 
 \hline 
 \end{tabular}
 
 \vspace{2em} 
 
\begin{tabular}{r|rrrrrr}
\hline\hline
 &\multicolumn{6}{c}{$\rho$} \\ 
 $\alpha = 0.2$ & $0$ & $0.1$ & $0.2$ & $0.3$ & $0.4$ & $0.5$ \\ 
 \hline$0.1$ & $83$ & $83$ & $83$ & $82$ & $79$ & $79$\\ 
$\pi^2\;\;\;$ $0.2$ & $82$ & $81$ & $80$ & $78$ & $80$ & $79$\\ 
$0.3$ & $80$ & $80$ & $80$ & $79$ & $81$ & $79$\\ 
$0.4$ & $80$ & $79$ & $79$ & $80$ & $81$ & $78$\\ 
 \hline 
 \end{tabular}
  \label{tab:CISim100TSLS_OLSvsIV}
  \caption{Coverage prob of TSLS Estimator, OLS vs IV Example, $N=100$}
\end{table}

\begin{table}[h]
  \centering
  \begin{tabular}{r|rrrrrr}
\hline\hline
 &\multicolumn{6}{c}{$\rho$} \\ 
 $\alpha = 0.05$ & $0$ & $0.1$ & $0.2$ & $0.3$ & $0.4$ & $0.5$ \\ 
 \hline$0.1$ & $99$ & $98$ & $98$ & $96$ & $93$ & $88$\\ 
$\pi^2\;\;\;$ $0.2$ & $96$ & $97$ & $96$ & $94$ & $93$ & $90$\\ 
$0.3$ & $96$ & $96$ & $96$ & $94$ & $93$ & $92$\\ 
$0.4$ & $96$ & $96$ & $95$ & $94$ & $93$ & $92$\\ 
 \hline 
 \end{tabular}
 
 \vspace{2em} 
 
\begin{tabular}{r|rrrrrr}
\hline\hline
 &\multicolumn{6}{c}{$\rho$} \\ 
 $\alpha = 0.1$ & $0$ & $0.1$ & $0.2$ & $0.3$ & $0.4$ & $0.5$ \\ 
 \hline$0.1$ & $96$ & $95$ & $93$ & $91$ & $89$ & $84$\\ 
$\pi^2\;\;\;$ $0.2$ & $93$ & $91$ & $92$ & $91$ & $89$ & $85$\\ 
$0.3$ & $91$ & $91$ & $90$ & $90$ & $90$ & $86$\\ 
$0.4$ & $90$ & $90$ & $89$ & $90$ & $89$ & $88$\\ 
 \hline 
 \end{tabular}
 
 \vspace{2em} 
 
\begin{tabular}{r|rrrrrr}
\hline\hline
 &\multicolumn{6}{c}{$\rho$} \\ 
 $\alpha = 0.2$ & $0$ & $0.1$ & $0.2$ & $0.3$ & $0.4$ & $0.5$ \\ 
 \hline$0.1$ & $87$ & $86$ & $84$ & $83$ & $80$ & $76$\\ 
$\pi^2\;\;\;$ $0.2$ & $82$ & $82$ & $83$ & $80$ & $81$ & $76$\\ 
$0.3$ & $83$ & $83$ & $80$ & $78$ & $77$ & $80$\\ 
$0.4$ & $80$ & $82$ & $78$ & $81$ & $79$ & $78$\\ 
 \hline 
 \end{tabular}
  \label{tab:CISim500TSLS_OLSvsIV}
  \caption{Coverage prob of TSLS Estimator, OLS vs IV Example, $N=500$}
\end{table}





\begin{table}[h]
  \centering
  \begin{subtable}{0.48\textwidth}
    \caption{Coverage Probability}
    \begin{tabular}{r|rrrrrr}
\hline\hline
 &\multicolumn{6}{c}{$\rho$} \\ 
 $\alpha = 0.05$ & $0$ & $0.1$ & $0.2$ & $0.3$ & $0.4$ & $0.5$ \\ 
 \hline$0.1$ & $94$ & $90$ & $70$ & $46$ & $34$ & $28$\\ 
$\pi^2\;\;\;$ $0.2$ & $93$ & $89$ & $73$ & $54$ & $50$ & $63$\\ 
$0.3$ & $93$ & $88$ & $78$ & $65$ & $70$ & $83$\\ 
$0.4$ & $92$ & $89$ & $79$ & $77$ & $83$ & $91$\\ 
 \hline 
 \end{tabular}
 
 \vspace{2em} 
 
\begin{tabular}{r|rrrrrr}
\hline\hline
 &\multicolumn{6}{c}{$\rho$} \\ 
 $\alpha = 0.1$ & $0$ & $0.1$ & $0.2$ & $0.3$ & $0.4$ & $0.5$ \\ 
 \hline$0.1$ & $88$ & $79$ & $61$ & $37$ & $26$ & $28$\\ 
$\pi^2\;\;\;$ $0.2$ & $85$ & $79$ & $62$ & $46$ & $49$ & $59$\\ 
$0.3$ & $85$ & $78$ & $66$ & $58$ & $67$ & $80$\\ 
$0.4$ & $86$ & $81$ & $69$ & $69$ & $80$ & $87$\\ 
 \hline 
 \end{tabular}
 
 \vspace{2em} 
 
\begin{tabular}{r|rrrrrr}
\hline\hline
 &\multicolumn{6}{c}{$\rho$} \\ 
 $\alpha = 0.2$ & $0$ & $0.1$ & $0.2$ & $0.3$ & $0.4$ & $0.5$ \\ 
 \hline$0.1$ & $76$ & $67$ & $46$ & $28$ & $20$ & $25$\\ 
$\pi^2\;\;\;$ $0.2$ & $73$ & $66$ & $50$ & $38$ & $46$ & $55$\\ 
$0.3$ & $74$ & $66$ & $52$ & $48$ & $60$ & $74$\\ 
$0.4$ & $73$ & $68$ & $56$ & $59$ & $72$ & $77$\\ 
 \hline 
 \end{tabular}
  \end{subtable}
  ~
  \begin{subtable}{0.48\textwidth}
    \caption{Relative Width}
    \begin{tabular}{r|rrrrrr}
\hline\hline
 &\multicolumn{6}{c}{$\rho$} \\ 
 $\alpha = 0.05$ & $0$ & $0.1$ & $0.2$ & $0.3$ & $0.4$ & $0.5$ \\ 
 \hline$0.1$ & $33$ & $33$ & $34$ & $34$ & $41$ & $43$\\ 
$\pi^2\;\;\;$ $0.2$ & $50$ & $53$ & $53$ & $59$ & $68$ & $77$\\ 
$0.3$ & $61$ & $63$ & $67$ & $76$ & $85$ & $95$\\ 
$0.4$ & $68$ & $70$ & $77$ & $87$ & $94$ & $99$\\ 
 \hline 
 \end{tabular}
 
 \vspace{2em} 
 
\begin{tabular}{r|rrrrrr}
\hline\hline
 &\multicolumn{6}{c}{$\rho$} \\ 
 $\alpha = 0.1$ & $0$ & $0.1$ & $0.2$ & $0.3$ & $0.4$ & $0.5$ \\ 
 \hline$0.1$ & $33$ & $31$ & $34$ & $35$ & $40$ & $41$\\ 
$\pi^2\;\;\;$ $0.2$ & $49$ & $51$ & $53$ & $59$ & $66$ & $76$\\ 
$0.3$ & $61$ & $62$ & $68$ & $75$ & $86$ & $94$\\ 
$0.4$ & $69$ & $72$ & $77$ & $87$ & $95$ & $99$\\ 
 \hline 
 \end{tabular}
 
 \vspace{2em} 
 
\begin{tabular}{r|rrrrrr}
\hline\hline
 &\multicolumn{6}{c}{$\rho$} \\ 
 $\alpha = 0.2$ & $0$ & $0.1$ & $0.2$ & $0.3$ & $0.4$ & $0.5$ \\ 
 \hline$0.1$ & $34$ & $33$ & $32$ & $36$ & $38$ & $42$\\ 
$\pi^2\;\;\;$ $0.2$ & $50$ & $52$ & $54$ & $61$ & $68$ & $74$\\ 
$0.3$ & $60$ & $61$ & $67$ & $76$ & $86$ & $94$\\ 
$0.4$ & $68$ & $72$ & $78$ & $87$ & $95$ & $99$\\ 
 \hline 
 \end{tabular}
  \end{subtable}
  \label{tab:CISim50Naive_OLSvsIV}
  \caption{Naive CI, OLS vs IV Example, $N=50$}
\end{table}

\begin{table}[h]
  \centering
  \begin{subtable}{0.48\textwidth}
    \caption{Coverage Probability}
    \begin{tabular}{r|rrrrrr}
\hline\hline
 &\multicolumn{6}{c}{$\rho$} \\ 
 $\alpha = 0.05$ & $0$ & $0.1$ & $0.2$ & $0.3$ & $0.4$ & $0.5$ \\ 
 \hline$0.1$ & $92$ & $83$ & $56$ & $36$ & $38$ & $55$\\ 
$\pi^2\;\;\;$ $0.2$ & $91$ & $84$ & $63$ & $56$ & $73$ & $85$\\ 
$0.3$ & $92$ & $85$ & $70$ & $73$ & $88$ & $93$\\ 
$0.4$ & $93$ & $85$ & $76$ & $83$ & $93$ & $94$\\ 
 \hline 
 \end{tabular}
 
 \vspace{2em} 
 
\begin{tabular}{r|rrrrrr}
\hline\hline
 &\multicolumn{6}{c}{$\rho$} \\ 
 $\alpha = 0.1$ & $0$ & $0.1$ & $0.2$ & $0.3$ & $0.4$ & $0.5$ \\ 
 \hline$0.1$ & $87$ & $73$ & $42$ & $33$ & $41$ & $54$\\ 
$\pi^2\;\;\;$ $0.2$ & $85$ & $76$ & $50$ & $53$ & $73$ & $83$\\ 
$0.3$ & $86$ & $75$ & $63$ & $71$ & $86$ & $88$\\ 
$0.4$ & $86$ & $74$ & $68$ & $83$ & $89$ & $89$\\ 
 \hline 
 \end{tabular}
 
 \vspace{2em} 
 
\begin{tabular}{r|rrrrrr}
\hline\hline
 &\multicolumn{6}{c}{$\rho$} \\ 
 $\alpha = 0.2$ & $0$ & $0.1$ & $0.2$ & $0.3$ & $0.4$ & $0.5$ \\ 
 \hline$0.1$ & $72$ & $57$ & $29$ & $26$ & $38$ & $53$\\ 
$\pi^2\;\;\;$ $0.2$ & $74$ & $59$ & $40$ & $45$ & $65$ & $77$\\ 
$0.3$ & $74$ & $59$ & $46$ & $62$ & $78$ & $79$\\ 
$0.4$ & $75$ & $61$ & $59$ & $74$ & $81$ & $78$\\ 
 \hline 
 \end{tabular}
  \end{subtable}
  ~
  \begin{subtable}{0.48\textwidth}
    \caption{Relative Width}
    \begin{tabular}{r|rrrrrr}
\hline\hline
 &\multicolumn{6}{c}{$\rho$} \\ 
 $\alpha = 0.05$ & $0$ & $0.1$ & $0.2$ & $0.3$ & $0.4$ & $0.5$ \\ 
 \hline$0.1$ & $ 39$ & $ 37$ & $ 31$ & $ 50$ & $ 56$ & $ 69$\\ 
$\pi^2\;\;\;$ $0.2$ & $ 52$ & $ 54$ & $ 64$ & $ 74$ & $ 86$ & $ 95$\\ 
$0.3$ & $ 60$ & $ 65$ & $ 75$ & $ 87$ & $ 96$ & $100$\\ 
$0.4$ & $ 68$ & $ 73$ & $ 84$ & $ 95$ & $100$ & $100$\\ 
 \hline 
 \end{tabular}
 
 \vspace{2em} 
 
\begin{tabular}{r|rrrrrr}
\hline\hline
 &\multicolumn{6}{c}{$\rho$} \\ 
 $\alpha = 0.1$ & $0$ & $0.1$ & $0.2$ & $0.3$ & $0.4$ & $0.5$ \\ 
 \hline$0.1$ & $ 38$ & $ 39$ & $ 43$ & $ 49$ & $ 54$ & $ 69$\\ 
$\pi^2\;\;\;$ $0.2$ & $ 52$ & $ 55$ & $ 61$ & $ 74$ & $ 87$ & $ 95$\\ 
$0.3$ & $ 61$ & $ 65$ & $ 76$ & $ 88$ & $ 97$ & $100$\\ 
$0.4$ & $ 69$ & $ 74$ & $ 85$ & $ 95$ & $100$ & $100$\\ 
 \hline 
 \end{tabular}
 
 \vspace{2em} 
 
\begin{tabular}{r|rrrrrr}
\hline\hline
 &\multicolumn{6}{c}{$\rho$} \\ 
 $\alpha = 0.2$ & $0$ & $0.1$ & $0.2$ & $0.3$ & $0.4$ & $0.5$ \\ 
 \hline$0.1$ & $ 40$ & $ 41$ & $ 40$ & $ 49$ & $ 57$ & $ 68$\\ 
$\pi^2\;\;\;$ $0.2$ & $ 52$ & $ 55$ & $ 63$ & $ 73$ & $ 86$ & $ 95$\\ 
$0.3$ & $ 61$ & $ 65$ & $ 74$ & $ 88$ & $ 96$ & $ 99$\\ 
$0.4$ & $ 69$ & $ 73$ & $ 85$ & $ 95$ & $100$ & $100$\\ 
 \hline 
 \end{tabular}
  \end{subtable}
  \label{tab:CISim100Naive_OLSvsIV}
  \caption{Naive CI, OLS vs IV Example, $N=100$}
\end{table}

\begin{table}[h]
  \centering
  \begin{subtable}{0.48\textwidth}
    \caption{Coverage Probability}
    \begin{tabular}{r|rrrrrr}
\hline\hline
 &\multicolumn{6}{c}{$\rho$} \\ 
 $\alpha = 0.05$ & $0$ & $0.1$ & $0.2$ & $0.3$ & $0.4$ & $0.5$ \\ 
 \hline$0.1$ & $92$ & $51$ & $50$ & $79$ & $94$ & $94$\\ 
$\pi^2\;\;\;$ $0.2$ & $91$ & $59$ & $78$ & $95$ & $94$ & $94$\\ 
$0.3$ & $92$ & $69$ & $90$ & $96$ & $95$ & $94$\\ 
$0.4$ & $91$ & $76$ & $93$ & $95$ & $95$ & $94$\\ 
 \hline 
 \end{tabular}
 
 \vspace{2em} 
 
\begin{tabular}{r|rrrrrr}
\hline\hline
 &\multicolumn{6}{c}{$\rho$} \\ 
 $\alpha = 0.1$ & $0$ & $0.1$ & $0.2$ & $0.3$ & $0.4$ & $0.5$ \\ 
 \hline$0.1$ & $82$ & $42$ & $48$ & $78$ & $88$ & $90$\\ 
$\pi^2\;\;\;$ $0.2$ & $84$ & $51$ & $76$ & $89$ & $88$ & $89$\\ 
$0.3$ & $85$ & $57$ & $88$ & $90$ & $89$ & $90$\\ 
$0.4$ & $84$ & $69$ & $89$ & $91$ & $91$ & $91$\\ 
 \hline 
 \end{tabular}
 
 \vspace{2em} 
 
\begin{tabular}{r|rrrrrr}
\hline\hline
 &\multicolumn{6}{c}{$\rho$} \\ 
 $\alpha = 0.2$ & $0$ & $0.1$ & $0.2$ & $0.3$ & $0.4$ & $0.5$ \\ 
 \hline$0.1$ & $71$ & $27$ & $44$ & $72$ & $80$ & $79$\\ 
$\pi^2\;\;\;$ $0.2$ & $72$ & $37$ & $74$ & $78$ & $81$ & $80$\\ 
$0.3$ & $73$ & $49$ & $77$ & $79$ & $81$ & $81$\\ 
$0.4$ & $74$ & $62$ & $80$ & $79$ & $80$ & $81$\\ 
 \hline 
 \end{tabular}
  \end{subtable}
  ~
  \begin{subtable}{0.48\textwidth}
    \caption{Relative Width}
    \begin{tabular}{r|rrrrrr}
\hline\hline
 &\multicolumn{6}{c}{$\rho$} \\ 
 $\alpha = 0.05$ & $0$ & $0.1$ & $0.2$ & $0.3$ & $0.4$ & $0.5$ \\ 
 \hline$0.1$ & $ 42$ & $ 49$ & $ 68$ & $ 87$ & $ 98$ & $100$\\ 
$\pi^2\;\;\;$ $0.2$ & $ 53$ & $ 66$ & $ 88$ & $ 99$ & $100$ & $100$\\ 
$0.3$ & $ 62$ & $ 79$ & $ 98$ & $100$ & $100$ & $100$\\ 
$0.4$ & $ 69$ & $ 88$ & $100$ & $100$ & $100$ & $100$\\ 
 \hline 
 \end{tabular}
 
 \vspace{2em} 
 
\begin{tabular}{r|rrrrrr}
\hline\hline
 &\multicolumn{6}{c}{$\rho$} \\ 
 $\alpha = 0.1$ & $0$ & $0.1$ & $0.2$ & $0.3$ & $0.4$ & $0.5$ \\ 
 \hline$0.1$ & $ 42$ & $ 49$ & $ 68$ & $ 88$ & $ 97$ & $100$\\ 
$\pi^2\;\;\;$ $0.2$ & $ 53$ & $ 66$ & $ 89$ & $ 99$ & $100$ & $100$\\ 
$0.3$ & $ 62$ & $ 78$ & $ 98$ & $100$ & $100$ & $100$\\ 
$0.4$ & $ 69$ & $ 88$ & $100$ & $100$ & $100$ & $100$\\ 
 \hline 
 \end{tabular}
 
 \vspace{2em} 
 
\begin{tabular}{r|rrrrrr}
\hline\hline
 &\multicolumn{6}{c}{$\rho$} \\ 
 $\alpha = 0.2$ & $0$ & $0.1$ & $0.2$ & $0.3$ & $0.4$ & $0.5$ \\ 
 \hline$0.1$ & $ 42$ & $ 47$ & $ 68$ & $ 87$ & $ 98$ & $100$\\ 
$\pi^2\;\;\;$ $0.2$ & $ 54$ & $ 65$ & $ 90$ & $ 99$ & $100$ & $100$\\ 
$0.3$ & $ 62$ & $ 78$ & $ 97$ & $100$ & $100$ & $100$\\ 
$0.4$ & $ 69$ & $ 88$ & $100$ & $100$ & $100$ & $100$\\ 
 \hline 
 \end{tabular}
  \end{subtable}
  \label{tab:CISim500Naive_OLSvsIV}
  \caption{Naive CI, OLS vs IV Example, $N=500$}
\end{table}








\begin{table}[h]
  \centering
  \begin{subtable}{0.48\textwidth}
    \caption{Coverage Probability}
    \begin{tabular}{r|rrrrrr}
\hline\hline
 &\multicolumn{6}{c}{$\rho$} \\ 
 $\alpha = 0.05$ & $0$ & $0.1$ & $0.2$ & $0.3$ & $0.4$ & $0.5$ \\ 
 \hline$0.1$ & $99$ & $99$ & $99$ & $97$ & $92$ & $78$\\ 
$\pi^2\;\;\;$ $0.2$ & $97$ & $98$ & $97$ & $92$ & $84$ & $79$\\ 
$0.3$ & $97$ & $97$ & $96$ & $90$ & $84$ & $87$\\ 
$0.4$ & $97$ & $97$ & $93$ & $89$ & $90$ & $94$\\ 
 \hline 
 \end{tabular}
 
 \vspace{2em} 
 
\begin{tabular}{r|rrrrrr}
\hline\hline
 &\multicolumn{6}{c}{$\rho$} \\ 
 $\alpha = 0.1$ & $0$ & $0.1$ & $0.2$ & $0.3$ & $0.4$ & $0.5$ \\ 
 \hline$0.1$ & $98$ & $97$ & $96$ & $90$ & $80$ & $67$\\ 
$\pi^2\;\;\;$ $0.2$ & $95$ & $93$ & $92$ & $84$ & $76$ & $72$\\ 
$0.3$ & $94$ & $92$ & $88$ & $82$ & $79$ & $86$\\ 
$0.4$ & $92$ & $92$ & $86$ & $85$ & $86$ & $91$\\ 
 \hline 
 \end{tabular}
 
 \vspace{2em} 
 
\begin{tabular}{r|rrrrrr}
\hline\hline
 &\multicolumn{6}{c}{$\rho$} \\ 
 $\alpha = 0.2$ & $0$ & $0.1$ & $0.2$ & $0.3$ & $0.4$ & $0.5$ \\ 
 \hline$0.1$ & $92$ & $90$ & $84$ & $76$ & $64$ & $55$\\ 
$\pi^2\;\;\;$ $0.2$ & $89$ & $87$ & $79$ & $69$ & $67$ & $66$\\ 
$0.3$ & $86$ & $85$ & $75$ & $66$ & $69$ & $81$\\ 
$0.4$ & $85$ & $82$ & $73$ & $72$ & $77$ & $83$\\ 
 \hline 
 \end{tabular}
  \end{subtable}
  ~
  \begin{subtable}{0.48\textwidth}
    \caption{Relative Width}
    \begin{tabular}{r|rrrrrr}
\hline\hline
 &\multicolumn{6}{c}{$\rho$} \\ 
 $\alpha = 0.05$ & $0$ & $0.1$ & $0.2$ & $0.3$ & $0.4$ & $0.5$ \\ 
 \hline$0.1$ & $ 93$ & $ 93$ & $ 94$ & $ 94$ & $ 94$ & $ 94$\\ 
$\pi^2\;\;\;$ $0.2$ & $ 96$ & $ 97$ & $ 96$ & $ 98$ & $101$ & $104$\\ 
$0.3$ & $ 97$ & $ 98$ & $100$ & $103$ & $106$ & $111$\\ 
$0.4$ & $ 98$ & $ 99$ & $102$ & $106$ & $109$ & $111$\\ 
 \hline 
 \end{tabular}
 
 \vspace{2em} 
 
\begin{tabular}{r|rrrrrr}
\hline\hline
 &\multicolumn{6}{c}{$\rho$} \\ 
 $\alpha = 0.1$ & $0$ & $0.1$ & $0.2$ & $0.3$ & $0.4$ & $0.5$ \\ 
 \hline$0.1$ & $ 90$ & $ 90$ & $ 89$ & $ 90$ & $ 93$ & $ 94$\\ 
$\pi^2\;\;\;$ $0.2$ & $ 94$ & $ 95$ & $ 96$ & $ 99$ & $103$ & $106$\\ 
$0.3$ & $ 97$ & $ 97$ & $101$ & $104$ & $109$ & $113$\\ 
$0.4$ & $ 98$ & $ 99$ & $103$ & $108$ & $111$ & $111$\\ 
 \hline 
 \end{tabular}
 
 \vspace{2em} 
 
\begin{tabular}{r|rrrrrr}
\hline\hline
 &\multicolumn{6}{c}{$\rho$} \\ 
 $\alpha = 0.2$ & $0$ & $0.1$ & $0.2$ & $0.3$ & $0.4$ & $0.5$ \\ 
 \hline$0.1$ & $ 87$ & $ 88$ & $ 84$ & $ 90$ & $ 93$ & $ 97$\\ 
$\pi^2\;\;\;$ $0.2$ & $ 92$ & $ 94$ & $ 96$ & $101$ & $107$ & $111$\\ 
$0.3$ & $ 94$ & $ 97$ & $101$ & $107$ & $113$ & $117$\\ 
$0.4$ & $ 96$ & $100$ & $105$ & $111$ & $113$ & $110$\\ 
 \hline 
 \end{tabular}
  \end{subtable}
  \label{tab:CISim50_1stepEqual_OLSvsIV}
  \caption{1-step Equal-tailed CI, OLS vs IV Example, $N=50$}
\end{table}

\begin{table}[h]
  \centering
  \begin{subtable}{0.48\textwidth}
    \caption{Coverage Probability}
    \begin{tabular}{r|rrrrrr}
\hline\hline
 &\multicolumn{6}{c}{$\rho$} \\ 
 $\alpha = 0.05$ & $0$ & $0.1$ & $0.2$ & $0.3$ & $0.4$ & $0.5$ \\ 
 \hline$0.1$ & $97$ & $98$ & $98$ & $94$ & $82$ & $72$\\ 
$\pi^2\;\;\;$ $0.2$ & $96$ & $98$ & $95$ & $88$ & $84$ & $88$\\ 
$0.3$ & $96$ & $97$ & $94$ & $87$ & $91$ & $96$\\ 
$0.4$ & $97$ & $96$ & $91$ & $89$ & $96$ & $96$\\ 
 \hline 
 \end{tabular}
 
 \vspace{2em} 
 
\begin{tabular}{r|rrrrrr}
\hline\hline
 &\multicolumn{6}{c}{$\rho$} \\ 
 $\alpha = 0.1$ & $0$ & $0.1$ & $0.2$ & $0.3$ & $0.4$ & $0.5$ \\ 
 \hline$0.1$ & $96$ & $95$ & $94$ & $84$ & $71$ & $66$\\ 
$\pi^2\;\;\;$ $0.2$ & $93$ & $94$ & $87$ & $77$ & $79$ & $87$\\ 
$0.3$ & $93$ & $91$ & $86$ & $80$ & $89$ & $93$\\ 
$0.4$ & $92$ & $89$ & $84$ & $88$ & $93$ & $90$\\ 
 \hline 
 \end{tabular}
 
 \vspace{2em} 
 
\begin{tabular}{r|rrrrrr}
\hline\hline
 &\multicolumn{6}{c}{$\rho$} \\ 
 $\alpha = 0.2$ & $0$ & $0.1$ & $0.2$ & $0.3$ & $0.4$ & $0.5$ \\ 
 \hline$0.1$ & $89$ & $87$ & $79$ & $65$ & $58$ & $63$\\ 
$\pi^2\;\;\;$ $0.2$ & $88$ & $83$ & $69$ & $63$ & $71$ & $84$\\ 
$0.3$ & $87$ & $80$ & $69$ & $71$ & $83$ & $84$\\ 
$0.4$ & $85$ & $77$ & $71$ & $79$ & $85$ & $79$\\ 
 \hline 
 \end{tabular}
  \end{subtable}
  ~
  \begin{subtable}{0.48\textwidth}
    \caption{Relative Width}
    \begin{tabular}{r|rrrrrr}
\hline\hline
 &\multicolumn{6}{c}{$\rho$} \\ 
 $\alpha = 0.05$ & $0$ & $0.1$ & $0.2$ & $0.3$ & $0.4$ & $0.5$ \\ 
 \hline$0.1$ & $ 94$ & $ 93$ & $ 96$ & $ 96$ & $ 97$ & $101$\\ 
$\pi^2\;\;\;$ $0.2$ & $ 96$ & $ 97$ & $100$ & $103$ & $108$ & $112$\\ 
$0.3$ & $ 97$ & $ 99$ & $103$ & $108$ & $111$ & $111$\\ 
$0.4$ & $ 98$ & $100$ & $105$ & $109$ & $108$ & $102$\\ 
 \hline 
 \end{tabular}
 
 \vspace{2em} 
 
\begin{tabular}{r|rrrrrr}
\hline\hline
 &\multicolumn{6}{c}{$\rho$} \\ 
 $\alpha = 0.1$ & $0$ & $0.1$ & $0.2$ & $0.3$ & $0.4$ & $0.5$ \\ 
 \hline$0.1$ & $ 91$ & $ 92$ & $ 94$ & $ 96$ & $ 98$ & $105$\\ 
$\pi^2\;\;\;$ $0.2$ & $ 95$ & $ 97$ & $100$ & $106$ & $112$ & $115$\\ 
$0.3$ & $ 96$ & $ 99$ & $105$ & $111$ & $112$ & $108$\\ 
$0.4$ & $ 98$ & $101$ & $107$ & $110$ & $107$ & $102$\\ 
 \hline 
 \end{tabular}
 
 \vspace{2em} 
 
\begin{tabular}{r|rrrrrr}
\hline\hline
 &\multicolumn{6}{c}{$\rho$} \\ 
 $\alpha = 0.2$ & $0$ & $0.1$ & $0.2$ & $0.3$ & $0.4$ & $0.5$ \\ 
 \hline$0.1$ & $ 89$ & $ 91$ & $ 90$ & $ 98$ & $104$ & $110$\\ 
$\pi^2\;\;\;$ $0.2$ & $ 93$ & $ 95$ & $102$ & $109$ & $116$ & $117$\\ 
$0.3$ & $ 95$ & $ 98$ & $106$ & $112$ & $113$ & $106$\\ 
$0.4$ & $ 97$ & $100$ & $109$ & $111$ & $105$ & $101$\\ 
 \hline 
 \end{tabular}
  \end{subtable}
  \label{tab:CISim100_1stepEqual_OLSvsIV}
  \caption{1-step Equal-tailed, OLS vs IV Example, $N=100$}
\end{table}

\begin{table}[h]
  \centering
  \begin{subtable}{0.48\textwidth}
    \caption{Coverage Probability}
    \begin{tabular}{r|rrrrrr}
\hline\hline
 &\multicolumn{6}{c}{$\rho$} \\ 
 $\alpha = 0.05$ & $0$ & $0.1$ & $0.2$ & $0.3$ & $0.4$ & $0.5$ \\ 
 \hline$0.1$ & $96$ & $97$ & $84$ & $81$ & $95$ & $98$\\ 
$\pi^2\;\;\;$ $0.2$ & $96$ & $95$ & $85$ & $96$ & $97$ & $95$\\ 
$0.3$ & $96$ & $91$ & $92$ & $98$ & $95$ & $94$\\ 
$0.4$ & $96$ & $90$ & $96$ & $95$ & $95$ & $94$\\ 
 \hline 
 \end{tabular}
 
 \vspace{2em} 
 
\begin{tabular}{r|rrrrrr}
\hline\hline
 &\multicolumn{6}{c}{$\rho$} \\ 
 $\alpha = 0.1$ & $0$ & $0.1$ & $0.2$ & $0.3$ & $0.4$ & $0.5$ \\ 
 \hline$0.1$ & $92$ & $92$ & $71$ & $79$ & $91$ & $96$\\ 
$\pi^2\;\;\;$ $0.2$ & $92$ & $84$ & $78$ & $93$ & $91$ & $89$\\ 
$0.3$ & $93$ & $80$ & $90$ & $92$ & $89$ & $90$\\ 
$0.4$ & $91$ & $83$ & $93$ & $91$ & $91$ & $91$\\ 
 \hline 
 \end{tabular}
 
 \vspace{2em} 
 
\begin{tabular}{r|rrrrrr}
\hline\hline
 &\multicolumn{6}{c}{$\rho$} \\ 
 $\alpha = 0.2$ & $0$ & $0.1$ & $0.2$ & $0.3$ & $0.4$ & $0.5$ \\ 
 \hline$0.1$ & $87$ & $76$ & $57$ & $74$ & $88$ & $83$\\ 
$\pi^2\;\;\;$ $0.2$ & $86$ & $66$ & $76$ & $85$ & $82$ & $80$\\ 
$0.3$ & $87$ & $68$ & $82$ & $81$ & $81$ & $81$\\ 
$0.4$ & $85$ & $72$ & $83$ & $79$ & $80$ & $81$\\ 
 \hline 
 \end{tabular}
  \end{subtable}
  ~
  \begin{subtable}{0.48\textwidth}
    \caption{Relative Width}
    \begin{tabular}{r|rrrrrr}
\hline\hline
 &\multicolumn{6}{c}{$\rho$} \\ 
 $\alpha = 0.05$ & $0$ & $0.1$ & $0.2$ & $0.3$ & $0.4$ & $0.5$ \\ 
 \hline$0.1$ & $ 95$ & $ 97$ & $102$ & $110$ & $113$ & $107$\\ 
$\pi^2\;\;\;$ $0.2$ & $ 97$ & $101$ & $109$ & $109$ & $102$ & $100$\\ 
$0.3$ & $ 97$ & $104$ & $109$ & $102$ & $100$ & $100$\\ 
$0.4$ & $ 98$ & $107$ & $105$ & $100$ & $100$ & $100$\\ 
 \hline 
 \end{tabular}
 
 \vspace{2em} 
 
\begin{tabular}{r|rrrrrr}
\hline\hline
 &\multicolumn{6}{c}{$\rho$} \\ 
 $\alpha = 0.1$ & $0$ & $0.1$ & $0.2$ & $0.3$ & $0.4$ & $0.5$ \\ 
 \hline$0.1$ & $ 93$ & $ 96$ & $104$ & $112$ & $112$ & $105$\\ 
$\pi^2\;\;\;$ $0.2$ & $ 96$ & $102$ & $111$ & $108$ & $101$ & $100$\\ 
$0.3$ & $ 97$ & $105$ & $109$ & $101$ & $100$ & $100$\\ 
$0.4$ & $ 98$ & $107$ & $105$ & $100$ & $100$ & $100$\\ 
 \hline 
 \end{tabular}
 
 \vspace{2em} 
 
\begin{tabular}{r|rrrrrr}
\hline\hline
 &\multicolumn{6}{c}{$\rho$} \\ 
 $\alpha = 0.2$ & $0$ & $0.1$ & $0.2$ & $0.3$ & $0.4$ & $0.5$ \\ 
 \hline$0.1$ & $ 89$ & $ 95$ & $107$ & $114$ & $110$ & $103$\\ 
$\pi^2\;\;\;$ $0.2$ & $ 93$ & $102$ & $113$ & $107$ & $101$ & $100$\\ 
$0.3$ & $ 95$ & $107$ & $109$ & $101$ & $100$ & $100$\\ 
$0.4$ & $ 97$ & $109$ & $104$ & $100$ & $100$ & $100$\\ 
 \hline 
 \end{tabular}
  \end{subtable}
  \label{tab:CISim500_1stepEqual_OLSvsIV}
  \caption{1-step Equal-tailed CI, OLS vs IV Example, $N=500$}
\end{table}





\begin{table}[h]
  \centering
  \begin{subtable}{0.48\textwidth}
    \caption{Coverage Probability}
    \begin{tabular}{r|rrrrrr}
\hline\hline
 &\multicolumn{6}{c}{$\rho$} \\ 
 $\alpha = 0.05$ & $0$ & $0.1$ & $0.2$ & $0.3$ & $0.4$ & $0.5$ \\ 
 \hline$0.1$ & $98$ & $98$ & $98$ & $97$ & $90$ & $77$\\ 
$\pi^2\;\;\;$ $0.2$ & $97$ & $97$ & $96$ & $92$ & $84$ & $81$\\ 
$0.3$ & $96$ & $97$ & $96$ & $90$ & $84$ & $87$\\ 
$0.4$ & $96$ & $97$ & $94$ & $90$ & $90$ & $94$\\ 
 \hline 
 \end{tabular}
 
 \vspace{2em} 
 
\begin{tabular}{r|rrrrrr}
\hline\hline
 &\multicolumn{6}{c}{$\rho$} \\ 
 $\alpha = 0.1$ & $0$ & $0.1$ & $0.2$ & $0.3$ & $0.4$ & $0.5$ \\ 
 \hline$0.1$ & $95$ & $95$ & $96$ & $89$ & $79$ & $68$\\ 
$\pi^2\;\;\;$ $0.2$ & $93$ & $92$ & $92$ & $85$ & $78$ & $74$\\ 
$0.3$ & $93$ & $91$ & $88$ & $84$ & $81$ & $86$\\ 
$0.4$ & $92$ & $91$ & $86$ & $86$ & $88$ & $91$\\ 
 \hline 
 \end{tabular}
 
 \vspace{2em} 
 
\begin{tabular}{r|rrrrrr}
\hline\hline
 &\multicolumn{6}{c}{$\rho$} \\ 
 $\alpha = 0.2$ & $0$ & $0.1$ & $0.2$ & $0.3$ & $0.4$ & $0.5$ \\ 
 \hline$0.1$ & $88$ & $85$ & $78$ & $72$ & $64$ & $58$\\ 
$\pi^2\;\;\;$ $0.2$ & $86$ & $85$ & $78$ & $70$ & $69$ & $68$\\ 
$0.3$ & $84$ & $83$ & $76$ & $70$ & $72$ & $82$\\ 
$0.4$ & $83$ & $81$ & $74$ & $75$ & $80$ & $82$\\ 
 \hline 
 \end{tabular}
  \end{subtable}
  ~
  \begin{subtable}{0.48\textwidth}
    \caption{Relative Width}
    \begin{tabular}{r|rrrrrr}
\hline\hline
 &\multicolumn{6}{c}{$\rho$} \\ 
 $\alpha = 0.05$ & $0$ & $0.1$ & $0.2$ & $0.3$ & $0.4$ & $0.5$ \\ 
 \hline$0.1$ & $ 93$ & $ 95$ & $100$ & $108$ & $110$ & $105$\\ 
$\pi^2\;\;\;$ $0.2$ & $ 96$ & $100$ & $108$ & $108$ & $101$ & $100$\\ 
$0.3$ & $ 97$ & $104$ & $108$ & $102$ & $100$ & $100$\\ 
$0.4$ & $ 98$ & $106$ & $105$ & $100$ & $100$ & $100$\\ 
 \hline 
 \end{tabular}
 
 \vspace{2em} 
 
\begin{tabular}{r|rrrrrr}
\hline\hline
 &\multicolumn{6}{c}{$\rho$} \\ 
 $\alpha = 0.1$ & $0$ & $0.1$ & $0.2$ & $0.3$ & $0.4$ & $0.5$ \\ 
 \hline$0.1$ & $ 90$ & $ 93$ & $102$ & $109$ & $109$ & $103$\\ 
$\pi^2\;\;\;$ $0.2$ & $ 95$ & $101$ & $109$ & $107$ & $101$ & $100$\\ 
$0.3$ & $ 96$ & $105$ & $109$ & $101$ & $100$ & $100$\\ 
$0.4$ & $ 97$ & $107$ & $104$ & $100$ & $100$ & $100$\\ 
 \hline 
 \end{tabular}
 
 \vspace{2em} 
 
\begin{tabular}{r|rrrrrr}
\hline\hline
 &\multicolumn{6}{c}{$\rho$} \\ 
 $\alpha = 0.2$ & $0$ & $0.1$ & $0.2$ & $0.3$ & $0.4$ & $0.5$ \\ 
 \hline$0.1$ & $ 81$ & $ 88$ & $101$ & $110$ & $107$ & $102$\\ 
$\pi^2\;\;\;$ $0.2$ & $ 89$ & $ 99$ & $111$ & $105$ & $100$ & $100$\\ 
$0.3$ & $ 93$ & $105$ & $108$ & $101$ & $100$ & $100$\\ 
$0.4$ & $ 96$ & $108$ & $103$ & $100$ & $100$ & $100$\\ 
 \hline 
 \end{tabular}
  \end{subtable}
  \label{tab:CISim50_1stepShort_OLSvsIV}
  \caption{1-step Shortest CI, OLS vs IV Example, $N=50$}
\end{table}

\begin{table}[h]
  \centering
  \begin{subtable}{0.48\textwidth}
    \caption{Coverage Probability}
    \begin{tabular}{r|rrrrrr}
\hline\hline
 &\multicolumn{6}{c}{$\rho$} \\ 
 $\alpha = 0.05$ & $0$ & $0.1$ & $0.2$ & $0.3$ & $0.4$ & $0.5$ \\ 
 \hline$0.1$ & $95$ & $97$ & $97$ & $93$ & $80$ & $72$\\ 
$\pi^2\;\;\;$ $0.2$ & $95$ & $98$ & $94$ & $88$ & $84$ & $89$\\ 
$0.3$ & $96$ & $96$ & $95$ & $88$ & $91$ & $96$\\ 
$0.4$ & $96$ & $96$ & $92$ & $90$ & $96$ & $95$\\ 
 \hline 
 \end{tabular}
 
 \vspace{2em} 
 
\begin{tabular}{r|rrrrrr}
\hline\hline
 &\multicolumn{6}{c}{$\rho$} \\ 
 $\alpha = 0.1$ & $0$ & $0.1$ & $0.2$ & $0.3$ & $0.4$ & $0.5$ \\ 
 \hline$0.1$ & $94$ & $93$ & $91$ & $82$ & $73$ & $68$\\ 
$\pi^2\;\;\;$ $0.2$ & $91$ & $92$ & $87$ & $79$ & $81$ & $88$\\ 
$0.3$ & $93$ & $91$ & $86$ & $82$ & $90$ & $93$\\ 
$0.4$ & $91$ & $90$ & $85$ & $89$ & $92$ & $90$\\ 
 \hline 
 \end{tabular}
 
 \vspace{2em} 
 
\begin{tabular}{r|rrrrrr}
\hline\hline
 &\multicolumn{6}{c}{$\rho$} \\ 
 $\alpha = 0.2$ & $0$ & $0.1$ & $0.2$ & $0.3$ & $0.4$ & $0.5$ \\ 
 \hline$0.1$ & $83$ & $79$ & $72$ & $64$ & $61$ & $63$\\ 
$\pi^2\;\;\;$ $0.2$ & $86$ & $81$ & $70$ & $67$ & $75$ & $83$\\ 
$0.3$ & $84$ & $80$ & $73$ & $75$ & $84$ & $82$\\ 
$0.4$ & $83$ & $78$ & $75$ & $81$ & $84$ & $78$\\ 
 \hline 
 \end{tabular}
  \end{subtable}
  ~
  \begin{subtable}{0.48\textwidth}
    \caption{Relative Width}
    \begin{tabular}{r|rrrrrr}
\hline\hline
 &\multicolumn{6}{c}{$\rho$} \\ 
 $\alpha = 0.05$ & $0$ & $0.1$ & $0.2$ & $0.3$ & $0.4$ & $0.5$ \\ 
 \hline$0.1$ & $ 92$ & $ 91$ & $ 94$ & $ 94$ & $ 95$ & $ 99$\\ 
$\pi^2\;\;\;$ $0.2$ & $ 96$ & $ 96$ & $ 99$ & $102$ & $107$ & $112$\\ 
$0.3$ & $ 97$ & $ 98$ & $103$ & $107$ & $111$ & $110$\\ 
$0.4$ & $ 98$ & $100$ & $105$ & $109$ & $108$ & $102$\\ 
 \hline 
 \end{tabular}
 
 \vspace{2em} 
 
\begin{tabular}{r|rrrrrr}
\hline\hline
 &\multicolumn{6}{c}{$\rho$} \\ 
 $\alpha = 0.1$ & $0$ & $0.1$ & $0.2$ & $0.3$ & $0.4$ & $0.5$ \\ 
 \hline$0.1$ & $ 88$ & $ 89$ & $ 91$ & $ 93$ & $ 95$ & $102$\\ 
$\pi^2\;\;\;$ $0.2$ & $ 94$ & $ 96$ & $ 99$ & $105$ & $110$ & $114$\\ 
$0.3$ & $ 95$ & $ 98$ & $104$ & $110$ & $111$ & $107$\\ 
$0.4$ & $ 97$ & $101$ & $106$ & $110$ & $106$ & $101$\\ 
 \hline 
 \end{tabular}
 
 \vspace{2em} 
 
\begin{tabular}{r|rrrrrr}
\hline\hline
 &\multicolumn{6}{c}{$\rho$} \\ 
 $\alpha = 0.2$ & $0$ & $0.1$ & $0.2$ & $0.3$ & $0.4$ & $0.5$ \\ 
 \hline$0.1$ & $ 81$ & $ 83$ & $ 81$ & $ 91$ & $ 97$ & $105$\\ 
$\pi^2\;\;\;$ $0.2$ & $ 89$ & $ 91$ & $ 98$ & $106$ & $113$ & $114$\\ 
$0.3$ & $ 93$ & $ 96$ & $104$ & $111$ & $111$ & $105$\\ 
$0.4$ & $ 96$ & $ 99$ & $108$ & $110$ & $105$ & $101$\\ 
 \hline 
 \end{tabular}
  \end{subtable}
  \label{tab:CISim100_1stepShort_OLSvsIV}
  \caption{1-step Shortest, OLS vs IV Example, $N=100$}
\end{table}

\begin{table}[h]
  \centering
  \begin{subtable}{0.48\textwidth}
    \caption{Coverage Probability}
    \begin{tabular}{r|rrrrrr}
\hline\hline
 &\multicolumn{6}{c}{$\rho$} \\ 
 $\alpha = 0.05$ & $0$ & $0.1$ & $0.2$ & $0.3$ & $0.4$ & $0.5$ \\ 
 \hline$0.1$ & $98$ & $98$ & $98$ & $97$ & $90$ & $77$\\ 
$\pi^2\;\;\;$ $0.2$ & $97$ & $97$ & $96$ & $92$ & $84$ & $81$\\ 
$0.3$ & $96$ & $97$ & $96$ & $90$ & $84$ & $87$\\ 
$0.4$ & $96$ & $97$ & $94$ & $90$ & $90$ & $94$\\ 
 \hline 
 \end{tabular}
 
 \vspace{2em} 
 
\begin{tabular}{r|rrrrrr}
\hline\hline
 &\multicolumn{6}{c}{$\rho$} \\ 
 $\alpha = 0.1$ & $0$ & $0.1$ & $0.2$ & $0.3$ & $0.4$ & $0.5$ \\ 
 \hline$0.1$ & $95$ & $95$ & $96$ & $89$ & $79$ & $68$\\ 
$\pi^2\;\;\;$ $0.2$ & $93$ & $92$ & $92$ & $85$ & $78$ & $74$\\ 
$0.3$ & $93$ & $91$ & $88$ & $84$ & $81$ & $86$\\ 
$0.4$ & $92$ & $91$ & $86$ & $86$ & $88$ & $91$\\ 
 \hline 
 \end{tabular}
 
 \vspace{2em} 
 
\begin{tabular}{r|rrrrrr}
\hline\hline
 &\multicolumn{6}{c}{$\rho$} \\ 
 $\alpha = 0.2$ & $0$ & $0.1$ & $0.2$ & $0.3$ & $0.4$ & $0.5$ \\ 
 \hline$0.1$ & $88$ & $85$ & $78$ & $72$ & $64$ & $58$\\ 
$\pi^2\;\;\;$ $0.2$ & $86$ & $85$ & $78$ & $70$ & $69$ & $68$\\ 
$0.3$ & $84$ & $83$ & $76$ & $70$ & $72$ & $82$\\ 
$0.4$ & $83$ & $81$ & $74$ & $75$ & $80$ & $82$\\ 
 \hline 
 \end{tabular}
  \end{subtable}
  ~
  \begin{subtable}{0.48\textwidth}
    \caption{Relative Width}
    \begin{tabular}{r|rrrrrr}
\hline\hline
 &\multicolumn{6}{c}{$\rho$} \\ 
 $\alpha = 0.05$ & $0$ & $0.1$ & $0.2$ & $0.3$ & $0.4$ & $0.5$ \\ 
 \hline$0.1$ & $ 93$ & $ 95$ & $100$ & $108$ & $110$ & $105$\\ 
$\pi^2\;\;\;$ $0.2$ & $ 96$ & $100$ & $108$ & $108$ & $101$ & $100$\\ 
$0.3$ & $ 97$ & $104$ & $108$ & $102$ & $100$ & $100$\\ 
$0.4$ & $ 98$ & $106$ & $105$ & $100$ & $100$ & $100$\\ 
 \hline 
 \end{tabular}
 
 \vspace{2em} 
 
\begin{tabular}{r|rrrrrr}
\hline\hline
 &\multicolumn{6}{c}{$\rho$} \\ 
 $\alpha = 0.1$ & $0$ & $0.1$ & $0.2$ & $0.3$ & $0.4$ & $0.5$ \\ 
 \hline$0.1$ & $ 90$ & $ 93$ & $102$ & $109$ & $109$ & $103$\\ 
$\pi^2\;\;\;$ $0.2$ & $ 95$ & $101$ & $109$ & $107$ & $101$ & $100$\\ 
$0.3$ & $ 96$ & $105$ & $109$ & $101$ & $100$ & $100$\\ 
$0.4$ & $ 97$ & $107$ & $104$ & $100$ & $100$ & $100$\\ 
 \hline 
 \end{tabular}
 
 \vspace{2em} 
 
\begin{tabular}{r|rrrrrr}
\hline\hline
 &\multicolumn{6}{c}{$\rho$} \\ 
 $\alpha = 0.2$ & $0$ & $0.1$ & $0.2$ & $0.3$ & $0.4$ & $0.5$ \\ 
 \hline$0.1$ & $ 81$ & $ 88$ & $101$ & $110$ & $107$ & $102$\\ 
$\pi^2\;\;\;$ $0.2$ & $ 89$ & $ 99$ & $111$ & $105$ & $100$ & $100$\\ 
$0.3$ & $ 93$ & $105$ & $108$ & $101$ & $100$ & $100$\\ 
$0.4$ & $ 96$ & $108$ & $103$ & $100$ & $100$ & $100$\\ 
 \hline 
 \end{tabular}
  \end{subtable}
  \label{tab:CISim500_1stepShort_OLSvsIV}
  \caption{1-step Shortest CI, OLS vs IV Example, $N=500$}
\end{table}







\begin{table}[h]
  \centering
  \begin{subtable}{0.48\textwidth}
    \caption{Coverage Probability}
    \begin{tabular}{r|rrrrrr}
\hline\hline
 &\multicolumn{6}{c}{$\rho$} \\ 
 $\alpha = 0.05$ & $0$ & $0.1$ & $0.2$ & $0.3$ & $0.4$ & $0.5$ \\ 
 \hline$0.1$ & $100$ & $100$ & $100$ & $ 99$ & $ 98$ & $ 94$\\ 
$\pi^2\;\;\;$ $0.2$ & $ 99$ & $ 99$ & $ 99$ & $ 98$ & $ 96$ & $ 92$\\ 
$0.3$ & $ 99$ & $ 99$ & $ 99$ & $ 98$ & $ 95$ & $ 95$\\ 
$0.4$ & $ 99$ & $ 99$ & $ 99$ & $ 97$ & $ 95$ & $ 97$\\ 
 \hline 
 \end{tabular}
 
 \vspace{2em} 
 
\begin{tabular}{r|rrrrrr}
\hline\hline
 &\multicolumn{6}{c}{$\rho$} \\ 
 $\alpha = 0.1$ & $0$ & $0.1$ & $0.2$ & $0.3$ & $0.4$ & $0.5$ \\ 
 \hline$0.1$ & $99$ & $99$ & $99$ & $98$ & $96$ & $90$\\ 
$\pi^2\;\;\;$ $0.2$ & $98$ & $98$ & $98$ & $97$ & $94$ & $90$\\ 
$0.3$ & $97$ & $97$ & $97$ & $95$ & $93$ & $92$\\ 
$0.4$ & $96$ & $97$ & $96$ & $96$ & $94$ & $95$\\ 
 \hline 
 \end{tabular}
 
 \vspace{2em} 
 
\begin{tabular}{r|rrrrrr}
\hline\hline
 &\multicolumn{6}{c}{$\rho$} \\ 
 $\alpha = 0.2$ & $0$ & $0.1$ & $0.2$ & $0.3$ & $0.4$ & $0.5$ \\ 
 \hline$0.1$ & $96$ & $97$ & $97$ & $97$ & $93$ & $87$\\ 
$\pi^2\;\;\;$ $0.2$ & $97$ & $95$ & $96$ & $93$ & $90$ & $86$\\ 
$0.3$ & $95$ & $94$ & $95$ & $90$ & $87$ & $90$\\ 
$0.4$ & $94$ & $94$ & $91$ & $90$ & $89$ & $92$\\ 
 \hline 
 \end{tabular}
  \end{subtable}
  ~
  \begin{subtable}{0.48\textwidth}
    \caption{Relative Width}
    \begin{tabular}{r|rrrrrr}
\hline\hline
 &\multicolumn{6}{c}{$\rho$} \\ 
 $\alpha = 0.05$ & $0$ & $0.1$ & $0.2$ & $0.3$ & $0.4$ & $0.5$ \\ 
 \hline$0.1$ & $119$ & $119$ & $120$ & $120$ & $123$ & $123$\\ 
$\pi^2\;\;\;$ $0.2$ & $122$ & $123$ & $123$ & $126$ & $129$ & $131$\\ 
$0.3$ & $123$ & $124$ & $126$ & $128$ & $130$ & $132$\\ 
$0.4$ & $124$ & $125$ & $126$ & $128$ & $129$ & $129$\\ 
 \hline 
 \end{tabular}
 
 \vspace{2em} 
 
\begin{tabular}{r|rrrrrr}
\hline\hline
 &\multicolumn{6}{c}{$\rho$} \\ 
 $\alpha = 0.1$ & $0$ & $0.1$ & $0.2$ & $0.3$ & $0.4$ & $0.5$ \\ 
 \hline$0.1$ & $128$ & $128$ & $129$ & $130$ & $133$ & $133$\\ 
$\pi^2\;\;\;$ $0.2$ & $131$ & $132$ & $133$ & $135$ & $138$ & $140$\\ 
$0.3$ & $132$ & $133$ & $135$ & $136$ & $139$ & $139$\\ 
$0.4$ & $132$ & $133$ & $134$ & $136$ & $136$ & $137$\\ 
 \hline 
 \end{tabular}
 
 \vspace{2em} 
 
\begin{tabular}{r|rrrrrr}
\hline\hline
 &\multicolumn{6}{c}{$\rho$} \\ 
 $\alpha = 0.2$ & $0$ & $0.1$ & $0.2$ & $0.3$ & $0.4$ & $0.5$ \\ 
 \hline$0.1$ & $147$ & $147$ & $146$ & $149$ & $151$ & $154$\\ 
$\pi^2\;\;\;$ $0.2$ & $148$ & $149$ & $150$ & $152$ & $154$ & $157$\\ 
$0.3$ & $147$ & $148$ & $149$ & $151$ & $152$ & $154$\\ 
$0.4$ & $146$ & $147$ & $148$ & $149$ & $149$ & $150$\\ 
 \hline 
 \end{tabular}
  \end{subtable}
  \label{tab:CISim50_2stepEqual_OLSvsIV}
  \caption{2-step CI, $\alpha_1 = \alpha_2 = \alpha/2$ OLS vs IV Example, $N=50$}
\end{table}

\begin{table}[h]
  \centering
  \begin{subtable}{0.48\textwidth}
    \caption{Coverage Probability}
    \begin{tabular}{r|rrrrrr}
\hline\hline
 &\multicolumn{6}{c}{$\rho$} \\ 
 $\alpha = 0.05$ & $0$ & $0.1$ & $0.2$ & $0.3$ & $0.4$ & $0.5$ \\ 
 \hline$0.1$ & $ 99$ & $100$ & $ 99$ & $ 98$ & $ 96$ & $ 92$\\ 
$\pi^2\;\;\;$ $0.2$ & $ 98$ & $ 99$ & $ 99$ & $ 98$ & $ 94$ & $ 95$\\ 
$0.3$ & $ 98$ & $ 98$ & $ 99$ & $ 97$ & $ 97$ & $ 98$\\ 
$0.4$ & $ 98$ & $ 99$ & $ 98$ & $ 96$ & $ 97$ & $ 98$\\ 
 \hline 
 \end{tabular}
 
 \vspace{2em} 
 
\begin{tabular}{r|rrrrrr}
\hline\hline
 &\multicolumn{6}{c}{$\rho$} \\ 
 $\alpha = 0.1$ & $0$ & $0.1$ & $0.2$ & $0.3$ & $0.4$ & $0.5$ \\ 
 \hline$0.1$ & $98$ & $98$ & $98$ & $98$ & $94$ & $90$\\ 
$\pi^2\;\;\;$ $0.2$ & $96$ & $97$ & $97$ & $94$ & $92$ & $93$\\ 
$0.3$ & $96$ & $97$ & $97$ & $94$ & $95$ & $97$\\ 
$0.4$ & $97$ & $96$ & $95$ & $95$ & $98$ & $97$\\ 
 \hline 
 \end{tabular}
 
 \vspace{2em} 
 
\begin{tabular}{r|rrrrrr}
\hline\hline
 &\multicolumn{6}{c}{$\rho$} \\ 
 $\alpha = 0.2$ & $0$ & $0.1$ & $0.2$ & $0.3$ & $0.4$ & $0.5$ \\ 
 \hline$0.1$ & $94$ & $95$ & $96$ & $94$ & $90$ & $87$\\ 
$\pi^2\;\;\;$ $0.2$ & $94$ & $95$ & $93$ & $89$ & $90$ & $92$\\ 
$0.3$ & $94$ & $95$ & $92$ & $90$ & $91$ & $94$\\ 
$0.4$ & $94$ & $93$ & $90$ & $92$ & $95$ & $94$\\ 
 \hline 
 \end{tabular}
  \end{subtable}
  ~
  \begin{subtable}{0.48\textwidth}
    \caption{Relative Width}
    \begin{tabular}{r|rrrrrr}
\hline\hline
 &\multicolumn{6}{c}{$\rho$} \\ 
 $\alpha = 0.05$ & $0$ & $0.1$ & $0.2$ & $0.3$ & $0.4$ & $0.5$ \\ 
 \hline$0.1$ & $120$ & $120$ & $120$ & $125$ & $127$ & $131$\\ 
$\pi^2\;\;\;$ $0.2$ & $122$ & $123$ & $127$ & $130$ & $133$ & $134$\\ 
$0.3$ & $123$ & $124$ & $128$ & $130$ & $132$ & $132$\\ 
$0.4$ & $124$ & $125$ & $128$ & $129$ & $129$ & $129$\\ 
 \hline 
 \end{tabular}
 
 \vspace{2em} 
 
\begin{tabular}{r|rrrrrr}
\hline\hline
 &\multicolumn{6}{c}{$\rho$} \\ 
 $\alpha = 0.1$ & $0$ & $0.1$ & $0.2$ & $0.3$ & $0.4$ & $0.5$ \\ 
 \hline$0.1$ & $130$ & $131$ & $132$ & $136$ & $137$ & $143$\\ 
$\pi^2\;\;\;$ $0.2$ & $132$ & $134$ & $135$ & $139$ & $141$ & $143$\\ 
$0.3$ & $131$ & $133$ & $136$ & $139$ & $139$ & $140$\\ 
$0.4$ & $132$ & $133$ & $135$ & $136$ & $136$ & $136$\\ 
 \hline 
 \end{tabular}
 
 \vspace{2em} 
 
\begin{tabular}{r|rrrrrr}
\hline\hline
 &\multicolumn{6}{c}{$\rho$} \\ 
 $\alpha = 0.2$ & $0$ & $0.1$ & $0.2$ & $0.3$ & $0.4$ & $0.5$ \\ 
 \hline$0.1$ & $149$ & $149$ & $149$ & $155$ & $158$ & $160$\\ 
$\pi^2\;\;\;$ $0.2$ & $148$ & $149$ & $152$ & $155$ & $157$ & $158$\\ 
$0.3$ & $147$ & $148$ & $151$ & $153$ & $153$ & $154$\\ 
$0.4$ & $147$ & $147$ & $149$ & $150$ & $149$ & $147$\\ 
 \hline 
 \end{tabular}
  \end{subtable}
  \label{tab:CISim100_2stepEqual_OLSvsIV}
  \caption{2-step CI, $\alpha_1 = \alpha_2 = \alpha/2$, OLS vs IV Example, $N=100$}
\end{table}

\begin{table}[h]
  \centering
  \begin{subtable}{0.48\textwidth}
    \caption{Coverage Probability}
    \begin{tabular}{r|rrrrrr}
\hline\hline
 &\multicolumn{6}{c}{$\rho$} \\ 
 $\alpha = 0.05$ & $0$ & $0.1$ & $0.2$ & $0.3$ & $0.4$ & $0.5$ \\ 
 \hline$0.1$ & $100$ & $100$ & $100$ & $ 99$ & $ 98$ & $ 94$\\ 
$\pi^2\;\;\;$ $0.2$ & $ 99$ & $ 99$ & $ 99$ & $ 98$ & $ 96$ & $ 92$\\ 
$0.3$ & $ 99$ & $ 99$ & $ 99$ & $ 98$ & $ 95$ & $ 95$\\ 
$0.4$ & $ 99$ & $ 99$ & $ 99$ & $ 97$ & $ 95$ & $ 97$\\ 
 \hline 
 \end{tabular}
 
 \vspace{2em} 
 
\begin{tabular}{r|rrrrrr}
\hline\hline
 &\multicolumn{6}{c}{$\rho$} \\ 
 $\alpha = 0.1$ & $0$ & $0.1$ & $0.2$ & $0.3$ & $0.4$ & $0.5$ \\ 
 \hline$0.1$ & $99$ & $99$ & $99$ & $98$ & $96$ & $90$\\ 
$\pi^2\;\;\;$ $0.2$ & $98$ & $98$ & $98$ & $97$ & $94$ & $90$\\ 
$0.3$ & $97$ & $97$ & $97$ & $95$ & $93$ & $92$\\ 
$0.4$ & $96$ & $97$ & $96$ & $96$ & $94$ & $95$\\ 
 \hline 
 \end{tabular}
 
 \vspace{2em} 
 
\begin{tabular}{r|rrrrrr}
\hline\hline
 &\multicolumn{6}{c}{$\rho$} \\ 
 $\alpha = 0.2$ & $0$ & $0.1$ & $0.2$ & $0.3$ & $0.4$ & $0.5$ \\ 
 \hline$0.1$ & $96$ & $97$ & $97$ & $97$ & $93$ & $87$\\ 
$\pi^2\;\;\;$ $0.2$ & $97$ & $95$ & $96$ & $93$ & $90$ & $86$\\ 
$0.3$ & $95$ & $94$ & $95$ & $90$ & $87$ & $90$\\ 
$0.4$ & $94$ & $94$ & $91$ & $90$ & $89$ & $92$\\ 
 \hline 
 \end{tabular}
  \end{subtable}
  ~
  \begin{subtable}{0.48\textwidth}
    \caption{Relative Width}
    \begin{tabular}{r|rrrrrr}
\hline\hline
 &\multicolumn{6}{c}{$\rho$} \\ 
 $\alpha = 0.05$ & $0$ & $0.1$ & $0.2$ & $0.3$ & $0.4$ & $0.5$ \\ 
 \hline$0.1$ & $122$ & $124$ & $131$ & $136$ & $139$ & $139$\\ 
$\pi^2\;\;\;$ $0.2$ & $122$ & $127$ & $133$ & $135$ & $135$ & $129$\\ 
$0.3$ & $123$ & $129$ & $132$ & $131$ & $124$ & $115$\\ 
$0.4$ & $124$ & $128$ & $129$ & $124$ & $115$ & $114$\\ 
 \hline 
 \end{tabular}
 
 \vspace{2em} 
 
\begin{tabular}{r|rrrrrr}
\hline\hline
 &\multicolumn{6}{c}{$\rho$} \\ 
 $\alpha = 0.1$ & $0$ & $0.1$ & $0.2$ & $0.3$ & $0.4$ & $0.5$ \\ 
 \hline$0.1$ & $131$ & $134$ & $141$ & $146$ & $148$ & $148$\\ 
$\pi^2\;\;\;$ $0.2$ & $132$ & $136$ & $142$ & $143$ & $141$ & $128$\\ 
$0.3$ & $132$ & $136$ & $139$ & $137$ & $125$ & $119$\\ 
$0.4$ & $132$ & $136$ & $136$ & $127$ & $119$ & $119$\\ 
 \hline 
 \end{tabular}
 
 \vspace{2em} 
 
\begin{tabular}{r|rrrrrr}
\hline\hline
 &\multicolumn{6}{c}{$\rho$} \\ 
 $\alpha = 0.2$ & $0$ & $0.1$ & $0.2$ & $0.3$ & $0.4$ & $0.5$ \\ 
 \hline$0.1$ & $148$ & $152$ & $158$ & $163$ & $164$ & $163$\\ 
$\pi^2\;\;\;$ $0.2$ & $148$ & $152$ & $157$ & $157$ & $147$ & $132$\\ 
$0.3$ & $147$ & $151$ & $153$ & $144$ & $130$ & $128$\\ 
$0.4$ & $147$ & $149$ & $147$ & $132$ & $128$ & $128$\\ 
 \hline 
 \end{tabular}
  \end{subtable}
  \label{tab:CISim500_2stepEqual_OLSvsIV}
  \caption{2-step CI, $\alpha_1 = \alpha_2 = \alpha/2$, OLS vs IV Example, $N=500$}
\end{table}




\begin{table}[h]
  \centering
  \begin{subtable}{0.48\textwidth}
    \caption{Coverage Probability}
    \begin{tabular}{r|rrrrrr}
\hline\hline
 &\multicolumn{6}{c}{$\rho$} \\ 
 $\alpha = 0.05$ & $0$ & $0.1$ & $0.2$ & $0.3$ & $0.4$ & $0.5$ \\ 
 \hline$0.1$ & $99$ & $99$ & $99$ & $99$ & $97$ & $92$\\ 
$\pi^2\;\;\;$ $0.2$ & $99$ & $98$ & $98$ & $97$ & $96$ & $91$\\ 
$0.3$ & $98$ & $98$ & $98$ & $97$ & $93$ & $93$\\ 
$0.4$ & $98$ & $98$ & $97$ & $96$ & $94$ & $96$\\ 
 \hline 
 \end{tabular}
 
 \vspace{2em} 
 
\begin{tabular}{r|rrrrrr}
\hline\hline
 &\multicolumn{6}{c}{$\rho$} \\ 
 $\alpha = 0.1$ & $0$ & $0.1$ & $0.2$ & $0.3$ & $0.4$ & $0.5$ \\ 
 \hline$0.1$ & $99$ & $99$ & $98$ & $96$ & $94$ & $88$\\ 
$\pi^2\;\;\;$ $0.2$ & $96$ & $96$ & $97$ & $96$ & $93$ & $88$\\ 
$0.3$ & $95$ & $95$ & $95$ & $93$ & $91$ & $91$\\ 
$0.4$ & $95$ & $96$ & $94$ & $93$ & $92$ & $93$\\ 
 \hline 
 \end{tabular}
 
 \vspace{2em} 
 
\begin{tabular}{r|rrrrrr}
\hline\hline
 &\multicolumn{6}{c}{$\rho$} \\ 
 $\alpha = 0.2$ & $0$ & $0.1$ & $0.2$ & $0.3$ & $0.4$ & $0.5$ \\ 
 \hline$0.1$ & $95$ & $95$ & $96$ & $94$ & $90$ & $83$\\ 
$\pi^2\;\;\;$ $0.2$ & $94$ & $93$ & $93$ & $89$ & $87$ & $82$\\ 
$0.3$ & $92$ & $92$ & $90$ & $85$ & $82$ & $87$\\ 
$0.4$ & $91$ & $90$ & $87$ & $85$ & $86$ & $89$\\ 
 \hline 
 \end{tabular}
  \end{subtable}
  ~
  \begin{subtable}{0.48\textwidth}
    \caption{Relative Width}
    \begin{tabular}{r|rrrrrr}
\hline\hline
 &\multicolumn{6}{c}{$\rho$} \\ 
 $\alpha = 0.05$ & $0$ & $0.1$ & $0.2$ & $0.3$ & $0.4$ & $0.5$ \\ 
 \hline$0.1$ & $113$ & $113$ & $113$ & $113$ & $116$ & $117$\\ 
$\pi^2\;\;\;$ $0.2$ & $115$ & $116$ & $117$ & $119$ & $121$ & $123$\\ 
$0.3$ & $116$ & $117$ & $119$ & $121$ & $122$ & $123$\\ 
$0.4$ & $117$ & $118$ & $119$ & $120$ & $121$ & $121$\\ 
 \hline 
 \end{tabular}
 
 \vspace{2em} 
 
\begin{tabular}{r|rrrrrr}
\hline\hline
 &\multicolumn{6}{c}{$\rho$} \\ 
 $\alpha = 0.1$ & $0$ & $0.1$ & $0.2$ & $0.3$ & $0.4$ & $0.5$ \\ 
 \hline$0.1$ & $120$ & $119$ & $120$ & $121$ & $125$ & $125$\\ 
$\pi^2\;\;\;$ $0.2$ & $122$ & $123$ & $125$ & $126$ & $128$ & $129$\\ 
$0.3$ & $123$ & $124$ & $125$ & $126$ & $128$ & $128$\\ 
$0.4$ & $123$ & $123$ & $124$ & $125$ & $125$ & $125$\\ 
 \hline 
 \end{tabular}
 
 \vspace{2em} 
 
\begin{tabular}{r|rrrrrr}
\hline\hline
 &\multicolumn{6}{c}{$\rho$} \\ 
 $\alpha = 0.2$ & $0$ & $0.1$ & $0.2$ & $0.3$ & $0.4$ & $0.5$ \\ 
 \hline$0.1$ & $137$ & $138$ & $135$ & $139$ & $140$ & $144$\\ 
$\pi^2\;\;\;$ $0.2$ & $135$ & $137$ & $137$ & $137$ & $140$ & $141$\\ 
$0.3$ & $134$ & $135$ & $135$ & $136$ & $136$ & $138$\\ 
$0.4$ & $133$ & $133$ & $133$ & $133$ & $133$ & $133$\\ 
 \hline 
 \end{tabular}
  \end{subtable}
  \label{tab:CISim50_2stepWideTau_OLSvsIV}
  \caption{2-step CI, $\alpha_1 = \alpha/4, \alpha_2 = 3\alpha/4$ OLS vs IV Example, $N=50$}
\end{table}

\begin{table}[h]
  \centering
  \begin{subtable}{0.48\textwidth}
    \caption{Coverage Probability}
    \begin{tabular}{r|rrrrrr}
\hline\hline
 &\multicolumn{6}{c}{$\rho$} \\ 
 $\alpha = 0.05$ & $0$ & $0.1$ & $0.2$ & $0.3$ & $0.4$ & $0.5$ \\ 
 \hline$0.1$ & $98$ & $99$ & $99$ & $98$ & $95$ & $90$\\ 
$\pi^2\;\;\;$ $0.2$ & $97$ & $99$ & $99$ & $98$ & $94$ & $94$\\ 
$0.3$ & $98$ & $98$ & $98$ & $96$ & $95$ & $98$\\ 
$0.4$ & $97$ & $98$ & $97$ & $94$ & $96$ & $98$\\ 
 \hline 
 \end{tabular}
 
 \vspace{2em} 
 
\begin{tabular}{r|rrrrrr}
\hline\hline
 &\multicolumn{6}{c}{$\rho$} \\ 
 $\alpha = 0.1$ & $0$ & $0.1$ & $0.2$ & $0.3$ & $0.4$ & $0.5$ \\ 
 \hline$0.1$ & $97$ & $97$ & $98$ & $97$ & $92$ & $88$\\ 
$\pi^2\;\;\;$ $0.2$ & $95$ & $96$ & $95$ & $92$ & $90$ & $92$\\ 
$0.3$ & $95$ & $96$ & $95$ & $91$ & $94$ & $96$\\ 
$0.4$ & $95$ & $95$ & $92$ & $93$ & $95$ & $95$\\ 
 \hline 
 \end{tabular}
 
 \vspace{2em} 
 
\begin{tabular}{r|rrrrrr}
\hline\hline
 &\multicolumn{6}{c}{$\rho$} \\ 
 $\alpha = 0.2$ & $0$ & $0.1$ & $0.2$ & $0.3$ & $0.4$ & $0.5$ \\ 
 \hline$0.1$ & $92$ & $93$ & $93$ & $92$ & $86$ & $83$\\ 
$\pi^2\;\;\;$ $0.2$ & $93$ & $92$ & $89$ & $85$ & $85$ & $89$\\ 
$0.3$ & $91$ & $92$ & $87$ & $85$ & $88$ & $91$\\ 
$0.4$ & $92$ & $89$ & $84$ & $87$ & $90$ & $90$\\ 
 \hline 
 \end{tabular}
  \end{subtable}
  ~
  \begin{subtable}{0.48\textwidth}
    \caption{Relative Width}
    \begin{tabular}{r|rrrrrr}
\hline\hline
 &\multicolumn{6}{c}{$\rho$} \\ 
 $\alpha = 0.05$ & $0$ & $0.1$ & $0.2$ & $0.3$ & $0.4$ & $0.5$ \\ 
 \hline$0.1$ & $113$ & $114$ & $113$ & $119$ & $121$ & $124$\\ 
$\pi^2\;\;\;$ $0.2$ & $115$ & $117$ & $120$ & $123$ & $125$ & $126$\\ 
$0.3$ & $117$ & $117$ & $121$ & $122$ & $123$ & $124$\\ 
$0.4$ & $117$ & $118$ & $120$ & $121$ & $121$ & $121$\\ 
 \hline 
 \end{tabular}
 
 \vspace{2em} 
 
\begin{tabular}{r|rrrrrr}
\hline\hline
 &\multicolumn{6}{c}{$\rho$} \\ 
 $\alpha = 0.1$ & $0$ & $0.1$ & $0.2$ & $0.3$ & $0.4$ & $0.5$ \\ 
 \hline$0.1$ & $121$ & $123$ & $124$ & $127$ & $128$ & $133$\\ 
$\pi^2\;\;\;$ $0.2$ & $123$ & $125$ & $126$ & $129$ & $131$ & $132$\\ 
$0.3$ & $122$ & $123$ & $126$ & $128$ & $128$ & $128$\\ 
$0.4$ & $122$ & $124$ & $124$ & $125$ & $125$ & $125$\\ 
 \hline 
 \end{tabular}
 
 \vspace{2em} 
 
\begin{tabular}{r|rrrrrr}
\hline\hline
 &\multicolumn{6}{c}{$\rho$} \\ 
 $\alpha = 0.2$ & $0$ & $0.1$ & $0.2$ & $0.3$ & $0.4$ & $0.5$ \\ 
 \hline$0.1$ & $138$ & $139$ & $137$ & $142$ & $144$ & $146$\\ 
$\pi^2\;\;\;$ $0.2$ & $136$ & $137$ & $138$ & $140$ & $142$ & $142$\\ 
$0.3$ & $135$ & $135$ & $136$ & $137$ & $137$ & $137$\\ 
$0.4$ & $133$ & $133$ & $133$ & $133$ & $133$ & $132$\\ 
 \hline 
 \end{tabular}
  \end{subtable}
  \label{tab:CISim100_2stepWideTau_OLSvsIV}
  \caption{2-step CI, $\alpha_1 = \alpha/4,\alpha_2 = 3\alpha/4$, OLS vs IV Example, $N=100$}
\end{table}

\begin{table}[h]
  \centering
  \begin{subtable}{0.48\textwidth}
    \caption{Coverage Probability}
    \begin{tabular}{r|rrrrrr}
\hline\hline
 &\multicolumn{6}{c}{$\rho$} \\ 
 $\alpha = 0.05$ & $0$ & $0.1$ & $0.2$ & $0.3$ & $0.4$ & $0.5$ \\ 
 \hline$0.1$ & $97$ & $98$ & $97$ & $94$ & $97$ & $99$\\ 
$\pi^2\;\;\;$ $0.2$ & $97$ & $98$ & $95$ & $98$ & $99$ & $99$\\ 
$0.3$ & $97$ & $97$ & $96$ & $99$ & $99$ & $95$\\ 
$0.4$ & $96$ & $97$ & $96$ & $98$ & $97$ & $95$\\ 
 \hline 
 \end{tabular}
 
 \vspace{2em} 
 
\begin{tabular}{r|rrrrrr}
\hline\hline
 &\multicolumn{6}{c}{$\rho$} \\ 
 $\alpha = 0.1$ & $0$ & $0.1$ & $0.2$ & $0.3$ & $0.4$ & $0.5$ \\ 
 \hline$0.1$ & $94$ & $97$ & $94$ & $90$ & $94$ & $98$\\ 
$\pi^2\;\;\;$ $0.2$ & $94$ & $94$ & $90$ & $95$ & $96$ & $96$\\ 
$0.3$ & $94$ & $93$ & $94$ & $96$ & $96$ & $92$\\ 
$0.4$ & $94$ & $93$ & $94$ & $97$ & $94$ & $93$\\ 
 \hline 
 \end{tabular}
 
 \vspace{2em} 
 
\begin{tabular}{r|rrrrrr}
\hline\hline
 &\multicolumn{6}{c}{$\rho$} \\ 
 $\alpha = 0.2$ & $0$ & $0.1$ & $0.2$ & $0.3$ & $0.4$ & $0.5$ \\ 
 \hline$0.1$ & $89$ & $92$ & $85$ & $87$ & $92$ & $94$\\ 
$\pi^2\;\;\;$ $0.2$ & $89$ & $89$ & $86$ & $90$ & $92$ & $90$\\ 
$0.3$ & $91$ & $87$ & $88$ & $90$ & $88$ & $86$\\ 
$0.4$ & $91$ & $84$ & $90$ & $87$ & $86$ & $86$\\ 
 \hline 
 \end{tabular}
  \end{subtable}
  ~
  \begin{subtable}{0.48\textwidth}
    \caption{Relative Width}
    \begin{tabular}{r|rrrrrr}
\hline\hline
 &\multicolumn{6}{c}{$\rho$} \\ 
 $\alpha = 0.05$ & $0$ & $0.1$ & $0.2$ & $0.3$ & $0.4$ & $0.5$ \\ 
 \hline$0.1$ & $115$ & $117$ & $123$ & $129$ & $130$ & $130$\\ 
$\pi^2\;\;\;$ $0.2$ & $116$ & $120$ & $125$ & $126$ & $126$ & $118$\\ 
$0.3$ & $117$ & $121$ & $123$ & $123$ & $115$ & $106$\\ 
$0.4$ & $117$ & $120$ & $121$ & $116$ & $107$ & $106$\\ 
 \hline 
 \end{tabular}
 
 \vspace{2em} 
 
\begin{tabular}{r|rrrrrr}
\hline\hline
 &\multicolumn{6}{c}{$\rho$} \\ 
 $\alpha = 0.1$ & $0$ & $0.1$ & $0.2$ & $0.3$ & $0.4$ & $0.5$ \\ 
 \hline$0.1$ & $123$ & $125$ & $131$ & $135$ & $137$ & $137$\\ 
$\pi^2\;\;\;$ $0.2$ & $123$ & $127$ & $131$ & $132$ & $130$ & $117$\\ 
$0.3$ & $123$ & $126$ & $128$ & $126$ & $115$ & $108$\\ 
$0.4$ & $123$ & $125$ & $125$ & $116$ & $108$ & $108$\\ 
 \hline 
 \end{tabular}
 
 \vspace{2em} 
 
\begin{tabular}{r|rrrrrr}
\hline\hline
 &\multicolumn{6}{c}{$\rho$} \\ 
 $\alpha = 0.2$ & $0$ & $0.1$ & $0.2$ & $0.3$ & $0.4$ & $0.5$ \\ 
 \hline$0.1$ & $137$ & $140$ & $144$ & $147$ & $148$ & $147$\\ 
$\pi^2\;\;\;$ $0.2$ & $136$ & $138$ & $141$ & $141$ & $132$ & $116$\\ 
$0.3$ & $135$ & $136$ & $136$ & $129$ & $115$ & $112$\\ 
$0.4$ & $133$ & $133$ & $131$ & $117$ & $112$ & $112$\\ 
 \hline 
 \end{tabular}
  \end{subtable}
  \label{tab:CISim500_2stepWideTau_OLSvsIV}
  \caption{2-step CI, $\alpha_1 = \alpha/4,\alpha_2 = 3\alpha/4$, OLS vs IV Example, $N=500$}
\end{table}





\begin{table}[h]
  \centering
  \begin{subtable}{0.48\textwidth}
    \caption{Coverage Probability}
    \begin{tabular}{r|rrrrrr}
\hline\hline
 &\multicolumn{6}{c}{$\rho$} \\ 
 $\alpha = 0.05$ & $0$ & $0.1$ & $0.2$ & $0.3$ & $0.4$ & $0.5$ \\ 
 \hline$0.1$ & $100$ & $100$ & $100$ & $100$ & $ 98$ & $ 96$\\ 
$\pi^2\;\;\;$ $0.2$ & $100$ & $100$ & $100$ & $ 99$ & $ 98$ & $ 94$\\ 
$0.3$ & $ 99$ & $100$ & $100$ & $ 99$ & $ 97$ & $ 96$\\ 
$0.4$ & $ 99$ & $100$ & $ 99$ & $ 98$ & $ 97$ & $ 98$\\ 
 \hline 
 \end{tabular}
 
 \vspace{2em} 
 
\begin{tabular}{r|rrrrrr}
\hline\hline
 &\multicolumn{6}{c}{$\rho$} \\ 
 $\alpha = 0.1$ & $0$ & $0.1$ & $0.2$ & $0.3$ & $0.4$ & $0.5$ \\ 
 \hline$0.1$ & $100$ & $100$ & $100$ & $ 99$ & $ 97$ & $ 93$\\ 
$\pi^2\;\;\;$ $0.2$ & $ 99$ & $ 99$ & $ 99$ & $ 98$ & $ 96$ & $ 93$\\ 
$0.3$ & $ 98$ & $ 99$ & $ 98$ & $ 97$ & $ 95$ & $ 94$\\ 
$0.4$ & $ 98$ & $ 99$ & $ 98$ & $ 97$ & $ 97$ & $ 96$\\ 
 \hline 
 \end{tabular}
 
 \vspace{2em} 
 
\begin{tabular}{r|rrrrrr}
\hline\hline
 &\multicolumn{6}{c}{$\rho$} \\ 
 $\alpha = 0.2$ & $0$ & $0.1$ & $0.2$ & $0.3$ & $0.4$ & $0.5$ \\ 
 \hline$0.1$ & $99$ & $98$ & $98$ & $98$ & $96$ & $90$\\ 
$\pi^2\;\;\;$ $0.2$ & $98$ & $98$ & $98$ & $96$ & $94$ & $90$\\ 
$0.3$ & $97$ & $97$ & $98$ & $94$ & $92$ & $94$\\ 
$0.4$ & $97$ & $96$ & $95$ & $95$ & $94$ & $96$\\ 
 \hline 
 \end{tabular}
  \end{subtable}
  ~
  \begin{subtable}{0.48\textwidth}
    \caption{Relative Width}
    \begin{tabular}{r|rrrrrr}
\hline\hline
 &\multicolumn{6}{c}{$\rho$} \\ 
 $\alpha = 0.05$ & $0$ & $0.1$ & $0.2$ & $0.3$ & $0.4$ & $0.5$ \\ 
 \hline$0.1$ & $130$ & $130$ & $131$ & $131$ & $133$ & $134$\\ 
$\pi^2\;\;\;$ $0.2$ & $133$ & $134$ & $134$ & $137$ & $140$ & $143$\\ 
$0.3$ & $134$ & $136$ & $138$ & $140$ & $143$ & $145$\\ 
$0.4$ & $136$ & $136$ & $139$ & $141$ & $142$ & $143$\\ 
 \hline 
 \end{tabular}
 
 \vspace{2em} 
 
\begin{tabular}{r|rrrrrr}
\hline\hline
 &\multicolumn{6}{c}{$\rho$} \\ 
 $\alpha = 0.1$ & $0$ & $0.1$ & $0.2$ & $0.3$ & $0.4$ & $0.5$ \\ 
 \hline$0.1$ & $141$ & $141$ & $142$ & $143$ & $146$ & $146$\\ 
$\pi^2\;\;\;$ $0.2$ & $145$ & $145$ & $147$ & $149$ & $152$ & $155$\\ 
$0.3$ & $147$ & $147$ & $149$ & $152$ & $155$ & $156$\\ 
$0.4$ & $147$ & $148$ & $150$ & $153$ & $154$ & $154$\\ 
 \hline 
 \end{tabular}
 
 \vspace{2em} 
 
\begin{tabular}{r|rrrrrr}
\hline\hline
 &\multicolumn{6}{c}{$\rho$} \\ 
 $\alpha = 0.2$ & $0$ & $0.1$ & $0.2$ & $0.3$ & $0.4$ & $0.5$ \\ 
 \hline$0.1$ & $164$ & $164$ & $164$ & $166$ & $168$ & $171$\\ 
$\pi^2\;\;\;$ $0.2$ & $166$ & $167$ & $169$ & $172$ & $176$ & $178$\\ 
$0.3$ & $167$ & $168$ & $170$ & $174$ & $177$ & $179$\\ 
$0.4$ & $167$ & $169$ & $171$ & $174$ & $175$ & $175$\\ 
 \hline 
 \end{tabular}
  \end{subtable}
  \label{tab:CISim50_2stepNarrowTau_OLSvsIV}
  \caption{2-step CI, $\alpha_1 = 3\alpha/4, \alpha_2 = \alpha/4$ OLS vs IV Example, $N=50$}
\end{table}

\begin{table}[h]
  \centering
  \begin{subtable}{0.48\textwidth}
    \caption{Coverage Probability}
    \begin{tabular}{r|rrrrrr}
\hline\hline
 &\multicolumn{6}{c}{$\rho$} \\ 
 $\alpha = 0.05$ & $0$ & $0.1$ & $0.2$ & $0.3$ & $0.4$ & $0.5$ \\ 
 \hline$0.1$ & $100$ & $100$ & $100$ & $ 99$ & $ 98$ & $ 94$\\ 
$\pi^2\;\;\;$ $0.2$ & $ 99$ & $100$ & $100$ & $ 99$ & $ 96$ & $ 96$\\ 
$0.3$ & $ 99$ & $ 99$ & $100$ & $ 98$ & $ 98$ & $ 99$\\ 
$0.4$ & $ 98$ & $100$ & $ 99$ & $ 98$ & $ 99$ & $ 99$\\ 
 \hline 
 \end{tabular}
 
 \vspace{2em} 
 
\begin{tabular}{r|rrrrrr}
\hline\hline
 &\multicolumn{6}{c}{$\rho$} \\ 
 $\alpha = 0.1$ & $0$ & $0.1$ & $0.2$ & $0.3$ & $0.4$ & $0.5$ \\ 
 \hline$0.1$ & $ 99$ & $100$ & $ 99$ & $ 99$ & $ 96$ & $ 92$\\ 
$\pi^2\;\;\;$ $0.2$ & $ 98$ & $ 98$ & $ 99$ & $ 97$ & $ 95$ & $ 95$\\ 
$0.3$ & $ 99$ & $ 99$ & $ 98$ & $ 96$ & $ 97$ & $ 98$\\ 
$0.4$ & $ 98$ & $ 98$ & $ 97$ & $ 97$ & $ 99$ & $ 99$\\ 
 \hline 
 \end{tabular}
 
 \vspace{2em} 
 
\begin{tabular}{r|rrrrrr}
\hline\hline
 &\multicolumn{6}{c}{$\rho$} \\ 
 $\alpha = 0.2$ & $0$ & $0.1$ & $0.2$ & $0.3$ & $0.4$ & $0.5$ \\ 
 \hline$0.1$ & $98$ & $98$ & $98$ & $98$ & $93$ & $90$\\ 
$\pi^2\;\;\;$ $0.2$ & $97$ & $97$ & $97$ & $95$ & $93$ & $94$\\ 
$0.3$ & $97$ & $98$ & $97$ & $94$ & $94$ & $96$\\ 
$0.4$ & $97$ & $97$ & $95$ & $96$ & $97$ & $98$\\ 
 \hline 
 \end{tabular}
  \end{subtable}
  ~
  \begin{subtable}{0.48\textwidth}
    \caption{Relative Width}
    \begin{tabular}{r|rrrrrr}
\hline\hline
 &\multicolumn{6}{c}{$\rho$} \\ 
 $\alpha = 0.05$ & $0$ & $0.1$ & $0.2$ & $0.3$ & $0.4$ & $0.5$ \\ 
 \hline$0.1$ & $131$ & $131$ & $131$ & $136$ & $137$ & $142$\\ 
$\pi^2\;\;\;$ $0.2$ & $134$ & $134$ & $138$ & $141$ & $145$ & $147$\\ 
$0.3$ & $135$ & $136$ & $140$ & $143$ & $145$ & $146$\\ 
$0.4$ & $135$ & $137$ & $141$ & $142$ & $143$ & $143$\\ 
 \hline 
 \end{tabular}
 
 \vspace{2em} 
 
\begin{tabular}{r|rrrrrr}
\hline\hline
 &\multicolumn{6}{c}{$\rho$} \\ 
 $\alpha = 0.1$ & $0$ & $0.1$ & $0.2$ & $0.3$ & $0.4$ & $0.5$ \\ 
 \hline$0.1$ & $142$ & $144$ & $145$ & $149$ & $150$ & $157$\\ 
$\pi^2\;\;\;$ $0.2$ & $145$ & $148$ & $149$ & $154$ & $158$ & $160$\\ 
$0.3$ & $146$ & $148$ & $152$ & $155$ & $157$ & $157$\\ 
$0.4$ & $147$ & $149$ & $152$ & $154$ & $154$ & $154$\\ 
 \hline 
 \end{tabular}
 
 \vspace{2em} 
 
\begin{tabular}{r|rrrrrr}
\hline\hline
 &\multicolumn{6}{c}{$\rho$} \\ 
 $\alpha = 0.2$ & $0$ & $0.1$ & $0.2$ & $0.3$ & $0.4$ & $0.5$ \\ 
 \hline$0.1$ & $166$ & $167$ & $167$ & $172$ & $176$ & $181$\\ 
$\pi^2\;\;\;$ $0.2$ & $167$ & $168$ & $172$ & $177$ & $181$ & $183$\\ 
$0.3$ & $167$ & $169$ & $173$ & $177$ & $179$ & $179$\\ 
$0.4$ & $168$ & $169$ & $173$ & $175$ & $175$ & $173$\\ 
 \hline 
 \end{tabular}
  \end{subtable}
  \label{tab:CISim100_2stepNarrowTau_OLSvsIV}
  \caption{2-step CI, $\alpha_1 = 3\alpha/4,\alpha_2 = \alpha/4$, OLS vs IV Example, $N=100$}
\end{table}

\begin{table}[h]
  \centering
  \begin{subtable}{0.48\textwidth}
    \caption{Coverage Probability}
    \begin{tabular}{r|rrrrrr}
\hline\hline
 &\multicolumn{6}{c}{$\rho$} \\ 
 $\alpha = 0.05$ & $0$ & $0.1$ & $0.2$ & $0.3$ & $0.4$ & $0.5$ \\ 
 \hline$0.1$ & $ 99$ & $ 99$ & $100$ & $ 96$ & $ 97$ & $100$\\ 
$\pi^2\;\;\;$ $0.2$ & $ 99$ & $ 99$ & $ 97$ & $ 99$ & $100$ & $100$\\ 
$0.3$ & $ 99$ & $ 99$ & $ 98$ & $100$ & $100$ & $ 98$\\ 
$0.4$ & $ 98$ & $ 99$ & $ 99$ & $100$ & $ 99$ & $ 98$\\ 
 \hline 
 \end{tabular}
 
 \vspace{2em} 
 
\begin{tabular}{r|rrrrrr}
\hline\hline
 &\multicolumn{6}{c}{$\rho$} \\ 
 $\alpha = 0.1$ & $0$ & $0.1$ & $0.2$ & $0.3$ & $0.4$ & $0.5$ \\ 
 \hline$0.1$ & $ 98$ & $ 98$ & $ 98$ & $ 94$ & $ 97$ & $100$\\ 
$\pi^2\;\;\;$ $0.2$ & $ 98$ & $ 98$ & $ 97$ & $ 98$ & $ 99$ & $ 99$\\ 
$0.3$ & $ 98$ & $ 97$ & $ 97$ & $ 99$ & $ 99$ & $ 97$\\ 
$0.4$ & $ 96$ & $ 98$ & $ 98$ & $100$ & $ 98$ & $ 98$\\ 
 \hline 
 \end{tabular}
 
 \vspace{2em} 
 
\begin{tabular}{r|rrrrrr}
\hline\hline
 &\multicolumn{6}{c}{$\rho$} \\ 
 $\alpha = 0.2$ & $0$ & $0.1$ & $0.2$ & $0.3$ & $0.4$ & $0.5$ \\ 
 \hline$0.1$ & $96$ & $98$ & $95$ & $92$ & $97$ & $98$\\ 
$\pi^2\;\;\;$ $0.2$ & $95$ & $96$ & $93$ & $96$ & $98$ & $98$\\ 
$0.3$ & $96$ & $96$ & $95$ & $98$ & $96$ & $96$\\ 
$0.4$ & $96$ & $95$ & $98$ & $96$ & $96$ & $95$\\ 
 \hline 
 \end{tabular}
  \end{subtable}
  ~
  \begin{subtable}{0.48\textwidth}
    \caption{Relative Width}
    \begin{tabular}{r|rrrrrr}
\hline\hline
 &\multicolumn{6}{c}{$\rho$} \\ 
 $\alpha = 0.05$ & $0$ & $0.1$ & $0.2$ & $0.3$ & $0.4$ & $0.5$ \\ 
 \hline$0.1$ & $132$ & $135$ & $141$ & $148$ & $152$ & $152$\\ 
$\pi^2\;\;\;$ $0.2$ & $133$ & $139$ & $146$ & $148$ & $148$ & $143$\\ 
$0.3$ & $134$ & $141$ & $145$ & $145$ & $137$ & $128$\\ 
$0.4$ & $135$ & $141$ & $143$ & $138$ & $128$ & $127$\\ 
 \hline 
 \end{tabular}
 
 \vspace{2em} 
 
\begin{tabular}{r|rrrrrr}
\hline\hline
 &\multicolumn{6}{c}{$\rho$} \\ 
 $\alpha = 0.1$ & $0$ & $0.1$ & $0.2$ & $0.3$ & $0.4$ & $0.5$ \\ 
 \hline$0.1$ & $144$ & $147$ & $155$ & $162$ & $165$ & $165$\\ 
$\pi^2\;\;\;$ $0.2$ & $146$ & $151$ & $158$ & $161$ & $159$ & $146$\\ 
$0.3$ & $146$ & $153$ & $157$ & $155$ & $143$ & $136$\\ 
$0.4$ & $147$ & $153$ & $154$ & $144$ & $136$ & $136$\\ 
 \hline 
 \end{tabular}
 
 \vspace{2em} 
 
\begin{tabular}{r|rrrrrr}
\hline\hline
 &\multicolumn{6}{c}{$\rho$} \\ 
 $\alpha = 0.2$ & $0$ & $0.1$ & $0.2$ & $0.3$ & $0.4$ & $0.5$ \\ 
 \hline$0.1$ & $166$ & $170$ & $179$ & $186$ & $189$ & $189$\\ 
$\pi^2\;\;\;$ $0.2$ & $167$ & $173$ & $182$ & $183$ & $174$ & $157$\\ 
$0.3$ & $167$ & $174$ & $178$ & $171$ & $155$ & $153$\\ 
$0.4$ & $168$ & $174$ & $173$ & $158$ & $153$ & $153$\\ 
 \hline 
 \end{tabular}
  \end{subtable}
  \label{tab:CISim500_2stepNarrowTau_OLSvsIV}
  \caption{2-step CI, $\alpha_1 = 3\alpha/4,\alpha_2 = \alpha/4$, OLS vs IV Example, $N=500$}
\end{table}






\clearpage




\begin{table}[h]
  \centering
  \begin{tabular}{r|rrrrrr}
\hline\hline
 &\multicolumn{6}{c}{$\rho$} \\ 
 $\alpha = 0.05$ & $0$ & $0.1$ & $0.2$ & $0.3$ & $0.4$ & $0.5$ \\ 
 \hline$0.1$ & $90$ & $90$ & $91$ & $89$ & $91$ & $92$\\ 
$\gamma^2\;\;\;$ $0.2$ & $91$ & $90$ & $90$ & $89$ & $90$ & $90$\\ 
$0.3$ & $91$ & $90$ & $90$ & $91$ & $89$ & $92$\\ 
$0.4$ & $90$ & $91$ & $91$ & $90$ & $90$ & $91$\\ 
 \hline 
 \end{tabular}
 
 \vspace{2em} 
 
\begin{tabular}{r|rrrrrr}
\hline\hline
 &\multicolumn{6}{c}{$\rho$} \\ 
 $\alpha = 0.1$ & $0$ & $0.1$ & $0.2$ & $0.3$ & $0.4$ & $0.5$ \\ 
 \hline$0.1$ & $84$ & $86$ & $85$ & $86$ & $84$ & $84$\\ 
$\gamma^2\;\;\;$ $0.2$ & $83$ & $81$ & $85$ & $84$ & $83$ & $85$\\ 
$0.3$ & $85$ & $85$ & $86$ & $86$ & $84$ & $86$\\ 
$0.4$ & $87$ & $86$ & $84$ & $85$ & $84$ & $86$\\ 
 \hline 
 \end{tabular}
 
 \vspace{2em} 
 
\begin{tabular}{r|rrrrrr}
\hline\hline
 &\multicolumn{6}{c}{$\rho$} \\ 
 $\alpha = 0.2$ & $0$ & $0.1$ & $0.2$ & $0.3$ & $0.4$ & $0.5$ \\ 
 \hline$0.1$ & $77$ & $75$ & $76$ & $77$ & $78$ & $77$\\ 
$\gamma^2\;\;\;$ $0.2$ & $75$ & $75$ & $76$ & $76$ & $76$ & $73$\\ 
$0.3$ & $71$ & $73$ & $77$ & $76$ & $76$ & $75$\\ 
$0.4$ & $75$ & $75$ & $74$ & $76$ & $74$ & $75$\\ 
 \hline 
 \end{tabular}
  \label{tab:CISim50Valid_ChooseIVs}
  \caption{Coverage prob of Valid Estimator, Choosing IVs Example, $N=50$}
\end{table}

\begin{table}[h]
  \centering
  \begin{tabular}{r|rrrrrr}
\hline\hline
 &\multicolumn{6}{c}{$\rho$} \\ 
 $\alpha = 0.05$ & $0$ & $0.1$ & $0.2$ & $0.3$ & $0.4$ & $0.5$ \\ 
 \hline$0.1$ & $93$ & $90$ & $92$ & $92$ & $91$ & $93$\\ 
$\gamma^2\;\;\;$ $0.2$ & $91$ & $91$ & $92$ & $92$ & $91$ & $94$\\ 
$0.3$ & $91$ & $93$ & $92$ & $93$ & $93$ & $92$\\ 
$0.4$ & $93$ & $92$ & $92$ & $92$ & $92$ & $91$\\ 
 \hline 
 \end{tabular}
 
 \vspace{2em} 
 
\begin{tabular}{r|rrrrrr}
\hline\hline
 &\multicolumn{6}{c}{$\rho$} \\ 
 $\alpha = 0.1$ & $0$ & $0.1$ & $0.2$ & $0.3$ & $0.4$ & $0.5$ \\ 
 \hline$0.1$ & $88$ & $85$ & $87$ & $87$ & $87$ & $89$\\ 
$\gamma^2\;\;\;$ $0.2$ & $87$ & $89$ & $87$ & $88$ & $88$ & $88$\\ 
$0.3$ & $89$ & $88$ & $87$ & $88$ & $88$ & $86$\\ 
$0.4$ & $86$ & $87$ & $87$ & $89$ & $86$ & $87$\\ 
 \hline 
 \end{tabular}
 
 \vspace{2em} 
 
\begin{tabular}{r|rrrrrr}
\hline\hline
 &\multicolumn{6}{c}{$\rho$} \\ 
 $\alpha = 0.2$ & $0$ & $0.1$ & $0.2$ & $0.3$ & $0.4$ & $0.5$ \\ 
 \hline$0.1$ & $78$ & $79$ & $77$ & $77$ & $78$ & $79$\\ 
$\gamma^2\;\;\;$ $0.2$ & $82$ & $79$ & $79$ & $79$ & $80$ & $78$\\ 
$0.3$ & $75$ & $74$ & $77$ & $79$ & $78$ & $79$\\ 
$0.4$ & $79$ & $79$ & $79$ & $79$ & $77$ & $79$\\ 
 \hline 
 \end{tabular}
  \label{tab:CISim100Valid_ChooseIVs}
  \caption{Coverage prob of Valid Estimator, Choosing IVs Example, $N=100$}
\end{table}

\begin{table}[h]
  \centering
  \begin{tabular}{r|rrrrrr}
\hline\hline
 &\multicolumn{6}{c}{$\rho$} \\ 
 $\alpha = 0.05$ & $0$ & $0.1$ & $0.2$ & $0.3$ & $0.4$ & $0.5$ \\ 
 \hline$0.1$ & $90$ & $90$ & $91$ & $89$ & $91$ & $92$\\ 
$\gamma^2\;\;\;$ $0.2$ & $91$ & $90$ & $90$ & $89$ & $90$ & $90$\\ 
$0.3$ & $91$ & $90$ & $90$ & $91$ & $89$ & $92$\\ 
$0.4$ & $90$ & $91$ & $91$ & $90$ & $90$ & $91$\\ 
 \hline 
 \end{tabular}
 
 \vspace{2em} 
 
\begin{tabular}{r|rrrrrr}
\hline\hline
 &\multicolumn{6}{c}{$\rho$} \\ 
 $\alpha = 0.1$ & $0$ & $0.1$ & $0.2$ & $0.3$ & $0.4$ & $0.5$ \\ 
 \hline$0.1$ & $84$ & $86$ & $85$ & $86$ & $84$ & $84$\\ 
$\gamma^2\;\;\;$ $0.2$ & $83$ & $81$ & $85$ & $84$ & $83$ & $85$\\ 
$0.3$ & $85$ & $85$ & $86$ & $86$ & $84$ & $86$\\ 
$0.4$ & $87$ & $86$ & $84$ & $85$ & $84$ & $86$\\ 
 \hline 
 \end{tabular}
 
 \vspace{2em} 
 
\begin{tabular}{r|rrrrrr}
\hline\hline
 &\multicolumn{6}{c}{$\rho$} \\ 
 $\alpha = 0.2$ & $0$ & $0.1$ & $0.2$ & $0.3$ & $0.4$ & $0.5$ \\ 
 \hline$0.1$ & $77$ & $75$ & $76$ & $77$ & $78$ & $77$\\ 
$\gamma^2\;\;\;$ $0.2$ & $75$ & $75$ & $76$ & $76$ & $76$ & $73$\\ 
$0.3$ & $71$ & $73$ & $77$ & $76$ & $76$ & $75$\\ 
$0.4$ & $75$ & $75$ & $74$ & $76$ & $74$ & $75$\\ 
 \hline 
 \end{tabular}
  \label{tab:CISim500Valid_ChooseIVs}
  \caption{Coverage prob of Valid Estimator, Choosing IVs Example, $N=500$}
\end{table}









\begin{table}[h]
  \centering
  \begin{subtable}{0.48\textwidth}
    \caption{Coverage Probability}
    \begin{tabular}{r|rrrrrr}
\hline\hline
 &\multicolumn{6}{c}{$\rho$} \\ 
 $\alpha = 0.05$ & $0$ & $0.1$ & $0.2$ & $0.3$ & $0.4$ & $0.5$ \\ 
 \hline$0.1$ & $85$ & $82$ & $75$ & $70$ & $78$ & $85$\\ 
$\gamma^2\;\;\;$ $0.2$ & $87$ & $77$ & $67$ & $61$ & $62$ & $75$\\ 
$0.3$ & $87$ & $77$ & $64$ & $54$ & $50$ & $66$\\ 
$0.4$ & $86$ & $78$ & $64$ & $51$ & $48$ & $58$\\ 
 \hline 
 \end{tabular}
 
 \vspace{2em} 
 
\begin{tabular}{r|rrrrrr}
\hline\hline
 &\multicolumn{6}{c}{$\rho$} \\ 
 $\alpha = 0.1$ & $0$ & $0.1$ & $0.2$ & $0.3$ & $0.4$ & $0.5$ \\ 
 \hline$0.1$ & $79$ & $75$ & $68$ & $67$ & $73$ & $78$\\ 
$\gamma^2\;\;\;$ $0.2$ & $79$ & $70$ & $60$ & $54$ & $58$ & $71$\\ 
$0.3$ & $78$ & $68$ & $57$ & $49$ & $49$ & $64$\\ 
$0.4$ & $81$ & $69$ & $52$ & $42$ & $41$ & $57$\\ 
 \hline 
 \end{tabular}
 
 \vspace{2em} 
 
\begin{tabular}{r|rrrrrr}
\hline\hline
 &\multicolumn{6}{c}{$\rho$} \\ 
 $\alpha = 0.2$ & $0$ & $0.1$ & $0.2$ & $0.3$ & $0.4$ & $0.5$ \\ 
 \hline$0.1$ & $72$ & $61$ & $55$ & $57$ & $67$ & $72$\\ 
$\gamma^2\;\;\;$ $0.2$ & $70$ & $59$ & $49$ & $43$ & $53$ & $60$\\ 
$0.3$ & $66$ & $57$ & $46$ & $38$ & $42$ & $55$\\ 
$0.4$ & $68$ & $58$ & $40$ & $33$ & $36$ & $46$\\ 
 \hline 
 \end{tabular}
  \end{subtable}
  ~
  \begin{subtable}{0.48\textwidth}
    \caption{Relative Width}
    \begin{tabular}{r|rrrrrr}
\hline\hline
 &\multicolumn{6}{c}{$\rho$} \\ 
 $\alpha = 0.05$ & $0$ & $0.1$ & $0.2$ & $0.3$ & $0.4$ & $0.5$ \\ 
 \hline$0.1$ & $67$ & $66$ & $71$ & $76$ & $85$ & $87$\\ 
$\gamma^2\;\;\;$ $0.2$ & $61$ & $56$ & $56$ & $61$ & $70$ & $75$\\ 
$0.3$ & $50$ & $23$ & $43$ & $53$ & $62$ & $73$\\ 
$0.4$ & $50$ & $44$ & $42$ & $48$ & $56$ & $64$\\ 
 \hline 
 \end{tabular}
 
 \vspace{2em} 
 
\begin{tabular}{r|rrrrrr}
\hline\hline
 &\multicolumn{6}{c}{$\rho$} \\ 
 $\alpha = 0.1$ & $0$ & $0.1$ & $0.2$ & $0.3$ & $0.4$ & $0.5$ \\ 
 \hline$0.1$ & $66$ & $64$ & $70$ & $75$ & $83$ & $82$\\ 
$\gamma^2\;\;\;$ $0.2$ & $55$ & $42$ & $57$ & $61$ & $69$ & $78$\\ 
$0.3$ & $54$ & $47$ & $50$ & $54$ & $61$ & $71$\\ 
$0.4$ & $48$ & $46$ & $43$ & $52$ & $54$ & $68$\\ 
 \hline 
 \end{tabular}
 
 \vspace{2em} 
 
\begin{tabular}{r|rrrrrr}
\hline\hline
 &\multicolumn{6}{c}{$\rho$} \\ 
 $\alpha = 0.2$ & $0$ & $0.1$ & $0.2$ & $0.3$ & $0.4$ & $0.5$ \\ 
 \hline$0.1$ & $69$ & $64$ & $69$ & $71$ & $71$ & $89$\\ 
$\gamma^2\;\;\;$ $0.2$ & $61$ & $53$ & $58$ & $53$ & $73$ & $74$\\ 
$0.3$ & $54$ & $52$ & $52$ & $55$ & $59$ & $65$\\ 
$0.4$ & $51$ & $46$ & $42$ & $50$ & $54$ & $62$\\ 
 \hline 
 \end{tabular}
  \end{subtable}
  \label{tab:CISim50Naive_ChooseIVs}
  \caption{Naive CI, Choosing IVs Example, $N=50$}
\end{table}

\begin{table}[h]
  \centering
  \begin{subtable}{0.48\textwidth}
    \caption{Coverage Probability}
    \begin{tabular}{r|rrrrrr}
\hline\hline
 &\multicolumn{6}{c}{$\rho$} \\ 
 $\alpha = 0.05$ & $0$ & $0.1$ & $0.2$ & $0.3$ & $0.4$ & $0.5$ \\ 
 \hline$0.1$ & $90$ & $80$ & $77$ & $83$ & $88$ & $93$\\ 
$\gamma^2\;\;\;$ $0.2$ & $89$ & $78$ & $66$ & $69$ & $82$ & $92$\\ 
$0.3$ & $87$ & $78$ & $59$ & $60$ & $77$ & $86$\\ 
$0.4$ & $89$ & $77$ & $55$ & $50$ & $67$ & $81$\\ 
 \hline 
 \end{tabular}
 
 \vspace{2em} 
 
\begin{tabular}{r|rrrrrr}
\hline\hline
 &\multicolumn{6}{c}{$\rho$} \\ 
 $\alpha = 0.1$ & $0$ & $0.1$ & $0.2$ & $0.3$ & $0.4$ & $0.5$ \\ 
 \hline$0.1$ & $84$ & $72$ & $73$ & $77$ & $86$ & $89$\\ 
$\gamma^2\;\;\;$ $0.2$ & $83$ & $72$ & $58$ & $66$ & $78$ & $86$\\ 
$0.3$ & $84$ & $70$ & $51$ & $55$ & $73$ & $82$\\ 
$0.4$ & $81$ & $65$ & $46$ & $48$ & $63$ & $78$\\ 
 \hline 
 \end{tabular}
 
 \vspace{2em} 
 
\begin{tabular}{r|rrrrrr}
\hline\hline
 &\multicolumn{6}{c}{$\rho$} \\ 
 $\alpha = 0.2$ & $0$ & $0.1$ & $0.2$ & $0.3$ & $0.4$ & $0.5$ \\ 
 \hline$0.1$ & $76$ & $63$ & $60$ & $70$ & $77$ & $79$\\ 
$\gamma^2\;\;\;$ $0.2$ & $75$ & $59$ & $49$ & $58$ & $73$ & $77$\\ 
$0.3$ & $69$ & $54$ & $39$ & $48$ & $67$ & $76$\\ 
$0.4$ & $71$ & $53$ & $34$ & $40$ & $57$ & $74$\\ 
 \hline 
 \end{tabular}
  \end{subtable}
  ~
  \begin{subtable}{0.48\textwidth}
    \caption{Relative Width}
    \begin{tabular}{r|rrrrrr}
\hline\hline
 &\multicolumn{6}{c}{$\rho$} \\ 
 $\alpha = 0.05$ & $0$ & $0.1$ & $0.2$ & $0.3$ & $0.4$ & $0.5$ \\ 
 \hline$0.1$ & $72$ & $72$ & $80$ & $91$ & $97$ & $99$\\ 
$\gamma^2\;\;\;$ $0.2$ & $62$ & $61$ & $72$ & $83$ & $91$ & $98$\\ 
$0.3$ & $56$ & $54$ & $58$ & $71$ & $87$ & $94$\\ 
$0.4$ & $49$ & $47$ & $53$ & $67$ & $80$ & $87$\\ 
 \hline 
 \end{tabular}
 
 \vspace{2em} 
 
\begin{tabular}{r|rrrrrr}
\hline\hline
 &\multicolumn{6}{c}{$\rho$} \\ 
 $\alpha = 0.1$ & $0$ & $0.1$ & $0.2$ & $0.3$ & $0.4$ & $0.5$ \\ 
 \hline$0.1$ & $71$ & $67$ & $81$ & $91$ & $99$ & $98$\\ 
$\gamma^2\;\;\;$ $0.2$ & $60$ & $61$ & $67$ & $78$ & $89$ & $92$\\ 
$0.3$ & $55$ & $54$ & $59$ & $72$ & $82$ & $94$\\ 
$0.4$ & $51$ & $47$ & $55$ & $68$ & $78$ & $89$\\ 
 \hline 
 \end{tabular}
 
 \vspace{2em} 
 
\begin{tabular}{r|rrrrrr}
\hline\hline
 &\multicolumn{6}{c}{$\rho$} \\ 
 $\alpha = 0.2$ & $0$ & $0.1$ & $0.2$ & $0.3$ & $0.4$ & $0.5$ \\ 
 \hline$0.1$ & $71$ & $71$ & $79$ & $92$ & $96$ & $98$\\ 
$\gamma^2\;\;\;$ $0.2$ & $60$ & $60$ & $68$ & $81$ & $91$ & $97$\\ 
$0.3$ & $55$ & $56$ & $61$ & $72$ & $84$ & $95$\\ 
$0.4$ & $51$ & $48$ & $53$ & $64$ & $78$ & $89$\\ 
 \hline 
 \end{tabular}
  \end{subtable}
  \label{tab:CISim100Naive_ChooseIVs}
  \caption{Naive CI, Choosing IVs Example, $N=100$}
\end{table}


\begin{table}[h]
  \centering
  \begin{subtable}{0.48\textwidth}
    \caption{Coverage Probability}
    \begin{tabular}{r|rrrrrr}
\hline\hline
 &\multicolumn{6}{c}{$\rho$} \\ 
 $\alpha = 0.05$ & $0$ & $0.1$ & $0.2$ & $0.3$ & $0.4$ & $0.5$ \\ 
 \hline$0.1$ & $93$ & $81$ & $94$ & $95$ & $94$ & $94$\\ 
$\gamma^2\;\;\;$ $0.2$ & $91$ & $71$ & $90$ & $94$ & $94$ & $92$\\ 
$0.3$ & $92$ & $63$ & $83$ & $95$ & $94$ & $95$\\ 
$0.4$ & $91$ & $62$ & $77$ & $94$ & $95$ & $94$\\ 
 \hline 
 \end{tabular}
 
 \vspace{2em} 
 
\begin{tabular}{r|rrrrrr}
\hline\hline
 &\multicolumn{6}{c}{$\rho$} \\ 
 $\alpha = 0.1$ & $0$ & $0.1$ & $0.2$ & $0.3$ & $0.4$ & $0.5$ \\ 
 \hline$0.1$ & $86$ & $77$ & $90$ & $90$ & $90$ & $90$\\ 
$\gamma^2\;\;\;$ $0.2$ & $86$ & $63$ & $84$ & $90$ & $90$ & $88$\\ 
$0.3$ & $84$ & $57$ & $81$ & $89$ & $90$ & $89$\\ 
$0.4$ & $82$ & $50$ & $74$ & $90$ & $90$ & $89$\\ 
 \hline 
 \end{tabular}
 
 \vspace{2em} 
 
\begin{tabular}{r|rrrrrr}
\hline\hline
 &\multicolumn{6}{c}{$\rho$} \\ 
 $\alpha = 0.2$ & $0$ & $0.1$ & $0.2$ & $0.3$ & $0.4$ & $0.5$ \\ 
 \hline$0.1$ & $76$ & $64$ & $81$ & $80$ & $78$ & $80$\\ 
$\gamma^2\;\;\;$ $0.2$ & $74$ & $53$ & $79$ & $80$ & $80$ & $81$\\ 
$0.3$ & $70$ & $41$ & $76$ & $79$ & $80$ & $81$\\ 
$0.4$ & $71$ & $38$ & $68$ & $80$ & $79$ & $78$\\ 
 \hline 
 \end{tabular}
  \end{subtable}
  ~
  \begin{subtable}{0.48\textwidth}
    \caption{Relative Width}
    \begin{tabular}{r|rrrrrr}
\hline\hline
 &\multicolumn{6}{c}{$\rho$} \\ 
 $\alpha = 0.05$ & $0$ & $0.1$ & $0.2$ & $0.3$ & $0.4$ & $0.5$ \\ 
 \hline$0.1$ & $ 76$ & $ 89$ & $100$ & $100$ & $100$ & $100$\\ 
$\gamma^2\;\;\;$ $0.2$ & $ 65$ & $ 79$ & $ 97$ & $100$ & $100$ & $100$\\ 
$0.3$ & $ 58$ & $ 70$ & $ 93$ & $100$ & $100$ & $100$\\ 
$0.4$ & $ 54$ & $ 65$ & $ 89$ & $ 99$ & $100$ & $100$\\ 
 \hline 
 \end{tabular}
 
 \vspace{2em} 
 
\begin{tabular}{r|rrrrrr}
\hline\hline
 &\multicolumn{6}{c}{$\rho$} \\ 
 $\alpha = 0.1$ & $0$ & $0.1$ & $0.2$ & $0.3$ & $0.4$ & $0.5$ \\ 
 \hline$0.1$ & $ 76$ & $ 89$ & $100$ & $100$ & $100$ & $100$\\ 
$\gamma^2\;\;\;$ $0.2$ & $ 66$ & $ 78$ & $ 97$ & $100$ & $100$ & $100$\\ 
$0.3$ & $ 59$ & $ 71$ & $ 93$ & $100$ & $100$ & $100$\\ 
$0.4$ & $ 55$ & $ 64$ & $ 89$ & $ 99$ & $100$ & $100$\\ 
 \hline 
 \end{tabular}
 
 \vspace{2em} 
 
\begin{tabular}{r|rrrrrr}
\hline\hline
 &\multicolumn{6}{c}{$\rho$} \\ 
 $\alpha = 0.2$ & $0$ & $0.1$ & $0.2$ & $0.3$ & $0.4$ & $0.5$ \\ 
 \hline$0.1$ & $ 75$ & $ 88$ & $100$ & $100$ & $100$ & $100$\\ 
$\gamma^2\;\;\;$ $0.2$ & $ 65$ & $ 78$ & $ 98$ & $100$ & $100$ & $100$\\ 
$0.3$ & $ 59$ & $ 69$ & $ 94$ & $100$ & $100$ & $100$\\ 
$0.4$ & $ 55$ & $ 65$ & $ 88$ & $ 99$ & $100$ & $100$\\ 
 \hline 
 \end{tabular}
  \end{subtable}
  \label{tab:CISim500Naive_ChooseIVs}
  \caption{Naive CI, Choosing IVs Example, $N=500$}
\end{table}






\begin{table}[h]
  \centering
  \begin{subtable}{0.48\textwidth}
    \caption{Coverage Probability}
    \begin{tabular}{r|rrrrrr}
\hline\hline
 &\multicolumn{6}{c}{$\rho$} \\ 
 $\alpha = 0.05$ & $0$ & $0.1$ & $0.2$ & $0.3$ & $0.4$ & $0.5$ \\ 
 \hline$0.1$ & $92$ & $90$ & $88$ & $83$ & $89$ & $91$\\ 
$\gamma^2\;\;\;$ $0.2$ & $93$ & $91$ & $89$ & $83$ & $84$ & $87$\\ 
$0.3$ & $92$ & $92$ & $91$ & $87$ & $83$ & $83$\\ 
$0.4$ & $92$ & $94$ & $92$ & $88$ & $82$ & $81$\\ 
 \hline 
 \end{tabular}
 
 \vspace{2em} 
 
\begin{tabular}{r|rrrrrr}
\hline\hline
 &\multicolumn{6}{c}{$\rho$} \\ 
 $\alpha = 0.1$ & $0$ & $0.1$ & $0.2$ & $0.3$ & $0.4$ & $0.5$ \\ 
 \hline$0.1$ & $86$ & $85$ & $82$ & $80$ & $82$ & $85$\\ 
$\gamma^2\;\;\;$ $0.2$ & $86$ & $84$ & $82$ & $78$ & $76$ & $83$\\ 
$0.3$ & $88$ & $88$ & $84$ & $78$ & $74$ & $80$\\ 
$0.4$ & $89$ & $89$ & $83$ & $78$ & $74$ & $76$\\ 
 \hline 
 \end{tabular}
 
 \vspace{2em} 
 
\begin{tabular}{r|rrrrrr}
\hline\hline
 &\multicolumn{6}{c}{$\rho$} \\ 
 $\alpha = 0.2$ & $0$ & $0.1$ & $0.2$ & $0.3$ & $0.4$ & $0.5$ \\ 
 \hline$0.1$ & $79$ & $74$ & $71$ & $72$ & $76$ & $77$\\ 
$\gamma^2\;\;\;$ $0.2$ & $79$ & $76$ & $72$ & $66$ & $69$ & $72$\\ 
$0.3$ & $78$ & $75$ & $73$ & $65$ & $62$ & $68$\\ 
$0.4$ & $79$ & $80$ & $71$ & $66$ & $60$ & $64$\\ 
 \hline 
 \end{tabular}
  \end{subtable}
  ~
  \begin{subtable}{0.48\textwidth}
    \caption{Relative Width}
    \begin{tabular}{r|rrrrrr}
\hline\hline
 &\multicolumn{6}{c}{$\rho$} \\ 
 $\alpha = 0.05$ & $0$ & $0.1$ & $0.2$ & $0.3$ & $0.4$ & $0.5$ \\ 
 \hline$0.1$ & $ 97$ & $ 97$ & $ 99$ & $101$ & $104$ & $105$\\ 
$\gamma^2\;\;\;$ $0.2$ & $ 97$ & $ 96$ & $ 97$ & $ 99$ & $101$ & $104$\\ 
$0.3$ & $ 96$ & $ 98$ & $ 95$ & $ 97$ & $100$ & $103$\\ 
$0.4$ & $ 96$ & $ 95$ & $ 95$ & $ 96$ & $ 98$ & $101$\\ 
 \hline 
 \end{tabular}
 
 \vspace{2em} 
 
\begin{tabular}{r|rrrrrr}
\hline\hline
 &\multicolumn{6}{c}{$\rho$} \\ 
 $\alpha = 0.1$ & $0$ & $0.1$ & $0.2$ & $0.3$ & $0.4$ & $0.5$ \\ 
 \hline$0.1$ & $ 95$ & $ 96$ & $ 99$ & $102$ & $105$ & $105$\\ 
$\gamma^2\;\;\;$ $0.2$ & $ 93$ & $ 96$ & $ 96$ & $ 98$ & $102$ & $106$\\ 
$0.3$ & $ 93$ & $ 93$ & $ 95$ & $ 97$ & $100$ & $103$\\ 
$0.4$ & $ 93$ & $ 93$ & $ 92$ & $ 96$ & $ 97$ & $103$\\ 
 \hline 
 \end{tabular}
 
 \vspace{2em} 
 
\begin{tabular}{r|rrrrrr}
\hline\hline
 &\multicolumn{6}{c}{$\rho$} \\ 
 $\alpha = 0.2$ & $0$ & $0.1$ & $0.2$ & $0.3$ & $0.4$ & $0.5$ \\ 
 \hline$0.1$ & $ 94$ & $ 95$ & $ 99$ & $102$ & $ 95$ & $108$\\ 
$\gamma^2\;\;\;$ $0.2$ & $ 93$ & $ 93$ & $ 96$ & $ 96$ & $105$ & $106$\\ 
$0.3$ & $ 93$ & $ 93$ & $ 95$ & $100$ & $102$ & $102$\\ 
$0.4$ & $ 92$ & $ 90$ & $ 91$ & $ 97$ & $100$ & $105$\\ 
 \hline 
 \end{tabular}
  \end{subtable}
  \label{tab:CISim50_1StepEqual_ChooseIVs}
  \caption{1-step Equal-tailed CI, Choosing IVs Example, $N=50$}
\end{table}

\begin{table}[h]
  \centering
  \begin{subtable}{0.48\textwidth}
    \caption{Coverage Probability}
    \begin{tabular}{r|rrrrrr}
\hline\hline
 &\multicolumn{6}{c}{$\rho$} \\ 
 $\alpha = 0.05$ & $0$ & $0.1$ & $0.2$ & $0.3$ & $0.4$ & $0.5$ \\ 
 \hline$0.1$ & $93$ & $91$ & $88$ & $89$ & $92$ & $95$\\ 
$\gamma^2\;\;\;$ $0.2$ & $93$ & $93$ & $88$ & $85$ & $89$ & $95$\\ 
$0.3$ & $93$ & $94$ & $87$ & $85$ & $87$ & $92$\\ 
$0.4$ & $95$ & $95$ & $90$ & $83$ & $82$ & $87$\\ 
 \hline 
 \end{tabular}
 
 \vspace{2em} 
 
\begin{tabular}{r|rrrrrr}
\hline\hline
 &\multicolumn{6}{c}{$\rho$} \\ 
 $\alpha = 0.1$ & $0$ & $0.1$ & $0.2$ & $0.3$ & $0.4$ & $0.5$ \\ 
 \hline$0.1$ & $89$ & $84$ & $84$ & $84$ & $90$ & $91$\\ 
$\gamma^2\;\;\;$ $0.2$ & $90$ & $89$ & $81$ & $80$ & $85$ & $90$\\ 
$0.3$ & $91$ & $89$ & $83$ & $76$ & $81$ & $88$\\ 
$0.4$ & $88$ & $90$ & $83$ & $75$ & $75$ & $84$\\ 
 \hline 
 \end{tabular}
 
 \vspace{2em} 
 
\begin{tabular}{r|rrrrrr}
\hline\hline
 &\multicolumn{6}{c}{$\rho$} \\ 
 $\alpha = 0.2$ & $0$ & $0.1$ & $0.2$ & $0.3$ & $0.4$ & $0.5$ \\ 
 \hline$0.1$ & $82$ & $75$ & $71$ & $75$ & $80$ & $81$\\ 
$\gamma^2\;\;\;$ $0.2$ & $85$ & $77$ & $70$ & $73$ & $80$ & $82$\\ 
$0.3$ & $81$ & $77$ & $67$ & $66$ & $76$ & $84$\\ 
$0.4$ & $84$ & $79$ & $68$ & $64$ & $68$ & $80$\\ 
 \hline 
 \end{tabular}
  \end{subtable}
  ~
  \begin{subtable}{0.48\textwidth}
    \caption{Relative Width}
    \begin{tabular}{r|rrrrrr}
\hline\hline
 &\multicolumn{6}{c}{$\rho$} \\ 
 $\alpha = 0.05$ & $0$ & $0.1$ & $0.2$ & $0.3$ & $0.4$ & $0.5$ \\ 
 \hline$0.1$ & $ 98$ & $ 99$ & $102$ & $106$ & $108$ & $106$\\ 
$\gamma^2\;\;\;$ $0.2$ & $ 97$ & $ 97$ & $101$ & $105$ & $108$ & $110$\\ 
$0.3$ & $ 96$ & $ 96$ & $ 98$ & $102$ & $107$ & $110$\\ 
$0.4$ & $ 95$ & $ 95$ & $ 97$ & $100$ & $105$ & $109$\\ 
 \hline 
 \end{tabular}
 
 \vspace{2em} 
 
\begin{tabular}{r|rrrrrr}
\hline\hline
 &\multicolumn{6}{c}{$\rho$} \\ 
 $\alpha = 0.1$ & $0$ & $0.1$ & $0.2$ & $0.3$ & $0.4$ & $0.5$ \\ 
 \hline$0.1$ & $ 97$ & $ 96$ & $104$ & $108$ & $108$ & $105$\\ 
$\gamma^2\;\;\;$ $0.2$ & $ 95$ & $ 97$ & $101$ & $106$ & $110$ & $108$\\ 
$0.3$ & $ 95$ & $ 95$ & $ 99$ & $104$ & $108$ & $112$\\ 
$0.4$ & $ 94$ & $ 94$ & $ 97$ & $103$ & $107$ & $112$\\ 
 \hline 
 \end{tabular}
 
 \vspace{2em} 
 
\begin{tabular}{r|rrrrrr}
\hline\hline
 &\multicolumn{6}{c}{$\rho$} \\ 
 $\alpha = 0.2$ & $0$ & $0.1$ & $0.2$ & $0.3$ & $0.4$ & $0.5$ \\ 
 \hline$0.1$ & $ 96$ & $100$ & $106$ & $110$ & $109$ & $105$\\ 
$\gamma^2\;\;\;$ $0.2$ & $ 93$ & $ 97$ & $104$ & $111$ & $114$ & $111$\\ 
$0.3$ & $ 94$ & $ 96$ & $101$ & $108$ & $113$ & $115$\\ 
$0.4$ & $ 91$ & $ 93$ & $ 98$ & $106$ & $112$ & $116$\\ 
 \hline 
 \end{tabular}
  \end{subtable}
  \label{tab:CISim100_1StepEqual_ChooseIVs}
  \caption{1-step Equal-tailed CI, Choosing IVs Example, $N=100$}
\end{table}

\begin{table}[h]
  \centering
  \begin{subtable}{0.48\textwidth}
    \caption{Coverage Probability}
    \begin{tabular}{r|rrrrrr}
\hline\hline
 &\multicolumn{6}{c}{$\rho$} \\ 
 $\alpha = 0.05$ & $0$ & $0.1$ & $0.2$ & $0.3$ & $0.4$ & $0.5$ \\ 
 \hline$0.1$ & $96$ & $92$ & $96$ & $96$ & $94$ & $94$\\ 
$\gamma^2\;\;\;$ $0.2$ & $96$ & $90$ & $93$ & $96$ & $94$ & $92$\\ 
$0.3$ & $96$ & $90$ & $88$ & $97$ & $95$ & $95$\\ 
$0.4$ & $95$ & $93$ & $84$ & $96$ & $97$ & $95$\\ 
 \hline 
 \end{tabular}
 
 \vspace{2em} 
 
\begin{tabular}{r|rrrrrr}
\hline\hline
 &\multicolumn{6}{c}{$\rho$} \\ 
 $\alpha = 0.1$ & $0$ & $0.1$ & $0.2$ & $0.3$ & $0.4$ & $0.5$ \\ 
 \hline$0.1$ & $91$ & $86$ & $93$ & $90$ & $90$ & $90$\\ 
$\gamma^2\;\;\;$ $0.2$ & $92$ & $82$ & $89$ & $92$ & $90$ & $88$\\ 
$0.3$ & $93$ & $84$ & $85$ & $93$ & $91$ & $89$\\ 
$0.4$ & $91$ & $86$ & $80$ & $93$ & $92$ & $89$\\ 
 \hline 
 \end{tabular}
 
 \vspace{2em} 
 
\begin{tabular}{r|rrrrrr}
\hline\hline
 &\multicolumn{6}{c}{$\rho$} \\ 
 $\alpha = 0.2$ & $0$ & $0.1$ & $0.2$ & $0.3$ & $0.4$ & $0.5$ \\ 
 \hline$0.1$ & $85$ & $74$ & $84$ & $81$ & $78$ & $80$\\ 
$\gamma^2\;\;\;$ $0.2$ & $84$ & $70$ & $84$ & $82$ & $80$ & $81$\\ 
$0.3$ & $85$ & $65$ & $80$ & $83$ & $80$ & $81$\\ 
$0.4$ & $84$ & $68$ & $74$ & $86$ & $81$ & $78$\\ 
 \hline 
 \end{tabular}
  \end{subtable}
  ~
  \begin{subtable}{0.48\textwidth}
    \caption{Relative Width}
    \begin{tabular}{r|rrrrrr}
\hline\hline
 &\multicolumn{6}{c}{$\rho$} \\ 
 $\alpha = 0.05$ & $0$ & $0.1$ & $0.2$ & $0.3$ & $0.4$ & $0.5$ \\ 
 \hline$0.1$ & $ 98$ & $105$ & $106$ & $100$ & $100$ & $100$\\ 
$\gamma^2\;\;\;$ $0.2$ & $ 98$ & $104$ & $111$ & $104$ & $100$ & $100$\\ 
$0.3$ & $ 97$ & $101$ & $111$ & $108$ & $101$ & $100$\\ 
$0.4$ & $ 97$ & $100$ & $110$ & $111$ & $102$ & $100$\\ 
 \hline 
 \end{tabular}
 
 \vspace{2em} 
 
\begin{tabular}{r|rrrrrr}
\hline\hline
 &\multicolumn{6}{c}{$\rho$} \\ 
 $\alpha = 0.1$ & $0$ & $0.1$ & $0.2$ & $0.3$ & $0.4$ & $0.5$ \\ 
 \hline$0.1$ & $ 98$ & $107$ & $105$ & $100$ & $100$ & $100$\\ 
$\gamma^2\;\;\;$ $0.2$ & $ 97$ & $105$ & $111$ & $103$ & $100$ & $100$\\ 
$0.3$ & $ 96$ & $103$ & $113$ & $106$ & $101$ & $100$\\ 
$0.4$ & $ 96$ & $101$ & $112$ & $110$ & $102$ & $100$\\ 
 \hline 
 \end{tabular}
 
 \vspace{2em} 
 
\begin{tabular}{r|rrrrrr}
\hline\hline
 &\multicolumn{6}{c}{$\rho$} \\ 
 $\alpha = 0.2$ & $0$ & $0.1$ & $0.2$ & $0.3$ & $0.4$ & $0.5$ \\ 
 \hline$0.1$ & $ 98$ & $108$ & $105$ & $100$ & $100$ & $100$\\ 
$\gamma^2\;\;\;$ $0.2$ & $ 96$ & $108$ & $111$ & $102$ & $100$ & $100$\\ 
$0.3$ & $ 95$ & $104$ & $114$ & $104$ & $100$ & $100$\\ 
$0.4$ & $ 93$ & $103$ & $116$ & $108$ & $101$ & $100$\\ 
 \hline 
 \end{tabular}
  \end{subtable}
  \label{tab:CISim500_1StepEqual_ChooseIVs}
  \caption{1-step Equal-tailed CI, Choosing IVs Example, $N=500$}
\end{table}






\begin{table}[h]
  \centering
  \begin{subtable}{0.48\textwidth}
    \caption{Coverage Probability}
    \begin{tabular}{r|rrrrrr}
\hline\hline
 &\multicolumn{6}{c}{$\rho$} \\ 
 $\alpha = 0.05$ & $0$ & $0.1$ & $0.2$ & $0.3$ & $0.4$ & $0.5$ \\ 
 \hline$0.1$ & $92$ & $90$ & $88$ & $84$ & $89$ & $91$\\ 
$\gamma^2\;\;\;$ $0.2$ & $93$ & $91$ & $89$ & $84$ & $84$ & $87$\\ 
$0.3$ & $92$ & $92$ & $91$ & $87$ & $83$ & $83$\\ 
$0.4$ & $92$ & $93$ & $92$ & $88$ & $83$ & $82$\\ 
 \hline 
 \end{tabular}
 
 \vspace{2em} 
 
\begin{tabular}{r|rrrrrr}
\hline\hline
 &\multicolumn{6}{c}{$\rho$} \\ 
 $\alpha = 0.1$ & $0$ & $0.1$ & $0.2$ & $0.3$ & $0.4$ & $0.5$ \\ 
 \hline$0.1$ & $86$ & $85$ & $82$ & $81$ & $83$ & $85$\\ 
$\gamma^2\;\;\;$ $0.2$ & $85$ & $84$ & $83$ & $79$ & $77$ & $83$\\ 
$0.3$ & $87$ & $88$ & $85$ & $79$ & $75$ & $81$\\ 
$0.4$ & $88$ & $89$ & $84$ & $79$ & $75$ & $78$\\ 
 \hline 
 \end{tabular}
 
 \vspace{2em} 
 
\begin{tabular}{r|rrrrrr}
\hline\hline
 &\multicolumn{6}{c}{$\rho$} \\ 
 $\alpha = 0.2$ & $0$ & $0.1$ & $0.2$ & $0.3$ & $0.4$ & $0.5$ \\ 
 \hline$0.1$ & $78$ & $72$ & $72$ & $72$ & $76$ & $77$\\ 
$\gamma^2\;\;\;$ $0.2$ & $77$ & $75$ & $72$ & $68$ & $70$ & $72$\\ 
$0.3$ & $76$ & $74$ & $72$ & $68$ & $65$ & $69$\\ 
$0.4$ & $76$ & $76$ & $69$ & $67$ & $63$ & $67$\\ 
 \hline 
 \end{tabular}
  \end{subtable}
  ~
  \begin{subtable}{0.48\textwidth}
    \caption{Relative Width}
    \begin{tabular}{r|rrrrrr}
\hline\hline
 &\multicolumn{6}{c}{$\rho$} \\ 
 $\alpha = 0.05$ & $0$ & $0.1$ & $0.2$ & $0.3$ & $0.4$ & $0.5$ \\ 
 \hline$0.1$ & $ 97$ & $ 97$ & $ 99$ & $101$ & $104$ & $104$\\ 
$\gamma^2\;\;\;$ $0.2$ & $ 96$ & $ 95$ & $ 96$ & $ 98$ & $100$ & $101$\\ 
$0.3$ & $ 94$ & $ 97$ & $ 95$ & $ 96$ & $ 99$ & $102$\\ 
$0.4$ & $ 94$ & $ 94$ & $ 93$ & $ 94$ & $ 96$ & $ 99$\\ 
 \hline 
 \end{tabular}
 
 \vspace{2em} 
 
\begin{tabular}{r|rrrrrr}
\hline\hline
 &\multicolumn{6}{c}{$\rho$} \\ 
 $\alpha = 0.1$ & $0$ & $0.1$ & $0.2$ & $0.3$ & $0.4$ & $0.5$ \\ 
 \hline$0.1$ & $ 95$ & $ 95$ & $ 98$ & $102$ & $105$ & $103$\\ 
$\gamma^2\;\;\;$ $0.2$ & $ 92$ & $ 85$ & $ 95$ & $ 97$ & $101$ & $104$\\ 
$0.3$ & $ 92$ & $ 91$ & $ 93$ & $ 95$ & $ 99$ & $102$\\ 
$0.4$ & $ 91$ & $ 91$ & $ 91$ & $ 95$ & $ 95$ & $101$\\ 
 \hline 
 \end{tabular}
 
 \vspace{2em} 
 
\begin{tabular}{r|rrrrrr}
\hline\hline
 &\multicolumn{6}{c}{$\rho$} \\ 
 $\alpha = 0.2$ & $0$ & $0.1$ & $0.2$ & $0.3$ & $0.4$ & $0.5$ \\ 
 \hline$0.1$ & $ 92$ & $ 90$ & $ 97$ & $100$ & $ 94$ & $106$\\ 
$\gamma^2\;\;\;$ $0.2$ & $ 90$ & $ 87$ & $ 93$ & $ 87$ & $102$ & $102$\\ 
$0.3$ & $ 87$ & $ 90$ & $ 91$ & $ 96$ & $ 97$ & $ 94$\\ 
$0.4$ & $ 87$ & $ 84$ & $ 84$ & $ 92$ & $ 96$ & $102$\\ 
 \hline 
 \end{tabular}
  \end{subtable}
  \label{tab:CISim50_1StepShort_ChooseIVs}
  \caption{1-step Shortest CI, Choosing IVs Example, $N=50$}
\end{table}

\begin{table}[h]
  \centering
  \begin{subtable}{0.48\textwidth}
    \caption{Coverage Probability}
    \begin{tabular}{r|rrrrrr}
\hline\hline
 &\multicolumn{6}{c}{$\rho$} \\ 
 $\alpha = 0.05$ & $0$ & $0.1$ & $0.2$ & $0.3$ & $0.4$ & $0.5$ \\ 
 \hline$0.1$ & $93$ & $90$ & $89$ & $89$ & $92$ & $94$\\ 
$\gamma^2\;\;\;$ $0.2$ & $93$ & $92$ & $89$ & $85$ & $90$ & $95$\\ 
$0.3$ & $92$ & $93$ & $88$ & $86$ & $88$ & $92$\\ 
$0.4$ & $94$ & $94$ & $90$ & $83$ & $83$ & $87$\\ 
 \hline 
 \end{tabular}
 
 \vspace{2em} 
 
\begin{tabular}{r|rrrrrr}
\hline\hline
 &\multicolumn{6}{c}{$\rho$} \\ 
 $\alpha = 0.1$ & $0$ & $0.1$ & $0.2$ & $0.3$ & $0.4$ & $0.5$ \\ 
 \hline$0.1$ & $89$ & $84$ & $84$ & $85$ & $90$ & $91$\\ 
$\gamma^2\;\;\;$ $0.2$ & $90$ & $89$ & $82$ & $81$ & $86$ & $90$\\ 
$0.3$ & $90$ & $89$ & $83$ & $78$ & $82$ & $88$\\ 
$0.4$ & $87$ & $89$ & $82$ & $78$ & $77$ & $85$\\ 
 \hline 
 \end{tabular}
 
 \vspace{2em} 
 
\begin{tabular}{r|rrrrrr}
\hline\hline
 &\multicolumn{6}{c}{$\rho$} \\ 
 $\alpha = 0.2$ & $0$ & $0.1$ & $0.2$ & $0.3$ & $0.4$ & $0.5$ \\ 
 \hline$0.1$ & $81$ & $76$ & $73$ & $76$ & $80$ & $80$\\ 
$\gamma^2\;\;\;$ $0.2$ & $83$ & $77$ & $74$ & $75$ & $80$ & $80$\\ 
$0.3$ & $80$ & $74$ & $69$ & $69$ & $78$ & $82$\\ 
$0.4$ & $81$ & $76$ & $68$ & $67$ & $70$ & $80$\\ 
 \hline 
 \end{tabular}
  \end{subtable}
  ~
  \begin{subtable}{0.48\textwidth}
    \caption{Relative Width}
    \begin{tabular}{r|rrrrrr}
\hline\hline
 &\multicolumn{6}{c}{$\rho$} \\ 
 $\alpha = 0.05$ & $0$ & $0.1$ & $0.2$ & $0.3$ & $0.4$ & $0.5$ \\ 
 \hline$0.1$ & $ 97$ & $ 99$ & $102$ & $106$ & $107$ & $106$\\ 
$\gamma^2\;\;\;$ $0.2$ & $ 97$ & $ 97$ & $101$ & $104$ & $108$ & $109$\\ 
$0.3$ & $ 96$ & $ 95$ & $ 97$ & $101$ & $106$ & $110$\\ 
$0.4$ & $ 94$ & $ 94$ & $ 96$ & $ 99$ & $104$ & $108$\\ 
 \hline 
 \end{tabular}
 
 \vspace{2em} 
 
\begin{tabular}{r|rrrrrr}
\hline\hline
 &\multicolumn{6}{c}{$\rho$} \\ 
 $\alpha = 0.1$ & $0$ & $0.1$ & $0.2$ & $0.3$ & $0.4$ & $0.5$ \\ 
 \hline$0.1$ & $ 97$ & $ 96$ & $104$ & $108$ & $108$ & $105$\\ 
$\gamma^2\;\;\;$ $0.2$ & $ 95$ & $ 96$ & $100$ & $105$ & $109$ & $107$\\ 
$0.3$ & $ 94$ & $ 94$ & $ 98$ & $103$ & $107$ & $111$\\ 
$0.4$ & $ 93$ & $ 92$ & $ 96$ & $102$ & $106$ & $110$\\ 
 \hline 
 \end{tabular}
 
 \vspace{2em} 
 
\begin{tabular}{r|rrrrrr}
\hline\hline
 &\multicolumn{6}{c}{$\rho$} \\ 
 $\alpha = 0.2$ & $0$ & $0.1$ & $0.2$ & $0.3$ & $0.4$ & $0.5$ \\ 
 \hline$0.1$ & $ 95$ & $ 98$ & $104$ & $110$ & $109$ & $104$\\ 
$\gamma^2\;\;\;$ $0.2$ & $ 90$ & $ 95$ & $102$ & $110$ & $113$ & $109$\\ 
$0.3$ & $ 90$ & $ 93$ & $ 99$ & $105$ & $111$ & $113$\\ 
$0.4$ & $ 87$ & $ 89$ & $ 95$ & $103$ & $109$ & $113$\\ 
 \hline 
 \end{tabular}
  \end{subtable}
  \label{tab:CISim100_1StepShort_ChooseIVs}
  \caption{1-step Shortest CI, Choosing IVs Example, $N=100$}
\end{table}

\begin{table}[h]
  \centering
  \begin{subtable}{0.48\textwidth}
    \caption{Coverage Probability}
    \begin{tabular}{r|rrrrrr}
\hline\hline
 &\multicolumn{6}{c}{$\rho$} \\ 
 $\alpha = 0.05$ & $0$ & $0.1$ & $0.2$ & $0.3$ & $0.4$ & $0.5$ \\ 
 \hline$0.1$ & $92$ & $90$ & $88$ & $84$ & $89$ & $91$\\ 
$\gamma^2\;\;\;$ $0.2$ & $93$ & $91$ & $89$ & $84$ & $84$ & $87$\\ 
$0.3$ & $92$ & $92$ & $91$ & $87$ & $83$ & $83$\\ 
$0.4$ & $92$ & $93$ & $92$ & $88$ & $83$ & $82$\\ 
 \hline 
 \end{tabular}
 
 \vspace{2em} 
 
\begin{tabular}{r|rrrrrr}
\hline\hline
 &\multicolumn{6}{c}{$\rho$} \\ 
 $\alpha = 0.1$ & $0$ & $0.1$ & $0.2$ & $0.3$ & $0.4$ & $0.5$ \\ 
 \hline$0.1$ & $86$ & $85$ & $82$ & $81$ & $83$ & $85$\\ 
$\gamma^2\;\;\;$ $0.2$ & $85$ & $84$ & $83$ & $79$ & $77$ & $83$\\ 
$0.3$ & $87$ & $88$ & $85$ & $79$ & $75$ & $81$\\ 
$0.4$ & $88$ & $89$ & $84$ & $79$ & $75$ & $78$\\ 
 \hline 
 \end{tabular}
 
 \vspace{2em} 
 
\begin{tabular}{r|rrrrrr}
\hline\hline
 &\multicolumn{6}{c}{$\rho$} \\ 
 $\alpha = 0.2$ & $0$ & $0.1$ & $0.2$ & $0.3$ & $0.4$ & $0.5$ \\ 
 \hline$0.1$ & $78$ & $72$ & $72$ & $72$ & $76$ & $77$\\ 
$\gamma^2\;\;\;$ $0.2$ & $77$ & $75$ & $72$ & $68$ & $70$ & $72$\\ 
$0.3$ & $76$ & $74$ & $72$ & $68$ & $65$ & $69$\\ 
$0.4$ & $76$ & $76$ & $69$ & $67$ & $63$ & $67$\\ 
 \hline 
 \end{tabular}
  \end{subtable}
  ~
  \begin{subtable}{0.48\textwidth}
    \caption{Relative Width}
    \begin{tabular}{r|rrrrrr}
\hline\hline
 &\multicolumn{6}{c}{$\rho$} \\ 
 $\alpha = 0.05$ & $0$ & $0.1$ & $0.2$ & $0.3$ & $0.4$ & $0.5$ \\ 
 \hline$0.1$ & $ 98$ & $105$ & $106$ & $100$ & $100$ & $100$\\ 
$\gamma^2\;\;\;$ $0.2$ & $ 97$ & $103$ & $110$ & $103$ & $100$ & $100$\\ 
$0.3$ & $ 97$ & $101$ & $110$ & $107$ & $101$ & $100$\\ 
$0.4$ & $ 96$ & $ 99$ & $109$ & $110$ & $102$ & $100$\\ 
 \hline 
 \end{tabular}
 
 \vspace{2em} 
 
\begin{tabular}{r|rrrrrr}
\hline\hline
 &\multicolumn{6}{c}{$\rho$} \\ 
 $\alpha = 0.1$ & $0$ & $0.1$ & $0.2$ & $0.3$ & $0.4$ & $0.5$ \\ 
 \hline$0.1$ & $ 98$ & $107$ & $105$ & $100$ & $100$ & $100$\\ 
$\gamma^2\;\;\;$ $0.2$ & $ 96$ & $104$ & $110$ & $102$ & $100$ & $100$\\ 
$0.3$ & $ 95$ & $103$ & $112$ & $105$ & $100$ & $100$\\ 
$0.4$ & $ 95$ & $100$ & $111$ & $108$ & $101$ & $100$\\ 
 \hline 
 \end{tabular}
 
 \vspace{2em} 
 
\begin{tabular}{r|rrrrrr}
\hline\hline
 &\multicolumn{6}{c}{$\rho$} \\ 
 $\alpha = 0.2$ & $0$ & $0.1$ & $0.2$ & $0.3$ & $0.4$ & $0.5$ \\ 
 \hline$0.1$ & $ 97$ & $107$ & $104$ & $100$ & $100$ & $100$\\ 
$\gamma^2\;\;\;$ $0.2$ & $ 94$ & $106$ & $110$ & $101$ & $100$ & $100$\\ 
$0.3$ & $ 92$ & $102$ & $112$ & $104$ & $100$ & $100$\\ 
$0.4$ & $ 90$ & $100$ & $113$ & $107$ & $101$ & $100$\\ 
 \hline 
 \end{tabular}
  \end{subtable}
  \label{tab:CISim500_1StepShort_ChooseIVs}
  \caption{1-step Shortest CI, Choosing IVs Example, $N=500$}
\end{table}









\begin{table}[h]
  \centering
  \begin{subtable}{0.48\textwidth}
    \caption{Coverage Probability}
    \begin{tabular}{r|rrrrrr}
\hline\hline
 &\multicolumn{6}{c}{$\rho$} \\ 
 $\alpha = 0.05$ & $0$ & $0.1$ & $0.2$ & $0.3$ & $0.4$ & $0.5$ \\ 
 \hline$0.1$ & $95$ & $96$ & $94$ & $94$ & $94$ & $96$\\ 
$\gamma^2\;\;\;$ $0.2$ & $95$ & $96$ & $95$ & $94$ & $94$ & $95$\\ 
$0.3$ & $95$ & $96$ & $96$ & $95$ & $93$ & $94$\\ 
$0.4$ & $94$ & $96$ & $96$ & $94$ & $94$ & $94$\\ 
 \hline 
 \end{tabular}
 
 \vspace{2em} 
 
\begin{tabular}{r|rrrrrr}
\hline\hline
 &\multicolumn{6}{c}{$\rho$} \\ 
 $\alpha = 0.1$ & $0$ & $0.1$ & $0.2$ & $0.3$ & $0.4$ & $0.5$ \\ 
 \hline$0.1$ & $93$ & $93$ & $92$ & $92$ & $91$ & $92$\\ 
$\gamma^2\;\;\;$ $0.2$ & $91$ & $91$ & $93$ & $90$ & $89$ & $92$\\ 
$0.3$ & $92$ & $95$ & $95$ & $92$ & $90$ & $92$\\ 
$0.4$ & $93$ & $95$ & $94$ & $93$ & $92$ & $91$\\ 
 \hline 
 \end{tabular}
 
 \vspace{2em} 
 
\begin{tabular}{r|rrrrrr}
\hline\hline
 &\multicolumn{6}{c}{$\rho$} \\ 
 $\alpha = 0.2$ & $0$ & $0.1$ & $0.2$ & $0.3$ & $0.4$ & $0.5$ \\ 
 \hline$0.1$ & $89$ & $88$ & $86$ & $86$ & $87$ & $89$\\ 
$\gamma^2\;\;\;$ $0.2$ & $88$ & $89$ & $89$ & $86$ & $86$ & $86$\\ 
$0.3$ & $87$ & $88$ & $90$ & $88$ & $86$ & $86$\\ 
$0.4$ & $86$ & $90$ & $90$ & $89$ & $86$ & $84$\\ 
 \hline 
 \end{tabular}
  \end{subtable}
  ~
  \begin{subtable}{0.48\textwidth}
    \caption{Relative Width}
    \begin{tabular}{r|rrrrrr}
\hline\hline
 &\multicolumn{6}{c}{$\rho$} \\ 
 $\alpha = 0.05$ & $0$ & $0.1$ & $0.2$ & $0.3$ & $0.4$ & $0.5$ \\ 
 \hline$0.1$ & $121$ & $122$ & $123$ & $124$ & $125$ & $125$\\ 
$\gamma^2\;\;\;$ $0.2$ & $121$ & $122$ & $124$ & $124$ & $125$ & $126$\\ 
$0.3$ & $121$ & $118$ & $122$ & $124$ & $126$ & $128$\\ 
$0.4$ & $121$ & $120$ & $121$ & $123$ & $125$ & $127$\\ 
 \hline 
 \end{tabular}
 
 \vspace{2em} 
 
\begin{tabular}{r|rrrrrr}
\hline\hline
 &\multicolumn{6}{c}{$\rho$} \\ 
 $\alpha = 0.1$ & $0$ & $0.1$ & $0.2$ & $0.3$ & $0.4$ & $0.5$ \\ 
 \hline$0.1$ & $129$ & $130$ & $131$ & $131$ & $132$ & $131$\\ 
$\gamma^2\;\;\;$ $0.2$ & $129$ & $127$ & $132$ & $132$ & $134$ & $135$\\ 
$0.3$ & $129$ & $130$ & $132$ & $132$ & $134$ & $135$\\ 
$0.4$ & $129$ & $130$ & $130$ & $133$ & $134$ & $137$\\ 
 \hline 
 \end{tabular}
 
 \vspace{2em} 
 
\begin{tabular}{r|rrrrrr}
\hline\hline
 &\multicolumn{6}{c}{$\rho$} \\ 
 $\alpha = 0.2$ & $0$ & $0.1$ & $0.2$ & $0.3$ & $0.4$ & $0.5$ \\ 
 \hline$0.1$ & $143$ & $143$ & $144$ & $145$ & $144$ & $145$\\ 
$\gamma^2\;\;\;$ $0.2$ & $145$ & $145$ & $147$ & $146$ & $148$ & $150$\\ 
$0.3$ & $146$ & $147$ & $148$ & $150$ & $150$ & $149$\\ 
$0.4$ & $146$ & $146$ & $148$ & $150$ & $152$ & $154$\\ 
 \hline 
 \end{tabular}
  \end{subtable}
  \label{tab:CISim50_2StepEqual_ChooseIVs}
  \caption{2-step CI, $\alpha_1 = \alpha_2 = \alpha/2$, Choosing IVs Example, $N=50$}
\end{table}

\begin{table}[h]
  \centering
  \begin{subtable}{0.48\textwidth}
    \caption{Coverage Probability}
    \begin{tabular}{r|rrrrrr}
\hline\hline
 &\multicolumn{6}{c}{$\rho$} \\ 
 $\alpha = 0.05$ & $0$ & $0.1$ & $0.2$ & $0.3$ & $0.4$ & $0.5$ \\ 
 \hline$0.1$ & $96$ & $96$ & $94$ & $95$ & $96$ & $97$\\ 
$\gamma^2\;\;\;$ $0.2$ & $96$ & $96$ & $96$ & $95$ & $95$ & $98$\\ 
$0.3$ & $96$ & $97$ & $96$ & $95$ & $96$ & $96$\\ 
$0.4$ & $96$ & $97$ & $96$ & $95$ & $95$ & $94$\\ 
 \hline 
 \end{tabular}
 
 \vspace{2em} 
 
\begin{tabular}{r|rrrrrr}
\hline\hline
 &\multicolumn{6}{c}{$\rho$} \\ 
 $\alpha = 0.1$ & $0$ & $0.1$ & $0.2$ & $0.3$ & $0.4$ & $0.5$ \\ 
 \hline$0.1$ & $94$ & $94$ & $92$ & $92$ & $95$ & $96$\\ 
$\gamma^2\;\;\;$ $0.2$ & $94$ & $96$ & $94$ & $92$ & $95$ & $95$\\ 
$0.3$ & $95$ & $96$ & $95$ & $92$ & $93$ & $94$\\ 
$0.4$ & $92$ & $95$ & $95$ & $93$ & $90$ & $92$\\ 
 \hline 
 \end{tabular}
 
 \vspace{2em} 
 
\begin{tabular}{r|rrrrrr}
\hline\hline
 &\multicolumn{6}{c}{$\rho$} \\ 
 $\alpha = 0.2$ & $0$ & $0.1$ & $0.2$ & $0.3$ & $0.4$ & $0.5$ \\ 
 \hline$0.1$ & $90$ & $91$ & $87$ & $87$ & $91$ & $92$\\ 
$\gamma^2\;\;\;$ $0.2$ & $93$ & $92$ & $89$ & $88$ & $91$ & $92$\\ 
$0.3$ & $90$ & $91$ & $90$ & $89$ & $88$ & $92$\\ 
$0.4$ & $90$ & $93$ & $91$ & $89$ & $88$ & $90$\\ 
 \hline 
 \end{tabular}
  \end{subtable}
  ~
  \begin{subtable}{0.48\textwidth}
    \caption{Relative Width}
    \begin{tabular}{r|rrrrrr}
\hline\hline
 &\multicolumn{6}{c}{$\rho$} \\ 
 $\alpha = 0.05$ & $0$ & $0.1$ & $0.2$ & $0.3$ & $0.4$ & $0.5$ \\ 
 \hline$0.1$ & $123$ & $124$ & $126$ & $126$ & $127$ & $127$\\ 
$\gamma^2\;\;\;$ $0.2$ & $122$ & $124$ & $127$ & $129$ & $130$ & $130$\\ 
$0.3$ & $121$ & $123$ & $126$ & $129$ & $131$ & $132$\\ 
$0.4$ & $121$ & $122$ & $125$ & $128$ & $132$ & $133$\\ 
 \hline 
 \end{tabular}
 
 \vspace{2em} 
 
\begin{tabular}{r|rrrrrr}
\hline\hline
 &\multicolumn{6}{c}{$\rho$} \\ 
 $\alpha = 0.1$ & $0$ & $0.1$ & $0.2$ & $0.3$ & $0.4$ & $0.5$ \\ 
 \hline$0.1$ & $131$ & $131$ & $133$ & $133$ & $133$ & $133$\\ 
$\gamma^2\;\;\;$ $0.2$ & $131$ & $133$ & $135$ & $136$ & $138$ & $137$\\ 
$0.3$ & $131$ & $132$ & $136$ & $137$ & $139$ & $140$\\ 
$0.4$ & $130$ & $131$ & $135$ & $138$ & $140$ & $142$\\ 
 \hline 
 \end{tabular}
 
 \vspace{2em} 
 
\begin{tabular}{r|rrrrrr}
\hline\hline
 &\multicolumn{6}{c}{$\rho$} \\ 
 $\alpha = 0.2$ & $0$ & $0.1$ & $0.2$ & $0.3$ & $0.4$ & $0.5$ \\ 
 \hline$0.1$ & $144$ & $146$ & $146$ & $146$ & $147$ & $145$\\ 
$\gamma^2\;\;\;$ $0.2$ & $146$ & $148$ & $150$ & $152$ & $152$ & $152$\\ 
$0.3$ & $148$ & $149$ & $151$ & $153$ & $154$ & $156$\\ 
$0.4$ & $146$ & $148$ & $151$ & $155$ & $156$ & $156$\\ 
 \hline 
 \end{tabular}
  \end{subtable}
  \label{tab:CISim100_2StepEqual_ChooseIVs}
  \caption{2-step CI, $\alpha_1 = \alpha_2 = \alpha/2$, Choosing IVs Example, $N=100$}
\end{table}

\begin{table}[h]
  \centering
  \begin{subtable}{0.48\textwidth}
    \caption{Coverage Probability}
    \begin{tabular}{r|rrrrrr}
\hline\hline
 &\multicolumn{6}{c}{$\rho$} \\ 
 $\alpha = 0.05$ & $0$ & $0.1$ & $0.2$ & $0.3$ & $0.4$ & $0.5$ \\ 
 \hline$0.1$ & $95$ & $96$ & $94$ & $94$ & $94$ & $96$\\ 
$\gamma^2\;\;\;$ $0.2$ & $95$ & $96$ & $95$ & $94$ & $94$ & $95$\\ 
$0.3$ & $95$ & $96$ & $96$ & $95$ & $93$ & $94$\\ 
$0.4$ & $94$ & $96$ & $96$ & $94$ & $94$ & $94$\\ 
 \hline 
 \end{tabular}
 
 \vspace{2em} 
 
\begin{tabular}{r|rrrrrr}
\hline\hline
 &\multicolumn{6}{c}{$\rho$} \\ 
 $\alpha = 0.1$ & $0$ & $0.1$ & $0.2$ & $0.3$ & $0.4$ & $0.5$ \\ 
 \hline$0.1$ & $93$ & $93$ & $92$ & $92$ & $91$ & $92$\\ 
$\gamma^2\;\;\;$ $0.2$ & $91$ & $91$ & $93$ & $90$ & $89$ & $92$\\ 
$0.3$ & $92$ & $95$ & $95$ & $92$ & $90$ & $92$\\ 
$0.4$ & $93$ & $95$ & $94$ & $93$ & $92$ & $91$\\ 
 \hline 
 \end{tabular}
 
 \vspace{2em} 
 
\begin{tabular}{r|rrrrrr}
\hline\hline
 &\multicolumn{6}{c}{$\rho$} \\ 
 $\alpha = 0.2$ & $0$ & $0.1$ & $0.2$ & $0.3$ & $0.4$ & $0.5$ \\ 
 \hline$0.1$ & $89$ & $88$ & $86$ & $86$ & $87$ & $89$\\ 
$\gamma^2\;\;\;$ $0.2$ & $88$ & $89$ & $89$ & $86$ & $86$ & $86$\\ 
$0.3$ & $87$ & $88$ & $90$ & $88$ & $86$ & $86$\\ 
$0.4$ & $86$ & $90$ & $90$ & $89$ & $86$ & $84$\\ 
 \hline 
 \end{tabular}
  \end{subtable}
  ~
  \begin{subtable}{0.48\textwidth}
    \caption{Relative Width}
    \begin{tabular}{r|rrrrrr}
\hline\hline
 &\multicolumn{6}{c}{$\rho$} \\ 
 $\alpha = 0.05$ & $0$ & $0.1$ & $0.2$ & $0.3$ & $0.4$ & $0.5$ \\ 
 \hline$0.1$ & $124$ & $126$ & $126$ & $125$ & $118$ & $114$\\ 
$\gamma^2\;\;\;$ $0.2$ & $124$ & $128$ & $130$ & $130$ & $128$ & $118$\\ 
$0.3$ & $123$ & $128$ & $132$ & $133$ & $133$ & $126$\\ 
$0.4$ & $122$ & $127$ & $133$ & $134$ & $134$ & $132$\\ 
 \hline 
 \end{tabular}
 
 \vspace{2em} 
 
\begin{tabular}{r|rrrrrr}
\hline\hline
 &\multicolumn{6}{c}{$\rho$} \\ 
 $\alpha = 0.1$ & $0$ & $0.1$ & $0.2$ & $0.3$ & $0.4$ & $0.5$ \\ 
 \hline$0.1$ & $132$ & $133$ & $133$ & $130$ & $120$ & $119$\\ 
$\gamma^2\;\;\;$ $0.2$ & $132$ & $136$ & $138$ & $138$ & $130$ & $120$\\ 
$0.3$ & $131$ & $137$ & $140$ & $141$ & $138$ & $126$\\ 
$0.4$ & $132$ & $136$ & $142$ & $143$ & $142$ & $134$\\ 
 \hline 
 \end{tabular}
 
 \vspace{2em} 
 
\begin{tabular}{r|rrrrrr}
\hline\hline
 &\multicolumn{6}{c}{$\rho$} \\ 
 $\alpha = 0.2$ & $0$ & $0.1$ & $0.2$ & $0.3$ & $0.4$ & $0.5$ \\ 
 \hline$0.1$ & $145$ & $146$ & $145$ & $136$ & $129$ & $128$\\ 
$\gamma^2\;\;\;$ $0.2$ & $147$ & $150$ & $151$ & $149$ & $134$ & $129$\\ 
$0.3$ & $147$ & $151$ & $155$ & $154$ & $145$ & $131$\\ 
$0.4$ & $148$ & $152$ & $157$ & $158$ & $153$ & $137$\\ 
 \hline 
 \end{tabular}
  \end{subtable}
  \label{tab:CISim500_2StepEqual_ChooseIVs}
  \caption{2-step CI, $\alpha_1 = \alpha_2 = \alpha/2$, Choosing IVs Example, $N=500$}
\end{table}






\begin{table}[h]
  \centering
  \begin{subtable}{0.48\textwidth}
    \caption{Coverage Probability}
    \begin{tabular}{r|rrrrrr}
\hline\hline
 &\multicolumn{6}{c}{$\rho$} \\ 
 $\alpha = 0.05$ & $0$ & $0.1$ & $0.2$ & $0.3$ & $0.4$ & $0.5$ \\ 
 \hline$0.1$ & $94$ & $94$ & $93$ & $91$ & $93$ & $94$\\ 
$\gamma^2\;\;\;$ $0.2$ & $94$ & $94$ & $94$ & $92$ & $92$ & $93$\\ 
$0.3$ & $94$ & $94$ & $96$ & $94$ & $91$ & $93$\\ 
$0.4$ & $93$ & $96$ & $95$ & $94$ & $92$ & $93$\\ 
 \hline 
 \end{tabular}
 
 \vspace{2em} 
 
\begin{tabular}{r|rrrrrr}
\hline\hline
 &\multicolumn{6}{c}{$\rho$} \\ 
 $\alpha = 0.1$ & $0$ & $0.1$ & $0.2$ & $0.3$ & $0.4$ & $0.5$ \\ 
 \hline$0.1$ & $91$ & $91$ & $89$ & $88$ & $88$ & $90$\\ 
$\gamma^2\;\;\;$ $0.2$ & $89$ & $89$ & $91$ & $87$ & $86$ & $90$\\ 
$0.3$ & $91$ & $93$ & $93$ & $90$ & $88$ & $90$\\ 
$0.4$ & $91$ & $93$ & $91$ & $92$ & $89$ & $89$\\ 
 \hline 
 \end{tabular}
 
 \vspace{2em} 
 
\begin{tabular}{r|rrrrrr}
\hline\hline
 &\multicolumn{6}{c}{$\rho$} \\ 
 $\alpha = 0.2$ & $0$ & $0.1$ & $0.2$ & $0.3$ & $0.4$ & $0.5$ \\ 
 \hline$0.1$ & $85$ & $84$ & $81$ & $82$ & $83$ & $85$\\ 
$\gamma^2\;\;\;$ $0.2$ & $85$ & $87$ & $86$ & $82$ & $82$ & $81$\\ 
$0.3$ & $84$ & $85$ & $88$ & $84$ & $81$ & $82$\\ 
$0.4$ & $84$ & $88$ & $87$ & $86$ & $81$ & $80$\\ 
 \hline 
 \end{tabular}
  \end{subtable}
  ~
  \begin{subtable}{0.48\textwidth}
    \caption{Relative Width}
    \begin{tabular}{r|rrrrrr}
\hline\hline
 &\multicolumn{6}{c}{$\rho$} \\ 
 $\alpha = 0.05$ & $0$ & $0.1$ & $0.2$ & $0.3$ & $0.4$ & $0.5$ \\ 
 \hline$0.1$ & $114$ & $115$ & $116$ & $116$ & $117$ & $117$\\ 
$\gamma^2\;\;\;$ $0.2$ & $114$ & $115$ & $116$ & $117$ & $118$ & $118$\\ 
$0.3$ & $114$ & $110$ & $115$ & $117$ & $119$ & $120$\\ 
$0.4$ & $114$ & $114$ & $115$ & $117$ & $119$ & $119$\\ 
 \hline 
 \end{tabular}
 
 \vspace{2em} 
 
\begin{tabular}{r|rrrrrr}
\hline\hline
 &\multicolumn{6}{c}{$\rho$} \\ 
 $\alpha = 0.1$ & $0$ & $0.1$ & $0.2$ & $0.3$ & $0.4$ & $0.5$ \\ 
 \hline$0.1$ & $119$ & $120$ & $121$ & $121$ & $122$ & $121$\\ 
$\gamma^2\;\;\;$ $0.2$ & $120$ & $118$ & $123$ & $122$ & $124$ & $125$\\ 
$0.3$ & $120$ & $121$ & $122$ & $123$ & $125$ & $125$\\ 
$0.4$ & $121$ & $121$ & $121$ & $125$ & $124$ & $126$\\ 
 \hline 
 \end{tabular}
 
 \vspace{2em} 
 
\begin{tabular}{r|rrrrrr}
\hline\hline
 &\multicolumn{6}{c}{$\rho$} \\ 
 $\alpha = 0.2$ & $0$ & $0.1$ & $0.2$ & $0.3$ & $0.4$ & $0.5$ \\ 
 \hline$0.1$ & $130$ & $129$ & $130$ & $130$ & $128$ & $129$\\ 
$\gamma^2\;\;\;$ $0.2$ & $133$ & $132$ & $133$ & $131$ & $133$ & $134$\\ 
$0.3$ & $133$ & $135$ & $135$ & $137$ & $136$ & $134$\\ 
$0.4$ & $135$ & $135$ & $136$ & $138$ & $139$ & $140$\\ 
 \hline 
 \end{tabular}
  \end{subtable}
  \label{tab:CISim50_2StepWideTau_ChooseIVs}
  \caption{2-step CI, $\alpha_1 = \alpha/4,  \alpha_2 = 3\alpha/4$, Choosing IVs Example, $N=50$}
\end{table}


\begin{table}[h]
  \centering
  \begin{subtable}{0.48\textwidth}
    \caption{Coverage Probability}
    \begin{tabular}{r|rrrrrr}
\hline\hline
 &\multicolumn{6}{c}{$\rho$} \\ 
 $\alpha = 0.05$ & $0$ & $0.1$ & $0.2$ & $0.3$ & $0.4$ & $0.5$ \\ 
 \hline$0.1$ & $96$ & $95$ & $94$ & $94$ & $95$ & $96$\\ 
$\gamma^2\;\;\;$ $0.2$ & $95$ & $95$ & $94$ & $93$ & $93$ & $97$\\ 
$0.3$ & $94$ & $97$ & $94$ & $94$ & $94$ & $96$\\ 
$0.4$ & $95$ & $96$ & $95$ & $93$ & $94$ & $94$\\ 
 \hline 
 \end{tabular}
 
 \vspace{2em} 
 
\begin{tabular}{r|rrrrrr}
\hline\hline
 &\multicolumn{6}{c}{$\rho$} \\ 
 $\alpha = 0.1$ & $0$ & $0.1$ & $0.2$ & $0.3$ & $0.4$ & $0.5$ \\ 
 \hline$0.1$ & $92$ & $90$ & $90$ & $89$ & $93$ & $95$\\ 
$\gamma^2\;\;\;$ $0.2$ & $92$ & $94$ & $91$ & $90$ & $92$ & $93$\\ 
$0.3$ & $93$ & $93$ & $93$ & $90$ & $90$ & $93$\\ 
$0.4$ & $90$ & $94$ & $93$ & $91$ & $87$ & $90$\\ 
 \hline 
 \end{tabular}
 
 \vspace{2em} 
 
\begin{tabular}{r|rrrrrr}
\hline\hline
 &\multicolumn{6}{c}{$\rho$} \\ 
 $\alpha = 0.2$ & $0$ & $0.1$ & $0.2$ & $0.3$ & $0.4$ & $0.5$ \\ 
 \hline$0.1$ & $88$ & $87$ & $83$ & $82$ & $88$ & $90$\\ 
$\gamma^2\;\;\;$ $0.2$ & $91$ & $88$ & $86$ & $85$ & $87$ & $89$\\ 
$0.3$ & $87$ & $88$ & $87$ & $84$ & $86$ & $89$\\ 
$0.4$ & $88$ & $91$ & $88$ & $84$ & $82$ & $88$\\ 
 \hline 
 \end{tabular}
  \end{subtable}
  ~
  \begin{subtable}{0.48\textwidth}
    \caption{Relative Width}
    \begin{tabular}{r|rrrrrr}
\hline\hline
 &\multicolumn{6}{c}{$\rho$} \\ 
 $\alpha = 0.05$ & $0$ & $0.1$ & $0.2$ & $0.3$ & $0.4$ & $0.5$ \\ 
 \hline$0.1$ & $116$ & $117$ & $118$ & $118$ & $118$ & $118$\\ 
$\gamma^2\;\;\;$ $0.2$ & $116$ & $117$ & $120$ & $121$ & $121$ & $122$\\ 
$0.3$ & $115$ & $116$ & $119$ & $121$ & $123$ & $124$\\ 
$0.4$ & $114$ & $115$ & $119$ & $121$ & $124$ & $125$\\ 
 \hline 
 \end{tabular}
 
 \vspace{2em} 
 
\begin{tabular}{r|rrrrrr}
\hline\hline
 &\multicolumn{6}{c}{$\rho$} \\ 
 $\alpha = 0.1$ & $0$ & $0.1$ & $0.2$ & $0.3$ & $0.4$ & $0.5$ \\ 
 \hline$0.1$ & $121$ & $121$ & $122$ & $122$ & $122$ & $122$\\ 
$\gamma^2\;\;\;$ $0.2$ & $122$ & $123$ & $125$ & $126$ & $127$ & $125$\\ 
$0.3$ & $122$ & $123$ & $126$ & $127$ & $128$ & $129$\\ 
$0.4$ & $122$ & $123$ & $126$ & $128$ & $130$ & $131$\\ 
 \hline 
 \end{tabular}
 
 \vspace{2em} 
 
\begin{tabular}{r|rrrrrr}
\hline\hline
 &\multicolumn{6}{c}{$\rho$} \\ 
 $\alpha = 0.2$ & $0$ & $0.1$ & $0.2$ & $0.3$ & $0.4$ & $0.5$ \\ 
 \hline$0.1$ & $131$ & $131$ & $130$ & $130$ & $131$ & $129$\\ 
$\gamma^2\;\;\;$ $0.2$ & $134$ & $134$ & $135$ & $136$ & $136$ & $135$\\ 
$0.3$ & $135$ & $136$ & $137$ & $138$ & $139$ & $139$\\ 
$0.4$ & $135$ & $137$ & $139$ & $140$ & $140$ & $141$\\ 
 \hline 
 \end{tabular}
  \end{subtable}
  \label{tab:CISim100_2StepWideTau_ChooseIVs}
  \caption{2-step CI, $\alpha_1 = \alpha/4,  \alpha_2 = 3\alpha/4$, Choosing IVs Example, $N=100$}
\end{table}


\begin{table}[h]
  \centering
  \begin{subtable}{0.48\textwidth}
    \caption{Coverage Probability}
    \begin{tabular}{r|rrrrrr}
\hline\hline
 &\multicolumn{6}{c}{$\rho$} \\ 
 $\alpha = 0.05$ & $0$ & $0.1$ & $0.2$ & $0.3$ & $0.4$ & $0.5$ \\ 
 \hline$0.1$ & $94$ & $94$ & $93$ & $91$ & $93$ & $94$\\ 
$\gamma^2\;\;\;$ $0.2$ & $94$ & $94$ & $94$ & $92$ & $92$ & $93$\\ 
$0.3$ & $94$ & $94$ & $96$ & $94$ & $91$ & $93$\\ 
$0.4$ & $93$ & $96$ & $95$ & $94$ & $92$ & $93$\\ 
 \hline 
 \end{tabular}
 
 \vspace{2em} 
 
\begin{tabular}{r|rrrrrr}
\hline\hline
 &\multicolumn{6}{c}{$\rho$} \\ 
 $\alpha = 0.1$ & $0$ & $0.1$ & $0.2$ & $0.3$ & $0.4$ & $0.5$ \\ 
 \hline$0.1$ & $91$ & $91$ & $89$ & $88$ & $88$ & $90$\\ 
$\gamma^2\;\;\;$ $0.2$ & $89$ & $89$ & $91$ & $87$ & $86$ & $90$\\ 
$0.3$ & $91$ & $93$ & $93$ & $90$ & $88$ & $90$\\ 
$0.4$ & $91$ & $93$ & $91$ & $92$ & $89$ & $89$\\ 
 \hline 
 \end{tabular}
 
 \vspace{2em} 
 
\begin{tabular}{r|rrrrrr}
\hline\hline
 &\multicolumn{6}{c}{$\rho$} \\ 
 $\alpha = 0.2$ & $0$ & $0.1$ & $0.2$ & $0.3$ & $0.4$ & $0.5$ \\ 
 \hline$0.1$ & $85$ & $84$ & $81$ & $82$ & $83$ & $85$\\ 
$\gamma^2\;\;\;$ $0.2$ & $85$ & $87$ & $86$ & $82$ & $82$ & $81$\\ 
$0.3$ & $84$ & $85$ & $88$ & $84$ & $81$ & $82$\\ 
$0.4$ & $84$ & $88$ & $87$ & $86$ & $81$ & $80$\\ 
 \hline 
 \end{tabular}
  \end{subtable}
  ~
  \begin{subtable}{0.48\textwidth}
    \caption{Relative Width}
    \begin{tabular}{r|rrrrrr}
\hline\hline
 &\multicolumn{6}{c}{$\rho$} \\ 
 $\alpha = 0.05$ & $0$ & $0.1$ & $0.2$ & $0.3$ & $0.4$ & $0.5$ \\ 
 \hline$0.1$ & $117$ & $118$ & $118$ & $117$ & $110$ & $106$\\ 
$\gamma^2\;\;\;$ $0.2$ & $117$ & $120$ & $122$ & $122$ & $120$ & $110$\\ 
$0.3$ & $116$ & $121$ & $124$ & $124$ & $124$ & $119$\\ 
$0.4$ & $115$ & $120$ & $125$ & $126$ & $126$ & $124$\\ 
 \hline 
 \end{tabular}
 
 \vspace{2em} 
 
\begin{tabular}{r|rrrrrr}
\hline\hline
 &\multicolumn{6}{c}{$\rho$} \\ 
 $\alpha = 0.1$ & $0$ & $0.1$ & $0.2$ & $0.3$ & $0.4$ & $0.5$ \\ 
 \hline$0.1$ & $122$ & $122$ & $122$ & $119$ & $110$ & $108$\\ 
$\gamma^2\;\;\;$ $0.2$ & $123$ & $125$ & $126$ & $126$ & $120$ & $110$\\ 
$0.3$ & $123$ & $127$ & $129$ & $129$ & $128$ & $116$\\ 
$0.4$ & $123$ & $127$ & $131$ & $131$ & $131$ & $124$\\ 
 \hline 
 \end{tabular}
 
 \vspace{2em} 
 
\begin{tabular}{r|rrrrrr}
\hline\hline
 &\multicolumn{6}{c}{$\rho$} \\ 
 $\alpha = 0.2$ & $0$ & $0.1$ & $0.2$ & $0.3$ & $0.4$ & $0.5$ \\ 
 \hline$0.1$ & $131$ & $129$ & $129$ & $121$ & $113$ & $112$\\ 
$\gamma^2\;\;\;$ $0.2$ & $134$ & $135$ & $135$ & $133$ & $119$ & $113$\\ 
$0.3$ & $135$ & $137$ & $138$ & $138$ & $130$ & $115$\\ 
$0.4$ & $135$ & $138$ & $140$ & $141$ & $137$ & $122$\\ 
 \hline 
 \end{tabular}
  \end{subtable}
  \label{tab:CISim500_2StepWideTau_ChooseIVs}
  \caption{2-step CI, $\alpha_1 = \alpha/4,  \alpha_2 = 3\alpha/4$, Choosing IVs Example, $N=500$}
\end{table}






\begin{table}[h]
  \centering
  \begin{subtable}{0.48\textwidth}
    \caption{Coverage Probability}
    \begin{tabular}{r|rrrrrr}
\hline\hline
 &\multicolumn{6}{c}{$\rho$} \\ 
 $\alpha = 0.05$ & $0$ & $0.1$ & $0.2$ & $0.3$ & $0.4$ & $0.5$ \\ 
 \hline$0.1$ & $97$ & $97$ & $96$ & $96$ & $96$ & $97$\\ 
$\gamma^2\;\;\;$ $0.2$ & $97$ & $97$ & $96$ & $96$ & $96$ & $96$\\ 
$0.3$ & $96$ & $97$ & $97$ & $98$ & $95$ & $96$\\ 
$0.4$ & $97$ & $97$ & $97$ & $96$ & $96$ & $95$\\ 
 \hline 
 \end{tabular}
 
 \vspace{2em} 
 
\begin{tabular}{r|rrrrrr}
\hline\hline
 &\multicolumn{6}{c}{$\rho$} \\ 
 $\alpha = 0.1$ & $0$ & $0.1$ & $0.2$ & $0.3$ & $0.4$ & $0.5$ \\ 
 \hline$0.1$ & $95$ & $96$ & $94$ & $95$ & $94$ & $94$\\ 
$\gamma^2\;\;\;$ $0.2$ & $95$ & $94$ & $96$ & $93$ & $92$ & $95$\\ 
$0.3$ & $95$ & $96$ & $97$ & $95$ & $94$ & $94$\\ 
$0.4$ & $95$ & $96$ & $95$ & $96$ & $95$ & $94$\\ 
 \hline 
 \end{tabular}
 
 \vspace{2em} 
 
\begin{tabular}{r|rrrrrr}
\hline\hline
 &\multicolumn{6}{c}{$\rho$} \\ 
 $\alpha = 0.2$ & $0$ & $0.1$ & $0.2$ & $0.3$ & $0.4$ & $0.5$ \\ 
 \hline$0.1$ & $93$ & $92$ & $91$ & $91$ & $92$ & $92$\\ 
$\gamma^2\;\;\;$ $0.2$ & $93$ & $93$ & $94$ & $91$ & $91$ & $90$\\ 
$0.3$ & $91$ & $92$ & $93$ & $93$ & $90$ & $90$\\ 
$0.4$ & $91$ & $93$ & $93$ & $93$ & $90$ & $89$\\ 
 \hline 
 \end{tabular}
  \end{subtable}
  ~
  \begin{subtable}{0.48\textwidth}
    \caption{Relative Width}
    \begin{tabular}{r|rrrrrr}
\hline\hline
 &\multicolumn{6}{c}{$\rho$} \\ 
 $\alpha = 0.05$ & $0$ & $0.1$ & $0.2$ & $0.3$ & $0.4$ & $0.5$ \\ 
 \hline$0.1$ & $134$ & $134$ & $135$ & $136$ & $138$ & $138$\\ 
$\gamma^2\;\;\;$ $0.2$ & $133$ & $133$ & $135$ & $136$ & $138$ & $139$\\ 
$0.3$ & $132$ & $130$ & $133$ & $135$ & $138$ & $139$\\ 
$0.4$ & $132$ & $132$ & $132$ & $134$ & $137$ & $139$\\ 
 \hline 
 \end{tabular}
 
 \vspace{2em} 
 
\begin{tabular}{r|rrrrrr}
\hline\hline
 &\multicolumn{6}{c}{$\rho$} \\ 
 $\alpha = 0.1$ & $0$ & $0.1$ & $0.2$ & $0.3$ & $0.4$ & $0.5$ \\ 
 \hline$0.1$ & $144$ & $145$ & $146$ & $148$ & $148$ & $148$\\ 
$\gamma^2\;\;\;$ $0.2$ & $144$ & $142$ & $146$ & $148$ & $150$ & $151$\\ 
$0.3$ & $144$ & $144$ & $146$ & $147$ & $150$ & $151$\\ 
$0.4$ & $143$ & $143$ & $144$ & $147$ & $148$ & $152$\\ 
 \hline 
 \end{tabular}
 
 \vspace{2em} 
 
\begin{tabular}{r|rrrrrr}
\hline\hline
 &\multicolumn{6}{c}{$\rho$} \\ 
 $\alpha = 0.2$ & $0$ & $0.1$ & $0.2$ & $0.3$ & $0.4$ & $0.5$ \\ 
 \hline$0.1$ & $163$ & $164$ & $166$ & $167$ & $167$ & $170$\\ 
$\gamma^2\;\;\;$ $0.2$ & $165$ & $165$ & $167$ & $167$ & $171$ & $172$\\ 
$0.3$ & $165$ & $165$ & $167$ & $170$ & $171$ & $171$\\ 
$0.4$ & $165$ & $164$ & $165$ & $169$ & $171$ & $174$\\ 
 \hline 
 \end{tabular}
  \end{subtable}
  \label{tab:CISim50_2StepNarrowTau_ChooseIVs}
  \caption{2-step CI, $\alpha_1 = 3\alpha/4,  \alpha_2 = \alpha/4$, Choosing IVs Example, $N=50$}
\end{table}


\begin{table}[h]
  \centering
  \begin{subtable}{0.48\textwidth}
    \caption{Coverage Probability}
    \begin{tabular}{r|rrrrrr}
\hline\hline
 &\multicolumn{6}{c}{$\rho$} \\ 
 $\alpha = 0.05$ & $0$ & $0.1$ & $0.2$ & $0.3$ & $0.4$ & $0.5$ \\ 
 \hline$0.1$ & $98$ & $97$ & $96$ & $97$ & $97$ & $98$\\ 
$\gamma^2\;\;\;$ $0.2$ & $98$ & $98$ & $98$ & $97$ & $96$ & $99$\\ 
$0.3$ & $98$ & $98$ & $97$ & $96$ & $97$ & $98$\\ 
$0.4$ & $97$ & $98$ & $97$ & $97$ & $96$ & $95$\\ 
 \hline 
 \end{tabular}
 
 \vspace{2em} 
 
\begin{tabular}{r|rrrrrr}
\hline\hline
 &\multicolumn{6}{c}{$\rho$} \\ 
 $\alpha = 0.1$ & $0$ & $0.1$ & $0.2$ & $0.3$ & $0.4$ & $0.5$ \\ 
 \hline$0.1$ & $96$ & $97$ & $95$ & $96$ & $97$ & $98$\\ 
$\gamma^2\;\;\;$ $0.2$ & $97$ & $97$ & $96$ & $95$ & $96$ & $97$\\ 
$0.3$ & $96$ & $97$ & $98$ & $95$ & $95$ & $96$\\ 
$0.4$ & $94$ & $97$ & $97$ & $95$ & $92$ & $93$\\ 
 \hline 
 \end{tabular}
 
 \vspace{2em} 
 
\begin{tabular}{r|rrrrrr}
\hline\hline
 &\multicolumn{6}{c}{$\rho$} \\ 
 $\alpha = 0.2$ & $0$ & $0.1$ & $0.2$ & $0.3$ & $0.4$ & $0.5$ \\ 
 \hline$0.1$ & $93$ & $94$ & $92$ & $92$ & $94$ & $94$\\ 
$\gamma^2\;\;\;$ $0.2$ & $95$ & $94$ & $93$ & $91$ & $94$ & $95$\\ 
$0.3$ & $93$ & $94$ & $94$ & $92$ & $91$ & $94$\\ 
$0.4$ & $94$ & $95$ & $95$ & $93$ & $91$ & $92$\\ 
 \hline 
 \end{tabular}
  \end{subtable}
  ~
  \begin{subtable}{0.48\textwidth}
    \caption{Relative Width}
    \begin{tabular}{r|rrrrrr}
\hline\hline
 &\multicolumn{6}{c}{$\rho$} \\ 
 $\alpha = 0.05$ & $0$ & $0.1$ & $0.2$ & $0.3$ & $0.4$ & $0.5$ \\ 
 \hline$0.1$ & $135$ & $137$ & $138$ & $139$ & $140$ & $140$\\ 
$\gamma^2\;\;\;$ $0.2$ & $134$ & $135$ & $139$ & $141$ & $143$ & $144$\\ 
$0.3$ & $133$ & $134$ & $137$ & $141$ & $144$ & $145$\\ 
$0.4$ & $132$ & $132$ & $135$ & $140$ & $143$ & $145$\\ 
 \hline 
 \end{tabular}
 
 \vspace{2em} 
 
\begin{tabular}{r|rrrrrr}
\hline\hline
 &\multicolumn{6}{c}{$\rho$} \\ 
 $\alpha = 0.1$ & $0$ & $0.1$ & $0.2$ & $0.3$ & $0.4$ & $0.5$ \\ 
 \hline$0.1$ & $146$ & $146$ & $150$ & $150$ & $151$ & $151$\\ 
$\gamma^2\;\;\;$ $0.2$ & $145$ & $147$ & $150$ & $152$ & $155$ & $154$\\ 
$0.3$ & $144$ & $146$ & $150$ & $153$ & $155$ & $157$\\ 
$0.4$ & $144$ & $144$ & $148$ & $153$ & $156$ & $158$\\ 
 \hline 
 \end{tabular}
 
 \vspace{2em} 
 
\begin{tabular}{r|rrrrrr}
\hline\hline
 &\multicolumn{6}{c}{$\rho$} \\ 
 $\alpha = 0.2$ & $0$ & $0.1$ & $0.2$ & $0.3$ & $0.4$ & $0.5$ \\ 
 \hline$0.1$ & $166$ & $168$ & $170$ & $171$ & $172$ & $170$\\ 
$\gamma^2\;\;\;$ $0.2$ & $166$ & $168$ & $172$ & $175$ & $177$ & $177$\\ 
$0.3$ & $166$ & $168$ & $172$ & $175$ & $178$ & $180$\\ 
$0.4$ & $165$ & $166$ & $170$ & $176$ & $179$ & $181$\\ 
 \hline 
 \end{tabular}
  \end{subtable}
  \label{tab:CISim100_2StepNarrowTau_ChooseIVs}
  \caption{2-step CI, $\alpha_1 = 3\alpha/4,  \alpha_2 = \alpha/4$, Choosing IVs Example, $N=100$}
\end{table}


\begin{table}[h]
  \centering
  \begin{subtable}{0.48\textwidth}
    \caption{Coverage Probability}
    \begin{tabular}{r|rrrrrr}
\hline\hline
 &\multicolumn{6}{c}{$\rho$} \\ 
 $\alpha = 0.05$ & $0$ & $0.1$ & $0.2$ & $0.3$ & $0.4$ & $0.5$ \\ 
 \hline$0.1$ & $97$ & $97$ & $96$ & $96$ & $96$ & $97$\\ 
$\gamma^2\;\;\;$ $0.2$ & $97$ & $97$ & $96$ & $96$ & $96$ & $96$\\ 
$0.3$ & $96$ & $97$ & $97$ & $98$ & $95$ & $96$\\ 
$0.4$ & $97$ & $97$ & $97$ & $96$ & $96$ & $95$\\ 
 \hline 
 \end{tabular}
 
 \vspace{2em} 
 
\begin{tabular}{r|rrrrrr}
\hline\hline
 &\multicolumn{6}{c}{$\rho$} \\ 
 $\alpha = 0.1$ & $0$ & $0.1$ & $0.2$ & $0.3$ & $0.4$ & $0.5$ \\ 
 \hline$0.1$ & $95$ & $96$ & $94$ & $95$ & $94$ & $94$\\ 
$\gamma^2\;\;\;$ $0.2$ & $95$ & $94$ & $96$ & $93$ & $92$ & $95$\\ 
$0.3$ & $95$ & $96$ & $97$ & $95$ & $94$ & $94$\\ 
$0.4$ & $95$ & $96$ & $95$ & $96$ & $95$ & $94$\\ 
 \hline 
 \end{tabular}
 
 \vspace{2em} 
 
\begin{tabular}{r|rrrrrr}
\hline\hline
 &\multicolumn{6}{c}{$\rho$} \\ 
 $\alpha = 0.2$ & $0$ & $0.1$ & $0.2$ & $0.3$ & $0.4$ & $0.5$ \\ 
 \hline$0.1$ & $93$ & $92$ & $91$ & $91$ & $92$ & $92$\\ 
$\gamma^2\;\;\;$ $0.2$ & $93$ & $93$ & $94$ & $91$ & $91$ & $90$\\ 
$0.3$ & $91$ & $92$ & $93$ & $93$ & $90$ & $90$\\ 
$0.4$ & $91$ & $93$ & $93$ & $93$ & $90$ & $89$\\ 
 \hline 
 \end{tabular}
  \end{subtable}
  ~
  \begin{subtable}{0.48\textwidth}
    \caption{Relative Width}
    \begin{tabular}{r|rrrrrr}
\hline\hline
 &\multicolumn{6}{c}{$\rho$} \\ 
 $\alpha = 0.05$ & $0$ & $0.1$ & $0.2$ & $0.3$ & $0.4$ & $0.5$ \\ 
 \hline$0.1$ & $136$ & $139$ & $140$ & $139$ & $131$ & $128$\\ 
$\gamma^2\;\;\;$ $0.2$ & $135$ & $141$ & $144$ & $144$ & $141$ & $131$\\ 
$0.3$ & $134$ & $140$ & $146$ & $146$ & $146$ & $140$\\ 
$0.4$ & $133$ & $138$ & $146$ & $148$ & $148$ & $146$\\ 
 \hline 
 \end{tabular}
 
 \vspace{2em} 
 
\begin{tabular}{r|rrrrrr}
\hline\hline
 &\multicolumn{6}{c}{$\rho$} \\ 
 $\alpha = 0.1$ & $0$ & $0.1$ & $0.2$ & $0.3$ & $0.4$ & $0.5$ \\ 
 \hline$0.1$ & $147$ & $150$ & $151$ & $147$ & $138$ & $136$\\ 
$\gamma^2\;\;\;$ $0.2$ & $146$ & $152$ & $155$ & $155$ & $148$ & $138$\\ 
$0.3$ & $145$ & $152$ & $158$ & $158$ & $156$ & $145$\\ 
$0.4$ & $146$ & $151$ & $159$ & $160$ & $160$ & $153$\\ 
 \hline 
 \end{tabular}
 
 \vspace{2em} 
 
\begin{tabular}{r|rrrrrr}
\hline\hline
 &\multicolumn{6}{c}{$\rho$} \\ 
 $\alpha = 0.2$ & $0$ & $0.1$ & $0.2$ & $0.3$ & $0.4$ & $0.5$ \\ 
 \hline$0.1$ & $167$ & $170$ & $170$ & $162$ & $153$ & $153$\\ 
$\gamma^2\;\;\;$ $0.2$ & $167$ & $173$ & $177$ & $174$ & $160$ & $153$\\ 
$0.3$ & $167$ & $173$ & $180$ & $180$ & $172$ & $156$\\ 
$0.4$ & $167$ & $173$ & $181$ & $183$ & $179$ & $163$\\ 
 \hline 
 \end{tabular}
  \end{subtable}
  \label{tab:CISim500_2StepNarrowTau_ChooseIVs}
  \caption{2-step CI, $\alpha_1 = 3\alpha/4,  \alpha_2 = \alpha/4$, Choosing IVs Example, $N=500$}
\end{table}

\end{document}
