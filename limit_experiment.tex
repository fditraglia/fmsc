%!TEX root = main.tex
\subsection{A Special Case of Post-FMSC Inference}
\label{sec:limitexperiment}
Corollary \ref{cor:momentavg} is sufficiently general to cover a wide range of moment selection and averaging procedures, but this same generality makes the confidence interval procedure given in Algorithm \ref{alg:conf} somewhat less than intuitive.
In this section I specialize the results on moment average estimators from above to the two examples of FMSC selection that appear in the simulation studies described below in Section \ref{sec:simulations}: OLS versus TSLS and choosing instrumental variables.
The structure of these examples allows us to bypass Algorithm \ref{alg:conf} and characterize the asymptotic properties of various proposals for post-FMSC inference \emph{directly}, clearly illustrating the relevant trade-offs between coverage and width.
Because this section presents asymptotic results, I treat any consistently estimable quantity that appears in a limit distribution as known.
Unfortunately the bias parameter $\tau$ remains unknown \emph{even in the limit}.
This is the main complication of post-FMSC inference and the focus of this section.

First recall the OLS versus TSLS example from Section \ref{sec:OLSvsIVExample}.
The joint limit distribution for this case is as follows
\begin{equation*}
  \left[ 
  \begin{array}{c}
    \sqrt{n} \left( \widehat{\beta}_{OLS} - \beta \right)\\
    \sqrt{n} \left( \widetilde{\beta}_{TSLS} - \beta \right)\\
    \widehat{\tau}
\end{array}
\right] \overset{d}{\rightarrow} N\left( \left[
\begin{array}{c}
  \tau/\sigma_x^2 \\ 0 \\ \tau
\end{array}
\right], \sigma_{\epsilon}^2 
\left[
\begin{array}{ccc}
  1/\sigma_x^2 & 1/\sigma_x^2 & 0\\
  1/\sigma_x^2 & 1/\gamma^2 & -\sigma_v^2/\gamma^2\\
  0 & -\sigma_v^2/\gamma^2 & \sigma_x^2 \sigma_v^2/\gamma^2
\end{array}
\right]\right).
\end{equation*}
Second, consider a slightly simplified version of the choosing instrumental variables example from Section \ref{sec:chooseIVexample}, namely
\begin{eqnarray*}
  y_{ni} &=& \beta x_{ni} + \epsilon_{ni}\\
  x_{ni} &=& \gamma w_{ni} + \mathbf{z}_{ni}' \boldsymbol{\pi} + v_{ni}
\end{eqnarray*}
where $x$ is the endogenous regressor of interest, $\mathbf{z}$ is a vector of exogenous instruments, and $w$ is a single potentially endogenous instrument.
Without loss of generality I assume that $w$ and $\mathbf{z}$ are uncorrelated and that all random variables are mean zero.
For simplicity, I further assume that the errors satisfy the same kind of asymptotic homoskedasticity condition used in the OLS versus TSLS example so that TSLS is the efficient GMM estimator.
Let the ``Full'' estimator denote the TSLS estimator using $w$ and $\mathbf{z}$ and the ``Valid'' estimator denote the TSLS estimator using only $\mathbf{z}$.
Then we have,
\begin{equation*}
  \left[ 
  \begin{array}{c}
    \sqrt{n} \left( \widehat{\beta}_{Full} - \beta \right)\\
    \sqrt{n} \left( \widetilde{\beta}_{Valid} - \beta \right)\\
    \widehat{\tau}
\end{array}
\right] \overset{d}{\rightarrow} N\left( \left[
\begin{array}{c}
  \tau\gamma/q^2_{F} \\ 0 \\ \tau
\end{array}
\right], \sigma_{\epsilon}^2 
\left[
\begin{array}{ccc}
  1/q_{F}^2 & 1/q_{F}^2 & 0\\
  1/q_{F}^2 & 1/q_{V}^2 & -\gamma\sigma_w^2/q^2_{V}\\ 
  0 & -\gamma\sigma_w^2/q^2_{V} & \sigma_w^2 q^2_{F}/q^2_{V}
\end{array}
\right]\right)
\end{equation*}
where $q^2_{F} = \gamma^2 \sigma_w^2 + q^2_{V}$, $q^2_{V} = \boldsymbol{\pi}'\Sigma_{zz}\boldsymbol{\pi}$, $\Sigma_{zz}$ is the covariance matrix of the valid instruments $\mathbf{z}$, and $\sigma_w^2$ is the variance of the ``suspect'' instrument $w$.
After some algebraic manipulations we see that both of these examples share the same structure, namely
\begin{equation}
  \left[ 
  \begin{array}{c}
    \sqrt{n} \left( \widehat{\beta} - \beta \right)\\
    \sqrt{n} \left( \widetilde{\beta} - \beta \right)\\
    \widehat{\tau}
\end{array}
\right] \overset{d}{\rightarrow} 
\left[
\begin{array}{c}
  U \\ V \\ T
\end{array}
\right] \sim
N\left( \left[
\begin{array}{c}
  c\tau\\ 0 \\ \tau
\end{array}
\right], 
\left[
\begin{array}{ccc}
  \eta^2 & \eta^2 & 0\\
  \eta^2 & \eta^2 + c^2 \sigma^2 & -c\sigma^2\\ 
  0 & -c\sigma^2 & \sigma^2 
\end{array}
\right]\right)
\label{eq:LimitExperiment}
\end{equation}
where $\widehat{\beta}$ denotes the low variance but possibly biased estimator, and $\widetilde{\beta}$ denotes the higher variance but unbiased estimator.
For any example with a limit distribution that takes this form, simple algebra shows that FMSC selection amounts to choosing $\widehat{\beta}$ whenever $|\widehat{\tau}|<\sqrt{2}\sigma$, and choosing $\widetilde{\beta}$ otherwise, in other words
\begin{eqnarray*}
  \sqrt{n}(\widehat{\beta}_{FMSC} - \beta) = \mathbf{1}\left\{ |\widehat{\tau}|<\sigma \sqrt{2} \right\} \sqrt{n}(\widehat{\beta} - \beta) +  \mathbf{1}\left\{ |\widehat{\tau}|\geq\sigma \sqrt{2} \right\}\sqrt{n}(\widetilde{\beta} - \beta)
\end{eqnarray*}
and so by the Continuous Mapping Theorem,
\begin{equation}
  \sqrt{n}(\widehat{\beta}_{FMSC} - \beta) \overset{d}{\rightarrow}  \mathbf{1}\left\{ |T|<\sigma \sqrt{2} \right\} U +  \mathbf{1}\left\{ |T|\geq\sigma \sqrt{2} \right\} V.
  \label{eq:FMSCLimitExperiment}
\end{equation}

To better understand the implications of Equation \ref{eq:FMSCLimitExperiment}, it is helpful to re-express the limit distribution from Equation \ref{eq:LimitExperiment} in terms of the marginal distribution of $T$ and the conditional distribution of $U$ and $V$ given $T$. 
We have $T \sim N(\tau, \sigma^2)$ and by direct calculation
\begin{equation}
  \left.\left[
  \begin{array}{c}
   U \\ V 
  \end{array}
\right]\right| (T = t) \sim N\left(
\left[
\begin{array}{c}
  c \tau \\ c\tau - ct
\end{array}
\right], \eta^2
\left[
\begin{array}{cc}
  1 & 1 \\ 1 & 1
\end{array}
\right]
\right)
\label{eq:LimitExperimentConditional}
\end{equation}
which is a \emph{singular distribution}.
Crucially $U$ is independent of $T$, but conditional on $T$ the random variables $U$ and $V$ are perfectly correlated with the same variance.
Given $T$, the only difference between $U$ and $V$ is that the mean of $V$ is shifted by a distance that depends on the realization $t$ of $T$.
Thus, the conditional distribution of $V$ shows a \emph{random bias}: on average $V$ has mean zero because the mean of $T$ is $\tau$ but any particular realization $t$ of $T$ will not in general equal $\tau$.
Using the form of the conditional distributions we can express the distribution of $(U,V,T)'$ from Equation \ref{eq:LimitExperiment} in a more transparent form as
\begin{eqnarray*}
  T &=& \sigma Z_1 + \tau\\
  U &=& \eta Z_2 + c\tau\\
  V &=& \eta Z_2 - c\sigma Z_1
\end{eqnarray*}
where $Z_1, Z_2$ are iid standard normal random variables.
Combined with Equation \ref{eq:FMSCLimitExperiment}, this representation allows us to tabulate the asymptotic distribution, $F_{FMSC}$, of the post-FMSC estimator \emph{directly} for any example that takes the form of Equation \ref{eq:LimitExperiment} rather than approximating it by simulation as in Algorithm \ref{alg:conf}.
We have:
\begin{eqnarray}
  F_{FMSC}(x) &=& G(x) + H_1(x) + H_2(x) \\
  \label{eq:FFMSC}
  G(x) &=& \Phi\left( \frac{x - c\tau}{\eta} \right)\left[ \Phi( \sqrt{2} - \tau/\sigma) -  \Phi( -\sqrt{2} - \tau/\sigma )\right]\\
  \label{eq:GFMSC}
  H_1(x) &=& \frac{1}{\sigma}\int_{-\infty}^{-\sigma\sqrt{2} - \tau} \Phi\left( \frac{x + ct}{\eta}\right)\varphi(t/\sigma)\; dt\\
  \label{eq:H1FMSC}
  H_2(x) &=& \frac{1}{\sigma}\int^{+\infty}_{\sigma\sqrt{2} - \tau} \Phi\left( \frac{x + ct}{\eta}\right)\varphi(t/\sigma)\; dt
  \label{eq:H2FMSC}
\end{eqnarray}
where $\Phi$ is the CDF and $\varphi$ the pdf of a standard normal random variable.
Note that the limit distribution of the post-FMSC distribution depends on $\tau$ in addition to the consistently estimable quantities $\sigma, \eta, c$ although I supress this dependence to simplify the notation.
While these expressions lack a closed form, there are various fast and extremely accurate numerical routines for evaluating integrals of the form taken by $H_1$ and $H_2$.
Armed with a numerical routine to evaluate $F_{FMSC}$ it is straightforward to evaluate the corresponding quantile function $Q_{FMSC}$ using a root-finder by analytically bounding $G$, $H_1$ and $H_2$.
I provide numerical routines to evaluate both $F_{FMSC}$ and $Q_{FMSC}$ along with various related quantities in my R package \texttt{fmscr}.\footnote{This package is open-source and freely available from \url{https://github.com/fditraglia/fmscr}.}

The ability to compute $F_{FMSC}$ and $Q_{FMSC}$ allows us to answer a number of important questions about post-FMSC inference for any example of the form given in Equation \ref{eq:LimitExperiment}.
To begin, suppose that we were to carry out FMSC selection and then construct a $(1 - \alpha) \times 100\%$ confidence interval \emph{conditional} in the selected estimator, completely ignoring the effects of the moment selection step.
What would be the resulting asymptotic coverage probability and width of such a ``na\"{i}ve'' confidence interval procedure?
Using calculations similar to those used above in the expression for $F_{FMSC}$, we find that the coverage probability of this na\"{i}ve interval is given by
\begin{eqnarray*}
  \mbox{CP}_{\mbox{Na\"{i}ve}}(\alpha) &=& G(u_\alpha) - G(-u_\alpha) +  H_1(\ell_\alpha) - H_2(-\ell_\alpha) +  H_2(\ell_\alpha) - H_2(-\ell_\alpha) \\
  u_\alpha &=& z_{1-\alpha/2}\; \eta\\
  \ell_{\alpha} &=& z_{1-\alpha/2} \sqrt{\eta^2 + c^2\sigma^2}
\end{eqnarray*}
where $z_{1-\alpha/2} = \Phi^{-1}(1 -\alpha/2)$ and $G$, $H_1$, $H_2$ are as defined in Equations \ref{eq:GFMSC}--\ref{eq:H2FMSC}.
And since the width of this na\"{i}ve CI equals that of the textbook interval for $\widehat{\beta}$ when $|\widehat{\tau}|<\sigma\sqrt{2}$ and that of the textbook interval for $\widetilde{\beta}$ otherwise, we have
\begin{equation*}
  \frac{E\left[ \mbox{Width}_{\mbox{Na\"{i}ve}}(\alpha) \right]}{\mbox{Width}_{Valid}(\alpha)} = 1 + \left[ \Phi( \sqrt{2} - \tau/\sigma) -  \Phi( -\sqrt{2} - \tau/\sigma )\right]\left( \sqrt{\frac{\eta^2}{\eta^2 + c^2 \sigma^2}} - 1 \right)
\end{equation*}
where $\mbox{Width}_{Valid}(\alpha)$ is the width of a standard, textbook confidence interval for $\widetilde{\beta}$.

To evaluate these expressions we need values for $c, \eta^2, \sigma^2$ and $\tau$.
For the remainder of this section I will consider the parameter values that correspond to the simulation experiments presented below in Section \ref{sec:simulations}.
For the OLS versus TSLS example we have $c=1$, $\eta^2=1$ and $\sigma^2 = (1-\pi^2)/\pi^2$ where $\pi^2$ denotes the population first-stage R-squared for the TSLS estimator. 
For the choosing IVs example we have $c =\gamma/(\gamma^2 +1/9)$, $\eta^2 = 1/(\gamma^2 + 1/9)$ and $\sigma^2 = 1 + 9\gamma^2$ where $\gamma^2$ is the increase in the population first-stage R-squared of the TSLS estimator from \emph{adding} $w$ to the instrument set.\footnote{The population first-stage R-squared with only $\mathbf{z}$ in the instument set $1/9$.}

Table \ref{tab:LimitNaiveCover} presents the asymptotic coverage probability and Table \ref{tab:LimitNaiveWidth} the expected relative width of the na\"{i}ve confidence interval procedure for a variety of values of $\tau$ and $\alpha$ for each of the two examples.
For the OLS versus TSLS example, I allow $\pi^2$ to vary while for the choosing IVs example I allow $\gamma^2$ to vary.
Note that the relative expected width does not depend on $\alpha$.
In terms of coverage probability, the na\"{i}ve interval performs very poorly: in some regions of the parameter space the actual coverage is very close to the nominal level, while in others it is far lower.
These striking size distortions, which echo the findings of \cite{Guggenberger2010} and \cite{Guggenberger2012}, provide a strong argument against the use of the na\"{i}ve interval.
Its attraction, of course, is width: the na\"{i}ve interval can be dramatically shorter than the corresponding ``textbook'' confidence interval for the valid estimator.

\begin{table}[h]
  \centering
  \begin{subtable}{0.48\textwidth}
    \caption{OLS versus TSLS}
    \begin{tabular}{r|rrrrrr}
\hline\hline
 &\multicolumn{6}{c}{$\tau$} \\ 
 $\alpha = 0.05$ & $0$ & $1$ & $2$ & $3$ & $4$ & $5$ \\ 
 \hline$0.1$ & $91$ & $81$ & $57$ & $41$ & $45$ & $58$\\ 
$\pi^2\;\;\;$ $0.2$ & $91$ & $83$ & $63$ & $58$ & $70$ & $84$\\ 
$0.3$ & $92$ & $84$ & $69$ & $73$ & $86$ & $93$\\ 
$0.4$ & $92$ & $85$ & $76$ & $84$ & $93$ & $95$\\ 
 \hline 
 \end{tabular}
 
 \vspace{2em} 
 
\begin{tabular}{r|rrrrrr}
\hline\hline
 &\multicolumn{6}{c}{$\tau$} \\ 
 $\alpha = 0.1$ & $0$ & $1$ & $2$ & $3$ & $4$ & $5$ \\ 
 \hline$0.1$ & $83$ & $70$ & $45$ & $35$ & $42$ & $55$\\ 
$\pi^2\;\;\;$ $0.2$ & $84$ & $72$ & $53$ & $52$ & $67$ & $81$\\ 
$0.3$ & $85$ & $74$ & $60$ & $68$ & $83$ & $89$\\ 
$0.4$ & $86$ & $76$ & $68$ & $80$ & $89$ & $90$\\ 
 \hline 
 \end{tabular}
 
 \vspace{2em} 
 
\begin{tabular}{r|rrrrrr}
\hline\hline
 &\multicolumn{6}{c}{$\tau$} \\ 
 $\alpha = 0.2$ & $0$ & $1$ & $2$ & $3$ & $4$ & $5$ \\ 
 \hline$0.1$ & $70$ & $54$ & $31$ & $27$ & $37$ & $50$\\ 
$\pi^2\;\;\;$ $0.2$ & $71$ & $57$ & $39$ & $45$ & $62$ & $74$\\ 
$0.3$ & $73$ & $59$ & $49$ & $61$ & $75$ & $79$\\ 
$0.4$ & $74$ & $62$ & $58$ & $72$ & $79$ & $80$\\ 
 \hline 
 \end{tabular}
  \end{subtable}
  ~
  \begin{subtable}{0.48\textwidth}
    \caption{Choosing IVs}
    \begin{tabular}{r|rrrrrr}
\hline\hline
 &\multicolumn{6}{c}{$\tau$} \\ 
 $\alpha = 0.05$ & $0$ & $1$ & $2$ & $3$ & $4$ & $5$ \\ 
 \hline$0.1$ & $93$ & $89$ & $84$ & $85$ & $91$ & $94$\\ 
$\gamma^2\;\;\;$ $0.2$ & $92$ & $87$ & $76$ & $74$ & $83$ & $91$\\ 
$0.3$ & $92$ & $85$ & $71$ & $65$ & $74$ & $86$\\ 
$0.4$ & $91$ & $85$ & $68$ & $59$ & $67$ & $80$\\ 
 \hline 
 \end{tabular}
 
 \vspace{2em} 
 
\begin{tabular}{r|rrrrrr}
\hline\hline
 &\multicolumn{6}{c}{$\tau$} \\ 
 $\alpha = 0.1$ & $0$ & $1$ & $2$ & $3$ & $4$ & $5$ \\ 
 \hline$0.1$ & $87$ & $82$ & $76$ & $79$ & $86$ & $89$\\ 
$\gamma^2\;\;\;$ $0.2$ & $85$ & $78$ & $66$ & $67$ & $79$ & $87$\\ 
$0.3$ & $84$ & $76$ & $61$ & $59$ & $71$ & $82$\\ 
$0.4$ & $84$ & $75$ & $57$ & $52$ & $63$ & $77$\\ 
 \hline 
 \end{tabular}
 
 \vspace{2em} 
 
\begin{tabular}{r|rrrrrr}
\hline\hline
 &\multicolumn{6}{c}{$\tau$} \\ 
 $\alpha = 0.2$ & $0$ & $1$ & $2$ & $3$ & $4$ & $5$ \\ 
 \hline$0.1$ & $75$ & $69$ & $64$ & $70$ & $77$ & $80$\\ 
$\gamma^2\;\;\;$ $0.2$ & $73$ & $64$ & $53$ & $59$ & $71$ & $78$\\ 
$0.3$ & $72$ & $62$ & $47$ & $50$ & $64$ & $75$\\ 
$0.4$ & $72$ & $60$ & $43$ & $44$ & $58$ & $71$\\ 
 \hline 
 \end{tabular}
  \end{subtable}
  \caption{Asymptotic coverage probability of Na\"{i}ve $(1-\alpha)\times 100\%$ confidence interval.}
  \label{tab:LimitNaiveCover}
\end{table}

\begin{table}[h]
  \centering
  \begin{subtable}{0.48\textwidth}
    \caption{OLS versus TSLS}
    \begin{tabular}{r|rrrrrr}
\hline\hline
 &\multicolumn{6}{c}{$\tau$} \\ 
  & $0$ & $1$ & $2$ & $3$ & $4$ & $5$ \\ 
 \hline$0.1$ & $ 42$ & $ 44$ & $ 48$ & $ 55$ & $ 64$ & $ 73$\\ 
$\pi^2\;\;\;$ $0.2$ & $ 53$ & $ 56$ & $ 64$ & $ 74$ & $ 85$ & $ 92$\\ 
$0.3$ & $ 62$ & $ 66$ & $ 76$ & $ 87$ & $ 95$ & $ 99$\\ 
$0.4$ & $ 69$ & $ 74$ & $ 85$ & $ 94$ & $ 99$ & $100$\\ 
 \hline 
 \end{tabular}
  \end{subtable}
  ~
  \begin{subtable}{0.48\textwidth}
    \caption{Choosing IVs}
    \begin{tabular}{r|rrrrrr}
\hline\hline
 &\multicolumn{6}{c}{$\tau$} \\ 
  & $0$ & $1$ & $2$ & $3$ & $4$ & $5$ \\ 
 \hline$0.1$ & $ 77$ & $ 80$ & $ 87$ & $ 94$ & $ 98$ & $100$\\ 
$\gamma^2\;\;\;$ $0.2$ & $ 66$ & $ 69$ & $ 77$ & $ 86$ & $ 93$ & $ 98$\\ 
$0.3$ & $ 60$ & $ 62$ & $ 69$ & $ 79$ & $ 88$ & $ 94$\\ 
$0.4$ & $ 55$ & $ 57$ & $ 64$ & $ 73$ & $ 83$ & $ 90$\\ 
 \hline 
 \end{tabular}
  \end{subtable}
  \caption{Asymptotic expected width of na\"{i}ve confidence interval relative to that of the valid estimator. Values are given in percentage points.}
  \label{tab:LimitNaiveWidth}
\end{table}

Is there any way to construct a post-FMSC confidence interval that does not suffer from the egregious size distortions of the na\"{i}ve interval but is still shorter than the textbook interval for the valid estimator?
As a first step towards answering this question, Table \ref{tab:WidthInfeasible} presents the relative width of the shortest possible \emph{infeasible} post-FMSC confidence interval constructed directly from $Q_{FMSC}$.
This interval has asymptotic coverage probability \emph{exactly} equal to its nominal level as it correctly accounts for the effect of moment selection on the asymptotic distribution of the estimators.
Unfortunately it cannot be used in practice because it requires knowledge of $\tau$, for which no consistent estimator exists.
As such, this interval serves as a benchmark against which to judge various feasible procedures that do not require knowledge of $\tau$.
For certain parameter values this interval is indeed shorter than the valid interval but the improvement is not uniform.
Just as the FMSC itself cannot provide a uniform reduction in AMSE relative to the valid estimator, the infeasible post-FMSC cannot provide a corresponding reduction in width.
In both cases, however, improvements are possible when $\tau$ is expected to be small, the setting in which this paper assumes that an applied researcher finds herself. 
The potential reductions in width can be particularly dramatic for larger values of $\alpha$.
The question remains: is there any way to capture these gains using a \emph{feasible} procedure?


\begin{table}[h]
  \centering
  \begin{subtable}{0.48\textwidth}
    \caption{OLS versus TSLS}
    \begin{tabular}{r|rrrrrr}
\hline\hline
 &\multicolumn{6}{c}{$\tau$} \\ 
 $\alpha = 0.05$ & $0$ & $1$ & $2$ & $3$ & $4$ & $5$ \\ 
 \hline$0.1$ & $ 99$ & $ 92$ & $ 85$ & $ 89$ & $ 95$ & $101$\\ 
$\pi^2\;\;\;$ $0.2$ & $ 97$ & $ 91$ & $ 94$ & $102$ & $110$ & $117$\\ 
$0.3$ & $ 94$ & $ 94$ & $102$ & $111$ & $117$ & $109$\\ 
$0.4$ & $ 92$ & $ 97$ & $107$ & $114$ & $107$ & $100$\\ 
 \hline 
 \end{tabular}
 
 \vspace{2em} 
 
\begin{tabular}{r|rrrrrr}
\hline\hline
 &\multicolumn{6}{c}{$\tau$} \\ 
 $\alpha = 0.1$ & $0$ & $1$ & $2$ & $3$ & $4$ & $5$ \\ 
 \hline$0.1$ & $ 88$ & $ 81$ & $ 85$ & $ 91$ & $ 99$ & $107$\\ 
$\pi^2\;\;\;$ $0.2$ & $ 89$ & $ 88$ & $ 97$ & $107$ & $116$ & $123$\\ 
$0.3$ & $ 86$ & $ 93$ & $105$ & $115$ & $119$ & $103$\\ 
$0.4$ & $ 87$ & $ 98$ & $111$ & $116$ & $104$ & $100$\\ 
 \hline 
 \end{tabular}
 
 \vspace{2em} 
 
\begin{tabular}{r|rrrrrr}
\hline\hline
 &\multicolumn{6}{c}{$\tau$} \\ 
 $\alpha = 0.2$ & $0$ & $1$ & $2$ & $3$ & $4$ & $5$ \\ 
 \hline$0.1$ & $ 48$ & $ 55$ & $ 84$ & $ 96$ & $106$ & $116$\\ 
$\pi^2\;\;\;$ $0.2$ & $ 65$ & $ 80$ & $101$ & $114$ & $125$ & $117$\\ 
$0.3$ & $ 74$ & $ 90$ & $111$ & $123$ & $112$ & $101$\\ 
$0.4$ & $ 80$ & $ 97$ & $116$ & $115$ & $102$ & $100$\\ 
 \hline 
 \end{tabular}
  \end{subtable}
  ~
  \begin{subtable}{0.48\textwidth}
    \caption{Choosing IVs}
    \begin{tabular}{r|rrrrrr}
\hline\hline
 &\multicolumn{6}{c}{$\tau$} \\ 
 $\alpha = 0.05$ & $0$ & $1$ & $2$ & $3$ & $4$ & $5$ \\ 
 \hline$0.1$ & $ 92$ & $ 97$ & $106$ & $111$ & $109$ & $102$\\ 
$\gamma^2\;\;\;$ $0.2$ & $ 93$ & $ 94$ & $101$ & $109$ & $115$ & $114$\\ 
$0.3$ & $ 95$ & $ 93$ & $ 97$ & $105$ & $112$ & $117$\\ 
$0.4$ & $ 97$ & $ 92$ & $ 94$ & $101$ & $108$ & $115$\\ 
 \hline 
 \end{tabular}
 
 \vspace{2em} 
 
\begin{tabular}{r|rrrrrr}
\hline\hline
 &\multicolumn{6}{c}{$\tau$} \\ 
 $\alpha = 0.1$ & $0$ & $1$ & $2$ & $3$ & $4$ & $5$ \\ 
 \hline$0.1$ & $ 89$ & $ 97$ & $108$ & $113$ & $108$ & $101$\\ 
$\gamma^2\;\;\;$ $0.2$ & $ 86$ & $ 93$ & $104$ & $113$ & $118$ & $109$\\ 
$0.3$ & $ 86$ & $ 90$ & $100$ & $109$ & $117$ & $121$\\ 
$0.4$ & $ 88$ & $ 88$ & $ 96$ & $105$ & $114$ & $121$\\ 
 \hline 
 \end{tabular}
 
 \vspace{2em} 
 
\begin{tabular}{r|rrrrrr}
\hline\hline
 &\multicolumn{6}{c}{$\tau$} \\ 
 $\alpha = 0.2$ & $0$ & $1$ & $2$ & $3$ & $4$ & $5$ \\ 
 \hline$0.1$ & $ 86$ & $ 96$ & $111$ & $115$ & $105$ & $101$\\ 
$\gamma^2\;\;\;$ $0.2$ & $ 78$ & $ 89$ & $108$ & $119$ & $118$ & $104$\\ 
$0.3$ & $ 72$ & $ 84$ & $103$ & $116$ & $125$ & $112$\\ 
$0.4$ & $ 67$ & $ 79$ & $ 99$ & $112$ & $123$ & $128$\\ 
 \hline 
 \end{tabular}
  \end{subtable}
  \caption{Width of shortest possible $(1-\alpha)\times 100\%$ post-FMSC confidence interval constructed directly from $Q_{FMSC}$ using knowledge of $\tau$. This interval is infeasible as no consistent estimator of $\tau$ exists. Values are given in percentage points.}
  \label{tab:WidthInfeasible}
\end{table}

Although no consistent estimator of $\tau$ exists, $\widehat{\tau}$ provides an asymptotically unbiased estimator.
I now consider two different ways to use $\widehat{\tau}$ to construct a post-FMSC confidence interval.
The first is equivalent to the two-step procedure given in Algorithm \ref{alg:conf} but uses exact calculations rather than simulations to evaluate $Q_{FMSC}$, following the derivations given above.
This procedure first constructs a $(1-\alpha_1)\times 100\%$ confidence interval for $\widehat{\tau}$.
For each $\tau^*$ in this interval it then constructs a $(1-\alpha_2)\times 100\%$ based on $Q_{FMSC}$ and then takes the upper and lower bounds over all of the resulting intervals.
This interval is guaranteed to have asymptotic coverage probability of at least $1 - (\alpha_1 + \alpha_2)$ by an argument essentially identical to the proof of Theorem \ref{thm:sim}.
As we are free when using this method to choose any pair $(\alpha_1, \alpha_2)$ such that $\alpha_1 + \alpha_2 = \alpha$, I experimented with three possibilities: $\alpha_1 = \alpha_2 = \alpha/2$, followed by $\alpha_1 = \alpha/4, \alpha_2 = 3\alpha/4$ and $\alpha_1 = 3\alpha/4, \alpha_2 = \alpha/4$.
Setting $\alpha_1 = \alpha/4$ produced the shortest intervals so I report only results for the middle configuration here.\footnote{For the two-step procedure I take lower and upper bounds over a collection of equal-tailed intervals. It does not necessarily follow that the bounds over these intervals would be tigher if each interval in the collection were constructed to be a short as possible.}
Additional results are available on request.
As we see from Table \ref{tab:Limit2StepWideTauOLSvsIV} for the OLS versus TSLS example and Table \ref{tab:Limit2StepWideTauChooseIVs} for the choosing IVs example, this procedure delivers on its promise that asymptotic coverage will never fall below $1-\alpha$.
Protection against under-coverage, however, comes at the expense of extreme conservatism, particularly for larger values of $\alpha$.
The two-step procedure yields asymptotic coverage that is systematically \emph{too large} and hence \emph{cannot} produce an interval shorter than the textbook CI for the valid estimator.


\begin{table}[h]
  \centering
  \begin{subtable}{0.48\textwidth}
    \caption{Coverage Probability}
    \begin{tabular}{r|rrrrrr}
\hline\hline
 &\multicolumn{6}{c}{$\tau$} \\ 
 $\alpha = 0.05$ & $0$ & $1$ & $2$ & $3$ & $4$ & $5$ \\ 
 \hline$0.1$ & $97$ & $97$ & $97$ & $98$ & $98$ & $98$\\ 
$\pi^2\;\;\;$ $0.2$ & $97$ & $97$ & $98$ & $97$ & $97$ & $97$\\ 
$0.3$ & $98$ & $98$ & $98$ & $97$ & $96$ & $97$\\ 
$0.4$ & $98$ & $98$ & $97$ & $96$ & $97$ & $98$\\ 
 \hline 
 \end{tabular}
 
 \vspace{2em} 
 
\begin{tabular}{r|rrrrrr}
\hline\hline
 &\multicolumn{6}{c}{$\tau$} \\ 
 $\alpha = 0.1$ & $0$ & $1$ & $2$ & $3$ & $4$ & $5$ \\ 
 \hline$0.1$ & $94$ & $95$ & $96$ & $96$ & $95$ & $94$\\ 
$\pi^2\;\;\;$ $0.2$ & $95$ & $96$ & $96$ & $95$ & $94$ & $93$\\ 
$0.3$ & $96$ & $96$ & $95$ & $94$ & $92$ & $94$\\ 
$0.4$ & $96$ & $95$ & $94$ & $92$ & $94$ & $95$\\ 
 \hline 
 \end{tabular}
 
 \vspace{2em} 
 
\begin{tabular}{r|rrrrrr}
\hline\hline
 &\multicolumn{6}{c}{$\tau$} \\ 
 $\alpha = 0.2$ & $0$ & $1$ & $2$ & $3$ & $4$ & $5$ \\ 
 \hline$0.1$ & $91$ & $92$ & $92$ & $91$ & $90$ & $90$\\ 
$\pi^2\;\;\;$ $0.2$ & $93$ & $92$ & $91$ & $89$ & $87$ & $85$\\ 
$0.3$ & $93$ & $92$ & $89$ & $86$ & $85$ & $89$\\ 
$0.4$ & $93$ & $91$ & $86$ & $85$ & $88$ & $89$\\ 
 \hline 
 \end{tabular}
  \end{subtable}
  ~
  \begin{subtable}{0.48\textwidth}
    \caption{Relative Width}
    \begin{tabular}{r|rrrrrr}
\hline\hline
 &\multicolumn{6}{c}{$\tau$} \\ 
 $\alpha = 0.05$ & $0$ & $1$ & $2$ & $3$ & $4$ & $5$ \\ 
 \hline$0.1$ & $114$ & $115$ & $117$ & $119$ & $123$ & $126$\\ 
$\pi^2\;\;\;$ $0.2$ & $116$ & $117$ & $120$ & $121$ & $125$ & $126$\\ 
$0.3$ & $117$ & $117$ & $120$ & $122$ & $123$ & $123$\\ 
$0.4$ & $116$ & $118$ & $120$ & $121$ & $121$ & $120$\\ 
 \hline 
 \end{tabular}
 
 \vspace{2em} 
 
\begin{tabular}{r|rrrrrr}
\hline\hline
 &\multicolumn{6}{c}{$\tau$} \\ 
 $\alpha = 0.1$ & $0$ & $1$ & $2$ & $3$ & $4$ & $5$ \\ 
 \hline$0.1$ & $121$ & $123$ & $125$ & $128$ & $129$ & $131$\\ 
$\pi^2\;\;\;$ $0.2$ & $122$ & $124$ & $126$ & $129$ & $130$ & $131$\\ 
$0.3$ & $123$ & $125$ & $126$ & $127$ & $128$ & $128$\\ 
$0.4$ & $123$ & $123$ & $124$ & $125$ & $125$ & $123$\\ 
 \hline 
 \end{tabular}
 
 \vspace{2em} 
 
\begin{tabular}{r|rrrrrr}
\hline\hline
 &\multicolumn{6}{c}{$\tau$} \\ 
 $\alpha = 0.2$ & $0$ & $1$ & $2$ & $3$ & $4$ & $5$ \\ 
 \hline$0.1$ & $135$ & $139$ & $140$ & $140$ & $144$ & $145$\\ 
$\pi^2\;\;\;$ $0.2$ & $136$ & $136$ & $137$ & $139$ & $141$ & $141$\\ 
$0.3$ & $135$ & $135$ & $136$ & $137$ & $136$ & $135$\\ 
$0.4$ & $133$ & $133$ & $133$ & $133$ & $131$ & $128$\\ 
 \hline 
 \end{tabular}
  \end{subtable}
  \caption{OLS versus TSLS Example: Asymptotic coverage and expected relative width of two-step confidence interval with $\alpha_1 = \alpha/4,  \alpha_2 = 3\alpha/4$.}
  \label{tab:Limit2StepWideTauOLSvsIV}
\end{table}

\begin{table}[h]
  \centering
  \begin{subtable}{0.48\textwidth}
    \caption{Coverage Probability}
    \begin{tabular}{r|rrrrrr}
\hline\hline
 &\multicolumn{6}{c}{$\tau$} \\ 
 $\alpha = 0.05$ & $0$ & $1$ & $2$ & $3$ & $4$ & $5$ \\ 
 \hline$0.1$ & $98$ & $98$ & $97$ & $96$ & $96$ & $97$\\ 
$\gamma^2\;\;\;$ $0.2$ & $98$ & $98$ & $98$ & $97$ & $96$ & $96$\\ 
$0.3$ & $98$ & $98$ & $98$ & $97$ & $97$ & $96$\\ 
$0.4$ & $97$ & $97$ & $98$ & $98$ & $97$ & $97$\\ 
 \hline 
 \end{tabular}
 
 \vspace{2em} 
 
\begin{tabular}{r|rrrrrr}
\hline\hline
 &\multicolumn{6}{c}{$\tau$} \\ 
 $\alpha = 0.1$ & $0$ & $1$ & $2$ & $3$ & $4$ & $5$ \\ 
 \hline$0.1$ & $96$ & $96$ & $94$ & $93$ & $93$ & $94$\\ 
$\gamma^2\;\;\;$ $0.2$ & $96$ & $96$ & $95$ & $94$ & $93$ & $93$\\ 
$0.3$ & $96$ & $96$ & $95$ & $95$ & $93$ & $92$\\ 
$0.4$ & $95$ & $96$ & $96$ & $95$ & $94$ & $93$\\ 
 \hline 
 \end{tabular}
 
 \vspace{2em} 
 
\begin{tabular}{r|rrrrrr}
\hline\hline
 &\multicolumn{6}{c}{$\tau$} \\ 
 $\alpha = 0.2$ & $0$ & $1$ & $2$ & $3$ & $4$ & $5$ \\ 
 \hline$0.1$ & $93$ & $91$ & $87$ & $85$ & $87$ & $88$\\ 
$\gamma^2\;\;\;$ $0.2$ & $93$ & $92$ & $89$ & $86$ & $85$ & $87$\\ 
$0.3$ & $93$ & $92$ & $90$ & $88$ & $85$ & $85$\\ 
$0.4$ & $93$ & $92$ & $91$ & $89$ & $87$ & $85$\\ 
 \hline 
 \end{tabular}
  \end{subtable}
  ~
  \begin{subtable}{0.48\textwidth}
    \caption{Relative Width}
    \begin{tabular}{r|rrrrrr}
\hline\hline
 &\multicolumn{6}{c}{$\tau$} \\ 
 $\alpha = 0.05$ & $0$ & $1$ & $2$ & $3$ & $4$ & $5$ \\ 
 \hline$0.1$ & $114$ & $115$ & $117$ & $119$ & $123$ & $126$\\ 
$\gamma^2\;\;\;$ $0.2$ & $116$ & $117$ & $120$ & $121$ & $125$ & $126$\\ 
$0.3$ & $117$ & $117$ & $120$ & $122$ & $123$ & $123$\\ 
$0.4$ & $116$ & $118$ & $120$ & $121$ & $121$ & $120$\\ 
 \hline 
 \end{tabular}
 
 \vspace{2em} 
 
\begin{tabular}{r|rrrrrr}
\hline\hline
 &\multicolumn{6}{c}{$\tau$} \\ 
 $\alpha = 0.1$ & $0$ & $1$ & $2$ & $3$ & $4$ & $5$ \\ 
 \hline$0.1$ & $121$ & $123$ & $125$ & $128$ & $129$ & $131$\\ 
$\gamma^2\;\;\;$ $0.2$ & $122$ & $124$ & $126$ & $129$ & $130$ & $131$\\ 
$0.3$ & $123$ & $125$ & $126$ & $127$ & $128$ & $128$\\ 
$0.4$ & $123$ & $123$ & $124$ & $125$ & $125$ & $123$\\ 
 \hline 
 \end{tabular}
 
 \vspace{2em} 
 
\begin{tabular}{r|rrrrrr}
\hline\hline
 &\multicolumn{6}{c}{$\tau$} \\ 
 $\alpha = 0.2$ & $0$ & $1$ & $2$ & $3$ & $4$ & $5$ \\ 
 \hline$0.1$ & $135$ & $139$ & $140$ & $140$ & $144$ & $145$\\ 
$\gamma^2\;\;\;$ $0.2$ & $136$ & $136$ & $137$ & $139$ & $141$ & $141$\\ 
$0.3$ & $135$ & $135$ & $136$ & $137$ & $136$ & $135$\\ 
$0.4$ & $133$ & $133$ & $133$ & $133$ & $131$ & $128$\\ 
 \hline 
 \end{tabular}
  \end{subtable}
  \caption{Choosing IVs Example: Asymptotic coverage and expected relative width of two-step confidence interval with $\alpha_1 = \alpha/4,  \alpha_2 = 3\alpha/4$.}
  \label{tab:Limit2StepWideTauChooseIVs}
\end{table}

To address this problem, I now consider an alternative one-step procedure.
Rather than first constructing a confidence region for $\tau$ and then taking upper and lower bounds, this method simply assumes that $\widehat{\tau}$ is exactly equal to $\tau$ and then constructs a confidence interval from $Q_{FMSC}$ exactly as in the infeasible interval described above.\footnote{As in the construction of the na\"{i}ve interval, I take the shortest possible interval based on $Q_{FMSC}$ rather than an equal-tailed interval. Additional results for an equal-tailed version of this one-step procedure are available upon request. Their performance is similar.}
Note that Theorem \ref{thm:sim} does \emph{not} apply to this one-step interval: it comes with no generic guarantee of uniform coverage performance.
For any example that take the form of Equation \ref{eq:LimitExperiment}, however, we can directly calculate the asymptotic coverage and expected relative width of this procedure to determine precisely how it performs over relevant regions of the parameter space.
Tables \ref{tab:Limit1StepShortOLSvsIV} and \ref{tab:Limit1StepShortChooseIVs} do exactly this.
We see that the one-step interval effectively ``splits the difference'' between the two-step interval and the na\"{i}ve procedure. 
While it can under-cover, the size distortions are fairly small, particularly for $\alpha=0.1$ and $0.05$.
At the same time, when $\tau$ is relatively small this procedure can yield shorter intervals.
While a full investigation of this phenomenon is beyond the scope of the present paper, these calculations suggest a plausible way forward for post-FMSC inference that is less conservative than the two-step procedure from Algorithm \ref{alg:conf} by directly calculating the relevant quantities from the limit distribution of interest.
One could imagine specifying a maximum allowable size distortion over some relevant region of the parameter space and then designing a confidence interval to minimize width.
This could be a one-step interval or possibly a two-step interval in which one allows the sum $\alpha_1 + \alpha_2$ to be \emph{less than} $\alpha$.
Just as the FMSC aims to achieve a favorable trade-off between bias and variance, such a confidence interval procedure could aim to achieve a favorable trade-off between width and coverage.
It would also be interesting to pursue analogous calculations for the minimum AMSE averaging estimator from Section \ref{sec:momentavgexample}.

\begin{table}[h]
  \centering
  \begin{subtable}{0.48\textwidth}
    \caption{Coverage Probability}
    \begin{tabular}{r|rrrrrr}
\hline\hline
 &\multicolumn{6}{c}{$\tau$} \\ 
 $\alpha = 0.05$ & $0$ & $1$ & $2$ & $3$ & $4$ & $5$ \\ 
 \hline$0.1$ & $93$ & $94$ & $95$ & $94$ & $91$ & $90$\\ 
$\pi^2\;\;\;$ $0.2$ & $95$ & $95$ & $95$ & $93$ & $91$ & $91$\\ 
$0.3$ & $95$ & $96$ & $94$ & $92$ & $92$ & $94$\\ 
$0.4$ & $96$ & $95$ & $94$ & $93$ & $95$ & $95$\\ 
 \hline 
 \end{tabular}
 
 \vspace{2em} 
 
\begin{tabular}{r|rrrrrr}
\hline\hline
 &\multicolumn{6}{c}{$\tau$} \\ 
 $\alpha = 0.1$ & $0$ & $1$ & $2$ & $3$ & $4$ & $5$ \\ 
 \hline$0.1$ & $89$ & $89$ & $88$ & $86$ & $82$ & $80$\\ 
$\pi^2\;\;\;$ $0.2$ & $91$ & $91$ & $88$ & $85$ & $83$ & $85$\\ 
$0.3$ & $92$ & $91$ & $87$ & $85$ & $87$ & $90$\\ 
$0.4$ & $92$ & $90$ & $87$ & $87$ & $90$ & $91$\\ 
 \hline 
 \end{tabular}
 
 \vspace{2em} 
 
\begin{tabular}{r|rrrrrr}
\hline\hline
 &\multicolumn{6}{c}{$\tau$} \\ 
 $\alpha = 0.2$ & $0$ & $1$ & $2$ & $3$ & $4$ & $5$ \\ 
 \hline$0.1$ & $84$ & $80$ & $71$ & $67$ & $65$ & $64$\\ 
$\pi^2\;\;\;$ $0.2$ & $85$ & $80$ & $71$ & $70$ & $70$ & $76$\\ 
$0.3$ & $84$ & $79$ & $73$ & $72$ & $78$ & $81$\\ 
$0.4$ & $84$ & $79$ & $74$ & $77$ & $81$ & $81$\\ 
 \hline 
 \end{tabular}
  \end{subtable}
  ~
  \begin{subtable}{0.48\textwidth}
    \caption{Relative Width}
    \begin{tabular}{r|rrrrrr}
\hline\hline
 &\multicolumn{6}{c}{$\tau$} \\ 
 $\alpha = 0.05$ & $0$ & $1$ & $2$ & $3$ & $4$ & $5$ \\ 
 \hline$0.1$ & $ 93$ & $ 93$ & $ 95$ & $ 97$ & $ 99$ & $102$\\ 
$\pi^2\;\;\;$ $0.2$ & $ 96$ & $ 97$ & $ 99$ & $104$ & $106$ & $109$\\ 
$0.3$ & $ 97$ & $ 99$ & $102$ & $106$ & $108$ & $107$\\ 
$0.4$ & $ 98$ & $100$ & $105$ & $108$ & $106$ & $103$\\ 
 \hline 
 \end{tabular}
 
 \vspace{2em} 
 
\begin{tabular}{r|rrrrrr}
\hline\hline
 &\multicolumn{6}{c}{$\tau$} \\ 
 $\alpha = 0.1$ & $0$ & $1$ & $2$ & $3$ & $4$ & $5$ \\ 
 \hline$0.1$ & $ 90$ & $ 91$ & $ 92$ & $ 97$ & $ 99$ & $102$\\ 
$\pi^2\;\;\;$ $0.2$ & $ 94$ & $ 96$ & $100$ & $105$ & $108$ & $110$\\ 
$0.3$ & $ 96$ & $100$ & $104$ & $108$ & $109$ & $106$\\ 
$0.4$ & $ 97$ & $101$ & $106$ & $108$ & $106$ & $103$\\ 
 \hline 
 \end{tabular}
 
 \vspace{2em} 
 
\begin{tabular}{r|rrrrrr}
\hline\hline
 &\multicolumn{6}{c}{$\tau$} \\ 
 $\alpha = 0.2$ & $0$ & $1$ & $2$ & $3$ & $4$ & $5$ \\ 
 \hline$0.1$ & $ 83$ & $ 84$ & $ 87$ & $ 93$ & $ 99$ & $103$\\ 
$\pi^2\;\;\;$ $0.2$ & $ 91$ & $ 92$ & $ 96$ & $105$ & $109$ & $110$\\ 
$0.3$ & $ 93$ & $ 97$ & $104$ & $109$ & $108$ & $106$\\ 
$0.4$ & $ 95$ & $100$ & $107$ & $108$ & $105$ & $102$\\ 
 \hline 
 \end{tabular}
  \end{subtable}
  \caption{OLS vs TSLS Example: Asymptotic coverage probability and expected relative width of 1-step confidence interval constructed by substituting $\tau =\widehat{\tau}$ into $Q_{FMSC}$. Results are for the shortest possible interval, as in the infeasible procedure.}
  \label{tab:Limit1StepShortOLSvsIV}
\end{table}


\begin{table}[h]
  \centering
  \begin{subtable}{0.48\textwidth}
    \caption{Coverage Probability}
    \begin{tabular}{r|rrrrrr}
\hline\hline
 &\multicolumn{6}{c}{$\tau$} \\ 
 $\alpha = 0.05$ & $0$ & $1$ & $2$ & $3$ & $4$ & $5$ \\ 
 \hline$0.1$ & $96$ & $95$ & $94$ & $93$ & $94$ & $95$\\ 
$\gamma^2\;\;\;$ $0.2$ & $96$ & $96$ & $95$ & $93$ & $93$ & $94$\\ 
$0.3$ & $95$ & $95$ & $95$ & $93$ & $92$ & $92$\\ 
$0.4$ & $95$ & $95$ & $95$ & $94$ & $92$ & $91$\\ 
 \hline 
 \end{tabular}
 
 \vspace{2em} 
 
\begin{tabular}{r|rrrrrr}
\hline\hline
 &\multicolumn{6}{c}{$\tau$} \\ 
 $\alpha = 0.1$ & $0$ & $1$ & $2$ & $3$ & $4$ & $5$ \\ 
 \hline$0.1$ & $92$ & $90$ & $88$ & $88$ & $89$ & $91$\\ 
$\gamma^2\;\;\;$ $0.2$ & $92$ & $91$ & $88$ & $86$ & $86$ & $89$\\ 
$0.3$ & $92$ & $91$ & $89$ & $86$ & $85$ & $87$\\ 
$0.4$ & $91$ & $91$ & $89$ & $86$ & $84$ & $84$\\ 
 \hline 
 \end{tabular}
 
 \vspace{2em} 
 
\begin{tabular}{r|rrrrrr}
\hline\hline
 &\multicolumn{6}{c}{$\tau$} \\ 
 $\alpha = 0.2$ & $0$ & $1$ & $2$ & $3$ & $4$ & $5$ \\ 
 \hline$0.1$ & $83$ & $80$ & $76$ & $77$ & $80$ & $81$\\ 
$\gamma^2\;\;\;$ $0.2$ & $84$ & $80$ & $75$ & $73$ & $76$ & $80$\\ 
$0.3$ & $85$ & $81$ & $75$ & $71$ & $72$ & $77$\\ 
$0.4$ & $84$ & $80$ & $73$ & $72$ & $69$ & $74$\\ 
 \hline 
 \end{tabular}
  \end{subtable}
  ~
  \begin{subtable}{0.48\textwidth}
    \caption{Relative Width}
    \begin{tabular}{r|rrrrrr}
\hline\hline
 &\multicolumn{6}{c}{$\tau$} \\ 
 $\alpha = 0.05$ & $0$ & $1$ & $2$ & $3$ & $4$ & $5$ \\ 
 \hline$0.1$ & $ 98$ & $100$ & $104$ & $106$ & $106$ & $104$\\ 
$\gamma^2\;\;\;$ $0.2$ & $ 97$ & $ 99$ & $103$ & $106$ & $108$ & $108$\\ 
$0.3$ & $ 97$ & $ 98$ & $101$ & $104$ & $107$ & $109$\\ 
$0.4$ & $ 97$ & $ 97$ & $ 99$ & $103$ & $106$ & $108$\\ 
 \hline 
 \end{tabular}
 
 \vspace{2em} 
 
\begin{tabular}{r|rrrrrr}
\hline\hline
 &\multicolumn{6}{c}{$\tau$} \\ 
 $\alpha = 0.1$ & $0$ & $1$ & $2$ & $3$ & $4$ & $5$ \\ 
 \hline$0.1$ & $ 98$ & $100$ & $104$ & $107$ & $106$ & $104$\\ 
$\gamma^2\;\;\;$ $0.2$ & $ 97$ & $ 98$ & $103$ & $107$ & $109$ & $107$\\ 
$0.3$ & $ 96$ & $ 97$ & $101$ & $105$ & $108$ & $109$\\ 
$0.4$ & $ 95$ & $ 96$ & $100$ & $103$ & $107$ & $109$\\ 
 \hline 
 \end{tabular}
 
 \vspace{2em} 
 
\begin{tabular}{r|rrrrrr}
\hline\hline
 &\multicolumn{6}{c}{$\tau$} \\ 
 $\alpha = 0.2$ & $0$ & $1$ & $2$ & $3$ & $4$ & $5$ \\ 
 \hline$0.1$ & $ 98$ & $100$ & $105$ & $107$ & $106$ & $103$\\ 
$\gamma^2\;\;\;$ $0.2$ & $ 94$ & $ 97$ & $104$ & $108$ & $109$ & $107$\\ 
$0.3$ & $ 93$ & $ 96$ & $101$ & $106$ & $109$ & $109$\\ 
$0.4$ & $ 89$ & $ 93$ & $ 97$ & $105$ & $108$ & $110$\\ 
 \hline 
 \end{tabular}
  \end{subtable}
  \caption{Choosing IVs Example: Asymptotic coverage probability and expected relative width of 1-step confidence interval constructed by substituting $\tau =\widehat{\tau}$ into $Q_{FMSC}$. Results are for the shortest possible interval, as in the infeasible procedure.}
  \label{tab:Limit1StepShortChooseIVs}
\end{table}

