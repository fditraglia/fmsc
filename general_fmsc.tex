%!TEX root = main.tex
\section{The Focused Moment Selection Criterion}
\label{sec:FMSC}

\subsection{The General Case}
The FMSC chooses among the potentially invalid moment conditions contained in $h$ based on the estimator AMSE of a user-specified scalar target parameter.\footnote{Although I focus on the case of a scalar target parameter in the body of the paper, the same idea can be applied to a vector of target parameters. For details see Online Appendix \ref{append:mult}.}
Denote this target parameter by $\mu$, a real-valued, $Z$-almost continuous function of the parameter vector $\theta$ that is differentiable in a neighborhood of $\theta_0$. 
Further, define the GMM estimator of $\mu$ based on $\widehat{\theta}_S$ by $\widehat{\mu}_S = \mu(\widehat{\theta}_S)$ and the true value of $\mu$ by $\mu_0 = \mu(\theta_0)$. 
Applying the Delta Method to Theorem \ref{thm:normality} gives the AMSE of $\widehat{\mu}_S$.

\begin{cor}[AMSE of Target Parameter]
\label{cor:target}
Under the hypotheses of Theorem \ref{thm:normality}, 
$$\sqrt{n}\left(\widehat{\mu}_S - \mu_0\right)\rightarrow_d-\nabla_\theta\mu(\theta_0)'K_S \Xi_S \left(M +  \left[\begin{array}
	{c} 0 \\ \tau
\end{array} \right]\right)$$ 
where $M$ is defined in Theorem \ref{thm:normality}.
Hence,
	$$\mbox{AMSE}\left(\widehat{\mu}_S\right) = \nabla_\theta\mu(\theta_0)'K_S \Xi_S \left\{\left[\begin{array}{cc}0&0\\0&\tau\tau'\end{array}\right] + \Omega\right\}\Xi_S'K_S'\nabla_\theta\mu(\theta_0).$$
\end{cor}

For the valid estimator $\widehat{\theta}_v$ we have $K_v = \left[G'W_{v}G\right]^{-1}G' W_{v}$ and $\Xi_v =\left[\begin{array}{cc} \mathbf{I}_p& \mathbf{0}_{p\times q} \end{array} \right]$. 
Thus, the valid estimator $\widehat{\mu}_v$ of $\mu$ has zero asymptotic bias. 
In contrast, any candidate estimator $\widehat{\mu}_S$ that includes moment conditions from $h$ inherits an asymptotic bias from the corresponding elements of $\tau$, the extent and direction of which depends both on $K_S$ and $\nabla_\theta\mu(\theta_0)$. 
The setting considered here, however, is one in which using moment conditions from $h$ in estimation will reduce the asymptotic variance.
In the nested case, where moment conditions from $h$ are \emph{added} to those of $g$, this follows automatically.
The usual proof that adding moment conditions cannot increase asymptotic variance under efficient GMM \citep[see for example][ch.\ 6]{Hallbook} continues to hold under local mis-specification, because all moment conditions are correctly specified in the limit.
In non-nested examples, for example when $h$ contains OLS moment conditions and $g$ contains IV moment conditions, however, this result does not apply because one would use $h$ \emph{instead of} $g$.
In such examples, one must establish an analogous ordering of asymptotic variances by direct calculation, as I do below for the OLS versus IV example.

Using this framework for moment selection requires estimators of the unknown quantities: $\theta_0$, $K_S$, $\Omega$, and $\tau$. 
Under local mis-specification, the estimator of $\theta$ under \emph{any} moment set is consistent. 
A natural estimator is $\widehat{\theta}_v$, although there are other possibilities. 
Recall that $K_S = [F_S'W_SF_S]^{-1} F_S'W_S \Xi_S$.
Because it is simply the selection matrix defining moment set $S$, $\Xi_S$ is known.  
The remaining quantities $F_S$ and $W_S$ that make up $K_S$ are consistently estimated by their sample analogues under Assumption \ref{assump:highlevel}.
Similarly, consistent estimators of $\Omega$ are readily available under local mis-specification, although the precise form depends on the situation.\footnote{See Sections \ref{sec:OLSvsIVExample} and \ref{sec:chooseIVexample} for discussion of this point for the two running examples.}
The only remaining unknown is $\tau$. Local mis-specification is essential for making meaningful comparisons of AMSE because it prevents the bias term from dominating the comparison. 
Unfortunately, it also prevents consistent estimation of the asymptotic bias parameter.
Under Assumption \ref{assump:Identification}, however, it remains possible to construct an \emph{asymptotically unbiased} estimator $\widehat{\tau}$ of $\tau$ by substituting $\widehat{\theta}_v$, the estimator of $\theta_0$ that uses only correctly specified moment conditions, into $h_n$, the sample analogue of the potentially mis-specified moment conditions. 
In other words,  $\widehat{\tau} = \sqrt{n} h_n(\widehat{\theta}_v)$. 

\begin{thm}[Asymptotic Distribution of $\widehat{\tau}$] 
\label{thm:tau}
Let $\widehat{\tau} = \sqrt{n} h_n(\widehat{\theta}_v)$ where $\widehat{\theta}_v$ is the valid estimator, based only on the moment conditions contained in $g$. 
Then under Assumptions \ref{assump:drift}, \ref{assump:highlevel} and \ref{assump:Identification}
$$\widehat{\tau} \rightarrow_d \Psi\left( M + \left[\begin{array}
	{c} 0 \\ \tau
\end{array} \right]\right), \quad \Psi = \left[\begin{array}{cc} -HK_v & \mathbf{I}_q \end{array}\right]$$ 
where $K_v$ is defined in Corollary \ref{cor:valid} and $\mathbf{I}_q$ denotes the $(q\times q)$ identity matrix.
Thus, $\widehat{\tau}\rightarrow_d (\Psi M + \tau) \sim N_q(\tau, \Psi \Omega \Psi')$.
\end{thm}

Returning to Corollary $\ref{cor:target}$, however, we see that it is $\tau \tau'$ rather than $\tau$ that enters the expression for AMSE. 
Although $\widehat{\tau}$ is an asymptotically unbiased estimator of $\tau$, the limiting expectation of $\widehat{\tau} \widehat{\tau}'$ is not $\tau\tau'$ because $\widehat{\tau}$ has an asymptotic variance.  
Subtracting a consistent estimate of the asymptotic variance removes this asymptotic bias.

\begin{cor}[Asymptotically Unbiased Estimator of $\tau \tau'$]
\label{cor:tautau}
If $\widehat{\Omega}$ and $\widehat{\Psi}$ are consistent for $\Omega$ and $\Psi$, then $ \widehat{\tau}\widehat{\tau}' - \widehat{\Psi}\widehat{\Omega}\widehat{\Psi}$ is an asymptotically unbiased estimator of $\tau\tau'$.
\end{cor}
It follows that
\begin{equation}
\label{eq:fmsc}
	\mbox{FMSC}_n(S) = \nabla_\theta\mu(\widehat{\theta})'\widehat{K}_S\Xi_S \left\{\left[\begin{array}{cc}0&0\\0&\widehat{\tau}\widehat{\tau}' - \widehat{\Psi}\widehat{\Omega}\widehat{\Psi}'\end{array}\right] + \widehat{\Omega}\right\}\Xi_S'\widehat{K}_S' \nabla_\theta\mu(\widehat{\theta})
\end{equation}
provides an asymptotically unbiased estimator of AMSE.
Given a set $\mathscr{S}$ of candidate specifications, the FMSC selects the candidate $S^*$ that \emph{minimizes} the expression given in Equation \ref{eq:fmsc}, that is $S^*_{FMSC} =  \arg \min_{S\in \mathscr{S}} \;\mbox{FMSC}_n(S)$.

In summary, the FMSC aims to choose the moment conditions that provide the lowest risk estimator of a target parameter $\mu$ where risk is defined as MSE.\footnote{One could choose a different risk function and proceed similarly, although I do not consider this idea further below. See, e.g., \cite{Claeskens2006} and \cite{ClaeskensHjort2008}.}
Because finite-sample MSE is unavailable, AMSE in a local-to-zero asymptotic framework serves in its stead.
Since no consistent estimator of AMSE exists in this setting, FMSC uses an asymptotically unbiased estimator.
This is the same idea that underlies the classical AIC and TIC model selection criteria as well as more recent procedures such as those described in \cite{ClaeskensHjort2003} and \cite{Schorfheide2005}.

