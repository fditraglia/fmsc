%!TEX root = main.tex
\subsection{Valid Confidence Intervals}
\label{sec:CIsim}
I now revisit the simulation experiments introduced above in Sections \ref{sec:OLSvsIVsim} and \ref{sec:chooseIVsim} to evaluate the finite-sample performance of confidence intervals contructed according to Algorithm \ref{alg:conf}.
All results are based on 10,000 simulation replications from the appropriate DGP with $\alpha = \delta = 0.05$.
For more computational details, see Appendix \ref{append:comp}.
Coverage probabilities and relative widths are all given in percentage points, rounded to the nearest whole percent.

Table \ref{tab:OLSvsIVsim_cover_naiveFMSC} shows the problem of ignoring moment selection by presenting the the actual coverage probability of a na\"{i}ve 90\%, post-FMSC confidence interval for the OLS versus TSLS simulation experiment.
The na\"{i}ve procedure simply constructs a textbook 90\% interval around the FMSC-selected estimator.
Unsurprisingly, it performs poorly: coverage probabilities can be made \emph{arbitrarily} close to zero by choosing appropriate parameter values, a problem that persists even for large $N$.
At other parameter values, however, the intervals are close to their nominal level.
This is precisely the lack of uniformity described by \cite{LeebPoetscher2005}.
A similar pattern emerges in the choosing instrumental variables simulation: see Table \ref{tab:chooseIVsim_cover_naiveFMSC} in Appendix \ref{sec:CIsupplement}.

\begin{table}[h]
\footnotesize
\centering
	\begin{subtable}{0.48\textwidth}
		\caption{Two-Stage Least Squares}
		\begin{tabular}{r|rrrrrr}
\hline\hline
 &\multicolumn{6}{c}{$\rho$} \\ 
 $N = 50$ & $0$ & $0.1$ & $0.2$ & $0.3$ & $0.4$ & $0.5$ \\ 
 \hline$0.1$ & $98$ & $98$ & $96$ & $93$ & $89$ & $82$\\ 
$0.2$ & $97$ & $97$ & $95$ & $93$ & $88$ & $83$\\ 
$\pi\quad$$0.3$ & $96$ & $96$ & $94$ & $92$ & $88$ & $85$\\ 
$0.4$ & $94$ & $93$ & $93$ & $91$ & $89$ & $87$\\ 
$0.5$ & $92$ & $92$ & $92$ & $91$ & $90$ & $88$\\ 
$0.6$ & $91$ & $91$ & $90$ & $90$ & $90$ & $88$\\ 
 \hline 
 \end{tabular}
 
 \vspace{2em} 
 
\begin{tabular}{r|rrrrrr}
\hline\hline
 &\multicolumn{6}{c}{$\rho$} \\ 
 $N = 100$ & $0$ & $0.1$ & $0.2$ & $0.3$ & $0.4$ & $0.5$ \\ 
 \hline$0.1$ & $98$ & $98$ & $97$ & $94$ & $89$ & $83$\\ 
$0.2$ & $96$ & $96$ & $95$ & $92$ & $89$ & $85$\\ 
$\pi\quad$$0.3$ & $94$ & $94$ & $93$ & $91$ & $89$ & $87$\\ 
$0.4$ & $92$ & $92$ & $92$ & $91$ & $90$ & $88$\\ 
$0.5$ & $91$ & $91$ & $90$ & $90$ & $89$ & $89$\\ 
$0.6$ & $90$ & $90$ & $90$ & $90$ & $90$ & $89$\\ 
 \hline 
 \end{tabular}
 
 \vspace{2em} 
 
\begin{tabular}{r|rrrrrr}
\hline\hline
 &\multicolumn{6}{c}{$\rho$} \\ 
 $N = 500$ & $0$ & $0.1$ & $0.2$ & $0.3$ & $0.4$ & $0.5$ \\ 
 \hline$0.1$ & $96$ & $96$ & $94$ & $93$ & $90$ & $86$\\ 
$0.2$ & $92$ & $92$ & $91$ & $91$ & $90$ & $89$\\ 
$\pi\quad$$0.3$ & $91$ & $91$ & $91$ & $91$ & $90$ & $90$\\ 
$0.4$ & $90$ & $90$ & $91$ & $90$ & $90$ & $90$\\ 
$0.5$ & $90$ & $90$ & $90$ & $90$ & $90$ & $90$\\ 
$0.6$ & $90$ & $91$ & $90$ & $90$ & $90$ & $90$\\ 
 \hline 
 \end{tabular}
		\label{tab:OLSvsIVsim_cover_TSLS}
	\end{subtable}	
	~
	\begin{subtable}{0.48\textwidth}
		\caption{Na\"{i}ve post-FMSC}
		\begin{tabular}{r|rrrrrr}
\hline\hline
 &\multicolumn{6}{c}{$\rho$} \\ 
 $N = 50$ & $0$ & $0.1$ & $0.2$ & $0.3$ & $0.4$ & $0.5$ \\ 
 \hline$0.1$ & $88$ & $80$ & $58$ & $30$ & $11$ & $4$\\ 
$0.2$ & $88$ & $79$ & $59$ & $34$ & $15$ & $10$\\ 
$\pi\quad$$0.3$ & $87$ & $81$ & $62$ & $39$ & $25$ & $23$\\ 
$0.4$ & $86$ & $80$ & $66$ & $46$ & $38$ & $43$\\ 
$0.5$ & $86$ & $81$ & $68$ & $56$ & $54$ & $62$\\ 
$0.6$ & $85$ & $81$ & $72$ & $66$ & $67$ & $75$\\ 
 \hline 
 \end{tabular}
 
 \vspace{2em} 
 
\begin{tabular}{r|rrrrrr}
\hline\hline
 &\multicolumn{6}{c}{$\rho$} \\ 
 $N = 100$ & $0$ & $0.1$ & $0.2$ & $0.3$ & $0.4$ & $0.5$ \\ 
 \hline$0.1$ & $88$ & $72$ & $36$ & $10$ & $4$ & $4$\\ 
$0.2$ & $87$ & $74$ & $40$ & $17$ & $13$ & $19$\\ 
$\pi\quad$$0.3$ & $86$ & $74$ & $45$ & $29$ & $32$ & $45$\\ 
$0.4$ & $85$ & $74$ & $51$ & $43$ & $54$ & $70$\\ 
$0.5$ & $85$ & $76$ & $59$ & $57$ & $70$ & $84$\\ 
$0.6$ & $85$ & $78$ & $66$ & $68$ & $81$ & $88$\\ 
 \hline 
 \end{tabular}
 
 \vspace{2em} 
 
\begin{tabular}{r|rrrrrr}
\hline\hline
 &\multicolumn{6}{c}{$\rho$} \\ 
 $N = 500$ & $0$ & $0.1$ & $0.2$ & $0.3$ & $0.4$ & $0.5$ \\ 
 \hline$0.1$ & $87$ & $31$ & $8$ & $12$ & $16$ & $24$\\ 
$0.2$ & $84$ & $35$ & $24$ & $42$ & $62$ & $80$\\ 
$\pi\quad$$0.3$ & $83$ & $42$ & $43$ & $70$ & $87$ & $90$\\ 
$0.4$ & $84$ & $49$ & $62$ & $86$ & $90$ & $90$\\ 
$0.5$ & $84$ & $57$ & $76$ & $89$ & $90$ & $90$\\ 
$0.6$ & $86$ & $66$ & $84$ & $90$ & $90$ & $90$\\ 
 \hline 
 \end{tabular}
		\label{tab:OLSvsIVsim_cover_naiveFMSC}
	\end{subtable}
	\caption{Coverage probabilities of nominal 90\% CIs for the OLS versus TSLS simulation experiment from Section \ref{sec:OLSvsIVsim}. Values are given in percentage points, rounded to the nearest whole percent, based on 10,000 simulation draws from the DGP given in Equations \ref{eq:OLSvsIVDGP1}--\ref{eq:OLSvsIVDGP3}.}
\end{table}


Table \ref{tab:OLSvsIVsim_cover_FMSC} gives the actual coverage probability of the conservative, 90\% post-FMSC confidence interval, constructed according to Algorithm \ref{alg:conf}, for the OLS versus TSLS example.
These intervals achieve their nominal minimum coverage for all parameter values but can be quite conservative, particularly for smaller values of $\pi, \rho$ and $N$. 
In particular, coverage never falls below 94\% but occasionally exceeds $99.5\%$.
% \footnote{Recall that coverage probabilities are given in percentage points, rounded to the nearest whole percent. Thus a value of 100, for example, in fact means $\geq 99.5$.}
Some conservatism is inevitable given the procedure, which takes which takes \emph{worst-case} bounds over a collection of intervals.
The real culprit in this example, however, is the TSLS estimator, as we see from Table \ref{tab:OLSvsIVsim_cover_TSLS}.
Although this estimator is correctly specified and is not subject to model selection uncertainty, its textbook 90\% confidence interval dramatically overcovers for smaller values of $\pi$ even if $N$ is fairly large.
This is a manifestation of the weak instruments problem.
This additional source of conservatism is inherited by the two-step post-FMSC intervals.
Results for the minimum-AMSE moment average estimator, given in Table \ref{tab:OLSvsIVsim_cover_AVG}, are similar.

\begin{table}[h]
\footnotesize
\centering
	\begin{subtable}{0.48\textwidth}
		\caption{FMSC}
		\begin{tabular}{r|rrrrrr}
\hline\hline
 &\multicolumn{6}{c}{$\rho$} \\ 
 $N = 50$ & $0$ & $0.1$ & $0.2$ & $0.3$ & $0.4$ & $0.5$ \\ 
 \hline$0.1$ & $100$ & $100$ & $99$ & $99$ & $98$ & $97$\\ 
$0.2$ & $99$ & $99$ & $99$ & $99$ & $98$ & $97$\\ 
$\pi\quad$$0.3$ & $99$ & $99$ & $99$ & $99$ & $98$ & $96$\\ 
$0.4$ & $98$ & $98$ & $98$ & $98$ & $98$ & $95$\\ 
$0.5$ & $97$ & $98$ & $98$ & $98$ & $97$ & $94$\\ 
$0.6$ & $97$ & $97$ & $97$ & $97$ & $96$ & $94$\\ 
 \hline 
 \end{tabular}
 
 \vspace{2em} 
 
\begin{tabular}{r|rrrrrr}
\hline\hline
 &\multicolumn{6}{c}{$\rho$} \\ 
 $N = 100$ & $0$ & $0.1$ & $0.2$ & $0.3$ & $0.4$ & $0.5$ \\ 
 \hline$0.1$ & $100$ & $99$ & $99$ & $99$ & $99$ & $98$\\ 
$0.2$ & $99$ & $99$ & $99$ & $99$ & $99$ & $97$\\ 
$\pi\quad$$0.3$ & $98$ & $98$ & $99$ & $99$ & $98$ & $95$\\ 
$0.4$ & $97$ & $97$ & $98$ & $98$ & $97$ & $94$\\ 
$0.5$ & $97$ & $97$ & $98$ & $97$ & $95$ & $95$\\ 
$0.6$ & $97$ & $97$ & $97$ & $96$ & $95$ & $96$\\ 
 \hline 
 \end{tabular}
 
 \vspace{2em} 
 
\begin{tabular}{r|rrrrrr}
\hline\hline
 &\multicolumn{6}{c}{$\rho$} \\ 
 $N = 500$ & $0$ & $0.1$ & $0.2$ & $0.3$ & $0.4$ & $0.5$ \\ 
 \hline$0.1$ & $99$ & $99$ & $99$ & $99$ & $99$ & $96$\\ 
$0.2$ & $97$ & $97$ & $98$ & $99$ & $97$ & $94$\\ 
$\pi\quad$$0.3$ & $96$ & $97$ & $98$ & $97$ & $95$ & $98$\\ 
$0.4$ & $96$ & $97$ & $97$ & $95$ & $98$ & $98$\\ 
$0.5$ & $96$ & $97$ & $96$ & $97$ & $98$ & $98$\\ 
$0.6$ & $96$ & $97$ & $95$ & $97$ & $97$ & $96$\\ 
 \hline 
 \end{tabular}
		\label{tab:OLSvsIVsim_cover_FMSC}
	\end{subtable}	
	~
	\begin{subtable}{0.48\textwidth}
		\caption{AMSE-Averaging Estimator}
		\begin{tabular}{r|rrrrrr}
\hline\hline
 &\multicolumn{6}{c}{$\rho$} \\ 
 $N = 250$ & $0$ & $0.1$ & $0.2$ & $0.3$ & $0.4$ & $0.5$ \\ 
 \hline$0.1$ & $100$ & $100$ & $100$ & $99$ & $99$ & $97$\\ 
$0.2$ & $99$ & $99$ & $99$ & $99$ & $97$ & $94$\\ 
$\pi\quad$$0.3$ & $97$ & $98$ & $98$ & $98$ & $95$ & $93$\\ 
$0.4$ & $97$ & $97$ & $97$ & $95$ & $92$ & $93$\\ 
$0.5$ & $96$ & $96$ & $94$ & $92$ & $91$ & $91$\\ 
$0.6$ & $95$ & $94$ & $92$ & $88$ & $89$ & $89$\\ 
 \hline 
 \end{tabular}
 
 \vspace{2em} 
 
\begin{tabular}{r|rrrrrr}
\hline\hline
 &\multicolumn{6}{c}{$\rho$} \\ 
 $N = 500$ & $0$ & $0.1$ & $0.2$ & $0.3$ & $0.4$ & $0.5$ \\ 
 \hline$0.1$ & $100$ & $100$ & $100$ & $99$ & $98$ & $96$\\ 
$0.2$ & $99$ & $99$ & $99$ & $98$ & $96$ & $94$\\ 
$\pi\quad$$0.3$ & $98$ & $97$ & $98$ & $96$ & $94$ & $95$\\ 
$0.4$ & $97$ & $97$ & $96$ & $93$ & $94$ & $95$\\ 
$0.5$ & $96$ & $95$ & $93$ & $91$ & $93$ & $92$\\ 
$0.6$ & $95$ & $94$ & $90$ & $89$ & $90$ & $90$\\ 
 \hline 
 \end{tabular}
 
 \vspace{2em} 
 
\begin{tabular}{r|rrrrrr}
\hline\hline
 &\multicolumn{6}{c}{$\rho$} \\ 
 $N = 1000$ & $0$ & $0.1$ & $0.2$ & $0.3$ & $0.4$ & $0.5$ \\ 
 \hline$0.1$ & $100$ & $99$ & $99$ & $99$ & $97$ & $95$\\ 
$0.2$ & $98$ & $99$ & $98$ & $96$ & $95$ & $96$\\ 
$\pi\quad$$0.3$ & $97$ & $98$ & $96$ & $94$ & $96$ & $97$\\ 
$0.4$ & $97$ & $97$ & $94$ & $94$ & $95$ & $95$\\ 
$0.5$ & $96$ & $95$ & $92$ & $93$ & $94$ & $93$\\ 
$0.6$ & $95$ & $92$ & $89$ & $90$ & $91$ & $87$\\ 
 \hline 
 \end{tabular}
		\label{tab:OLSvsIVsim_cover_AVG}
	\end{subtable}
	\caption{Coverage probabilities of simulation-based conservative $90\%$ CIs for the OLS versus TSLS simulation experiment from Section \ref{sec:OLSvsIVsim}. Values are given in percentage points, rounded to the nearest whole percent, based on 10,000 simulation draws from the DGP given in Equations \ref{eq:OLSvsIVDGP1}--\ref{eq:OLSvsIVDGP3}.}
\end{table}

The worry, of course, is not conservatism as such but the attendant increase in confidence interval width.
Accordingly, Tables \ref{tab:OLSvsIVsim_width_FMSC} and \ref{tab:OLSvsIVsim_width_AVG} compare the median width of the simulation-based post-FMSC and minimum-AMSE intervals to that of the TSLS estimator.
A value of 25, for example indicates that the simulation-based interval is 25\% wider than the corresponding interval for the TSLS estimator.
This comparison shows us the inferential cost of carrying out moment selection relative to simply using the correctly-specified TSLS estimator and calling it a day.
Moment selection is not a free lunch: the averaging and post-selection intervals are wider than those of the TSLS estimator, sometimes considerably so.
Intriguingly, the minimum-AMSE intervals are generally much shorter than the post-FMSC intervals in spite of being somewhat more conservative.

\begin{table}[h]
\footnotesize
\centering
	\begin{subtable}{0.48\textwidth}
		\caption{post-FMSC Estimator}
		\begin{tabular}{r|rrrrrr}
\hline\hline
 &\multicolumn{6}{c}{$\rho$} \\ 
 $N = 250$ & $0$ & $0.1$ & $0.2$ & $0.3$ & $0.4$ & $0.5$ \\ 
 \hline$0.1$ & $40$ & $40$ & $41$ & $41$ & $42$ & $42$\\ 
$0.2$ & $42$ & $42$ & $43$ & $45$ & $46$ & $48$\\ 
$\pi\quad$$0.3$ & $43$ & $43$ & $44$ & $46$ & $48$ & $49$\\ 
$0.4$ & $43$ & $44$ & $44$ & $45$ & $45$ & $46$\\ 
$0.5$ & $43$ & $43$ & $42$ & $42$ & $43$ & $42$\\ 
$0.6$ & $42$ & $41$ & $39$ & $39$ & $38$ & $35$\\ 
 \hline 
 \end{tabular}
 
 \vspace{2em} 
 
\begin{tabular}{r|rrrrrr}
\hline\hline
 &\multicolumn{6}{c}{$\rho$} \\ 
 $N = 500$ & $0$ & $0.1$ & $0.2$ & $0.3$ & $0.4$ & $0.5$ \\ 
 \hline$0.1$ & $40$ & $41$ & $41$ & $42$ & $44$ & $46$\\ 
$0.2$ & $42$ & $43$ & $44$ & $47$ & $50$ & $52$\\ 
$\pi\quad$$0.3$ & $43$ & $43$ & $46$ & $48$ & $49$ & $49$\\ 
$0.4$ & $43$ & $44$ & $45$ & $45$ & $46$ & $44$\\ 
$0.5$ & $43$ & $43$ & $42$ & $42$ & $39$ & $27$\\ 
$0.6$ & $42$ & $40$ & $39$ & $37$ & $28$ & $19$\\ 
 \hline 
 \end{tabular}
 
 \vspace{2em} 
 
\begin{tabular}{r|rrrrrr}
\hline\hline
 &\multicolumn{6}{c}{$\rho$} \\ 
 $N = 1000$ & $0$ & $0.1$ & $0.2$ & $0.3$ & $0.4$ & $0.5$ \\ 
 \hline$0.1$ & $41$ & $41$ & $43$ & $45$ & $47$ & $50$\\ 
$0.2$ & $42$ & $44$ & $47$ & $50$ & $52$ & $53$\\ 
$\pi\quad$$0.3$ & $43$ & $44$ & $48$ & $49$ & $49$ & $45$\\ 
$0.4$ & $43$ & $44$ & $46$ & $45$ & $37$ & $21$\\ 
$0.5$ & $43$ & $43$ & $42$ & $37$ & $21$ & $19$\\ 
$0.6$ & $42$ & $39$ & $38$ & $24$ & $19$ & $19$\\ 
 \hline 
 \end{tabular}
		\label{tab:OLSvsIVsim_width_FMSC}
	\end{subtable}	
	~
	\begin{subtable}{0.48\textwidth}
		\caption{AMSE-Averaging Estimator}
		\begin{tabular}{r|rrrrrr}
\hline\hline
 &\multicolumn{6}{c}{$\rho$} \\ 
 $N = 250$ & $0$ & $0.1$ & $0.2$ & $0.3$ & $0.4$ & $0.5$ \\ 
 \hline$0.1$ & $30$ & $35$ & $36$ & $35$ & $35$ & $35$\\ 
$0.2$ & $31$ & $34$ & $35$ & $32$ & $38$ & $35$\\ 
$\pi\quad$$0.3$ & $35$ & $33$ & $34$ & $35$ & $35$ & $35$\\ 
$0.4$ & $34$ & $34$ & $35$ & $35$ & $35$ & $37$\\ 
$0.5$ & $35$ & $34$ & $34$ & $32$ & $35$ & $34$\\ 
$0.6$ & $36$ & $33$ & $31$ & $32$ & $33$ & $30$\\ 
 \hline 
 \end{tabular}
 
 \vspace{2em} 
 
\begin{tabular}{r|rrrrrr}
\hline\hline
 &\multicolumn{6}{c}{$\rho$} \\ 
 $N = 500$ & $0$ & $0.1$ & $0.2$ & $0.3$ & $0.4$ & $0.5$ \\ 
 \hline$0.1$ & $32$ & $27$ & $33$ & $37$ & $30$ & $36$\\ 
$0.2$ & $33$ & $37$ & $33$ & $36$ & $36$ & $39$\\ 
$\pi\quad$$0.3$ & $34$ & $34$ & $34$ & $38$ & $37$ & $38$\\ 
$0.4$ & $35$ & $35$ & $36$ & $37$ & $37$ & $35$\\ 
$0.5$ & $37$ & $33$ & $34$ & $35$ & $33$ & $31$\\ 
$0.6$ & $35$ & $32$ & $31$ & $32$ & $30$ & $25$\\ 
 \hline 
 \end{tabular}
 
 \vspace{2em} 
 
\begin{tabular}{r|rrrrrr}
\hline\hline
 &\multicolumn{6}{c}{$\rho$} \\ 
 $N = 1000$ & $0$ & $0.1$ & $0.2$ & $0.3$ & $0.4$ & $0.5$ \\ 
 \hline$0.1$ & $30$ & $30$ & $37$ & $34$ & $37$ & $37$\\ 
$0.2$ & $35$ & $37$ & $35$ & $36$ & $41$ & $41$\\ 
$\pi\quad$$0.3$ & $34$ & $34$ & $36$ & $36$ & $37$ & $38$\\ 
$0.4$ & $37$ & $34$ & $37$ & $35$ & $34$ & $28$\\ 
$0.5$ & $36$ & $34$ & $34$ & $34$ & $27$ & $22$\\ 
$0.6$ & $37$ & $32$ & $31$ & $28$ & $23$ & $20$\\ 
 \hline 
 \end{tabular}
		\label{tab:OLSvsIVsim_width_AVG}
	\end{subtable}
	\caption{Median width of two-step, simulation-based conservative $90\%$ CI relative to that of a traditional 90\% CI for the TSLS estimator in the OLS versus TSLS example from Section \ref{sec:OLSvsIVsim}. Values are given in percentage points, rounded to the nearest whole percent, based on 10,000 simulation draws from the DGP given in Equations \ref{eq:OLSvsIVDGP1}--\ref{eq:OLSvsIVDGP3}.}
\end{table}

Turning our attention now to the choosing instrumental variables simulation experiment from Section \ref{sec:chooseIVsim}, Table \ref{tab:chooseIVsim_cover_FMSC} gives the coverage probability and Table \ref{tab:chooseIVsim_width_FMSC} the median relative width of the conservative, 90\%, simulation-based, post-FMSC confidence interval.
In this case, the width calculation is relative to the valid estimator, the TSLS estimator that includes the exogenous instruments $z_1, z_2, z_3$ but excludes the potentially endogenous instrument $w$.
Here  the simulation-based intervals are far less conservative and occasionally undercover slightly.
The worst case, 81\% actual coverage compared to 90\% nominal coverage, occurs when $N=50, \gamma = 0.6, \rho = 0.5$.
This problem stems from the fact that traditional interval for the valid estimator systematically under-covers when $N = 50$ or 100.\footnote{For details, see Table \ref{tab:chooseIVsim_cover_Valid} in Appendix \ref{sec:CIsupplement}.}
Nevertheless, the simulation-based interval works well in this example: in the worst case, its median width is only 22\% greater than that of the valid estimator.

\begin{table}[h]
\footnotesize
\centering
	\begin{subtable}{0.48\textwidth}
		\caption{Coverage Probability}
		\label{tab:chooseIVsim_cover_FMSC}
		\begin{tabular}{r|rrrrrr}
\hline\hline
 &\multicolumn{6}{c}{$\rho$} \\ 
 $N = 50$ & $0$ & $0.1$ & $0.2$ & $0.3$ & $0.4$ & $0.5$ \\ 
 \hline$0$ & $89$ & $89$ & $89$ & $89$ & $90$ & $89$\\ 
$0.1$ & $90$ & $89$ & $87$ & $88$ & $89$ & $89$\\ 
$\gamma\quad$$0.2$ & $90$ & $88$ & $87$ & $86$ & $88$ & $90$\\ 
$0.3$ & $91$ & $89$ & $87$ & $85$ & $86$ & $89$\\ 
$0.4$ & $92$ & $90$ & $87$ & $84$ & $84$ & $86$\\ 
$0.5$ & $92$ & $91$ & $89$ & $85$ & $82$ & $83$\\ 
 \hline 
 \end{tabular}
 
 \vspace{2em} 
 
\begin{tabular}{r|rrrrrr}
\hline\hline
 &\multicolumn{6}{c}{$\rho$} \\ 
 $N = 100$ & $0$ & $0.1$ & $0.2$ & $0.3$ & $0.4$ & $0.5$ \\ 
 \hline$0$ & $91$ & $91$ & $91$ & $91$ & $91$ & $92$\\ 
$0.1$ & $92$ & $90$ & $90$ & $91$ & $91$ & $92$\\ 
$\gamma\quad$$0.2$ & $93$ & $90$ & $88$ & $89$ & $91$ & $91$\\ 
$0.3$ & $94$ & $90$ & $86$ & $87$ & $90$ & $91$\\ 
$0.4$ & $94$ & $92$ & $86$ & $85$ & $88$ & $91$\\ 
$0.5$ & $93$ & $93$ & $87$ & $82$ & $85$ & $89$\\ 
 \hline 
 \end{tabular}
 
 \vspace{2em} 
 
\begin{tabular}{r|rrrrrr}
\hline\hline
 &\multicolumn{6}{c}{$\rho$} \\ 
 $N = 250$ & $0$ & $0.1$ & $0.2$ & $0.3$ & $0.4$ & $0.5$ \\ 
 \hline$0$ & $93$ & $94$ & $93$ & $94$ & $93$ & $94$\\ 
$0.1$ & $94$ & $93$ & $92$ & $92$ & $92$ & $91$\\ 
$\gamma\quad$$0.2$ & $95$ & $90$ & $91$ & $91$ & $91$ & $92$\\ 
$0.3$ & $95$ & $90$ & $88$ & $91$ & $92$ & $92$\\ 
$0.4$ & $95$ & $91$ & $85$ & $90$ & $91$ & $92$\\ 
$0.5$ & $95$ & $92$ & $83$ & $88$ & $91$ & $92$\\ 
 \hline 
 \end{tabular}
	\end{subtable}	
	~
	\begin{subtable}{0.48\textwidth}
		\caption{Relative Median Width}
		\label{tab:chooseIVsim_width_FMSC}
		\begin{tabular}{r|rrrrrr}
\hline\hline
 &\multicolumn{6}{c}{$\rho$} \\ 
 $N = 50$ & $0$ & $0.1$ & $0.2$ & $0.3$ & $0.4$ & $0.5$ \\ 
 \hline$0$ & $19$ & $19$ & $20$ & $20$ & $21$ & $23$\\ 
$0.1$ & $20$ & $18$ & $17$ & $17$ & $17$ & $18$\\ 
$\gamma\quad$$0.2$ & $21$ & $18$ & $16$ & $16$ & $16$ & $17$\\ 
$0.3$ & $21$ & $17$ & $15$ & $14$ & $15$ & $17$\\ 
$0.4$ & $22$ & $17$ & $14$ & $12$ & $13$ & $15$\\ 
$0.5$ & $22$ & $17$ & $13$ & $12$ & $12$ & $13$\\ 
 \hline 
 \end{tabular}
 
 \vspace{2em} 
 
\begin{tabular}{r|rrrrrr}
\hline\hline
 &\multicolumn{6}{c}{$\rho$} \\ 
 $N = 100$ & $0$ & $0.1$ & $0.2$ & $0.3$ & $0.4$ & $0.5$ \\ 
 \hline$0$ & $20$ & $19$ & $20$ & $20$ & $20$ & $22$\\ 
$0.1$ & $20$ & $18$ & $16$ & $15$ & $14$ & $14$\\ 
$\gamma\quad$$0.2$ & $21$ & $17$ & $14$ & $13$ & $13$ & $14$\\ 
$0.3$ & $22$ & $16$ & $13$ & $10$ & $11$ & $14$\\ 
$0.4$ & $21$ & $15$ & $11$ & $10$ & $10$ & $13$\\ 
$0.5$ & $21$ & $15$ & $11$ & $9$ & $10$ & $12$\\ 
 \hline 
 \end{tabular}
 
 \vspace{2em} 
 
\begin{tabular}{r|rrrrrr}
\hline\hline
 &\multicolumn{6}{c}{$\rho$} \\ 
 $N = 250$ & $0$ & $0.1$ & $0.2$ & $0.3$ & $0.4$ & $0.5$ \\ 
 \hline$0$ & $20$ & $19$ & $19$ & $19$ & $19$ & $19$\\ 
$0.1$ & $21$ & $18$ & $14$ & $12$ & $10$ & $10$\\ 
$\gamma\quad$$0.2$ & $22$ & $16$ & $12$ & $9$ & $9$ & $10$\\ 
$0.3$ & $21$ & $14$ & $10$ & $7$ & $9$ & $12$\\ 
$0.4$ & $20$ & $13$ & $9$ & $7$ & $8$ & $12$\\ 
$0.5$ & $20$ & $13$ & $9$ & $7$ & $7$ & $10$\\ 
 \hline 
 \end{tabular}
	\end{subtable}
\caption{Performance of the simulation-based, conservative 90\% post-FMSC confidence interval in the choosing instrumental variables simulation from Section \ref{sec:chooseIVsim}. The left panel gives coverage probabilities, and the right panel gives median widths relative to that of a traditional 90\% interval for the valid estimator. Values are given in percentage points, rounded to the nearest whole percent, based on 10,000 simulation draws from the DGP given in Equations \ref{eq:chooseIVDGP1}--\ref{eq:chooseIVDGP3}.}
\end{table}

Although the simulation-based intervals work fairly well, two caveats are in order.
First, when the usual first-order asymptotic theory begins to break down, such as a weak instruments example, the simulation-based intervals can inherit an under-- or over--coverage problem from the valid estimator.
Second, moment selection comes with a cost: the simulation-based intervals are on average wider than a textbook confidence interval for the valid estimator, as we would expect given the impossibility results for post-selection inference outlined in \cite{LeebPoetscher2005}.\footnote{The intervals presented here could potentially be shortened by optimizing width over $\alpha$ while holding $\alpha + \delta$ fixed at 0.1. For more discussion of this idea, see \cite{ClaeskensHjortbook} and \cite{McCloskey}.}
As described above, the primary goal of the the FMSC is \emph{estimation} rather than inference.
Once the decision to carry out moment selection has been taken, however, one cannot simply ignore this fact and report the usual confidence intervals.
Algorithm \ref{alg:conf} provides a way to carry out honest inference post-selection and construct confidence intervals for complicated objects such as the minimum-AMSE averaging estimator from Section \ref{sec:momentavgexample}.
More to the point, although formal moment selection is relatively rare, \emph{informal} moment selection is extremely common in applied work.
Downward $J$-tests, DHW tests and the like are a standard part of the applied econometrician's toolkit.
Because it can be employed to construct confidence intervals that account for the effects of specification searches, Algorithm \ref{alg:conf} can provide a valuable robustness check, as I explore in the empirical example that follows.
