%!TEX root = main.tex
\section{Introduction}
In finite samples, the addition of a slightly endogenous but highly relevant instrument can reduce estimator variance by far more than bias is increased. 
Building on this observation, I propose a novel moment selection criterion for generalized method of moments (GMM) estimation: the focused moment selection criterion (FMSC). 
Rather than selecting only valid moment conditions, the FMSC chooses from a set of potentially mis-specified moment conditions based on the asymptotic mean squared error (AMSE) of their associated GMM estimators of a user-specified scalar target parameter $\mu$.
To ensure a meaningful bias-variance tradeoff in the limit, I employ a drifting asymptotic framework in which mis-specification, while present for any fixed sample size, vanishes asymptotically.
In the presence of such \emph{locally mis-specified} moment conditions, GMM remains consistent although, centered and rescaled, its limiting distribution displays an asymptotic bias. Adding an additional mis-specified moment condition introduces a further source of bias while reducing asymptotic variance. 
The idea behind the FMSC is to trade off these two effects in the limit as an approximation to finite sample behavior.\footnote{When finite-sample MSE is undefined, AMSE comparisons remain meaningful: see Appendix \ref{append:trim}.}

 
I suppose that two blocks of moment conditions are available: one that is assumed correctly specified, and another that may not be.
This mimics the situation faced by an applied researcher who begins with a ``baseline'' set of relatively weak maintained assumptions and must decide whether to impose any of a collection of stronger but also more controversial ``suspect'' assumptions.
When the (correctly specified) baseline moment conditions identify the model, the FMSC provides an asymptotically unbiased estimator of AMSE, allowing us select over the suspect moment conditions.\footnote{When this is not the case, it remains possible to use the AMSE framework to carry out a sensitivity analysis: see Appendix \ref{sec:digress}.}

Continuing under the local mis-specification framework, I go on to derive the limit distribution of ``moment average estimators,'' data-dependent weighted averages of estimators based on different moment conditions.
These estimators are interesting in their own right and can be used to study the important problem of inference post-selection.
I propose two simulation-based procedures for constructing confidence intervals for moment average and post-selection estimators including the FMSC: a ``1-Step'' confidence interval and a ``2-Step'' confidence interval.
The 2-Step interval guarantees asymptotically valid inference in exchange for conservatism and a greater computational burden.
The 1-Step interval is easier to compute and much less conservative.
While this interval can suffer from small size distortions, it provides a dramatic improvement over the na\"{i}ve post-selection intervals typically reported in applied work.

While my methods apply to general GMM models, I focus on two simple but empirically relevant examples: choosing between ordinary least squares (OLS) and two-stage least squares (TSLS) estimators, and selecting instruments in linear instrumental variables (IV) models. 
In the OLS versus TSLS example the FMSC takes a particularly transparent form, providing a risk-based justification for the Durbin-Hausman-Wu test, and leading to a novel ``minimum-AMSE'' averaging estimator that combines OLS and TSLS.
The FMSC, averaging estimator, and related confidence interval procedures work well in practice, as I demonstrate in a series of simulation experiments and an empirical example from development economics.

The FMSC and minimum-AMSE averaging estimator considered here are derived for a scalar parameter interest, as this is the most common situation encountered in applied work.\footnote{For an extension of the FMSC to vector target parameters, see Appendix \ref{append:mult}.}
As a consequence, Stein-type results do not apply: it is impossible to construct an estimator with uniformly lower risk than the ``valid'' estimator that uses only the baseline moment conditions.
Nevertheless, as my simulation results show, selection and averaging can substantially outperform the valid estimator over large regions of the parameter space, particularly when the ``suspect'' moment conditions are highly informative and \emph{nearly} correct.
This is precisely the situation for which the FMSC is intended.

My approach to moment selection is inspired by the focused information criterion of \citet{ClaeskensHjort2003}, a model selection criterion for maximum likelihood estimation. 
Like \citet{ClaeskensHjort2003}, I study AMSE-based selection under mis-specification in a drifting asymptotic framework. 
In contradistinction, however, I consider moment rather than model selection, and general GMM rather than maximum likelihood estimation.
\cite{Schorfheide2005} uses a similar approach to select over forecasts constructed from mis-specified vector autoregression models, developed independently of the FIC. 
While the use of locally mis-specified moment conditions dates back at least as far as \cite{Newey1985}, the idea of using this framework for AMSE-based moment selection, however, is novel.

The existing literature on moment selection primarily aims to consistently select all correctly specified moment conditions while eliminating all invalid ones\footnote{Under the local mis-specification asymptotics considered below, consistent moment selection criteria simply choose \emph{all} available moment conditions. For details, see Theorem \ref{pro:andrews}.}
This idea begins with \cite{Andrews1999} and is extended by  \cite{AndrewsLu} and \cite{HongPrestonShum}.
More recently, \cite{Liao} proposes a shrinkage procedure for consistent GMM moment selection and estimation. 
In a similar vein, \cite{CanerHanLee} extend and generalize earlier work by \cite{Caner2009} on LASSO-type model selection for GMM to carry out simultaneous model and moment selection via an adaptive elastic net penalty. 
Whereas these proposals examine only the validity of the moment conditions under consideration, the FMSC balances validity against relevance to minimize AMSE.
Although \cite{HallPeixe2003} and \cite{ChengLiao} do consider relevance, their aim is to avoid including redundant moment conditions after consistently eliminating invalid ones.
Some other papers that propose choosing, or combining, instruments to minimize MSE include \cite{DonaldNewey2001}, \cite{DonaldImbensNewey2009}, and \cite{KuersteinerOkui2010}.
Unlike the FMSC, however, these papers consider the \emph{higher-order} bias that arises from including many valid instruments rather than the first-order bias that arises from the use of invalid instruments.

Another distinguishing feature of the FMSC is focus: rather than a one-size-fits-all criterion, the FMSC is really a method of constructing application-specific moment selection criteria.
Consider, for example, a dynamic panel model.
If your target parameter is a long-run effect while mine is a contemporaneous effect, there is no reason to suppose \emph{a priori} that we should use the same moment conditions in estimation, even if we share the same model and dataset.
The FMSC explicitly takes this difference of research goals into account.

Like Akaike's Information Criterion (AIC), the FMSC is a \emph{conservative} rather than consistent selection procedure, as it remains random even in the limit.	
While consistency is a desirable property in many settings, the situation is more complex for model and moment selection: consistent and conservative selection procedures have different strengths, but these strengths cannot be combined \citep{Yang2005}.
The goal of this paper is estimators with low risk.
Viewed from this perspective consistent selection criteria suffer from a serious defect: they exhibit unbounded minimax risk \citep{LeebPoetscher2008}.  
Conservative criteria such as the FMSC do not suffer from this shortcoming.
Moreover, as discussed in more detail below, the asymptotics of consistent selection paint a misleading picture of the effects of moment selection on inference.
For these reasons, the fact that the FMSC is conservative rather than consistent is an asset in the present context.

Because it studies inference post-moment selection, this paper relates to a vast literature on ``pre-test'' estimators.
For an overview, see \citet{LeebPoetscher2005, LeebPoetscher2009}.
There are several proposals to construct valid confidence intervals post-model selection, including \cite{Kabaila1998}, \cite{HjortClaeskens} and \cite{KabailaLeeb2006}. 
To my knowledge, however, this is the first paper to treat the problem in general for post-moment selection and moment average estimators in the presence of mis-specification.
Some related results appear in \cite{Berkowitz2008}, \cite{Berkowitz2012}, \cite{Guggenberger2010}, \cite{Guggenberger2012}, \cite{GuggenbergerKumar}, and \cite{Caner2014}.
While I developed the simulation-based, two-stage confidence interval procedure described below by analogy to a suggestion in \cite{ClaeskensHjortbook}, \cite{Leeb} kindly pointed out that similar constructions have appeared in \cite{Loh1985}, \cite{Berger1994}, and \cite{Silvapulle1996}. More recently, \cite{McCloskey} takes a similar approach to study a class of non-standard testing problems.

The framework within which I study moment averaging is related to the frequentist model average estimators of \cite{HjortClaeskens}.
Two other papers that consider weighting estimators based on different moment conditions are \cite{Xiao} and \cite{ChenChavezLinton}.
Whereas these papers combine estimators computed using valid moment conditions to achieve a minimum variance estimator, I combine estimators computed using potentially invalid conditions with the aim of reducing estimator AMSE.
A related idea underlies the combined moments (CM) estimator of \cite{Judge2007}.
For a different approach to combining OLS and TSLS estimators, similar in spirit to the Stein-estimator and developed independently of the work presented here, see \cite{HansenStein}. 
\cite{ChengLiaoShi} provide related results for Stein-type moment averaging in a GMM context with potentially mis-specified moment conditions.

A limitation of the results presented here is that they are based upon the assumption of strong identification and a fixed number of moment conditions.
Extending the idea behind the FMSC to allow for weak identification or a large number of moment conditions is a challenging topic that I leave for future research.


The remainder of the paper is organized as follows.
Section \ref{sec:asymp} describes the asymptotic framework and Section \ref{sec:FMSC} derives the FMSC, both in general and for two specific examples: OLS versus TSLS and choosing instrumental variables.
Section \ref{sec:avg} studies moment average estimators and shows how they can be used to construct valid confidence intervals post-moment selection.
Section \ref{sec:simulations} presents simulation results and Section \ref{sec:application} considers an empirical example from development economics.
Proofs appear at the end of the document; computational details and additional material appear in a Supplementary Appendix. 
