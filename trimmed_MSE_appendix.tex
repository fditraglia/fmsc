%!TEX root = main.tex
\section{Trimmed MSE}
\label{append:trim}
    Even in situations where finite sample MSE does not exist, it is still meaningful to consider comparisons of asymptotic MSE.
    To make the connection between the finite-sample and limit experiment a bit tidier in this case we can work in terms of \emph{trimmed} MSE,  following Hansen \cite{HansenShrink}. 
    To this end, define
    \begin{align*}
      MSE_n(\widehat{\mu}_S, \zeta) &= E\left[ \min\left\{ n(\widehat{\mu} - \mu_0)^2, \zeta  \right\} \right]\\
      AMSE(\widehat{\mu}_S) &= \lim_{\zeta \rightarrow \infty} \liminf_{n\rightarrow \infty} MSE_n(\widehat{\mu}_S, \zeta)
    \end{align*}
    where $\zeta$ is a positive constant that bounds the expectation for finite $n$.
    By Corollary 3.1 of my paper, $\sqrt{n}(\widehat{\mu}_S-\mu_0) \rightarrow_d \Lambda$ where $\Lambda$ is a normally distributed random variable.
    Thus, by Lemma 1 of \cite{HansenShrink}, we have $AMSE(\widehat{\mu}_S) = E[\Lambda^2]$.
    Thus, working with a sequence of trimmed MSE functions leaves AMSE unchanged while ensuring that finite-sample risk is bounded.
    This justifies the approximation $MSE_n(\widehat{\mu}_S, \zeta) \approx E[\Lambda^2]$ for large $n$ and $\zeta$.
    In a simulation exercise in which ordinary MSE does not exist, for example instrumental variables with a single instrument, one could remove the largest 1\% of simulation draws in absolute value and evaluate the performance of the FMSC against the empirical MSE calculated for the remaining draws.

