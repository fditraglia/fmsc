%!TEX root = main.tex
\subsection{Moment Averaging for the OLS versus TSLS Example}
\label{sec:momentavgexample}
Although it is a special case of moment averaging, moment selection is a somewhat crude procedure: it gives full weight to the estimator that minimizes the moment selection criterion no matter how close its nearest competitor lies. 
Accordingly, when competing moment sets have similar criterion values in the population, sampling variation can be \emph{magnified} in the selected estimator. 
In some settings it is possible to derive Stein-type results for moment averaging estimators (e.g.\ \cite{Hansen2014} and \cite{ChengLiaoShi}).
Such results lead to average schemes with uniformly lower risk than the ``valid'' estimator.
Stein-type results are unavailable, however, in the case of a scalar target parameter and hence cannot be used to guide the construction of moment averaging weights for the setting considered in this paper.

So how should one construct weights for this case?
One possibility is to adapt a proposal from \cite{Burnhametal}, who suggest averaging a collection of competing maximum likelihood estimator with weights of the form 
$w_k =\left. \exp(-I_k/2)\right/\sum_{i=1}^K \exp(-I_i/2)$
where $I_k$ is an information criterion evaluated for model $k$, and $i$ indexes the set of $K$ candidate models. 
This expression, constructed by an analogy with Bayesian model averaging, gives more weight to models with lower values of the information criterion but non-zero weight to all models. 
Applying a slightly more general form of this idea, suggested by \cite{ClaeskensHjortbook}, to the moment selection criteria examined above we might consider 
\[		\widehat{\omega}_S = \left.\exp\left\{-\frac{\kappa}{2} \mbox{MSC}(S)\right\}\right/\sum_{S' \in \mathscr{S}}\exp\left\{-\frac{\kappa}{2} \mbox{MSC}(S')\right\}\]
where MSC$(\cdot)$ is a moment selection criterion and the parameter $\kappa \geq 0$ varies the uniformity of the weighting. 
As $\kappa \rightarrow 0$ the weights become more uniform; as $\kappa \rightarrow \infty$ they approach the moment selection procedure given by minimizing the corresponding criterion. 
Setting $\kappa = 1$ gives the \cite{Burnhametal} weights.

Some preliminary simulation results, reported in an earlier draft of this paper, suggest that exponential weighting can indeed provide MSE improvements.
The difficulty, however, lies in choosing an appropriate value for $\kappa$.
In at least some applications, however, there is a compelling alternative to the exponential weighting scheme: one can instead derive weights \emph{analytically} to minimize AMSE within the FMSC framework.
This immediately suggests a plug-in estimator of the optimal weights along the lines of the FMSC estimate of AMSE.
To illustrate this idea, I revisit the OLS versus TSLS example from Section \ref{sec:OLSvsIVExample}.
Let $\widetilde{\beta}(\omega)$ be a convex combination of the OLS and TSLS estimators, namely  
\begin{equation}
	\widetilde{\beta}(\omega) = \omega \widehat{\beta}_{OLS} + (1 - \omega) \widetilde{\beta}_{TSLS}
\end{equation}
where $\omega \in [0,1]$ is the weight given to the OLS estimator.
\begin{thm}
	\label{thm:OLSvsIVavg} 
	Under the conditions of Theorem \ref{thm:OLSvsIV}, the AMSE of the weighted-average estimator $\sqrt{n}\left[\widehat{\beta}(\omega) - \beta \right]$ is minimized over $\omega \in [0,1]$ by taking $\omega = \omega^*$ where
	$$ \omega^* = \left[1 + \frac{\tau^2/\sigma_x^4}{\sigma_\epsilon^2(1/\gamma^2 - 1/\sigma_x^2)}\right]^{-1} = \left[1 + \frac{\mbox{ABIAS(OLS)}^2}{\mbox{AVAR(TSLS)}-\mbox{AVAR(OLS)}} \right]^{-1}.$$
\end{thm}

The preceding result has several important consequences. 
First, since the variance of the TSLS estimator is always strictly greater than that of the OLS estimator, the optimal value of $\omega$ \emph{cannot} be zero. 
No matter how strong the endogeneity of $x$ as measured by $\tau$, we should always give some weight to the OLS estimator. 
Second, when $\tau = 0$ the optimal value of $\omega$ is one. If $x$ is exogenous, OLS is strictly preferable to TSLS. 
Third, the optimal weights depend on the strength of the instruments $\mathbf{z}$ as measured by $\gamma$.
All else equal, the stronger the instruments, the less weight we should give to OLS.
To operationalize the AMSE-optimal averaging estimator suggested from Theorem \ref{thm:OLSvsIVavg}, I define the plug-in estimator 
\begin{equation}
	\widehat{\beta}^*_{AVG} = \widehat{\omega}^* \widehat{\beta}_{OLS} + (1 - \widehat{\omega}^*)\widetilde{\beta}_{TSLS}
	\label{eq:OLSvsIV_AVG1}
\end{equation}
where
\begin{equation}
\widehat{\omega }^* = \left[1 + \frac{\max \left\{0, \; \left(\widehat{\tau}^2 - \widehat{\sigma}_\epsilon^2\widehat{\sigma}_x^2  \left(\widehat{\sigma}_x^2/\widehat{\gamma}^2 - 1 \right) \right)/\;\widehat{\sigma}_x^4 \right\}}{\widehat{\sigma}_\epsilon^2 (1/\widehat{\gamma}^2 - 1/\widehat{\sigma}_x^2)}\right]^{-1}
	\label{eq:OLSvsIV_AVG2}
\end{equation}
This expression employs the same consistent estimators of $\sigma_x^2, \gamma$ and $\sigma_{\epsilon}$ as the FMSC expressions from Section \ref{sec:OLSvsIVExample}.
To ensure that $\widehat{\omega}^*$ lies in the interval $[0,1]$, however, I use a \emph{positive part} estimator for $\tau^2$, namely $\max\{0, \; \widehat{\tau}^2 - \widehat{V}\}$ rather than $\widehat{\tau}^2 - \widehat{V}$.\footnote{While $\widehat{\tau}^2 - \widehat{V}$ is an asymptotically unbiased estimator of $\tau^2$ it \emph{can} negative.}
In the following section I show how one can construct a valid confidence interval for $\widehat{\beta}^*$ and related estimators.
