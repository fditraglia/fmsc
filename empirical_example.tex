%!TEX root = main.tex
\section{Empirical Example: Geography or Institutions?}
\label{sec:application}
\cite{Carstensen2006} address a controversial question from the development literature: what is the causal effect of geography on income per capita after controlling for the quality of institutions?
A number of well-known studies find little or no direct effect of geographic endowments \citep{Acemoglu,Rodrik,Easterly}. \cite{Sachs}, on the other hand, shows that malaria transmission, a variable largely driven by ecological conditions, directly influences the level of per capita income, even after controlling for institutions.
Because malaria transmission is very likely endogenous, Sachs uses a measure of ``malaria ecology,'' constructed to be exogenous both to present economic conditions and public health interventions, as an instrument. 
\cite{Carstensen2006} extend Sachs's work using the following baseline regression for a sample of 44 countries:
\begin{equation}
	\mbox{ln\emph{gdpc}}_i = \beta_1 + \beta_2 \cdot \mbox{\emph{institutions}}_i + \beta_3 \cdot \mbox{\emph{malaria}}_i + \epsilon_i
\end{equation}
This model augments the baseline specification of \cite{Acemoglu} to include a direct effect of malaria transmission which, like institutions, is treated as endogenous.\footnote{Due to a lack of data for certain instruments, \cite{Carstensen2006} work with a smaller sample of countries than \cite{Acemoglu}.} 
Considering a variety of measures of both institutions and malaria transmission, and a number of instrument sets, \cite{Carstensen2006} find large negative effects of malaria transmission, lending support to Sach's conclusion.

In this section, I revisit and expand upon the instrument selection exercise given in Table 2 of \cite{Carstensen2006} using the FMSC and corrected confidence intervals described above. 
All results in this section are calculated by TSLS using the formulas from Section \ref{sec:chooseIVexample} and the variables described in Table \ref{tab:desc}, with ln\emph{gdpc} as the outcome  variable and \emph{rule} and \emph{malfal} as measures of institutions and malaria transmission.
I this exercise I imagine two hypothetical econometricians.
The first, like \cite{Sachs} and \cite{Carstensen2006}, seeks the best possible estimate of the causal effect of malaria transmission, $\beta_3$, after controlling for institutions by selecting over a number of possible instruments.
The second, in contrast, seeks the best possible estimate of the causal effect of \emph{institutions}, $\beta_2$, after controlling for malaria transmission by selecting over the same collection of instruments.
After estimating their desired target parameters, both econometricians also wish to report valid confidence intervals that account for the additional uncertainty introduced by instrument selection.
For the purposes of this example, to illustrate the results relevant to each of my hypothetical researchers, I take each of $\beta_2$ and $\beta_3$ \emph{in turn} as the target parameter.\footnote{A researcher interested in \emph{both} $\beta_2$ and $\beta_3$, however, should not proceed in this fashion, as it could lead to contradictory inferences. 
Instead, she should define a target parameter that includes both $\beta_2$ and $\beta_3$.}


\begin{table}[!tbp]
\small
\centering
\begin{tabular}{lll}
\hline \hline
Name& Description &\\
\hline
ln\emph{gdpc}&Real GDP/capita at PPP, 1995 International Dollars &Outcome\\
\emph{rule}&Institutional quality (Average Governance Indicator)&Regressor\\
\emph{malfal}&Fraction of population at risk of malaria transmission, 1994&Regressor\\
ln\emph{mort}&Log settler mortality (per 1000 settlers), early 19th century&Baseline\\
\emph{maleco}&Index of stability of malaria transmission&Baseline\\
\emph{frost}&Prop.\ of land receiving at least 5 days of frost in winter&Climate\\
\emph{humid}&Highest temp. in month with highest avg.\ afternoon humidity&Climate\\
\emph{latitude}&Distance from equator (absolute value of latitude in degrees)&Climate \\
\emph{eurfrac}&Fraction of pop.\ that speaks major West.\ European Language&Europe \\
\emph{engfrac}&Fraction of pop.\ that speaks English&Europe\\
\emph{coast}&Proportion of land area within 100km of sea coast&Openness\\
\emph{trade}&Log Frankel-Romer predicted trade share&Openness\\
\hline
\end{tabular}
\caption{Description of variables for Empirical Example.}
\label{tab:desc}
\end{table}

To apply the FMSC to the present example, we require a minimum of two valid instruments besides the constant term. 
Based on the arguments given by \cite{Acemoglu} and \cite{Sachs}, I proceed under the assumption that ln\emph{mort} and \emph{maleco}, measures of early settler mortality and malaria ecology, are exogenous.
Rather than selecting over all 128 possible instrument sets, I consider eight specifications formed from the four instrument blocks defined by \cite{Carstensen2006}.
The baseline block contains ln\emph{mort}, \emph{maleco} and a constant; the climate block contains \emph{frost}, \emph{humid}, and \emph{latitude}; the Europe block contains \emph{eurfrac} and \emph{engfrac}; and the openness block contains \emph{coast} and \emph{trade}. 
Full descriptions of these variables appear in Table \ref{tab:desc}.
Table \ref{tab:fullresults} lists the eight instrument sets considered here, along with TSLS estimates and traditional 95\% confidence intervals for each.\footnote{The results for the baseline instrument presented in panel 1 of Table \ref{tab:fullresults} are slightly different from those in \cite{Carstensen2006} as I exclude Vietnam to keep the sample fixed across instrument sets.} 



\begin{table}[h]
  \small
\centering
\begin{tabular}{lrrrrrrrr}
\hline \hline 
& \multicolumn{2}{c}{1} & \multicolumn{2}{c}{2} & \multicolumn{2}{c}{3} & \multicolumn{2}{c}{4}\\ 
& \multicolumn{1}{c}{\emph{rule}} & \multicolumn{1}{c}{\emph{malfal}} & \multicolumn{1}{c}{\emph{rule}} & \multicolumn{1}{c}{\emph{malfal}} & \multicolumn{1}{c}{\emph{rule}} & \multicolumn{1}{c}{\emph{malfal}} & \multicolumn{1}{c}{\emph{rule}} & \multicolumn{1}{c}{\emph{malfal}}\\ 
 \hline 
 
coeff. & $0.89$ & $-1.04$ & $0.97$ & $-0.90$ & $0.81$ & $-1.09$ & $0.86$ & $-1.14$\\ 
SE & $0.18$ & $0.31$ & $0.16$ & $0.29$ & $0.16$ & $0.29$ & $0.16$ & $0.27$\\ 
lower & $0.53$ & $-1.66$ & $0.65$ & $-1.48$ & $0.49$ & $-1.67$ & $0.55$ & $-1.69$\\ 
upper & $1.25$ & $-0.42$ & $1.30$ & $-0.32$ & $1.13$ & $-0.51$ & $1.18$ & $-0.59$\\ 
& \multicolumn{2}{c}{Baseline} & \multicolumn{2}{c}{Baseline} & \multicolumn{2}{c}{Baseline} & \multicolumn{2}{c}{Baseline}\\ 
& \multicolumn{2}{c}{} & \multicolumn{2}{c}{Climate} & \multicolumn{2}{c}{} & \multicolumn{2}{c}{}\\ 
& \multicolumn{2}{c}{} & \multicolumn{2}{c}{} & \multicolumn{2}{c}{Openness} & \multicolumn{2}{c}{}\\ 
& \multicolumn{2}{c}{} & \multicolumn{2}{c}{} & \multicolumn{2}{c}{} & \multicolumn{2}{c}{Europe}\\ 
 \hline
\end{tabular} 
 
 \vspace{2em} 
 
 \begin{tabular}{lrrrrrrrr}
\hline \hline 
& \multicolumn{2}{c}{5} & \multicolumn{2}{c}{6} & \multicolumn{2}{c}{7} & \multicolumn{2}{c}{8}\\ 
& \multicolumn{1}{c}{\emph{rule}} & \multicolumn{1}{c}{\emph{malfal}} & \multicolumn{1}{c}{\emph{rule}} & \multicolumn{1}{c}{\emph{malfal}} & \multicolumn{1}{c}{\emph{rule}} & \multicolumn{1}{c}{\emph{malfal}} & \multicolumn{1}{c}{\emph{rule}} & \multicolumn{1}{c}{\emph{malfal}}\\ 
 \hline 
 
coeff. & $0.93$ & $-1.02$ & $0.86$ & $-0.98$ & $0.81$ & $-1.16$ & $0.84$ & $-1.08$\\ 
SE & $0.15$ & $0.26$ & $0.14$ & $0.27$ & $0.15$ & $0.27$ & $0.13$ & $0.25$\\ 
lower & $0.63$ & $-1.54$ & $0.59$ & $-1.53$ & $0.51$ & $-1.70$ & $0.57$ & $-1.58$\\ 
upper & $1.22$ & $-0.49$ & $1.14$ & $-0.43$ & $1.11$ & $-0.62$ & $1.10$ & $-0.58$\\ 
& \multicolumn{2}{c}{Baseline} & \multicolumn{2}{c}{Baseline} & \multicolumn{2}{c}{Baseline} & \multicolumn{2}{c}{Baseline}\\ 
& \multicolumn{2}{c}{Climate} & \multicolumn{2}{c}{Climate} & \multicolumn{2}{c}{} & \multicolumn{2}{c}{Climate}\\ 
& \multicolumn{2}{c}{} & \multicolumn{2}{c}{Openness} & \multicolumn{2}{c}{Openness} & \multicolumn{2}{c}{Openness}\\ 
& \multicolumn{2}{c}{Europe} & \multicolumn{2}{c}{} & \multicolumn{2}{c}{Europe} & \multicolumn{2}{c}{Europe}\\ 
 \hline
\end{tabular}
\caption{Two-stage least squares estimation results for all instrument sets.}
\label{tab:fullresults}

\end{table}

Table \ref{tab:FMSC_values} presents FMSC and ``positive-part'' FMSC results for instrument sets 1--8.
The positive-part FMSC sets a negative squared bias estimate to zero when estimating AMSE.
If the squared bias estimate is positive, FMSC and positive-part FMSC coincide; if the squared bias estimate is negative, positive-part FMSC is strictly greater than FMSC. 
Additional simulation results for the choosing instrumental variables experiment from Section \ref{sec:chooseIVsim}, available upon request, reveal that the positive-part FMSC never performs worse than the ordinary FMSC and sometimes performs slightly better, suggesting that it may be preferable in real-world applications.
For each criterion the table presents two cases: the first takes the effect of \emph{malfal}, a measure of malaria transmission, as the target parameter while the second uses the effect of \emph{rule}, a measure of institutions. 
In each case the two best instrument sets are numbers 5 (baseline, climate and Europe) and 8 (all instruments).
When the target parameter is the coefficient on \emph{malfal}, 8 is the clear winner under both the plain-vanilla and positive-part FMSC, leading to an estimate of $-1.08$ for the effect of malaria transmission on per-capita income.
When the target parameter is the coefficient on \emph{rule}, however, instrument sets 5 and 8 are virtually indistinguishable.
Indeed, while the plain-vanilla FMSC selects instrument set 8, leading to an estimate of $0.84$ for the effect of instutitions on per-capita income, the positive-part FMSC selects instrument set 5, leading to an estimate of $0.93$. 
Thus the FMSC methodology shows that, while helpful for estimating the effect of malaria transmission, the openness instruments \emph{coast} and \emph{trade} provide essentially no additional information for studying the effect of institutions.

\begin{table}[htbp]
  \small
	\centering
	\begin{tabular}{lcccccc}
\hline\hline
 & \multicolumn{3}{c}{$\mu = malfal$}& \multicolumn{3}{c}{$\mu = rule$}\\ 
 & FMSC & posFMSC & $\widehat{\mu}$ & FMSC & posFMSC & $\widehat{\mu}$ \\ 
 \hline
 (1) Valid & $ 3.03$ & $ 3.03$ & $-1.04$ & $1.27$ & $1.27$ & $0.89$\\ 
(2) Climate & $ 3.07$ & $ 3.07$ & $-0.90$ & $1.00$ & $1.00$ & $0.97$\\ 
(3) Openness & $ 2.30$ & $ 2.42$ & $-1.09$ & $1.21$ & $1.21$ & $0.81$\\ 
(4) Europe & $ 1.82$ & $ 2.15$ & $-1.14$ & $0.52$ & $0.73$ & $0.86$\\ 
(5) Climate, Europe & $ 0.85$ & $ 2.03$ & $-1.02$ & $0.25$ & $0.59$ & $0.93$\\ 
(6) Climate, Openness & $ 1.85$ & $ 2.30$ & $-0.98$ & $0.45$ & $0.84$ & $0.86$\\ 
(7) Openness, Europe & $ 1.63$ & $ 1.80$ & $-1.16$ & $0.75$ & $0.75$ & $0.81$\\ 
(8) Full & $ 0.53$ & $ 1.69$ & $-1.08$ & $0.23$ & $0.62$ & $0.84$ \\ 
 \hline
 \end{tabular}
		\caption{FMSC and and positive-part FMSC values corresponding to the instrument sets from Table \ref{tab:fullresults}}
		\label{tab:FMSC_values}
\end{table}

Table \ref{tab:postFMSC_CIs} presents three alternative post-selection confidence intervals for each of the instrument selection exercises from Table \ref{tab:FMSC_values}: Na\"{i}ve, 1-Step, and 2-Step.
These intervals are constructed as described in the simulation experiments from Section \ref{sec:CIsim} above. 
The simulation-based intervals are based on 10,000 random draws.
For the two-step interval I take $\alpha = \delta = 0.025$ which guarantees asymptotic coverage of at least 95\%.
From the resulting intervals, we see that both of our two hypothetical econometricians would report a statistically significant result even after accounting for the effects of instrument selection on inference and in spite of the conservatism of the 2-Step interval.

\begin{table}[htbp]
  \small
	\centering
	\begin{tabular}{lcccc} 
 \hline \hline 
 & \multicolumn{2}{c}{\emph{$\mu=$malfal}} & \multicolumn{2}{c}{$\mu=$\emph{rule}}\\ 
 & FMSC & posFMSC & FMSC & posFMSC\\ 
 \hline 
Na\"{i}ve & $(-1.58, -0.58)$ & $(-1.58, -0.58)$ & $(0.57, 1.10)$ & $(0.63, 1.22)$ \\ 
 1-Step & $(-1.52, -0.67)$ & $(-1.51, -0.68)$ & $(0.57, 1.08)$ & $(0.68, 1.17)$ \\ 
 2-Step & $(-1.62, -0.54)$ & $(-1.62, -0.55)$ & $(0.49, 1.18)$ & $(0.58, 1.27)$\\ 
 \hline 
\end{tabular}
	\caption{Post-selection CIs for the instrument selection exercise from Table \ref{tab:FMSC_values}.}
	\label{tab:postFMSC_CIs}
\end{table}

Although this example uses a simple model and a relatively small number of observations, it nevertheless provides a realistic test for the 2-Step interval because the computational complexity of the procedure is determined almost entirely by the dimension of $\tau$.
With $q = 7$, this example presents the kind of computational challenge that one would reasonably expect to encounter in practice yet is well within the ability of a standard desktop computer using off-the-shelf optimization routines.
Running on a single core it took just over ten minutes to generate all of the results for the empirical example in this paper.
For more computational details, including a description of the packages used, see Appendix \ref{append:comp}. 
