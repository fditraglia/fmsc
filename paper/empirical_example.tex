%!TEX root = main.tex
\section{Empirical Example: Geography or Institutions?}
\label{sec:application}
\cite{Carstensen2006} address a controversial question from the development literature: does geography directly effect income after controlling for institutions? 
A number of well-known studies find little or no direct effect of geographic endowments \citep{Acemoglu,Rodrik,Easterly}. \cite{Sachs}, on the other hand, shows that malaria transmission, a variable largely driven by ecological conditions, directly influences the level of per capita income, even after controlling for institutions. Because malaria transmission is very likely endogenous, Sachs uses a measure of ``malaria ecology,'' constructed to be exogenous both to present economic conditions and public health interventions, as an instrument. 
\cite{Carstensen2006} address the robustness of Sachs's results using the following baseline regression for a sample of 44 countries:
\begin{equation}
	\mbox{ln\emph{gdpc}}_i = \beta_1 + \beta_2 \cdot \mbox{\emph{institutions}}_i + \beta_3 \cdot \mbox{\emph{malaria}}_i + \epsilon_i
\end{equation}
Treating both institutions and malaria transmission as endogenous, they consider a variety of measures of each and a number of instrument sets. 
In each case, they find large negative effects of malaria transmission, lending support to Sach's conclusion.

In this section, I expand on the instrument selection exercise given in Table 2 of \cite{Carstensen2006} using the FMSC and corrected confidence intervals described above. I consider two questions. 
First, based on the FMSC methodology, which instruments should we choose to produce the best estimate of $\beta_3$, the effect of malaria transmission on per capita income? 
Second, after correcting confidence intervals for instrument selection, do we still find evidence of large and negative effects of malaria transmission on income? 
All results given here are calculated by 2SLS using the formulas from Section \ref{sec:chooseIVFMSC} and the variables described in Table \ref{tab:desc}. 
In keeping with Table 2 of \cite{Carstensen2006}, I use ln\emph{gdpc} as the dependent variable and \emph{rule} and \emph{malfal} as measures of institutions and malaria transmission throughout.
\todo[inline]{Explain about the sample size: I use 44 whereas they use 45 when available.} 


\begin{table}[!tbp]
\small
\centering
\begin{tabular}{lll}
\hline \hline
Name& Description &\\
\hline
ln\emph{gdpc}&Real GDP/capita at PPP, 1995 International Dollars &Outcome\\
\emph{rule}&Institutional quality (Average Governance Indicator)&Regressor\\
\emph{malfal}&Fraction of population at risk of malaria transmission, 1994&Regressor\\
ln\emph{mort}&Log settler mortality (per 1000 settlers), early 19th century&Baseline\\
\emph{maleco}&Index of stability of malaria transmission&Baseline\\
\emph{frost}&Prop.\ of land receiving at least 5 days of frost in winter&Climate\\
\emph{humid}&Highest temp. in month with highest avg.\ afternoon humidity&Climate\\
\emph{latitude}&Distance from equator (absolute value of latitude in degrees)&Climate \\
\emph{eurfrac}&Fraction of pop.\ that speaks major West.\ European Language&Europe \\
\emph{engfrac}&Fraction of pop.\ that speaks English&Europe\\
\emph{coast}&Proportion of land area within 100km of sea coast&Openness\\
\emph{trade}&Log Frankel-Romer predicted trade share&Openness\\
\hline
\end{tabular}
\caption{Description of variables for Empirical Example.}
\label{tab:desc}
\end{table}

To apply the FMSC to the present example, we need a minimum of two valid instruments besides the constant term. 
Based on the arguments given in \cite{Acemoglu}, \cite{Carstensen2006} and \cite{Sachs}, I proceed under the assumption that ln\emph{mort} and \emph{maleco}, measures of early settler mortality and malaria ecology, are exogenous.
Rather than selecting over every possible subset of instruments, I consider a number of instrument blocks defined in \cite{Carstensen2006}.
The baseline block contains ln\emph{mort}, \emph{maleco} and a constant; the climate block contains \emph{frost}, \emph{humid}, and \emph{latitude}; the Europe block contains \emph{eurfrac} and \emph{engfrac}; and the openness block contains \emph{coast} and \emph{trade}. 
Full descriptions of these variables appear in Table \ref{tab:desc}.
Table \ref{tab:fullresults} gives 2SLS results and traditional 95\% confidence intervals for all instrument sets considered here.

\begin{table}[h]
\centering
\begin{tabular}{lrrrrrrrr}
\hline \hline 
& \multicolumn{2}{c}{1} & \multicolumn{2}{c}{2} & \multicolumn{2}{c}{3} & \multicolumn{2}{c}{4}\\ 
& \multicolumn{1}{c}{\emph{rule}} & \multicolumn{1}{c}{\emph{malfal}} & \multicolumn{1}{c}{\emph{rule}} & \multicolumn{1}{c}{\emph{malfal}} & \multicolumn{1}{c}{\emph{rule}} & \multicolumn{1}{c}{\emph{malfal}} & \multicolumn{1}{c}{\emph{rule}} & \multicolumn{1}{c}{\emph{malfal}}\\ 
 \hline 
 
coeff. & $0.89$ & $-1.04$ & $0.97$ & $-0.90$ & $0.81$ & $-1.09$ & $0.86$ & $-1.14$\\ 
SE & $0.18$ & $0.31$ & $0.16$ & $0.29$ & $0.16$ & $0.29$ & $0.16$ & $0.27$\\ 
lower & $0.53$ & $-1.66$ & $0.65$ & $-1.48$ & $0.49$ & $-1.67$ & $0.55$ & $-1.69$\\ 
upper & $1.25$ & $-0.42$ & $1.30$ & $-0.32$ & $1.13$ & $-0.51$ & $1.18$ & $-0.59$\\ 
& \multicolumn{2}{c}{Baseline} & \multicolumn{2}{c}{Baseline} & \multicolumn{2}{c}{Baseline} & \multicolumn{2}{c}{Baseline}\\ 
& \multicolumn{2}{c}{} & \multicolumn{2}{c}{Climate} & \multicolumn{2}{c}{} & \multicolumn{2}{c}{}\\ 
& \multicolumn{2}{c}{} & \multicolumn{2}{c}{} & \multicolumn{2}{c}{Openness} & \multicolumn{2}{c}{}\\ 
& \multicolumn{2}{c}{} & \multicolumn{2}{c}{} & \multicolumn{2}{c}{} & \multicolumn{2}{c}{Europe}\\ 
 \hline
\end{tabular} 
 
 \vspace{2em} 
 
 \begin{tabular}{lrrrrrrrr}
\hline \hline 
& \multicolumn{2}{c}{5} & \multicolumn{2}{c}{6} & \multicolumn{2}{c}{7} & \multicolumn{2}{c}{8}\\ 
& \multicolumn{1}{c}{\emph{rule}} & \multicolumn{1}{c}{\emph{malfal}} & \multicolumn{1}{c}{\emph{rule}} & \multicolumn{1}{c}{\emph{malfal}} & \multicolumn{1}{c}{\emph{rule}} & \multicolumn{1}{c}{\emph{malfal}} & \multicolumn{1}{c}{\emph{rule}} & \multicolumn{1}{c}{\emph{malfal}}\\ 
 \hline 
 
coeff. & $0.93$ & $-1.02$ & $0.86$ & $-0.98$ & $0.81$ & $-1.16$ & $0.84$ & $-1.08$\\ 
SE & $0.15$ & $0.26$ & $0.14$ & $0.27$ & $0.15$ & $0.27$ & $0.13$ & $0.25$\\ 
lower & $0.63$ & $-1.54$ & $0.59$ & $-1.53$ & $0.51$ & $-1.70$ & $0.57$ & $-1.58$\\ 
upper & $1.22$ & $-0.49$ & $1.14$ & $-0.43$ & $1.11$ & $-0.62$ & $1.10$ & $-0.58$\\ 
& \multicolumn{2}{c}{Baseline} & \multicolumn{2}{c}{Baseline} & \multicolumn{2}{c}{Baseline} & \multicolumn{2}{c}{Baseline}\\ 
& \multicolumn{2}{c}{Climate} & \multicolumn{2}{c}{Climate} & \multicolumn{2}{c}{} & \multicolumn{2}{c}{Climate}\\ 
& \multicolumn{2}{c}{} & \multicolumn{2}{c}{Openness} & \multicolumn{2}{c}{Openness} & \multicolumn{2}{c}{Openness}\\ 
& \multicolumn{2}{c}{Europe} & \multicolumn{2}{c}{} & \multicolumn{2}{c}{Europe} & \multicolumn{2}{c}{Europe}\\ 
 \hline
\end{tabular}
\caption{Two-stage least squares estimation results for all instrument sets.}
\label{tab:fullresults}

\end{table}

\todo[inline]{Need to update some of the text here since I'm no longer looking at instrument sets 9--12. The only remaining results to add are the CIs.}

Table \ref{tab:replicate} presents FMSC results for instrument sets 1--8  as defined in Table \ref{tab:fullresults}. Results are presented for two cases: the first takes the effect of \emph{malfal}, a measure of malaria transmission, as the target parameter while the second uses the effect of \emph{rule}, a measure of institutions. 
In each case, the FMSC selects instrument set 8: the full instrument set containing the baseline, climate, Europe and openness blocks. 
The rankings, however, differ depending on the target parameter.  
When the target is \emph{rule} instrument sets 8 and 5 are virtually identical in terms of FMSC: 0.26 versus 0.23. 
In Table 2 of their paper, \cite{Carstensen2006} report GMM-BIC and HQ results for selection over instrument sets 2--4 and 8 that also favor instrument set 8. 
However, the authors do not consider instrument sets 5--7. 
Although the FMSC also selects instrument set 8, the FMSC values of instrument set 5 are small enough to suggest that including the openness block does little to reduce MSE.


We find no evidence that accounting for the effects of instrument selection changes our conclusions about the sign or significance of \emph{malfal} or \emph{rule}.

\begin{table}[htbp]
	\centering
	\begin{tabular}{lcccccc}
\hline\hline
 & \multicolumn{3}{c}{$\mu = malfal$}& \multicolumn{3}{c}{$\mu = rule$}\\ 
 & FMSC & posFMSC & $\widehat{\mu}$ & FMSC & posFMSC & $\widehat{\mu}$ \\ 
 \hline
 (1) Valid & $ 3.03$ & $ 3.03$ & $-1.04$ & $1.27$ & $1.27$ & $0.89$\\ 
(2) Climate & $ 3.07$ & $ 3.07$ & $-0.90$ & $1.00$ & $1.00$ & $0.97$\\ 
(3) Openness & $ 2.30$ & $ 2.42$ & $-1.09$ & $1.21$ & $1.21$ & $0.81$\\ 
(4) Europe & $ 1.82$ & $ 2.15$ & $-1.14$ & $0.52$ & $0.73$ & $0.86$\\ 
(5) Climate, Europe & $ 0.85$ & $ 2.03$ & $-1.02$ & $0.25$ & $0.59$ & $0.93$\\ 
(6) Climate, Openness & $ 1.85$ & $ 2.30$ & $-0.98$ & $0.45$ & $0.84$ & $0.86$\\ 
(7) Openness, Europe & $ 1.63$ & $ 1.80$ & $-1.16$ & $0.75$ & $0.75$ & $0.81$\\ 
(8) Full & $ 0.53$ & $ 1.69$ & $-1.08$ & $0.23$ & $0.62$ & $0.84$ \\ 
 \hline
 \end{tabular}
		\caption{FMSC and and positive-part FMSC values for selection over the instrument sets presented in Table ???.}
\end{table}

\begin{table}[htbp]
	\centering
	\begin{tabular}{lcccc} 
 \hline \hline 
 & \multicolumn{2}{c}{\emph{$\mu=$malfal}} & \multicolumn{2}{c}{$\mu=$\emph{rule}}\\ 
 & FMSC & posFMSC & FMSC & posFMSC\\ 
 \hline 
Na\"{i}ve & $(-1.58, -0.58)$ & $(-1.58, -0.58)$ & $(0.57, 1.10)$ & $(0.63, 1.22)$ \\ 
 1-Step & $(-1.52, -0.67)$ & $(-1.51, -0.68)$ & $(0.57, 1.08)$ & $(0.68, 1.17)$ \\ 
 2-Step & $(-1.62, -0.54)$ & $(-1.62, -0.55)$ & $(0.49, 1.18)$ & $(0.58, 1.27)$\\ 
 \hline 
\end{tabular}
	\caption{Post-selection confidence intervals for the instrument selection exercise presented in Table ???. Explain the meaning of each row!}
\end{table}

\todo[inline]{There is a very slight discrepancy between the Na\"{i}ve CIs and the corresponding CIs in the table of 2SLS results. This is intentional. The 2SLS results are designed to be as close a possible to C\&G. Since they use a t-distribution to construct CIs, I do the same in that table.  who use a t-distribution to construct their CIs.}

Although it provides an asymptotically unbiased estimator of AMSE, the FMSC may be negative because it subtracts $\widehat{\Psi}\widehat{\Omega}\widehat{\Psi}'$ from $\widehat{\tau}\widehat{\tau}'$ to estimate squared bias.

These results lend support to the view of \cite{Carstensen2006} and \cite{Sachs} that malaria transmission has a direct effect on development.