%!TEX root = main.tex
\section{Empirical Example: Geography or Institutions?}
\label{sec:application}
\cite{Carstensen2006} address a controversial question from the development literature: does geography directly effect income after controlling for institutions? A number of well-known studies find little or no direct effect of geographic endowments. \cite{Acemoglu}, for example, find that countries nearer to the equator do not have lower incomes after controlling for institutions. \cite{Rodrik} report that geographic variables have only small direct effects on income, affecting development mainly through their influence on institutions. Similarly, \cite{Easterly} find no effect of ``tropics, germs and crops'' except through institutions. \cite{Sachs} responds directly to these three papers by showing that malaria transmission, a variable largely driven by ecological conditions, directly influences the level of per capita income, even after controlling for institutions. Because malaria transmission  is very likely endogenous, Sachs uses a measure of ``malaria ecology,'' constructed to be exogenous both to present economic conditions and public health interventions, as an instrument. \cite{Carstensen2006} address the robustness of Sachs's results using the following baseline regression for a sample of 45 countries:
\begin{equation}
	\mbox{ln\emph{gdpc}}_i = \beta_1 + \beta_2 \cdot \mbox{\emph{institutions}}_i + \beta_3 \cdot \mbox{\emph{malaria}}_i + \epsilon_i
\end{equation}
Treating both institutions and malaria transmission as endogenous, they consider a variety of measures of each and a number of instrument sets. In each case, they find large negative effects of malaria transmission, lending further support to Sach's conclusion. In this section, I expand on the instrument selection exercise given in Table 2 of \cite{Carstensen2006} using the FMSC and corrected confidence intervals described above. I consider two questions. First, based on the FMSC methodology, which instruments should we choose to produce the best estimate of $\beta_3$, the effect of malaria transmission on per capita income? Second, after correcting confidence intervals for instrument selection, do we still find evidence of large and negative effects of malaria transmission on income? All results given here are calculated by 2SLS using the formulas from Section \ref{sec:chooseIVFMSC} and the variables described in Table \ref{tab:desc}. In keeping with Table 2 of \cite{Carstensen2006}, I use ln\emph{gdpc} as the dependent variable and \emph{rule} and \emph{malfal} as measures of institutions and malaria transmission throughout. 


\begin{table}[!tbp]
\caption{Description of Variables}
\small
\label{tab:desc}
\begin{center}
\begin{tabular}{lll}
\hline \hline
Name& Description &\\
\hline
ln\emph{gdpc}&Real GDP/capita at PPP, 1995 International Dollars &Outcome\\
\emph{rule}&Institutional quality (Average Governance Indicator)&Regressor\\
\emph{malfal}&Fraction of population at risk of malaria transmission, 1994&Regressor\\
ln\emph{mort}&Log settler mortality (per 1000 settlers), early 19th century&Baseline\\
\emph{maleco}&Index of stability of malaria transmission&Baseline\\
\emph{frost}&Prop.\ of land receiving at least 5 days of frost in winter&Climate\\
\emph{humid}&Highest temp. in month with highest avg.\ afternoon humidity&Climate\\
\emph{latitude}&Distance from equator (absolute value of latitude in degrees)&Climate \\
\emph{eurfrac}&Fraction of pop.\ that speaks major West.\ European Language&Europe \\
\emph{engfrac}&Fraction of pop.\ that speaks English&Europe\\
\emph{coast}&Proportion of land area within 100km of sea coast&Openness\\
\emph{trade}&Log Frankel-Romer predicted trade share&Openness\\
\hline
\end{tabular}
\end{center}
\end{table}

To apply the FMSC to the present example, we need a minimum of two valid instruments besides the constant term. Based on the arguments given in \cite{Acemoglu}, \cite{Carstensen2006} and \cite{Sachs}, I proceed under the assumption that ln\emph{mort} and \emph{maleco}, measures of early settler mortality and malaria ecology, are exogenous. Rather than selecting over every possible subset of instruments, I consider a number of instrument blocks defined in \cite{Carstensen2006}. The baseline block contains ln\emph{mort}, \emph{maleco} and a constant; the climate block contains \emph{frost}, \emph{humid}, and \emph{latitude}; the Europe block contains \emph{eurfrac} and \emph{engfrac}; and the openness block contains \emph{coast} and \emph{trade}. Full descriptions of these variables appear in Table \ref{tab:desc}. Table \ref{tab:fullresults} gives 2SLS results and traditional 95\% confidence intervals for all instrument sets considered here.

\begin{sidewaystable}[!tbp]
\caption{2SLS Results for all Instrument Sets}
\label{tab:fullresults}
 \begin{center}

\begin{tabular}{lrrrrrrrr}
\hline \hline 
& \multicolumn{2}{c}{1} & \multicolumn{2}{c}{2} & \multicolumn{2}{c}{3} & \multicolumn{2}{c}{4}\\ 
& \multicolumn{1}{c}{\emph{rule}} & \multicolumn{1}{c}{\emph{malfal}} & \multicolumn{1}{c}{\emph{rule}} & \multicolumn{1}{c}{\emph{malfal}} & \multicolumn{1}{c}{\emph{rule}} & \multicolumn{1}{c}{\emph{malfal}} & \multicolumn{1}{c}{\emph{rule}} & \multicolumn{1}{c}{\emph{malfal}}\\ 
 \hline 
 
coeff. & $0.89$ & $-1.04$ & $0.97$ & $-0.90$ & $0.81$ & $-1.09$ & $0.86$ & $-1.14$\\ 
SE & $0.18$ & $0.31$ & $0.16$ & $0.29$ & $0.16$ & $0.29$ & $0.16$ & $0.27$\\ 
lower & $0.53$ & $-1.66$ & $0.65$ & $-1.48$ & $0.49$ & $-1.67$ & $0.55$ & $-1.69$\\ 
upper & $1.25$ & $-0.42$ & $1.30$ & $-0.32$ & $1.13$ & $-0.51$ & $1.18$ & $-0.59$\\ 
& \multicolumn{2}{c}{Baseline} & \multicolumn{2}{c}{Baseline} & \multicolumn{2}{c}{Baseline} & \multicolumn{2}{c}{Baseline}\\ 
& \multicolumn{2}{c}{} & \multicolumn{2}{c}{Climate} & \multicolumn{2}{c}{} & \multicolumn{2}{c}{}\\ 
& \multicolumn{2}{c}{} & \multicolumn{2}{c}{} & \multicolumn{2}{c}{Openness} & \multicolumn{2}{c}{}\\ 
& \multicolumn{2}{c}{} & \multicolumn{2}{c}{} & \multicolumn{2}{c}{} & \multicolumn{2}{c}{Europe}\\ 
 \hline
\end{tabular} 
 
 \vspace{2em} 
 
 \begin{tabular}{lrrrrrrrr}
\hline \hline 
& \multicolumn{2}{c}{5} & \multicolumn{2}{c}{6} & \multicolumn{2}{c}{7} & \multicolumn{2}{c}{8}\\ 
& \multicolumn{1}{c}{\emph{rule}} & \multicolumn{1}{c}{\emph{malfal}} & \multicolumn{1}{c}{\emph{rule}} & \multicolumn{1}{c}{\emph{malfal}} & \multicolumn{1}{c}{\emph{rule}} & \multicolumn{1}{c}{\emph{malfal}} & \multicolumn{1}{c}{\emph{rule}} & \multicolumn{1}{c}{\emph{malfal}}\\ 
 \hline 
 
coeff. & $0.93$ & $-1.02$ & $0.86$ & $-0.98$ & $0.81$ & $-1.16$ & $0.84$ & $-1.08$\\ 
SE & $0.15$ & $0.26$ & $0.14$ & $0.27$ & $0.15$ & $0.27$ & $0.13$ & $0.25$\\ 
lower & $0.63$ & $-1.54$ & $0.59$ & $-1.53$ & $0.51$ & $-1.70$ & $0.57$ & $-1.58$\\ 
upper & $1.22$ & $-0.49$ & $1.14$ & $-0.43$ & $1.11$ & $-0.62$ & $1.10$ & $-0.58$\\ 
& \multicolumn{2}{c}{Baseline} & \multicolumn{2}{c}{Baseline} & \multicolumn{2}{c}{Baseline} & \multicolumn{2}{c}{Baseline}\\ 
& \multicolumn{2}{c}{Climate} & \multicolumn{2}{c}{Climate} & \multicolumn{2}{c}{} & \multicolumn{2}{c}{Climate}\\ 
& \multicolumn{2}{c}{} & \multicolumn{2}{c}{Openness} & \multicolumn{2}{c}{Openness} & \multicolumn{2}{c}{Openness}\\ 
& \multicolumn{2}{c}{Europe} & \multicolumn{2}{c}{} & \multicolumn{2}{c}{Europe} & \multicolumn{2}{c}{Europe}\\ 
 \hline
\end{tabular}

\end{center}

\end{sidewaystable}



Table \ref{tab:replicate} presents FMSC results for instrument sets 1--8  as defined in Table \ref{tab:fullresults}. Results are presented for two cases: the first takes the effect of \emph{malfal}, a measure of malaria transmission, as the target parameter while the second uses the effect of \emph{rule}, a measure of institutions. In each case, the FMSC selects instrument set 8: the full instrument set containing the baseline, climate, Europe and openness blocks. The rankings, however, differ depending on the target parameter.  When the target is \emph{rule} instrument sets 8 and 5 are virtually identical in terms of FMSC: 0.26 versus 0.23. In Table 2 of their paper, \cite{Carstensen2006} report GMM-BIC and HQ results for selection over instrument sets 2--4 and 8 that also favor instrument set 8. However, the authors do not consider instrument sets 5--7. Although the FMSC also selects instrument set 8, the FMSC values of instrument set 5 are small enough to suggest that including the openness block does little to reduce MSE.

The bottom panel of Table \ref{tab:replicate} presents a number of alternative 95\% confidence intervals for the effects of \emph{malfal} and \emph{rule}, respectively. The first row gives the traditional asymptotic confidence interval from Table \ref{tab:fullresults}, while the following three give simulation-based intervals accounting for the effects of instrument selection. I do not present intervals for the conservative procedure given in Algorithm \ref{alg:conf} because the results in this example are so insensitive to the value of $\tau$ that the minimization and maximization problems given in Step 2 of the Algorithm are badly behaved. To illustrate this, I instead present intervals that use the same simulation procedure as Algorithm \ref{alg:conf} but treat $\tau$ as fixed. I consider four possible values of the bias parameter. When $\tau = \widehat{\tau}$, we have the one-step corrected interval considered in Table \ref{tab:FMSCnaive}. When $\tau = 0$, we have an interval that assumes all instruments are valid. The remaining two values $\widehat{\tau}_{min}$ and $\widehat{\tau}_{max}$ correspond to the lower and upper bounds of \emph{elementwise} 95\% confidence intervals for $\tau$ based on the distributional result given in Theorem \ref{thm:tau}. These result in a region with greater than 95\% coverage for $\tau$ considered jointly. Corrected 95\% intervals for the effect of \emph{malfal} are similar regardless of the value of $\tau$ used in the simulation, and the same is true for \emph{rule}. We find no evidence that accounting for the effects of instrument selection changes our conclusions about the sign or significance of \emph{malfal} or \emph{rule}.

\begin{table}[htbp]
\caption{FMSC values and confidence intervals for instrument sets 1--8.}
\label{tab:replicate}
\small
 \begin{center}
 \begin{tabular}{lcccc}\hline\hline
 & \multicolumn{2}{c}{$\mu=$\emph{malfal}}& \multicolumn{2}{c}{$\mu=$\emph{rule}}\\
&\multicolumn{1}{c}{FMSC}&\multicolumn{1}{c}{$\widehat{\mu}$}&\multicolumn{1}{c}{FMSC}&\multicolumn{1}{c}{$\widehat{\mu}$}\tabularnewline
\hline
Valid (1)&3.03&-1.04&1.27&0.89\tabularnewline
Climate (2)&2.67&-0.90&0.92&0.97\tabularnewline   
Openness (3)&2.31&-1.09&1.23&0.81\tabularnewline  
Europe (4)&1.83&-1.14&0.55&0.86\tabularnewline
Openness, Europe (7)&1.72&-1.16&0.77&0.81\tabularnewline  
Climate, Openness (6)&1.65&-0.98&0.43&0.86\tabularnewline
Climate, Europe (5)&0.71&-1.02&0.26&0.93\tabularnewline
Full (8)&0.53&-1.08&0.23&0.84\tabularnewline
\hline 
Traditional&\multicolumn{2}{c}{(-1.58, -0.58)}&\multicolumn{2}{c}{(0.57, 1.10)}\\
$\tau = \widehat{\tau}$&\multicolumn{2}{c}{(-1.54, -0.61)}&\multicolumn{2}{c}{(0.55, 1.13)}\\
$\tau = 0$&\multicolumn{2}{c}{(-1.53, -0.64)}&\multicolumn{2}{c}{(0.55, 1.12)}\\
$\tau = \widehat{\tau}_{max}$&\multicolumn{2}{c}{(-1.51, -0.55)}&\multicolumn{2}{c}{(0.55, 1.17)}\\
$\tau = \widehat{\tau}_{min}$&\multicolumn{2}{c}{(-1.61, -0.58)}&\multicolumn{2}{c}{(0.49, 1.15)}\\
\hline

\end{tabular}
\end{center}
\end{table}



FMSC is designed to include invalid instruments when doing so will reduce AMSE. Table \ref{tab:endog} considers adding two almost certainly invalid instruments to the baseline instrument set: \emph{rule}$^2$ and \emph{malfal}$^2$. Because they are constructed from the endogenous regressors, these instruments are likely to be highly relevant. Unless the effect of institutions and malaria transmission on GDP per capita is exactly linear, however, they are invalid. When the target is \emph{malfal}, we see that the FMSC selects an instrument set including \emph{malfal}$^2$ and the baseline instruments. FMSC is negative in this case. Although it provides an asymptotically unbiased estimator of AMSE, the FMSC may be negative because it subtracts $\widehat{\Psi}\widehat{\Omega}\widehat{\Psi}'$ from $\widehat{\tau}\widehat{\tau}'$ to estimate squared bias. When the target is \emph{rule}, FMSC chooses the full instrument set, including the baseline instruments along with \emph{rule}$^2$ and \emph{malfal}$^2$. While these instruments are likely invalid, FMSC chooses to include them because its estimate of the bias they induce is small compared to the reduction in variance they provide. Table \ref{tab:all} further expands the instrument sets under consideration to include 1--4 and 9--12. In this case, the FMSC chooses instrument set 12 for both target parameters. However, we see from the FMSC rankings that most of the reduction in MSE achieved by instrument set 12 comes from the inclusion of the squared endogenous regressors in the instrument set. Turning our attention to the confidence intervals in Tables \ref{tab:endog} and \ref{tab:all}, we again see that the simulation-based intervals are extremely insensitive to the value of $\tau$ used. Again, the sign and significance of \emph{malfal} and \emph{rule} is insensitive to the effects of instrument selection. These results lend support to the view of \cite{Carstensen2006} and \cite{Sachs} that malaria transmission has a direct effect on development.

\begin{table}[htbp]
\caption{FMSC values and confidence intervals for instrument sets 1 and 9--11}
\label{tab:endog}
\small
\centering
 \begin{tabular}{lrrrr}\hline\hline
 & \multicolumn{2}{c}{$\mu=$\emph{malfal}}& \multicolumn{2}{c}{$\mu=$\emph{rule}}\\
&\multicolumn{1}{c}{FMSC}&\multicolumn{1}{c}{$\widehat{\mu}$}&\multicolumn{1}{c}{FMSC}&\multicolumn{1}{c}{$\widehat{\mu}$}\tabularnewline
\hline
Valid (1)& 3.03&-1.04& 1.27&0.89\tabularnewline
\emph{rule}$^2$ (10)& 2.05&-0.84& 0.28&1.02\tabularnewline
Full (11)&-0.20&-0.85& -0.06&1.02\tabularnewline
\emph{malfal}$^2$ (9)&-0.41&-0.92&0.18&0.93\tabularnewline
\hline
Traditional&\multicolumn{2}{r}{(-1.39, -0.46)}&\multicolumn{2}{r}{(0.72, 1.32)}\\
$\tau = \widehat{\tau}$&\multicolumn{2}{r}{(-1.49, -0.38)}&\multicolumn{2}{r}{(0.68, 1.36)}\\
$\tau = 0$&\multicolumn{2}{r}{(-1.46, -0.38)}&\multicolumn{2}{r}{(0.71, 1.32)}\\
$\tau = \widehat{\tau}_{max}$&\multicolumn{2}{r}{(-1.51, -0.38)}&\multicolumn{2}{r}{(0.66, 1.37)}\\
$\tau = \widehat{\tau}_{min}$&\multicolumn{2}{r}{(-1.49, -0.38)}&\multicolumn{2}{r}{(0.71, 1.35)}\\
\hline
\end{tabular}


\footnotesize
\end{table}


\begin{table}[htbp]
\caption{FMSC values and confidence intervals for instrument sets 1--4 and 9--12}
\label{tab:all}
\small
\centering
 \begin{tabular}{lcccc}\hline\hline
 & \multicolumn{2}{c}{$\mu=$\emph{malfal}}& \multicolumn{2}{c}{$\mu=$\emph{rule}}\\
&\multicolumn{1}{c}{FMSC}&\multicolumn{1}{c}{$\widehat{\mu}$}&\multicolumn{1}{c}{FMSC}&\multicolumn{1}{c}{$\widehat{\mu}$}\tabularnewline
\hline
Valid (1)& 3.03&-1.04& 1.27&0.89\tabularnewline
Climate (2)& 2.85&-0.90& 0.95&0.97\tabularnewline
Openness (3)& 2.51&-1.09& 1.26&0.81\tabularnewline
Europe (4)& 1.94&-1.14& 0.58&0.86\tabularnewline
\emph{rule}$^2$ (10)& 1.88&-0.84& 0.25&1.02\tabularnewline
\emph{malfal}$^2$, \emph{rule}$^2$ (11)& 0.06&-0.85&-0.03&1.02\tabularnewline
\emph{malfal}$^2$ (9)&-0.20&-0.92& 0.15&0.93\tabularnewline
Full (12)&-1.38&-1.00&-0.61&0.88\tabularnewline
\hline
Traditional&\multicolumn{2}{c}{(-1.42, -0.57)}&\multicolumn{2}{c}{(0.63, 1.12)}\\
$\tau = \widehat{\tau}$&\multicolumn{2}{c}{(-1.51, -0.51)}&\multicolumn{2}{c}{(0.57, 1.17)}\\
$\tau = 0$&\multicolumn{2}{c}{(-1.48, -0.52)}&\multicolumn{2}{c}{(0.60, 1.15)}\\
$\tau = \widehat{\tau}_{max}$&\multicolumn{2}{c}{(-1.50, -0.50)}&\multicolumn{2}{c}{(0.55, 1.17)}\\
$\tau = \widehat{\tau}_{min}$&\multicolumn{2}{c}{(-1.50, -0.49)}&\multicolumn{2}{c}{(0.59, 1.18)}\\
\hline
\end{tabular}


\footnotesize
\end{table}