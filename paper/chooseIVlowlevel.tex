%!TEX root = main.tex
\subsection{Example: Choosing Instrumental Variables}
\label{sec:chooseIVlowlevel}
The preceding example was quite specific, but it a sense it amounted to a problem of instrument selection: if $x$ is exogenous, it is clearly its ``own best instrument.'' 
Viewed from this perspective, the FMSC amounted to trading off endogeneity against instrument strength. We now consider instrument selection \emph{in general} for linear GMM estimators in an iid setting. 
Consider the following model:
\begin{eqnarray}
    y_i &=& \mathbf{x}_i' \beta +  \epsilon_i\\
    \mathbf{x}_i &=&  \Pi_1' \mathbf{z}_{i}^{(1)} + \Pi_2'\mathbf{z}_{i}^{(2)} + \mathbf{v}_i
\end{eqnarray}
where $y$ is an outcome of interest, $\mathbf{x}$ is an $r$-vector of regressors, some of which are endogenous, $\mathbf{z}^{(1)}$ is a $p$-vector of instruments known to be exogenous, and $\mathbf{z}^{(2)}$ is a $q$-vector  of \emph{potentially endogenous} instruments. 
The $r$-vector $\beta$, $p\times r$ matrix $\Pi_1$, and $q\times r$ matrix $\Pi_2$ contain unknown constants. Stacking observations in the usual way, let $\mathbf{y}' = (y_1, \hdots, y_n)$, $X' = (\mathbf{x}_1, \hdots, \mathbf{x}_n)$, $\boldsymbol{\epsilon} = (\epsilon_1, \hdots, \epsilon_n)$, 
$Z_1' = (\mathbf{z}_{1}^{(1)}, \hdots, \mathbf{z}_{n}^{(1)})$, $Z_2' = (\mathbf{z}_{1}^{(2)}, \hdots, \mathbf{z}_{n}^{(2)})$, and $V' = (\mathbf{v}_1, \hdots, \mathbf{v}_n)$ where $n$ is the sample size.
Using this notation, $\mathbf{y} = X\beta +\boldsymbol{\epsilon}$ and $X =  Z \Pi + V$, where $Z = (Z_1, Z_2)$ and $\Pi = (\Pi_1', \Pi_2')'$. 

The idea behind this setup is that the instruments contained in $Z_2$ are expected to be strong. 
If we were confident that they were exogenous, we would certainly use them in estimation. 
Yet the very fact that we expect them to be strongly correlated with $\mathbf{x}$ gives us reason to fear that the instruments contained in $Z_2$ may be endogenous. 
The exact opposite is true of $Z_1$. These are the instruments that we are prepared to assume are exogenous. 
But when is such an assumption plausible? Precisely when the instruments contained in $Z_1$ are \emph{not especially strong}. 
\todo[inline]{Should briefly mention some situations in which this setup arises. Panel data, etc. Also refer to my empirical example.}
In this setting, the FMSC attempts to trade off a small increase in bias from using a \emph{slightly} endogenous instrument against a larger decrease in variance from increasing the overall strength of the instruments used in estimation.

To this end, consider a general linear GMM estimator of the form
$$\widehat{\beta}_S = (X'Z_S \widetilde{W}_S Z_S' X)^{-1}X'Z_S \widetilde{W}_S  Z_S' \mathbf{y}$$
where $S$ indexes the instruments used in estimation, $Z_S'  = \Xi_S Z'$ is the matrix containing only those instruments included in $S$, $|S|$ is the number of instruments used in estimation and $\widetilde{W}_S$ is an $|S|\times|S|$ positive semi-definite weighting matrix. In this example, the local mis-specification assumption is given by
\begin{equation}
  E\left[\begin{array}
    {c}
    \textbf{z}_{ni}^{(1)} (y_i - \textbf{x}_i\beta) \\
    \textbf{z}_{ni}^{(2)} (y_i - \textbf{x}_i \beta)
\end{array}\right] = \left[
  \begin{array}
    {c}
    \textbf{0} \\ \boldsymbol{\tau}/\sqrt{n}
  \end{array}
\right]
\end{equation}
where $\mathbf{0}$ is a $p$-vector of zeros, and $\boldsymbol{\tau}$ is a $q$-vector of unknown constants. 
The following low-level conditions are sufficient for the asymptotic normality of $\widehat{\beta}_S$.

\begin{assump}[Choosing IVs]
\label{assump:chooseIV} 
	Let $\{(\mathbf{z}_{ni}, \mathbf{v}_{ni}, \epsilon_{ni})\colon 1\leq i \leq n, n = 1, 2, \hdots\}$ be a triangular array of random variables with $\mathbf{z}_{ni} = (\mathbf{z}_{ni}^{(1)}$, $\mathbf{z}_{ni}^{(1)})$ such that
	\begin{enumerate}[(a)]
		\item $(\mathbf{z}_{ni}, \mathbf{v}_{ni}, \epsilon_{ni}) \sim$ iid within each row of the array (i.e.\ for fixed $n$)
		\item $E[\mathbf{v}_{ni}\mathbf{z}_{ni}']=\mathbf{0}$, $E[\mathbf{z}^{(1)}_{ni} \epsilon_{ni}]=\mathbf{0}$, and $E[\mathbf{z}^{(2)}_{ni} \epsilon_{ni}] = \boldsymbol{\tau}/\sqrt{n}$ for all $n$
		\item $E[\left|\mathbf{z}_{ni}\right|^{4+\eta}] <C$, $E[\left|\epsilon_{ni}\right|^{4+\eta}] <C$, and $E[\left|\mathbf{v}_{ni}\right|^{4+\eta}] <C$ for some $\eta >0$, $C <\infty$
		\item $E[\mathbf{z}_{ni} \mathbf{z}_{ni}'] \rightarrow Q>0$ and $E[\epsilon_{ni}^2 \mathbf{z}_{ni} \mathbf{z}_{ni}'] \rightarrow \Omega >0$ as $n\rightarrow \infty$
		\item $\mathbf{x}_{ni} =  \Pi_1' \mathbf{z}_{ni}^{(1)} + \Pi_2'\mathbf{z}_{ni}^{(2)} + \mathbf{v}_{ni}$ where $\Pi_1 \neq \mathbf{0}$, $\Pi_2 \neq \mathbf{0}$, and $y_i = \mathbf{x}_{ni}' \beta +  \epsilon_{ni}$
	\end{enumerate}
\end{assump}

The preceding conditions are similar to although more general than those contained in Assumption \ref{assump:OLSvsIV}. 
While we no longer assume homoskedasticity, for simplicity we retain the assumption that the triangular array is iid in each row. 

\begin{thm}[Choosing IVs Limit Distribution]
\label{thm:chooseIV} Suppose that $\widetilde{W}_S \rightarrow_p W_S >0$. Then, under Assumption \ref{assump:chooseIV}
$$\sqrt{n}\left(\widehat{\beta}_S - \beta \right) \overset{d}{\rightarrow} -K_S \Xi_S \left(\left[\begin{array}
           {c} \mathbf{0} \\ \boldsymbol{\tau}
         \end{array}\right] + M \right)$$
where
         $$-K_S = \left(\Pi' Q_S W_S Q_S'\Pi\right)^{-1} \Pi'Q_SW_S$$
$M \sim N(\mathbf{0}, \Omega)$, $Q_S = Q \Xi_S'$, and $Q$ and $\Omega$ are defined in Assumption \ref{assump:chooseIV}.
\end{thm}