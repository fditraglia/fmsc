%!TEX root = main.tex
\subsection{Moment Averaging Examples}
Although it is a special case of moment averaging, moment selection is a somewhat crude procedure: it gives full weight to the estimator that minimizes the moment selection criterion no matter how close its nearest competitor lies. 
Accordingly, when competing moment sets have similar criterion values in the population, sampling variation can be \emph{magnified} in the selected estimator. 
Thus, it may be possible to achieve better performance by using smooth weights rather than discrete selection.
In this section we explore this possibility via two examples: one based on a simple heursitic and another on more detailed analytical calculations.

\subsubsection{Exponential Weights}
\todo[inline]{Possibly remove this section later.}
In the context of maximum likelihood estimation, \cite{Burnhametal} suggest averaging the estimators resulting from a number of competing models using exponential weights of the form $w_k = \exp(-I_k/2)/\sum_{i=1}^K \exp(-I_i/2)$ where $I_k$ is an information criterion evaluated for model $k$, and $i$ indexes the set of $K$ candidate models. This expression, constructed by an analogy with Bayesian model averaging, gives more weight to models with lower values of the information criterion but non-zero weight to all models. Applying this idea to the moment selection criteria given above, consider
	\begin{equation}	
		\widehat{\omega}_S = \left.\exp\left\{-\frac{\kappa}{2} \mbox{MSC}(S)\right\}\right/\sum_{S' \in \mathscr{S}}\exp\left\{-\frac{\kappa}{2} \mbox{MSC}(S')\right\}
		\label{eq:expweight}
\end{equation}
where MSC$(\cdot)$ is a moment selection criterion and the parameter $\kappa$ varies the uniformity of the weighting. As $\kappa \rightarrow 0$ the weights become more uniform; as $\kappa \rightarrow \infty$ they approach the moment selection procedure given by minimizing the corresponding criterion. Table \ref{tab:avg} compares moment averaging against moment selection by substituting FMSC, GMM-AIC, BIC and HQ into Equation \ref{eq:expweight} using the simulation experiment described in Section \ref{sec:fmscsim}. Calculations are based on 10,000 replications, each with a sample size of 500. For FMSC averaging $\kappa = 1/100$ to account for the fact that the FMSC is generally more variable than criteria based on the $J$-test. Weights for GMM-BIC, HQ, and AIC averaging set $\kappa = 1$. Both in terms of average and worst-case RMSE, moment selection is inferior to moment averaging. The only exception is worst-case RMSE for the FMSC. (Pointwise comparisons are available upon request.) If our goal is estimators with low RMSE, moment averaging may be preferable to moment selection. 


% latex.default(RMSE.average.vs.select, file = outfilename, greek = TRUE,      numeric.dollar = FALSE, na.blank = TRUE, landscape = FALSE,      append = TRUE) 
%
\begin{table}[!tbp]
 \begin{center}
\caption{Average and worst-case RMSE of moment averaging versus selection.}
\label{tab:avg}
\small
 \begin{tabular}{lrr}\hline\hline
\multicolumn{1}{l}{Average RMSE}&\multicolumn{1}{c}{Averaging}&\multicolumn{1}{c}{Selection}\tabularnewline
\hline
FMSC&0.24&0.26\tabularnewline
GMM-BIC&0.26&0.29\tabularnewline
GMM-HQ&0.26&0.29\tabularnewline
GMM-AIC&0.26&0.28\tabularnewline
\hline
\multicolumn{1}{l}{Worst-Case RMSE}&\multicolumn{1}{c}{Averaging}&\multicolumn{1}{c}{Selection}\tabularnewline
\hline
FMSC&0.36&0.33\tabularnewline
GMM-BIC&0.41&0.47\tabularnewline
GMM-HQ&0.36&0.39\tabularnewline
GMM-AIC&0.33&0.35\tabularnewline
\hline
\end{tabular}
\end{center}
\footnotesize
\begin{tablenotes}
	\item  Averaging is based on $\kappa = 1/100$ for FMSC weights and $\kappa = 1$ for all other weights. Values are calculated by simulating from Equations \ref{eq:secondstage}--\ref{eq:varmatrix} with $10,000$ replications at each combination of parameter values from Table ??? and a sample size of $500$.
\end{tablenotes}
\end{table}

\subsubsection{Minimum AMSE Weights for OLS versus 2SLS Example}
The preceding example was based on the simple heuristic of ``exponential smoothing.'' 
In some applications, however, it is possible to \emph{analytically} derive weights that minimize AMSE.\footnote{I thank Bruce Hansen for suggesting this idea.} 
The OLS versus 2SLS example from Sections \ref{sec:OLSvsIVlowlevel} and \ref{sec:FMSCforOLSvsIV} is one such case. 

To begin, define an arbitrary weighted average of the OLS and 2SLS estimators from Equations \ref{eq:OLS} and \ref{eq:2SLS} by  
\begin{equation}
	\widetilde{\beta}(\omega) = \omega \widehat{\beta}_{OLS} + (1 - \omega) \widetilde{\beta}_{2SLS}
\end{equation}
where $\omega \in [0,1]$ is the weight given to the OLS estimator.
Since the weights sum to one, we have
\begin{eqnarray*}
	\sqrt{n}\left[\widehat{\beta}(\omega) - \beta \right] &=& \left[ \begin{array}
	{cc} \omega & (1 - \omega)
\end{array}\right] \left[
\begin{array}{c}
  \sqrt{n}(\widehat{\beta}_{OLS} - \beta) \\
  \sqrt{n}(\widetilde{\beta}_{2SLS} - \beta)
\end{array}
\right]\\
& \overset{d}{\rightarrow} & N\left(\mbox{Bias}\left[\widehat{\beta}(\omega)\right], Var\left[\widehat{\beta}(\omega)\right] \right)
\end{eqnarray*}
by Theorem \ref{thm:OLSvsIV}, where
\begin{eqnarray*}
	\mbox{Bias}\left[\widehat{\beta}(\omega)\right] &=& \omega \left( \frac{\tau}{\sigma_x^2} \right) \\
	 Var\left[\widehat{\beta}(\omega)\right] &=&  \frac{\sigma_\epsilon^2}{\sigma_x^2} \left[(2\omega^2 - \omega)\left( \frac{\sigma_x^2}{\gamma^2} - 1\right)+\frac{\sigma_x^2}{\gamma^2} \right]
\end{eqnarray*}
and accordingly
\begin{equation}
	\mbox{AMSE}\left[\widehat{\beta}(\omega) \right] =  \omega^2 \left(\frac{\tau^2}{\sigma_x^4} \right) + (\omega^2 - 2 \omega)\left(\frac{\sigma_\epsilon^2}{\sigma_x^2}\right)\left( \frac{\sigma_x^2}{\gamma^2} - 1\right) + \frac{\sigma_\epsilon^2}{\gamma^2}.
\end{equation}
The preceding is a globally convex function of $\omega$. 
Taking the first order condition and rearranging, we find that the unique global minimizer is
\begin{equation}
\label{eq:AMSEoptimal}
	\omega^* = \underset{\omega \in [0,1]}{\mbox{arg min}}\; \mbox{AMSE}\left[\widehat{\beta}(\omega) \right] 
	=\left[1 + \frac{\tau^2/\sigma_x^4}{\sigma_\epsilon^2(1/\gamma^2 - 1/\sigma_x^2)}\right]^{-1}
\end{equation}
In other words,
$$\omega^* = \left[1 + \frac{\mbox{ABIAS(OLS)}^2}{\mbox{AVAR(2SLS)}-\mbox{AVAR(OLS)}} \right]^{-1}$$

The preceding expression has several important consequences. 
First, since the variance of the 2SLS estimator is always strictly greater than that of the OLS estimator, the optimal value of $\omega$ \emph{cannot} be zero. 
No matter how strong the endogeneity of $x$ as measured by $\tau$, we should always give some weight to the OLS estimator. 
Second, when $\tau = 0$ the optimal value of $\omega$ is one. If $x$ is exogenous, OLS is strictly preferable to 2SLS. 
Third, the optimal weights depend on the strength of the instruments $\mathbf{z}$ as measured by $\gamma$. 
For a given value of $\tau\neq 0$, the stronger the instruments, the less weight we should give to OLS.

Equation \ref{eq:AMSEoptimal} gives the AMSE-optimal weighted average of the OLS and 2SLS estimators. 
To actually use the corresponding moment average estimator in practice, however, we need to estimate the unknowns.
As discussed above in Section \ref{sec:FMSCforOLSvsIV} the usual estimators of $\sigma_x^2$ and $\gamma$ remain consistent under local mis-specification, and the residuals from the 2SLS estimator provide a robust estimator of $\sigma_\epsilon^2$.
As before, the problem is estimating $\tau^2$.
A natural idea is to substitute the asymptotically unbiased estimator that arises from Theorem \ref{thm:tauOLSvsIV}, namely $\widehat{\tau}^2 - \widehat{V}$. 
The problem with this approach is that, while $\tau^2$ is always greater than or equal to zero as is $\widehat{\tau}^2$, the difference $\widehat{\tau}^2 - \widehat{V}$ \emph{can easily be negative}, yielding a \emph{negative} estimate of $\mbox{ABIAS(OLS)}^2$.
To solve this problem, we borrow an idea from the literature on shrinkage estimation and use the \emph{positive part} instead, namely $\max\left\{0, \; \widehat{\tau}^2 - \widehat{V}\right\}$, as in the positive-part James-Stein estimator.
This ensures that our estimator of $\omega^*$ lies inside the interval $[0,1]$.
Accordingly, we define 
\begin{equation}
	\widehat{\beta}^*_{AVG} = \widehat{\omega}^* \widehat{\beta}_{OLS} + (1 - \widehat{\omega}^*)\widetilde{\beta}_{2SLS}
\end{equation}
where
\begin{equation}
\widehat{\omega }^* = \left[1 + \frac{\max \left\{0, \; \left(\widehat{\tau}^2 - \widehat{\sigma}_\epsilon^2\widehat{\sigma}_x^2  \left(\widehat{\sigma}_x^2/\widehat{\gamma}^2 - 1 \right) \right)/\;\widehat{\sigma}_x^4 \right\}}{\widehat{\sigma}_\epsilon^2 (1/\widehat{\gamma}^2 - 1/\widehat{\sigma}_x^2)}\right]^{-1}
\end{equation}
 
\todo[inline]{Add simulation study. Also talk about how the estimator of the weight isn't asymptotically unbiased and how we might want to try other approaches, a question we leave for future research. This section is meant to illustrate the possibilities for moment averaging.
In the simulations, it works well. How to get a CI for such a procedure? We'll see in the next section.}