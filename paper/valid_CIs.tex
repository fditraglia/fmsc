%!TEX root = main.tex

\subsection{Valid Confidence Intervals}
While Corollary \ref{cor:momentavg} characterizes the limiting behavior of moment-average, and hence post-selection estimators, the limiting random variable $\Lambda(\tau)$ is a complicated function of the normal random vector $M$. 
Because this distribution is analytically intractable, I adapt a suggestion from \cite{ClaeskensHjortbook} and approximate it by simulation.
The result is a conservative procedure that provides asymptotically valid confidence intervals for moment average and hence post-conservative selection estimators.\footnote{Although I originally developed this procedure by analogy to \cite{ClaeskensHjortbook}, \cite{Leeb} kindly pointed out that constructions of the kind given here have appeared elsewhere in the statistics literature, notably in \cite{Loh1985}, \cite{Berger1994}, and \cite{Silvapulle1996}. 
More recently, \cite{McCloskey} uses a similar approach to study non-standard testing problems.}
 
First, suppose that $K_S$, $\varphi_S$, $\theta_0$, $\Omega$ and $\tau$ were known. 
Then, by simulating from $M$, as defined in Theorem \ref{thm:normality}, the distribution of $\Lambda(\tau)$, defined in Corollary \ref{cor:momentavg}, could be approximated to arbitrary precision. 
To operationalize this procedure, substitute consistent estimators of $K_S$, $\theta_0$, and $\Omega$, e.g.\ those used to calculate FMSC. 
To estimate $\varphi_S$, we first need to derive the limit distribution of $\widehat{\omega}_S$, the data-based weights specified by the user. 
As an example, consider the case of moment selection based on the FMSC. Here $\widehat{\omega}_S$ is simply the indicator function
\begin{equation}
	\label{eq:FMSCindicate}
	\widehat{\omega}_S = \mathbf{1}\left\{\mbox{FMSC}_n(S) = \min_{S'\in \mathscr{S}} \mbox{FMSC}_n(S')\right\}
\end{equation}
To estimate $\varphi_S$, we first substitute consistent estimators of $\Omega$, $K_S$ and $\theta_0$ into $\mbox{FMSC}_S(\tau,M)$, defined in Corollary \ref{cor:FMSClimit}, yielding,
\begin{equation}
	\widehat{\mbox{FMSC}}_S(\tau,M) = \nabla_\theta\mu(\widehat{\theta})'\widehat{K}_S\Xi_S \left\{\left[\begin{array}{cc}0&0\\0&\widehat{\mathcal{B}}(\tau,M) \end{array}\right] + \widehat{\Omega}\right\}\Xi_S'\widehat{K}_S'\nabla_\theta\mu(\widehat{\theta}).
\end{equation}
where
\begin{equation}
	\widehat{\mathcal{B}}(\tau,M) = (\widehat{\Psi} M + \tau)(\widehat{\Psi} M + \tau)' - \widehat{\Psi} \widehat{\Omega} \widehat{\Psi}
\end{equation}
Combining this with Equation \ref{eq:FMSCindicate},
\begin{equation}
\label{eq:omegahat}
	\widehat{\varphi}_S(\tau,M) = \mathbf{1}\left\{\widehat{\mbox{FMSC}}_S(\tau,M) = \min_{S'\in \mathscr{S}} \widehat{\mbox{FMSC}}_{S'}(\tau,M)\right\}
\end{equation}
For GMM-AIC moment selection or selection based on a downward $J$-test, $\varphi_S(\cdot,\cdot)$ may be estimated analogously, following  Theorem \ref{pro:jstat}. 

Although simulating draws from $M$, defined in Theorem \ref{thm:normality}, requires only an estimate of $\Omega$, the limit $\varphi_S$ of the weight function also depends on $\tau$. 
As discussed above, no consistent estimator of $\tau$ is available under local mis-specification: the estimator $\widehat{\tau}$ has a non-degenerate limit distribution (see Theorem \ref{thm:tau}). 
Thus, substituting $\widehat{\tau}$ for $\tau$ will give erroneous results by failing to account for the uncertainty that enters through $\widehat{\tau}$. 
The solution is to use a two-stage procedure. 
First construct a  $100(1-\delta)\%$ confidence region $\mathscr{T}(\widehat{\tau},\delta)$ for $\tau$ using Theorem \ref{thm:tau}. 
Then, for each $\tau^* \in \mathscr{T}(\widehat{\tau},\delta)$ simulate from the distribution of $\Lambda(\tau^*)$, defined in Corollary \ref{cor:momentavg}, to obtain a \emph{collection} of $(1-\alpha)\times 100\%$ confidence intervals indexed by $\tau^*$. 
Taking the lower and upper bounds of these yields a \emph{conservative} confidence interval for $\widehat{\mu}$, as defined in defined in Equation \ref{eq:avg}. 
This interval has asymptotic coverage probability of \emph{at least} $(1-\alpha-\delta)\times 100\%$.
The precise algorithm is as follows.
\begin{alg}[Simulation-based Confidence Interval for $\widehat{\mu}$]
\label{alg:conf}
\mbox{}
\begin{enumerate}
	\item For each $\tau^* \in \mathscr{T}(\widehat{\tau},\delta)$ 
		\begin{enumerate}[(i)]
			\item Generate $J$ independent draws $M_j \sim N_{p+q}( 0, \widehat{\Omega} )$
			\item Set $\Lambda_j(\tau^*) = -\nabla_\theta\mu(\widehat{\theta})'\left[\sum_{S \in \mathscr{S}} \widehat{\varphi}_S(\tau^*,M_j) \widehat{K}_S\Xi_S\right] (M_j + \tau^*)$
			\item Using the draws $\{\Lambda_j(\tau^*)\}_{j=1}^J$, calculate $\widehat{a}(\tau^*)$, $\widehat{b}(\tau^*)$ such that
		$$P\left\{ \widehat{a}(\tau^*) \leq\Lambda(\tau^*)\leq \widehat{b}(\tau^*) \right\} = 1 - \alpha$$
		\end{enumerate}
	\item Set $\displaystyle \widehat{a}_{min}(\widehat{\tau})=\min_{\tau^* \in \mathscr{T}(\widehat{\tau},\delta)} \widehat{a}(\tau^*)$ and $\displaystyle \widehat{b}_{max}(\widehat{\tau})= \max_{\tau^* \in \mathscr{T}(\widehat{\tau},\delta)} \widehat{b}(\tau^*)$ \vspace{0.5em}
	\item The confidence interval for $\mu$ is
				$\displaystyle \mbox{CI}_{sim}=\left[ \widehat{\mu} - \frac{\widehat{b}_{max}(\widehat{\tau})}{\sqrt{n}}, \;\;\; \widehat{\mu} - \frac{\widehat{a}_{min}(\widehat{\tau})}{\sqrt{n}} \right]$
\end{enumerate}
\end{alg}

\begin{thm}[Simulation-based Confidence Interval for $\widehat{\mu}$]
\label{pro:sim}
Let $\widehat{\Psi}$, $\widehat{\Omega}$, $\widehat{\theta}$, $\widehat{K}_S$, $\widehat{\varphi}_S$ be consistent estimators of $\Psi$, $\Omega$, $\theta_0$, $K_S$, $\varphi_S$ and define 
\begin{eqnarray*}
	\Delta_n(\widehat{\tau},\tau^*) &=& \left(\widehat{\tau} - \tau^*\right)' \left(\widehat{\Psi}\widehat{\Omega}\widehat{\Psi}'\right)^{-1} \left(\widehat{\tau} - \tau^*\right)\\
	\mathscr{T}(\widehat{\tau},\delta) &=& \left\{\tau^* \colon  \Delta_n(\widehat{\tau},\tau^*) \leq \chi^2_q(\delta)\right\}
\end{eqnarray*}
where $\chi^2_q(\delta)$ denotes the $1-\delta$ quantile of a $\chi^2$ distribution with $q$ degrees of freedom.
Then, the interval $\mbox{CI}_{sim}$ defined in Algorithm \ref{alg:conf} has asymptotic coverage probability no less than $1-(\alpha + \delta)$ as $J,n\rightarrow \infty$.
\end{thm}

\subsection{Simulation Study: Valid Confidence Intervals}
\todo[inline]{Need to talk about coverage versus width. Explore in the simulations and empirical example. There is a cost to moment selection. But this cost is still present when selection is carried out informally: we're just trying to make it formal here. Also talk about how we could use the same procedure to get an interval for more general moment average estimators.}

\subsubsection{OLS versus IV Example}

\begin{table}[h]
\footnotesize
\centering
	\begin{subtable}{0.48\textwidth}
		\caption{post-FMSC Estimator}
		\begin{tabular}{r|rrrrrr}
\hline\hline
 &\multicolumn{6}{c}{$\rho$} \\ 
 $N = 50$ & $0$ & $0.1$ & $0.2$ & $0.3$ & $0.4$ & $0.5$ \\ 
 \hline$0.1$ & $100$ & $100$ & $99$ & $99$ & $98$ & $97$\\ 
$0.2$ & $99$ & $99$ & $99$ & $99$ & $98$ & $97$\\ 
$\pi\quad$$0.3$ & $99$ & $99$ & $99$ & $99$ & $98$ & $96$\\ 
$0.4$ & $98$ & $98$ & $98$ & $98$ & $98$ & $95$\\ 
$0.5$ & $97$ & $98$ & $98$ & $98$ & $97$ & $94$\\ 
$0.6$ & $97$ & $97$ & $97$ & $97$ & $96$ & $94$\\ 
 \hline 
 \end{tabular}
 
 \vspace{2em} 
 
\begin{tabular}{r|rrrrrr}
\hline\hline
 &\multicolumn{6}{c}{$\rho$} \\ 
 $N = 100$ & $0$ & $0.1$ & $0.2$ & $0.3$ & $0.4$ & $0.5$ \\ 
 \hline$0.1$ & $100$ & $99$ & $99$ & $99$ & $99$ & $98$\\ 
$0.2$ & $99$ & $99$ & $99$ & $99$ & $99$ & $97$\\ 
$\pi\quad$$0.3$ & $98$ & $98$ & $99$ & $99$ & $98$ & $95$\\ 
$0.4$ & $97$ & $97$ & $98$ & $98$ & $97$ & $94$\\ 
$0.5$ & $97$ & $97$ & $98$ & $97$ & $95$ & $95$\\ 
$0.6$ & $97$ & $97$ & $97$ & $96$ & $95$ & $96$\\ 
 \hline 
 \end{tabular}
 
 \vspace{2em} 
 
\begin{tabular}{r|rrrrrr}
\hline\hline
 &\multicolumn{6}{c}{$\rho$} \\ 
 $N = 500$ & $0$ & $0.1$ & $0.2$ & $0.3$ & $0.4$ & $0.5$ \\ 
 \hline$0.1$ & $99$ & $99$ & $99$ & $99$ & $99$ & $96$\\ 
$0.2$ & $97$ & $97$ & $98$ & $99$ & $97$ & $94$\\ 
$\pi\quad$$0.3$ & $96$ & $97$ & $98$ & $97$ & $95$ & $98$\\ 
$0.4$ & $96$ & $97$ & $97$ & $95$ & $98$ & $98$\\ 
$0.5$ & $96$ & $97$ & $96$ & $97$ & $98$ & $98$\\ 
$0.6$ & $96$ & $97$ & $95$ & $97$ & $97$ & $96$\\ 
 \hline 
 \end{tabular}
	\end{subtable}	
	~
	\begin{subtable}{0.48\textwidth}
		\caption{AMSE-Averaging Estimator}
		\begin{tabular}{r|rrrrrr}
\hline\hline
 &\multicolumn{6}{c}{$\rho$} \\ 
 $N = 250$ & $0$ & $0.1$ & $0.2$ & $0.3$ & $0.4$ & $0.5$ \\ 
 \hline$0.1$ & $100$ & $100$ & $100$ & $99$ & $99$ & $97$\\ 
$0.2$ & $99$ & $99$ & $99$ & $99$ & $97$ & $94$\\ 
$\pi\quad$$0.3$ & $97$ & $98$ & $98$ & $98$ & $95$ & $93$\\ 
$0.4$ & $97$ & $97$ & $97$ & $95$ & $92$ & $93$\\ 
$0.5$ & $96$ & $96$ & $94$ & $92$ & $91$ & $91$\\ 
$0.6$ & $95$ & $94$ & $92$ & $88$ & $89$ & $89$\\ 
 \hline 
 \end{tabular}
 
 \vspace{2em} 
 
\begin{tabular}{r|rrrrrr}
\hline\hline
 &\multicolumn{6}{c}{$\rho$} \\ 
 $N = 500$ & $0$ & $0.1$ & $0.2$ & $0.3$ & $0.4$ & $0.5$ \\ 
 \hline$0.1$ & $100$ & $100$ & $100$ & $99$ & $98$ & $96$\\ 
$0.2$ & $99$ & $99$ & $99$ & $98$ & $96$ & $94$\\ 
$\pi\quad$$0.3$ & $98$ & $97$ & $98$ & $96$ & $94$ & $95$\\ 
$0.4$ & $97$ & $97$ & $96$ & $93$ & $94$ & $95$\\ 
$0.5$ & $96$ & $95$ & $93$ & $91$ & $93$ & $92$\\ 
$0.6$ & $95$ & $94$ & $90$ & $89$ & $90$ & $90$\\ 
 \hline 
 \end{tabular}
 
 \vspace{2em} 
 
\begin{tabular}{r|rrrrrr}
\hline\hline
 &\multicolumn{6}{c}{$\rho$} \\ 
 $N = 1000$ & $0$ & $0.1$ & $0.2$ & $0.3$ & $0.4$ & $0.5$ \\ 
 \hline$0.1$ & $100$ & $99$ & $99$ & $99$ & $97$ & $95$\\ 
$0.2$ & $98$ & $99$ & $98$ & $96$ & $95$ & $96$\\ 
$\pi\quad$$0.3$ & $97$ & $98$ & $96$ & $94$ & $96$ & $97$\\ 
$0.4$ & $97$ & $97$ & $94$ & $94$ & $95$ & $95$\\ 
$0.5$ & $96$ & $95$ & $92$ & $93$ & $94$ & $93$\\ 
$0.6$ & $95$ & $92$ & $89$ & $90$ & $91$ & $87$\\ 
 \hline 
 \end{tabular}
	\end{subtable}
	\caption{Actual coverage probabilities of simulation-based interval (nominal $>90\%$)}
\end{table}


\begin{table}[h]
\footnotesize
\centering
	\begin{subtable}{0.48\textwidth}
		\caption{Two-Stage Least Squares}
		\begin{tabular}{r|rrrrrr}
\hline\hline
 &\multicolumn{6}{c}{$\rho$} \\ 
 $N = 50$ & $0$ & $0.1$ & $0.2$ & $0.3$ & $0.4$ & $0.5$ \\ 
 \hline$0.1$ & $98$ & $98$ & $96$ & $93$ & $89$ & $82$\\ 
$0.2$ & $97$ & $97$ & $95$ & $93$ & $88$ & $83$\\ 
$\pi\quad$$0.3$ & $96$ & $96$ & $94$ & $92$ & $88$ & $85$\\ 
$0.4$ & $94$ & $93$ & $93$ & $91$ & $89$ & $87$\\ 
$0.5$ & $92$ & $92$ & $92$ & $91$ & $90$ & $88$\\ 
$0.6$ & $91$ & $91$ & $90$ & $90$ & $90$ & $88$\\ 
 \hline 
 \end{tabular}
 
 \vspace{2em} 
 
\begin{tabular}{r|rrrrrr}
\hline\hline
 &\multicolumn{6}{c}{$\rho$} \\ 
 $N = 100$ & $0$ & $0.1$ & $0.2$ & $0.3$ & $0.4$ & $0.5$ \\ 
 \hline$0.1$ & $98$ & $98$ & $97$ & $94$ & $89$ & $83$\\ 
$0.2$ & $96$ & $96$ & $95$ & $92$ & $89$ & $85$\\ 
$\pi\quad$$0.3$ & $94$ & $94$ & $93$ & $91$ & $89$ & $87$\\ 
$0.4$ & $92$ & $92$ & $92$ & $91$ & $90$ & $88$\\ 
$0.5$ & $91$ & $91$ & $90$ & $90$ & $89$ & $89$\\ 
$0.6$ & $90$ & $90$ & $90$ & $90$ & $90$ & $89$\\ 
 \hline 
 \end{tabular}
 
 \vspace{2em} 
 
\begin{tabular}{r|rrrrrr}
\hline\hline
 &\multicolumn{6}{c}{$\rho$} \\ 
 $N = 500$ & $0$ & $0.1$ & $0.2$ & $0.3$ & $0.4$ & $0.5$ \\ 
 \hline$0.1$ & $96$ & $96$ & $94$ & $93$ & $90$ & $86$\\ 
$0.2$ & $92$ & $92$ & $91$ & $91$ & $90$ & $89$\\ 
$\pi\quad$$0.3$ & $91$ & $91$ & $91$ & $91$ & $90$ & $90$\\ 
$0.4$ & $90$ & $90$ & $91$ & $90$ & $90$ & $90$\\ 
$0.5$ & $90$ & $90$ & $90$ & $90$ & $90$ & $90$\\ 
$0.6$ & $90$ & $91$ & $90$ & $90$ & $90$ & $90$\\ 
 \hline 
 \end{tabular}
	\end{subtable}	
	~
	\begin{subtable}{0.48\textwidth}
		\caption{Na\"{i}ve post-FMSC}
		\begin{tabular}{r|rrrrrr}
\hline\hline
 &\multicolumn{6}{c}{$\rho$} \\ 
 $N = 50$ & $0$ & $0.1$ & $0.2$ & $0.3$ & $0.4$ & $0.5$ \\ 
 \hline$0.1$ & $88$ & $80$ & $58$ & $30$ & $11$ & $4$\\ 
$0.2$ & $88$ & $79$ & $59$ & $34$ & $15$ & $10$\\ 
$\pi\quad$$0.3$ & $87$ & $81$ & $62$ & $39$ & $25$ & $23$\\ 
$0.4$ & $86$ & $80$ & $66$ & $46$ & $38$ & $43$\\ 
$0.5$ & $86$ & $81$ & $68$ & $56$ & $54$ & $62$\\ 
$0.6$ & $85$ & $81$ & $72$ & $66$ & $67$ & $75$\\ 
 \hline 
 \end{tabular}
 
 \vspace{2em} 
 
\begin{tabular}{r|rrrrrr}
\hline\hline
 &\multicolumn{6}{c}{$\rho$} \\ 
 $N = 100$ & $0$ & $0.1$ & $0.2$ & $0.3$ & $0.4$ & $0.5$ \\ 
 \hline$0.1$ & $88$ & $72$ & $36$ & $10$ & $4$ & $4$\\ 
$0.2$ & $87$ & $74$ & $40$ & $17$ & $13$ & $19$\\ 
$\pi\quad$$0.3$ & $86$ & $74$ & $45$ & $29$ & $32$ & $45$\\ 
$0.4$ & $85$ & $74$ & $51$ & $43$ & $54$ & $70$\\ 
$0.5$ & $85$ & $76$ & $59$ & $57$ & $70$ & $84$\\ 
$0.6$ & $85$ & $78$ & $66$ & $68$ & $81$ & $88$\\ 
 \hline 
 \end{tabular}
 
 \vspace{2em} 
 
\begin{tabular}{r|rrrrrr}
\hline\hline
 &\multicolumn{6}{c}{$\rho$} \\ 
 $N = 500$ & $0$ & $0.1$ & $0.2$ & $0.3$ & $0.4$ & $0.5$ \\ 
 \hline$0.1$ & $87$ & $31$ & $8$ & $12$ & $16$ & $24$\\ 
$0.2$ & $84$ & $35$ & $24$ & $42$ & $62$ & $80$\\ 
$\pi\quad$$0.3$ & $83$ & $42$ & $43$ & $70$ & $87$ & $90$\\ 
$0.4$ & $84$ & $49$ & $62$ & $86$ & $90$ & $90$\\ 
$0.5$ & $84$ & $57$ & $76$ & $89$ & $90$ & $90$\\ 
$0.6$ & $86$ & $66$ & $84$ & $90$ & $90$ & $90$\\ 
 \hline 
 \end{tabular}
	\end{subtable}
	\caption{Actual coverage probabilities nominal 90\% CIs}
\end{table}


\begin{table}[h]
\footnotesize
\centering
	\begin{subtable}{0.48\textwidth}
		\caption{post-FMSC Estimator}
		\begin{tabular}{r|rrrrrr}
\hline\hline
 &\multicolumn{6}{c}{$\rho$} \\ 
 $N = 250$ & $0$ & $0.1$ & $0.2$ & $0.3$ & $0.4$ & $0.5$ \\ 
 \hline$0.1$ & $40$ & $40$ & $41$ & $41$ & $42$ & $42$\\ 
$0.2$ & $42$ & $42$ & $43$ & $45$ & $46$ & $48$\\ 
$\pi\quad$$0.3$ & $43$ & $43$ & $44$ & $46$ & $48$ & $49$\\ 
$0.4$ & $43$ & $44$ & $44$ & $45$ & $45$ & $46$\\ 
$0.5$ & $43$ & $43$ & $42$ & $42$ & $43$ & $42$\\ 
$0.6$ & $42$ & $41$ & $39$ & $39$ & $38$ & $35$\\ 
 \hline 
 \end{tabular}
 
 \vspace{2em} 
 
\begin{tabular}{r|rrrrrr}
\hline\hline
 &\multicolumn{6}{c}{$\rho$} \\ 
 $N = 500$ & $0$ & $0.1$ & $0.2$ & $0.3$ & $0.4$ & $0.5$ \\ 
 \hline$0.1$ & $40$ & $41$ & $41$ & $42$ & $44$ & $46$\\ 
$0.2$ & $42$ & $43$ & $44$ & $47$ & $50$ & $52$\\ 
$\pi\quad$$0.3$ & $43$ & $43$ & $46$ & $48$ & $49$ & $49$\\ 
$0.4$ & $43$ & $44$ & $45$ & $45$ & $46$ & $44$\\ 
$0.5$ & $43$ & $43$ & $42$ & $42$ & $39$ & $27$\\ 
$0.6$ & $42$ & $40$ & $39$ & $37$ & $28$ & $19$\\ 
 \hline 
 \end{tabular}
 
 \vspace{2em} 
 
\begin{tabular}{r|rrrrrr}
\hline\hline
 &\multicolumn{6}{c}{$\rho$} \\ 
 $N = 1000$ & $0$ & $0.1$ & $0.2$ & $0.3$ & $0.4$ & $0.5$ \\ 
 \hline$0.1$ & $41$ & $41$ & $43$ & $45$ & $47$ & $50$\\ 
$0.2$ & $42$ & $44$ & $47$ & $50$ & $52$ & $53$\\ 
$\pi\quad$$0.3$ & $43$ & $44$ & $48$ & $49$ & $49$ & $45$\\ 
$0.4$ & $43$ & $44$ & $46$ & $45$ & $37$ & $21$\\ 
$0.5$ & $43$ & $43$ & $42$ & $37$ & $21$ & $19$\\ 
$0.6$ & $42$ & $39$ & $38$ & $24$ & $19$ & $19$\\ 
 \hline 
 \end{tabular}
	\end{subtable}	
	~
	\begin{subtable}{0.48\textwidth}
		\caption{AMSE-Averaging Estimator}
		\begin{tabular}{r|rrrrrr}
\hline\hline
 &\multicolumn{6}{c}{$\rho$} \\ 
 $N = 250$ & $0$ & $0.1$ & $0.2$ & $0.3$ & $0.4$ & $0.5$ \\ 
 \hline$0.1$ & $30$ & $35$ & $36$ & $35$ & $35$ & $35$\\ 
$0.2$ & $31$ & $34$ & $35$ & $32$ & $38$ & $35$\\ 
$\pi\quad$$0.3$ & $35$ & $33$ & $34$ & $35$ & $35$ & $35$\\ 
$0.4$ & $34$ & $34$ & $35$ & $35$ & $35$ & $37$\\ 
$0.5$ & $35$ & $34$ & $34$ & $32$ & $35$ & $34$\\ 
$0.6$ & $36$ & $33$ & $31$ & $32$ & $33$ & $30$\\ 
 \hline 
 \end{tabular}
 
 \vspace{2em} 
 
\begin{tabular}{r|rrrrrr}
\hline\hline
 &\multicolumn{6}{c}{$\rho$} \\ 
 $N = 500$ & $0$ & $0.1$ & $0.2$ & $0.3$ & $0.4$ & $0.5$ \\ 
 \hline$0.1$ & $32$ & $27$ & $33$ & $37$ & $30$ & $36$\\ 
$0.2$ & $33$ & $37$ & $33$ & $36$ & $36$ & $39$\\ 
$\pi\quad$$0.3$ & $34$ & $34$ & $34$ & $38$ & $37$ & $38$\\ 
$0.4$ & $35$ & $35$ & $36$ & $37$ & $37$ & $35$\\ 
$0.5$ & $37$ & $33$ & $34$ & $35$ & $33$ & $31$\\ 
$0.6$ & $35$ & $32$ & $31$ & $32$ & $30$ & $25$\\ 
 \hline 
 \end{tabular}
 
 \vspace{2em} 
 
\begin{tabular}{r|rrrrrr}
\hline\hline
 &\multicolumn{6}{c}{$\rho$} \\ 
 $N = 1000$ & $0$ & $0.1$ & $0.2$ & $0.3$ & $0.4$ & $0.5$ \\ 
 \hline$0.1$ & $30$ & $30$ & $37$ & $34$ & $37$ & $37$\\ 
$0.2$ & $35$ & $37$ & $35$ & $36$ & $41$ & $41$\\ 
$\pi\quad$$0.3$ & $34$ & $34$ & $36$ & $36$ & $37$ & $38$\\ 
$0.4$ & $37$ & $34$ & $37$ & $35$ & $34$ & $28$\\ 
$0.5$ & $36$ & $34$ & $34$ & $34$ & $27$ & $22$\\ 
$0.6$ & $37$ & $32$ & $31$ & $28$ & $23$ & $20$\\ 
 \hline 
 \end{tabular}
	\end{subtable}
	\caption{Median CI width relative to that of 2SLS (\%)}
\end{table}


\subsubsection{Choosing Instrumental Variables Example}

To evaluate the performance of the procedure given in Algorithm \ref{alg:conf}, we revisit the simulation experiment described in Section \ref{sec:fmscsim}, considering FMSC moment selection. The following results are based on 10,000 replications, each with a sample size of 500. Table \ref{tab:FMSCconf} gives the empirical coverage probabilities of traditional 95\% confidence intervals post-FMSC selection. These are far below the nominal level over the vast majority of the parameter space. Table \ref{tab:FMSCcorrect} presents the empirical coverage of conservative 90\% confidence intervals constructed according to Algorithm \ref{alg:conf}, with $B=1000$.\footnote{Because this simulation is computationally intensive, I use a reduced grid of parameter values.} The two-stage simulation procedure performs remarkably well, achieving a minimum coverage probability of $0.89$ relative to its nominal level of $0.9$. Moreover, a na\"{i}ve one-step procedure that omits the first-stage and simply simulates from $M$ based on $\widehat{\tau}$ performs surprisingly well; see Table \ref{tab:FMSCnaive}. While the empirical coverage probabilities of the one-step procedure are generally lower than the nominal level of $0.95$, they represent a substantial improvement over the traditional intervals given in Table \ref{tab:FMSCconf}, with a worst-case coverage of $0.72$ compared to $0.15$. This suggests that the one-step intervals might be used as a rough but useful approximation to the correct but more computationally intensive intervals constructed according to Algorithm \ref{alg:conf}.

% latex.default(cover.FMSC, file = outfilename, greek = TRUE, numeric.dollar = FALSE,      na.blank = TRUE, landscape = FALSE, rowname = NULL, append = TRUE) 
%
\begin{table}[!tbp]
 \begin{center}
 \caption{Coverage post-FMSC moment selection (nominal 95\%).}
\label{tab:FMSCconf}
\small
 \begin{tabular}{r|rrrrrrrrr}\hline\hline
&\multicolumn{9}{c}{$\rho = Cov(w,u)$}\\
\multicolumn{1}{c|}{$N=500$}&\multicolumn{1}{c}{0}&\multicolumn{1}{c}{0.05}&\multicolumn{1}{c}{0.10}&\multicolumn{1}{c}{0.15}&\multicolumn{1}{c}{0.20}&\multicolumn{1}{c}{0.25}&\multicolumn{1}{c}{0.30}&\multicolumn{1}{c}{0.35}&\multicolumn{1}{c}{0.40}\tabularnewline
\hline
0.0&0.92&0.93&0.93&0.93&0.93&0.93&0.93&0.93&0.93\tabularnewline
0.1&0.91&0.87&0.88&0.91&0.93&0.93&0.93&0.93&0.93\tabularnewline
0.2&0.90&0.79&0.72&0.82&0.90&0.93&0.92&0.93&0.93\tabularnewline
0.3&0.90&0.76&0.58&0.64&0.80&0.90&0.92&0.93&0.93\tabularnewline
0.4&0.89&0.75&0.50&0.47&0.64&0.80&0.88&0.91&0.92\tabularnewline
0.5&0.89&0.74&0.45&0.36&0.50&0.67&0.79&0.87&0.91\tabularnewline
0.6&0.89&0.74&0.43&0.30&0.38&0.54&0.68&0.78&0.85\tabularnewline
\multirow{4}{5mm}{\begin{sideways}\parbox{1mm}{$\gamma\;$=$\;Cov(w,x)$}\end{sideways}}
0.7&0.90&0.74&0.41&0.24&0.31&0.44&0.57&0.68&0.78\tabularnewline
0.8&0.89&0.74&0.41&0.22&0.25&0.36&0.48&0.59&0.70\tabularnewline
0.9&0.91&0.74&0.41&0.20&0.21&0.31&0.41&0.52&0.61\tabularnewline
1.0&0.90&0.75&0.40&0.18&0.19&0.25&0.35&0.45&0.53\tabularnewline
1.1&0.90&0.76&0.40&0.17&0.17&0.23&0.32&0.39&0.47\tabularnewline
1.2&0.91&0.76&0.41&0.17&0.15&0.20&0.27&0.34&0.42\tabularnewline
1.3&0.92&0.77&0.41&0.16&0.15&0.19&0.24&0.31&0.39\tabularnewline
\hline
\end{tabular}
\end{center}
\footnotesize
\begin{tablenotes}
\item Values are calculated by simulating from Equations \ref{eq:secondstage}--\ref{eq:varmatrix} with $10,000$ replications.
\end{tablenotes}
\end{table}



% latex.default(conf.refined, file = outfilename, greek = TRUE,      numeric.dollar = FALSE, na.blank = TRUE, landscape = FALSE,      rowname = NULL, append = FALSE) 
%
\begin{table}[!tbp]
 \begin{center}
\caption{Coverage of conservative two-step interval post-FMSC (nominal $>90\%$)}
\label{tab:FMSCcorrect}
\small
 \begin{tabular}{r|rrrrr}\hline\hline
&\multicolumn{5}{c}{$\rho = Cov(w,u)$}\\
\multicolumn{1}{c|}{$N=500$}&\multicolumn{1}{c}{0}&\multicolumn{1}{c}{0.1}&\multicolumn{1}{c}{0.2}&\multicolumn{1}{c}{0.3}&\multicolumn{1}{c}{0.4}\tabularnewline
\hline
0.0&0.92&0.93&0.93&0.93&0.94\tabularnewline
0.2&0.95&0.91&0.93&0.95&0.97\tabularnewline
0.4&0.95&0.95&0.90&0.93&0.97\tabularnewline
0.6&0.95&0.95&0.92&0.90&0.92\tabularnewline
\multirow{4}{5mm}{\begin{sideways}\parbox{1mm}{$\gamma\;$=$\;Cov(w,x)$}\end{sideways}}
0.8&0.94&0.95&0.96&0.90&0.89\tabularnewline
1.0&0.94&0.94&0.96&0.93&0.90\tabularnewline
1.2&0.94&0.94&0.96&0.95&0.92\tabularnewline
\hline
\end{tabular}
\end{center}
\footnotesize
\begin{tablenotes}
	\item Intervals are calculated using Algorithm \ref{alg:conf} with $B = 1000$. Simulations are generated from Equations \ref{eq:secondstage}--\ref{eq:varmatrix} with $10,000$ replications.
\end{tablenotes}
\end{table}


% latex.default(conf.naive, file = outfilename, greek = TRUE, numeric.dollar = FALSE,      na.blank = TRUE, landscape = FALSE, rowname = NULL, append = FALSE) 
%
\begin{table}[!tbp]
\caption{Coverage of na\"{i}ve one-step interval post-FMSC (nominal 95\%)}
\label{tab:FMSCnaive}
\small
 \begin{center}
 \begin{tabular}{r|rrrrrrrrr}\hline\hline
&\multicolumn{9}{c}{$\rho = Cov(w,u)$}\\
\multicolumn{1}{c|}{$N=500$}&\multicolumn{1}{c}{0}&\multicolumn{1}{c}{0.05}&\multicolumn{1}{c}{0.10}&\multicolumn{1}{c}{0.15}&\multicolumn{1}{c}{0.20}&\multicolumn{1}{c}{0.25}&\multicolumn{1}{c}{0.30}&\multicolumn{1}{c}{0.35}&\multicolumn{1}{c}{0.40}\tabularnewline
\hline
0.0&0.93&0.92&0.93&0.93&0.93&0.93&0.93&0.93&0.94\tabularnewline
0.1&0.93&0.91&0.91&0.92&0.92&0.92&0.93&0.94&0.95\tabularnewline
0.2&0.94&0.91&0.86&0.87&0.92&0.93&0.94&0.95&0.96\tabularnewline
0.3&0.95&0.94&0.87&0.81&0.85&0.91&0.94&0.96&0.96\tabularnewline
0.4&0.95&0.95&0.91&0.82&0.77&0.84&0.90&0.94&0.95\tabularnewline
0.5&0.95&0.95&0.93&0.86&0.76&0.76&0.82&0.88&0.92\tabularnewline
0.6&0.94&0.94&0.94&0.90&0.80&0.74&0.75&0.81&0.87\tabularnewline
\multirow{4}{5mm}{\begin{sideways}\parbox{1mm}{$\gamma\;$=$\;Cov(w,x)$}\end{sideways}}
0.7&0.94&0.94&0.95&0.93&0.85&0.74&0.73&0.75&0.81\tabularnewline
0.8&0.94&0.94&0.95&0.94&0.88&0.79&0.73&0.73&0.76\tabularnewline
0.9&0.95&0.94&0.94&0.94&0.91&0.83&0.76&0.72&0.73\tabularnewline
1.0&0.95&0.94&0.94&0.94&0.92&0.86&0.78&0.73&0.73\tabularnewline
1.1&0.95&0.94&0.94&0.95&0.94&0.89&0.81&0.76&0.73\tabularnewline
1.2&0.95&0.94&0.94&0.95&0.94&0.90&0.85&0.79&0.75\tabularnewline
1.3&0.95&0.94&0.94&0.95&0.95&0.92&0.87&0.81&0.78\tabularnewline
\hline
\end{tabular}
\end{center}
\footnotesize
	\begin{tablenotes}
		\item Intervals are calculated by simulation with $B=1000$ using $\widehat{\tau}$ rather than constructing a confidence interval for $\tau$ (c.f.\ Algorithm \ref{alg:conf}). Simulations are generated from Equations \ref{eq:secondstage}--\ref{eq:varmatrix} with $10,000$ replications.
	\end{tablenotes}
\end{table}