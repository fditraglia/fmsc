%!TEX root = main.tex
\subsection{OLS versus TSLS Example}
\label{sec:OLSvsIVExample}
The simplest interesting application of the FMSC is choosing between ordinary least squares (OLS) and two-stage least squares (TSLS) estimators of the effect $\beta$ of a single endogenous regressor $x$ on an outcome of interest $y$.
The intuition is straightforward: because TSLS is a high-variance estimator, OLS will have a lower mean-squared error provided that $x$ isn't \emph{too} endogenous.\footnote{Because the moments of the TSLS estimator only exist up to the order of overidentificiation \citep{Phillips1980, Kinal} mean-squared error should be understood to refer to ``trimmed'' mean-squared error when the number of instruments is two or fewer. For details, see Online Appendix \ref{append:trim}.}
To keep the presentation transparent, I work within an iid, homoskedastic setting for this example and assume, without loss of generality, that there are no exogenous regressors.\footnote{The homoskedasticity assumption concerns the \emph{limit} random variables: under local mis-specification there will be heteroskedasticity for fixed $n$. See Assumption \ref{assump:OLSvsIV} in Online Appendix \ref{sec:sufficient_conditions} for details.}
Equivalently we may suppose that any exogenous regressors, including a constant, have been ``projected out.''
Low-level sufficient conditions for all of the results in this section appear in Assumption \ref{assump:OLSvsIV} of Online Appendix \ref{sec:sufficient_conditions}.
The data generating process is
    \begin{eqnarray}
			y_{ni} &=& \beta x_{ni}  + \epsilon_{ni}\\
	x_{ni} &=& \mathbf{z}_{ni}' \boldsymbol{\pi} + v_{ni}
	\end{eqnarray}
where $\beta$ and $\boldsymbol{\pi}$ are unknown constants, $\mathbf{z}_{ni}$ is a vector of exogenous and relevant instruments, $x_{ni}$ is the endogenous regressor, $y_{ni}$ is the outcome of interest, and $\epsilon_{ni}, v_{ni}$ are unobservable error terms.
All random variables in this system are mean zero, or equivalently all constant terms have been projected out. 
Stacking observations in the usual way, the estimators under consideration are $\widehat{\beta}_{OLS} = \left(\mathbf{x}'\mathbf{x}\right)^{-1}\mathbf{x}'\mathbf{y}$ and
$\widetilde{\beta}_{TSLS} = \left(\mathbf{x}'P_Z\mathbf{x}\right)^{-1}\mathbf{x}'P_Z\mathbf{y}$ where we define $P_Z = Z(Z'Z)^{-1}Z'$. 

\begin{thm}[OLS and TSLS Limit Distributions]
	\label{thm:OLSvsIV} 
  Let $(\mathbf{z}_{ni}, v_{ni}, \epsilon_{ni})$ be a triangular array of random variables such that $E[\mathbf{z}_{ni} \epsilon_{ni}]=\mathbf{0}$, $E[\mathbf{z}_{ni} v_{ni}]=\mathbf{0}$, and $E[\epsilon_{ni}v_{ni}] = \tau/\sqrt{n}$ for all $n$. Then, under standard regularity conditions, e.g. Assumption \ref{assump:OLSvsIV} in Online Appendix \ref{sec:sufficient_conditions}, 
	$$
\left[
\begin{array}{c}
  \sqrt{n}(\widehat{\beta}_{OLS} - \beta) \\
  \sqrt{n}(\widetilde{\beta}_{TSLS} - \beta)
\end{array}
\right] \overset{d}{\rightarrow}
N\left(
\left[
\begin{array}{c}
\tau/\sigma_x^2 \\ 
0
\end{array}
\right],\;
\sigma_\epsilon^2 \left[ \begin{array}{cc}
  1/\sigma_x^2 & 1/\sigma_x^2\\
  1/\sigma_x^2 & 1/\gamma^2 
  \end{array}\right]
  \right)
$$
where $\sigma_x^2 = \gamma^2 + \sigma_v^2$, $\gamma^2 = \boldsymbol{\pi}'Q \boldsymbol{\pi}$, $E[\mathbf{z}_{ni} \mathbf{z}_{ni}'] \rightarrow Q$, $E[v_{ni}^2]\rightarrow \sigma_v^2$, and $E[\epsilon_{ni}^2] \rightarrow \sigma_\epsilon^2$ as $n\rightarrow \infty$.
\end{thm}
We see immediately that, as expected, the variance of the OLS estimator is always strictly lower than that of the TSLS estimator since $\sigma^2_\epsilon/\sigma_x^2 = \sigma^2_\epsilon/(\gamma^2 + \sigma_v^2)$. 
Unless $\tau = 0$, however, OLS shows an asymptotic bias. 
In contrast, the TSLS estimator is asymptotically unbiased regardless of the value of $\tau$.  
Thus,
$$\mbox{AMSE(OLS)} = \frac{\tau^2}{\sigma_x^4} + \frac{\sigma_\epsilon^2}{\sigma_x^2},\quad \quad
  \mbox{AMSE(TSLS)} = \frac{\sigma_\epsilon^2}{\gamma^2}.$$
 and rerranging, we see that the AMSE of the OLS estimator is strictly less than that of the TSLS estimator whenever $\tau^2  < \sigma_x^2 \sigma_\epsilon^2\sigma_v^2/\gamma^2$. 
To estimate the unknowns required to turn this inequality into a moment selection procedure, I set 
  $$\widehat{\sigma}_x^2 = n^{-1}\mathbf{x}'\mathbf{x}, \quad \widehat{\gamma}^2 = n^{-1}\mathbf{x}'Z(Z'Z)^{-1}Z'\mathbf{x}, \quad \widehat{\sigma}_v^2 =  \widehat{\sigma}_x^2 - \widehat{\gamma}^2$$
and define
$$\widehat{\sigma}_\epsilon^2 = n^{-1}\left(\textbf{y} - \textbf{x}\widetilde{\beta}_{TSLS} \right)'\left(\textbf{y} - \textbf{x}\widetilde{\beta}_{TSLS} \right)$$
Under local mis-specification each of these estimators is consistent for its population counterpart.\footnote{While using the OLS residuals to estimate $\sigma_\epsilon^2$ \emph{also} provides a consistent estimate under local mis-specification, the estimator based on the TSLS residuals should be more robust.} 
All that remains is to estimate $\tau^2$. Specializing Theorem \ref{thm:tau} and Corollary \ref{cor:tautau} to the present example gives the following result.
\begin{thm}
	\label{thm:tauOLSvsIV}
	Let $\widehat{\tau} =  n^{-1/2} \mathbf{x}'(\mathbf{y} - \mathbf{x}\widetilde{\beta}_{TSLS})$. Then, under the conditions of Theorem \ref{thm:OLSvsIV},
	$$\widehat{\tau}\rightarrow_d N(\tau,V), \quad V = \sigma_\epsilon^2 \sigma_x^2(\sigma_v^2/\gamma^2).$$ 
\end{thm}
It follows that $\widehat{\tau}^2 -  \widehat{\sigma}_\epsilon^2\widehat{\sigma}_x^2 \left(\widehat{\sigma}_v^2/\widehat{\gamma}^2\right)$ is an asymptotically unbiased estimator of $\tau^2$ and hence, substituting into the AMSE inequality from above and rearranging, the FMSC instructs us to choose OLS whenever $\widehat{T}_{FMSC} = \widehat{\tau}^2/\widehat{V} < 2$
where $\widehat{V} = \widehat{\sigma}_v^2 \widehat{\sigma}_\epsilon^2 \widehat{\sigma}_x^2/\widehat{\gamma}^2$. 
The quantity $\widehat{T}_{FMSC}$ looks very much like a test statistic and indeed it can be viewed as such. 
By Theorem \ref{thm:tauOLSvsIV} and the continuous mapping theorem, $\widehat{T}_{FMSC} \rightarrow_d \chi^2(1)$. 
Thus, the FMSC can be viewed as a test of the null hypothesis $H_0\colon \tau = 0$ against the two-sided alternative with a critical value of $2$. 
This corresponds to a significance level of $\alpha \approx 0.16$. 
But how does this novel ``test'' compare to something more familiar, say the Durbin-Hausman-Wu (DHW) test? 
It turns out that in this particular example, although not in general, the FMSC is \emph{numerically equivalent} to using OLS unless the DHW test rejects at the 16\% level. 
\begin{thm}
    \label{thm:DHW} Under the conditions of Theorem \ref{thm:OLSvsIV}, FMSC selection between the OLS and TSLS estimators is equivalent to a Durbin-Hausman-Wu pre-test with a critical value of $2$.
\end{thm}
The equivalence between FMSC selection and a DHW test in this example is helpful for two reasons. 
First, it provides a novel justification for the use of the DHW test to select between OLS and TSLS. So long as it is carried out with $\alpha \approx 16\%$, the DHW test is equivalent to selecting the estimator that minimizes an asymptotically unbiased estimator of AMSE. 
Note that this significance level differs from the more usual values of 5\% or 10\% in that it leads us to select TSLS \emph{more often}: OLS should indeed be given the benefit of the doubt, but not by so wide a margin as traditional practice suggests. 
Second, this equivalence shows that the FMSC can be viewed as an \emph{extension} of the idea behind the familiar DHW test to more general GMM environments.\footnote{Note that the FMSC in this example, characterized in Theorem \ref{thm:DHW}, chooses between OLS and IV to minimize estimator AMSE. If one wishes to carry out inference post-selection one must contend with the size distortions of the familiar ``textbook'' confidence interval procedure, as pointed out by \cite{Guggenberger2010}. I discuss this point extensively below in Section \ref{sec:avg} ans propose possible remedies.}
